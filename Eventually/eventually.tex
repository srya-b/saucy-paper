\documentclass[acmsmall, screen, review, anonymous]{acmart}
\settopmatter{printfolios=true,printccs=false,printacmref=false}
%\IEEEoverridecommandlockouts
% The preceding line is only needed to identify funding in the first footnote. If that is unneeded, please comment it out.
%\usepackage[dvipsnames]{xcolor}
%\usepackage{cite}
\usepackage[most]{tcolorbox}
%\usepackage{amsmath,amssymb,amsfonts,amsthm}
\usepackage{algorithmic}
\usepackage{graphicx}
\usepackage{subcaption}
\usepackage{textcomp}
\usepackage{mathtools}
\usepackage[shortlabels]{enumitem}
\usepackage[T1]{fontenc}
\usepackage{listings}
\usepackage{tabularx}
\usepackage{bbm}
%\usepackage{unicode-math}
\usepackage[utf8]{inputenc}
\usepackage{newunicodechar}
\usepackage{multirow}
\usepackage{booktabs}
\usepackage{adjustbox}

%% PL packages
\usepackage{stmaryrd} 
\usepackage{proof}
\usepackage{mathpartir}
%\usepackage{color}
\usepackage{xstring}
\usepackage{xspace}
\usepackage{turnstile}

\AtBeginDocument{%
  \providecommand\BibTeX{{%
    \normalfont B\kern-0.5em{\scshape i\kern-0.25em b}\kern-0.8em\TeX}}}

%% Rights management information.  This information is sent to you
%% when you complete the rights form.  These commands have SAMPLE
%% values in them; it is your responsibility as an author to replace
%% the commands and values with those provided to you when you
%% complete the rights form.
\setcopyright{none}


%%
%% These commands are for a JOURNAL article.
\acmJournal{PACMPL}
\acmVolume{1}
\acmNumber{ICFP} 
\acmArticle{1}
\acmYear{2022}
\acmMonth{9}
\acmDOI{} % \acmDOI{10.1145/nnnnnnn.nnnnnnn}
\startPage{1}

\citestyle{acmauthoryear}


\begin{document}
\usetikzlibrary{matrix, arrows.meta, calc, positioning}
\tikzset{myarrow/.style={-Latex, rounded corners},}

\newcommand*\emptycirc[1][1ex]{\tikz\draw (0,0) circle (#1);} 
\newcommand*\halfcircleft[1][1ex]{%
  \begin{tikzpicture}
  \draw[fill] (0,0)-- (90:#1) arc (90:270:#1) -- cycle ;
  \draw (0,0) circle (#1);
  \end{tikzpicture}}
\newcommand*\halfcircright[1][1ex]{%
  \begin{tikzpicture}
  \draw[fill] (0,0)-- (0:#1) arc (0:90:#1) -- cycle ;
  \draw[fill] (0,0)-- (270:#1) arc (270:360:#1) -- cycle;
  \draw (0,0) circle (#1);
  \end{tikzpicture}}
\newcommand*\fullcirc[1][1ex]{\tikz\fill (0,0) circle (#1);} 

\newcolumntype{R}[2]{
	>{\adjustbox{angle=#1, lap=\width-(#2)}\bgroup}
	c
	<{\egroup}
}
\newcommand*\rot{\multicolumn{1}{R{30}{1.5em}}}


\definecolor{vert}{RGB}{0,181,0}
\definecolor{oran}{RGB}{223,74,0}
\definecolor{viol}{RGB}{134,0,175}
\definecolor{roug}{RGB}{215,15,0}
\definecolor{bb}{RGB}{0,0,0}
\definecolor{gg}{RGB}{220,220,220}
\definecolor{royalblue}{rgb}{0.25, 0.41, 0.88}
\definecolor{forestgreen}{rgb}{0.13, 0.55, 0.13}
\definecolor{YellowOrange}{rgb}{0.98, 0.6, 0.01}
\definecolor{Red}{rgb}{0.89, 0.0, 0.13}
\definecolor{Black}{rgb}{0.0, 0.0, 0.0}
\definecolor{Purple}{rgb}{0.63, 0.36, 0.94}
\definecolor{purp}{rgb}{0.59, 0.48, 0.71}

\newcommand{\anote}[1]{{\color{magenta}{AM: {{#1}}}}}
\newcommand{\snote}[1]{{\color{green}{SB: {{#1}}}}}

\newtcolorbox[auto counter]{bbox}[2][]{%
    colback=white,
    colframe=bb,
    %colbacktitle=white!90!roug,
	colbacktitle=white!40!gg,
    coltitle=black,
    fonttitle=\small\bfseries, 
	fontupper=\small,
	fontlower=\small,
    enhanced,
    attach boxed title to top left={yshift=-2mm, xshift=0.5cm},%
    #1,% For possible options
}

\mathchardef\hyp="2D
\mathchardef\car="5E

\makeatletter
\newcommand\BeraMonottFamily{%
	\def\fvm@scale{0.85}%
	\fontfamily{fvm}\selectfont
}
\makeatother

\title{Eventually
}

\newcommand{\mc}[1]{\ensuremath{\mathcal{#1}}}
\newcommand{\msf}[1]{\ensuremath{{\mathsf {#1}}}}
\newcommand{\mathc}[1]{\ensuremath{\mathcal{#1}}}
\newcommand{\tsc}[1]{\textsc{#1}}
\newcommand{\f}[1]{\ensuremath{\mathcal{#1}}\xspace}
\newcommand{\F}{\f{F}}
\newcommand{\PI}{\ensuremath{\pi}\xspace}
\newcommand{\RHO}{\ensuremath{\rho}\xspace}
\newcommand{\achan}{\ensuremath{\F_{\msf{achan}}^{p_r,p_s}}}
%\newcommand{\C}{\mathcal{C}}
\newcommand{\con}[1]{\msf{Contract_{#1}}}
%\newcommand{\Fsync}[2]{\ensuremath{\F_{\msf{sync},#1,#2}}}
\newcommand{\Fsync}[2]{\ensuremath{\F_{\msf{BD-SEC}}(#1,#2)}}
\newcommand{\Fchan}[2]{\ensuremath{\F_{\msf{chan}}(#1,#2)}}
\newcommand{\Fbdsec}{\ensuremath{\F_{\msf{BD-SEC}}^{\delta,\ell}}}
\newcommand{\Fbc}{\ensuremath{\F_{\msf{broadcast}}}}
\newcommand{\Fsfe}{\ensuremath{\F_{\msf{SFE}}}}
\newcommand{\Fstate}{\ensuremath{\F_{\msf{state}}}}
\newcommand{\Fclock}{\ensuremath{\F_{\msf{clock}}}}
\newcommand{\Frbc}{\ensuremath{\F_{\msf{rbc}}}}
\newcommand{\Fpay}{\ensuremath{\F_{\msf{pay}}}}
\newcommand{\Fcom}{\ensuremath{\F_{\msf{com}}}\xspace}
\newcommand{\Fauth}{\ensuremath{\F_{\msf{auth}}}\xspace}
\newcommand{\Fflip}{\ensuremath{\F_{\msf{coinflip}}}\xspace}
\newcommand{\Fro}{\ensuremath{\F_{\msf{RO}}}\xspace}
\renewcommand{\O}[1]{\ensuremath{\mathcal{O}(#1)}\xspace}
\newcommand{\kbits}{\ensuremath{\{0,1\}^k}\xspace}
\newcommand{\samplek}{\ensuremath{\xleftarrow{\$} \kbits}\xspace}
\newcommand{\Fsmc}{\ensuremath{\F_{\msf{SMC}}}\xspace}
\newcommand{\Fropp}{\ensuremath{\F_{\msf{P2P\hyp RO}}}\xspace}
\newcommand{\Gledger}{\ensuremath{\f{G}_{\msf{ledger}}}}
\newcommand{\Wsync}{\ensuremath{\mathcal{W}_{\msf{sync}}}}
\newcommand{\Wasync}{\ensuremath{\mathcal{W}_{\msf{async}}}}
\newcommand{\Ssyncbracha}{\ensuremath{\mathc{S}_{\msf{sbracha}}}}
\newcommand{\Fbracha}{\ensuremath{\mathcal{F}_{\msf{bracha}}}}
\newcommand{\Schedule}{\tsc{Schedule}}
\newcommand{\Delay}{\tsc{Delay}}
\newcommand{\Advance}{\tsc{Advance}}
\newcommand{\Exec}{\tsc{Exec}}
%\newcommand{\Adversary}{\ensuremath{\mathcal{A}}\xspace}
\newcommand{\A}{\ensuremath{\mathcal{A}}\xspace}
\newcommand{\DummyAdv}{\ensuremath{\mathcal{A}_\mathcal{D}}\xspace}
\newcommand{\DA}{\ensuremath{\A_\mathcal{D}}\xspace}
\newcommand{\Sim}{\ensuremath{\mathcal{S}}\xspace}
\newcommand{\SIM}[1]{\ensuremath{\mathcal{S}_{#1}}\xspace}
\newcommand{\simcom}{\SIM{\msf{com}}}
\newcommand{\cf}{\ensuremath{\mathcal{C}}\xspace}
\newcommand{\ID}[1]{\ensuremath{\mathcal{I}(#1)}\xspace}
%\newcommand{\Sim}[1][]{\ifthenelse{\equal{#1}{}}{\ensuremath{\Simulator}}{\ensuremath{\Simulator_{#1}}}}
\newcommand{\DS}{\SIM{D}\xspace}
%\newcommand{\Environment}{\ensuremath{\mathcal{Z}}\xspace}
\newcommand{\Z}{\ensuremath{\mathcal{Z}}\xspace}
\newcommand{\Partyi}{\ensuremath{P_i}}
\newcommand{\Partyj}{\ensuremath{P_j}}
\newcommand{\partywrapper}{multiplexer\xspace}
\newcommand{\pw}{\PI}
\newcommand{\fwrapper}{\todo{fwrappername}\xspace}

\newcommand{\dealer}{\ensuremath{\mathcal{D}}}
\newcommand{\globalf}[1]{\ensuremath{{\overline{\mathcal{#1}}}}}
\newcommand{\todo}[1]{\textcolor{Red}{todo: #1}}
\newcommand{\edict}{\{\}}
\newcommand{\lar}{\leftarrow}
\newcommand{\rar}{\rightarrow}
\newcommand{\Init}{{\bf \color{NavyBlue} Init}~}
\newcommand{\OnInput}{{\bf \textcolor{Black} On input}~}
\newcommand{\Allinputs}{{\bf \color{Cerulean} All other input~}}
\newcommand{\OnAdvInput}{{\bf \color{BrickRed} On input}~}
\newcommand{\heading}[1]{\textbf{#1}}
\newcommand{\Type}{\ensuremath{\yo{type}}}
\newcommand{\Stype}{\ensuremath{\yo{stype}}}
\newcommand{\bangf}{\ensuremath{!\F}}
\newcommand{\execuc}{\ensuremath{\msf{execUC}}}
\newcommand{\iexecuc}{\inline{execUC}}
\newcommand{\UC}[4]{\ensuremath{\execuc #1  #2  #3  #4}}
\newcommand{\idealP}{\ensuremath{\mathbbm{1}_d}\xspace}
%\newcommand{\prot}[1][]{\ifthenelse{\equal{\ensuremath{#1}}{}}{\ensuremath{\Pi}}{\ensuremath{\Pi_{X #1}}}}
\newcommand{\prot}[1]{\ensuremath{\pi_{\msf{#1}}}}
\newcommand{\lla}{\leftarrow}
\newcommand{\lvd}{\vdash}
\newcommand{\tb}[1]{\text{\color{royalblue}{#1}}}
\newcommand{\tgr}[1]{\text{\color{forestgreen}{#1}}}
\newcommand{\tm}[1]{\text{\color{magenta}{#1}}}
\newcommand{\tg}[1]{\text{\color{gray}{#1}}}
\newcommand{\tp}[1]{\text{\color{purp}{#1}}}
\newcommand{\nparam}[1]{\tp{#1}}
\newcommand{\tr}[1]{\text{\color{Red}{#1}}}
\newcommand{\yo}[1]{\text{\color{YellowOrange}{#1}}}
\newcommand{\inline}[1]{\lstinline[basicstyle=\footnotesize\BeraMonottFamily, mathescape]!#1!}
\newcommand{\nrecv}{\tb{recv}}
\newcommand{\nsend}{\tb{send}}
\newcommand{\nget}{\tb{get}}
\newcommand{\npay}{\tb{pay}}
\newcommand{\nsimget}{\tm{simget}}
\newcommand{\nsimpay}{\tm{simpay}}
\newcommand{\ncase}{\tm{case}}
\newcommand{\nproc}{\tb{proc}}
\newcommand{\nwithdraw}{\tm{withdrawTokens}}
\newcommand{\nif}{\yo{if}}
\newcommand{\nthen}{\yo{then}}
\newcommand{\nend}{\yo{end}}
\newcommand{\nwhile}{\yo{while}}


%\newcommand{\pluseq}{\mathrel{+}=}
%\newcommand{\minuseq}{\mathrel{-}=}
\newcommand{\Assert}{{\bf \color{BrickRed} Assert }}
\newcommand{\Require}{{\bf \color{BrickRed} Require }}

%\theoremstyle{acmdefinition}
%\newtheorem{definition}{Definition}[section]
\newtheorem{ddef}{Definition}
%\newtheorem{theorem}{Theorem}
\newtheorem{claim}{Claim}
%\newtheorem{lemma}{Lemma}

%\newlist{renumerate}{enumerate}{1}
%\setlist[renumerate]{before=\setlength{\baselineskip}{20pt}, itemsep=-2ex, topsep=-2ex}
%\newenvironment{renumerate}{\begin{enumerate}[before=\setlength{\baselineskip}{20pt},itemsep=-2ex,topsep=0pt]}{\end{enumerate}}
\newenvironment{renumerate}{\begin{enumerate}[nosep]}{\end{enumerate}}
%\newenvironment{ritemize}{\begin{itemize}[before=\setlength{\baselineskip}{20pt},itemsep=-2ex,topsep=0pt]}{\end{itemize}}
\newenvironment{ritemize}{\begin{itemize}[nosep] \renewcommand\labelitemi{--}}{\end{itemize}}

\newenvironment{mylst}{\begin{lstlisting}[basicstyle=\small\BeraMonottFamily, frame=single, mathescape]}{\end{lstlisting}}

\makeatletter
\newcommand{\inmsg}[1]{%
(#1\checknextarg}
\newcommand{\checknextarg}{\@ifnextchar\bgroup{\gobblenextarg}{)~}}
\newcommand{\gobblenextarg}[1]{, #1\@ifnextchar\bgroup{\gobblenextarg}{)~}}
\makeatother


\newcommand{\transfermsg}{\inmsg{transfer}{to}{val}{data}{from}}
\newcommand{\createmsg}{\inmsg{contract \ create}{addr}{val}{data}{private}{from}}
\newcommand{\reject}{\textbf{reject}~}
\newcommand{\ignore}{\textbf{ignore}~}
%\newcommand{\For}{\textbf{For}~}
\newcommand{\Env}{\ensuremath{\mathcal{Z}}}
%\newcommand{\While}{\textbf{While}~}
\newcommand{\Buffer}{\textbf{Buffer}~}
\newcommand{\Send}{\textbf{Send}~}
\newcommand{\Output}{\emph{Output}~}
\newcommand{\Leak}{\textbf{Leak}}
\newcommand{\Eventually}{\textbf{Eventually}~}
\newcommand{\In}{\textbf{in}~}
\newcommand{\If}{\textbf{If}~}
\newcommand{\Else}{\textbf{Else}~}
%\newcommand{\Return}{\textbf{Return}~}

\newcommand{\pluseq}{\ensuremath{\mathrel{+}=}}
\newcommand{\minuseq}{\ensuremath{\mathrel{-}=}}
\newcommand{\Adv}{\ensuremath{\mathcal{A}}}
%\newcommand{\Partyi}{\ensuremath{\mathbf{P_i=(sid,pid)}}}
\newcommand{\sid}{\ensuremath{\msf{sid}}\xspace}
\newcommand{\pid}{\ensuremath{\msf{pid}}\xspace}
\newcommand{\dquad}{\quad \quad}
\newcommand{\qqquad}{\qquad \quad}
\newcommand{\qqqquad}{\qqquad \quad}
\newcommand{\qqqqquad}{\qqqquad \quad}

\newcommand*\circled[1]{\tikz[baseline=(char.base)]{
            \node[shape=circle,draw,inner sep=1pt] (char) {#1};}}

\newcommand*\token{~\circled{t}}

\DeclarePairedDelimiter{\ceil}{\lceil}{\rceil}


\newcommand{\spheading}[1]{ %
	\rotatebox{60}{\parbox{2.5cm}{\raggedright #1}}}




%Potential annotations
\newlength{\rWidth}

\newcommand{\funtype}[1]{%
    {\settowidth{\rWidth}{\ensuremath{#1}}%
        \;\ensuremath{{\xrightarrow{\hspace{\rWidth}}\hspace{-0.84\rWidth}}\!\!\!^%
         {#1}%{\BehindSubString{,}{#1} / \BeforeSubString{,}{#1}}%
         \hspace{0.2\rWidth}\;\;}}}


%% Notation
\newcommand{\m}[1]{\ensuremath{\mathsf{#1}}}
\newcommand{\mb}[1]{\ensuremath{\mathbf{#1}}}
\newenvironment{sill}{\begin{tabbing}}{\end{tabbing}}


%% Configuration
\newcommand{\conftree}[3]{\left[#1\right] \; \proc{#2}{#3}}
\newcommand{\confprovider}[2]{(#1)^{#2}}
\newcommand{\confset}[1]{\overline{#1}}
\newcommand{\esync}{\; \m{esync}}
\newcommand{\measure}{energy}
\newcommand{\measures}{energies}
% \newcommand{\mc}[1]{\mathcal{#1}}
\newcommand{\CC}{\mathcal{C}}
\newcommand{\DD}{\mathcal{D}}
\newcommand{\EE}{\mathcal{E}}
\newcommand{\FF}{\mathcal{F}}

%% Modes
\newcommand{\s}{\m{S}}
\newcommand{\li}{\m{L}}
\newcommand{\cl}{\m{C}}
\newcommand{\p}{\m{P}}

\newcommand{\lang}[1]{\mathbf{L}(#1)}

%% Contexts and Typing Judgment
\newcommand{\W}{\Omega}
\newcommand{\Sg}{\Sigma}
\newcommand{\xvdash}[2]{\sststile{#2}{#1}}
\newcommand{\xVdash}[1]{%
  \Vdash^{\mkern-8mu\scriptstyle\rule[-.9ex]{0pt}{0pt}#1}%
}
\newcommand{\confpot}[2]{\overset{#1}{\underset{#2}{\vDash}}}
\newcommand{\potconf}[1]{\overset{#1}{\vDash}}
\newcommand{\spanconf}{\vDash}
\newcommand{\confspan}[1]{\overset{(#1)}{\vDash}}
\newcommand{\confspanlocal}[1]{\overset{\langle #1 \rangle}{\vDash}}
\newcommand{\D}{\Delta}
%\newcommand{\G}{\Gamma}
\newcommand{\T}{\Theta}
\newcommand{\proves}{\vDash}
\newcommand{\w}{\omega}
%\newcommand{\Co}{\mathcal{C}}
%\renewcommand{\C}{\mathcal{C}}
\newcommand{\set}[1]{\lvert\lvert#1\rvert\rvert}

\newcommand{\lin}[1]{\m{lin}(\overline{#1})}
\newcommand{\shd}[1]{\m{shd}(\overline{#1})}
\newcommand{\slin}[1]{\m{slin}(\overline{#1})}
\newcommand{\plin}{\; \m{purelin}}

%% Operational Semantics Predicates
\newcommand{\proc}[2]{\m{proc}(#1, #2)}
\newcommand{\msg}[2]{\m{msg}(#1, #2)}
\newcommand{\ichan}[3]{\m{ichan}(#1, #2, #3)}
\newcommand{\ochan}[3]{\m{ochan}(#1, #2, #3)}
\newcommand{\unavail}[1]{\m{unavail}(#1)}

%% Semantics
\newcommand{\step}{\; \mapsto \;}
\newcommand{\zerostep}{\step^{0}}
\newcommand{\timed}[2]{\{#1\}_{#2}}
\newcommand{\unit}{M}
\newcommand{\Step}{\Longrightarrow}
\newcommand{\info}{\mapsto}
\newcommand{\andin}{\; \m{and} \;}
\newcommand{\minus}{\setminus}
\newcommand{\fresh}[1]{(#1 \text{ fresh})}
\newcommand{\eval}[1]{\Downarrow_{#1}}

%% Expressions Semantics
\newcommand{\val}{\; \m{val}}

%% Expressions
\newcommand{\lam}[3]{\lambda #1 : #2 . M_x}
\newcommand{\inl}[1]{l \cdot #1}
\newcommand{\inr}[1]{r \cdot #1}
\newcommand{\case}[3]{\m{case} \; #1 \; (l \hookrightarrow #2, r \hookrightarrow #3)}
\newcommand{\pair}[2]{\left\langle #1, #2 \right\rangle}
\newcommand{\projl}[1]{#1 \cdot l}
\newcommand{\projr}[1]{#1 \cdot r}
\newcommand{\match}[4]{\m{match} \; #1 \; ([] \rightarrow #2, #3 \rightarrow #4)}
\newcommand{\eproc}[3]{\{#1 \leftarrow #2 \leftarrow #3\}}


%% Proof Terms
\newcommand{\ecase}[3]{\m{case} \; #1 \; (#2 \Rightarrow #3)}
\newcommand{\ecasecf}[3]{\m{case^{cf}} \; #1 \; (#2 \Rightarrow #3)}
\newcommand{\erecvch}[2]{#2 \leftarrow \m{recv} \; #1}
\newcommand{\erecvchcf}[2]{#2 \leftarrow \m{recv^{cf}} \; #1}
\newcommand{\erecvshift}[1]{\m{shift} \leftarrow \m{recv} \; #1}
\newcommand{\esendch}[2]{\m{send} \; #1 \; #2}
\newcommand{\esendchcf}[2]{\m{send^{cf}} \; #1 \; #2}
\newcommand{\esendshift}[1]{\m{send} \; #1 \; \m{shift}}
\newcommand{\ewait}[1]{\m{wait} \; #1}
\newcommand{\ewaitcf}[1]{\m{wait^{cf}} \; #1}
\newcommand{\eclose}[1]{\m{close} \; #1}
\newcommand{\eclosecf}[1]{\m{close^{cf}} \; #1}
\newcommand{\fwd}[2]{#1 \leftarrow #2}
\newcommand{\fwdp}[2]{#1 \overset{+}{\leftarrow} #2}
\newcommand{\fwdn}[2]{#1 \overset{-}{\leftarrow} #2}
\newcommand{\esendl}[2]{#1.#2}
\newcommand{\esendlcf}[2]{(#1.#2)^{\m{cf}}}
\newcommand{\ecut}[4]{#1 \leftarrow #2 \leftarrow #3 \semi #4}
\newcommand{\ecutna}[3]{#1 \leftarrow #2 \semi #3}
\newcommand{\espawn}[4]{#1 \leftarrow #2 \leftarrow #3 = #4}
\newcommand{\procg}[3]{\m{proc}(#1, #2, \overline{#3})}
\newcommand{\edelay}[1]{\m{delay} \; (#1)}
\newcommand{\ewhen}[2]{\m{when?} \; (#1) ; #2}
\newcommand{\enow}[2]{\m{now!} \; (#1) ; #2}
\newcommand{\etick}[1]{\m{tick} \; (#1)}
\newcommand{\ework}[1]{\m{work} \; \{#1\}}
\newcommand{\eget}[2]{\m{get} \; #1 \; \{#2\}}
\newcommand{\epay}[2]{\m{pay} \; #1 \; \{#2\}}
\newcommand{\procdef}[3]{#3 \leftarrow #1 \; #2}
\newcommand{\procdefna}[2]{#2 \leftarrow #1}
\newcommand{\casedef}[1]{\m{case} \; #1}
\newcommand{\labdef}[1]{#1 \Rightarrow}
\newcommand{\wk}[1]{\m{work}(#1)}
\newcommand{\eassume}[2]{\m{assume} \; #1 \; \{#2\}}
\newcommand{\eassert}[2]{\m{assert} \; #1 \; \{#2\}}
\newcommand{\eimpos}[2]{\m{impossible} \; #1 \; \{#2\}}
\newcommand{\eif}[1]{\m{if} \; (#1)}
\newcommand{\ethen}{\; \m{then} \; }
\newcommand{\eelse}{\m{else} \; }

%% Type Constructors
\newcommand{\lolli}{\multimap}
\newcommand{\tensor}{\otimes}
\newcommand{\with}{\mathbin{\binampersand}}
\newcommand{\paar}{\mathbin{\bindnasrepma}}
\newcommand{\one}{\mathbf{1}}
\newcommand{\zero}{\mathbf{0}}
\newcommand{\bang}{{!}}
\newcommand{\whynot}{{?}}
\newcommand{\semi}{\, ; \,}
\newcommand{\ichoiceop}{\ensuremath{\oplus}}
\newcommand{\echoiceop}{\ensuremath{\with}}
\newcommand{\ichoice}[1]{\ichoiceop \{ #1 \}}
\newcommand{\echoice}[1]{\echoiceop \{ #1 \}}
\newcommand{\fuse}{\bullet}
\newcommand{\mi}[1]{\mbox{\it #1}}
\newcommand{\lunder}{\mathbin{\backslash}}
\newcommand{\tassertop}{?}
\newcommand{\tassumeop}{!}
\newcommand{\tassert}[1]{\; \tassertop\{#1\}. \;}
\newcommand{\tassume}[1]{\; \tassumeop\{#1\}. \;}
\newcommand{\arrow}{\rightarrow}
\newcommand{\product}{\times}

%% Functional Types
\newcommand{\tproc}[2]{\{#1 \leftarrow #2\}}

%% Types with Potential
\newcommand{\pot}[2]{#1^{#2}}
\newcommand{\lollipot}[1]{\overset{#1}{\lolli}}
\newcommand{\tensorpot}[1]{\overset{#1}{\tensor}}
\newcommand{\potfop}{\phi}
\newcommand{\potf}[1]{\potfop(#1)}
\newcommand{\mlab}{M^{\textsf{label}}}
\newcommand{\mchan}{M^{\textsf{channel}}}
\newcommand{\mcl}{M^{\textsf{close}}}
\newcommand{\mall}{M}
\newcommand{\mint}{M^{\textsf{internal}}}
\newcommand{\mval}{M^{\textsf{value}}}
\newcommand{\mshd}{M^{\textsf{share}}}
\newcommand{\ms}{M_s}
\newcommand{\mr}{M_r}
\newcommand{\entailpot}[2]{\xvdash{#1}{#2}}
\newcommand{\exppot}[1]{\xVdash{#1}}
\newcommand{\texp}{\Vdash}
\newcommand{\pexp}{\vdash}
\newcommand{\paypot}{\triangleright}
\newcommand{\getpot}{\triangleleft}
\newcommand{\tgetpot}[2]{\getpot^{\{#2\}} #1}
\newcommand{\tpaypot}[2]{\paypot^{\{#2\}} #1}
\newcommand{\bigeval}[3]{#1 \Downarrow #2 \mid #3}
\newcommand{\share}{\curlyveedownarrow}
\newcommand{\zpot}{\overline{0}}


%% Temporal Types
\newcommand{\entailpotcf}[1]{\underset{\m{cf}}{\entailpot{#1}}}
\newcommand{\entailspan}{\vdash}
\newcommand{\entailtype}{\vdash}
\newcommand{\fpot}{\; @ \;}
\newcommand{\pay}[1]{#1^{1}}
\newcommand{\sync}[1]{#1^{2}}
\newcommand{\spanpot}[1]{\langle \pay{#1}, \sync{#1} \rangle}
\newcommand{\ichoicepot}[2]{\overset{#1}{\ichoiceop} \{ #2 \}}
\newcommand{\echoicepot}[2]{\overset{#1}{\echoiceop} \{ #2 \}}
\newcommand{\tlist}[1]{\m{list}_{#1}}
\newcommand{\plist}[2]{\m{list}_{#1}^{#2}}
\newcommand{\tdia}[1]{\Diamond #1}
\newcommand{\tbox}[1]{\Box #1}
\newcommand{\tforall}[1]{\forall . #1}
\newcommand{\texists}[1]{\exists . #1}
\newcommand{\Dia}{\Diamond}
\newcommand{\Next}{\raisebox{0.3ex}{$\scriptstyle\bigcirc$}}
\newcommand{\tdelay}[2]{
    \IfEqCase{#2}{%
        {1}{\next{#1}}%
        % you can add more cases here as desired
    }[{\Next^{#2} (#1)}]%
}%
\newcommand{\sch}[1]{\tau(#1)}
\newcommand{\lforce}[2]{[#1]_L^{#2}}
\newcommand{\rforce}[2]{[#1]_R^{#2}}
\newcommand{\force}[2]{#1 \circ (#2)}

\setlength{\inferLineSkip}{4pt}
\newcommand{\blue}[1]{{\color{blue}#1}}
\newcommand{\red}[1]{{\color{red}#1}}
\newcommand{\green}[1]{{\color{green}#1}}
\newcommand{\tick}{\blue{\m{tick}}}
\newcommand{\delay}{\red{\m{delay}}}
\newcommand{\when}[1]{\red{\m{when?}\;#1}}
\newcommand{\now}[1]{\red{\m{now!}\;#1}}
\newcommand{\noww}{\red{\m{now!}}}
\newcommand{\whenn}{\red{\m{when?}}}
% \newcommand{\vdashi}{\vdash^{\!\!{}^i}}
\newcommand{\vdashi}{\vdash^{\!\!\scriptscriptstyle i}}
\newcommand{\tock}{`}


%% Indices
\newcommand{\indv}[1]{\overline{\{#1\}}}
\newcommand{\ind}[1]{\{#1\}}


%% Syntactic Sugar
\newcommand{\config}{\mathcal{C}}
\newcommand{\cost}[2]{\mathrm{cost}(\proc{#1}{#2})}
\newcommand{\tcost}[2]{\mathrm{cost}(#1 \mapsto #2)}
\newcommand{\ccost}[1]{\mathrm{cost}(#1)}
\newcommand{\dc}{\mathcal{D}}
\newcommand{\ec}{\mathcal{E}}
\newcommand{\ac}{\mathcal{A}}
\newcommand{\st}[1]{\m{store}_{#1}}
\newcommand{\stack}[1]{\m{stack}_{#1}}
\newcommand{\queue}[1]{\m{queue}_{#1}}
\newcommand{\mapper}[1]{\m{mapper}_{#1}}
\newcommand{\fdr}[1]{\m{folder}_{#1}}
\newcommand{\lt}[1]{\m{list}_{#1}}
%\newcommand{\bits}{\m{bits}}
\newcommand{\ctr}{\m{ctr}}
\newcommand{\trans}[2]{#1 \Longrightarrow #2}
\newcommand{\typetrans}[1]{\left\lvert{#1}\right\rvert}
\newcommand{\tree}{\m{tree}}
\newcommand{\bool}{\m{bool}}
\newcommand{\delayedbox}[1]{#1 \; \m{delayed}^{\Box}}
\newcommand{\delayeddia}[1]{#1 \; \m{delayed}^{\Diamond}}
\newcommand{\dom}[1]{\m{dom}(#1)}
\newcommand{\valid}[1]{#1 \; \m{valid}}
\newcommand{\invalid}[1]{#1 \; \m{invalid}}

%% Smart Contracts
\newcommand{\addr}{\m{addr}}
\newcommand{\ether}{\m{ether}}
\newcommand{\players}{\m{players}}
\newcommand{\lottery}{\m{lottery}}
\newcommand{\tint}{\m{int}}
\newcommand{\ballot}{\m{ballot}}
\newcommand{\tbool}{\m{bool}}
\newcommand{\lc}{\tlist{\m{coin}}}
\newcommand{\auction}{\m{auction}}
\newcommand{\object}{\m{object}}

%% Typing Judgments for Servers and Clients
\newcommand{\sentailpot}[1]{\prescript{}{S}{\xvdash{#1}} \hspace{2pt}}
\newcommand{\centailpot}[1]{\prescript{}{C}{\xvdash{#1}} \hspace{2pt}}


%% Sharing
\newcommand{\down}{\downarrow^{\m{S}}_{\m{L}}}
\newcommand{\up}{\uparrow^{\m{S}}_{\m{L}}}
\newcommand{\eacquire}[2]{#1 \leftarrow \m{acquire} \; #2}
\newcommand{\eaccept}[2]{#1 \leftarrow \m{accept} \; #2}
\newcommand{\erelease}[2]{#1 \leftarrow \m{release} \; #2}
\newcommand{\edetach}[2]{#1 \leftarrow \m{detach} \; #2}


%% Subtyping
\newcommand{\subt}[2]{#1 \leq #2}
\newcommand{\wsubt}{ <: }
\newcommand{\qsubt}[1]{\overset{#1}{\leq}}


%% Latex
%\newtheorem{theorem}{Theorem}
%\newtheorem{definition}{Definition}
%\newtheorem{lemma}{Lemma}
%\newtheorem{cor}{Corollary}


%%Global Semantics
\newcommand{\sinfer}[3]
{\inferrule
{#3}
{#2}
#1}
\newcommand{\enq}[2]{\m{enq}(#1, #2)}
\newcommand{\deq}[1]{\m{deq}(#1)}
\newcommand{\nil}{[]}
\newcommand{\elem}[1]{[#1]}


%% Channel typing
\newcommand{\eqdef}{\cong}


%% Types to Processes
\newcommand{\typeProc}[2]{#1 \Longrightarrow #2}

%% AARA
\newcommand{\abs}[1]{\left\lvert #1 \right\rvert}
\newcommand{\bin}[1]{(#1)_2}
% \newcommand{\ceil}[1]{\left\lceil #1 \right\rceil}
\newcommand{\bigO}[1]{\mathcal{O}(#1)}
% \newcommand{\ignore}[1]{\textcolor{red}{#1}}
% new \oset macro
\makeatletter
\newcommand{\oset}[3][-0.7ex]{%
  \mathrel{\mathop{#3}\limits^{
    \vbox to#1{\kern-2\ex@
    \hbox{$\scriptstyle#2$}\vss}}}}
\makeatother
\newcommand{\monus}{\oset{.}{-}}

%% Indexed Types
\newcommand{\cons}{\mathcal{C}}
\newcommand{\vars}{\mathfrak{v}}
\newcommand{\Vars}{\mathcal{V}}
\newcommand{\Cons}{\mathcal{C}}
\newcommand{\Tokens}{\Gamma}
\newcommand{\K}{\gamma}
\newcommand{\Tokentypes}{\mathcal{K}}
\newcommand{\VTokens}{\mathcal{V}}
\newcommand{\TokSig}{\mathcal{S}}
\newcommand{\exchange}[3]{#1 \overset{#2}{\longrightarrow} #3}
\newcommand{\GlobalF}{\ensuremath{\mathfrak{f}}\xspace}
\newcommand{\GlobalP}{\mathfrak{p}}
\newcommand{\depth}{\mathfrak{d}}

%% Two Counter Machines
\newcommand{\ins}{\iota}
\newcommand{\tcm}{\mathcal{M}}
\newcommand{\inc}[1]{\m{inc}(#1)}
\newcommand{\dec}[1]{\m{dec}(#1)}
\newcommand{\goto}{\m{goto}}
\newcommand{\zeroc}[1]{\m{zero}(#1) ?}
\newcommand{\halt}{\m{halt}}

%% UC stuff
\newcommand{\fcomm}{\mathcal{F}_{\msf{comm}}}
%\newcommand{\B}[1]{\colorbox{gray}{#1}}
%\newcommand{\hlc}[2][yellow]{{%
%    \colorlet{foo}{#1}%
%        \sethlcolor{foo}\hl{#2}}%
%        }
%\newcommand{\hlcyan}[1]{{\sethlcolor{cyan}\hl{#1}}}
%\newcommand{\B}[1]{\hlc[pink]{#1}}
\definecolor{airforceblue}{rgb}{0.36, 0.54, 0.66}
\newcommand{\B}[1]{{\color{airforceblue}{#1}}}
\newcommand{\wt}{\circled{w}}

%% TODO
\newcommand{\ankush}[1]{\textcolor{red}{\textbf{Ankush: #1}}}


%%% Local Variables:
%%% mode: plain-tex
%%% TeX-master: "pldi19"
%%% End:


\begin{abstract}
Software tooling for UC programming. Strongl typed language like Haskell we can define contexts for UC protocols/proofs in a modular way.
Results in on-paper definitions that can use clean abstractions through shell processes, and makes code easy.
We present Eventually and Optionally constructions and realize an ABA protocol using it. 
Show fuzz testing as tool for protocol analysis to discover buggy / in-secure code.
Fuzzing on ACast to detect problems with the code (relates directly to the ABA fuzzing because it is also an issue of thresholds).
Future work / discussion: contexts for more complex things like MPC. Write code directly but it translates to MPC instructions with the appropriate
communication assumptions.

\end{abstract}

\maketitle


%\begin{IEEEkeywords}
%component, formatting, style, styling, insert
%\end{IEEEkeywords}

\section{Introduction}
In this work we present a programming language design based on the Universal (UC) Composability framework from cryptography.

We build on existing work, ILC, which also aims to be a programming language for UC, but start from a more powerful language, called Nomos, which incorpoates session types and work aware resource types and has been previously used for  smart contracts and distributed applications.
Both of the features of Nomos turn out to have 

Second, Nomos features a notion of Work-aware types. This is useful for capturing the notion of “locally polynomial runtime.” This allows us to model UC more faithfully than any prior work to date

As a starting point, we build a language that merges types rules from ILC into Nomos. The main design idea of ILC is that it is uses static typing rules to encode the requirements of the Interacting Turing Machines (ITMs) model, a model that is uniquely associated with UC. The ILC rules roughly ensure that simulations of the language can be carried out by probabilistic Turing machines, which is necessary for reduction to computationally hard problems, required for cryptographic security proofs. The rules from ILC are compatible with session types, so it turns out to be straightforward to merge these into nomos. The result provides benefits associated with session types, namely that it avoids potential errors from internally-inconsistent programs.

   Beyond just session types, the Work-aware component of Nomos allows us to tackle a fundamental challenge in defining a programming language for UC that ILC (and all other related work) left unfulfilled, which is to express the notion of polynomially runtime.
   
   The challenge of “polynomial runtime” in UC is that individual processes must be judged as polynomial, but when eveluated in context with other concurrently running process it is difficult to assign blame.
       The current best way to define polynomial runtime, found in the 2019 and later version of UC, is based a concept of ``import tokens.''
   We identify how to relate the “Potential” concept from Nomos, to the import tokens from UC. 
	The result is a deep connection between session type semantics and the formal foundation of UC.
	The Preservation theorem we prove associated with our type system and operational semantics proves the following: 
well-typed terms in Nomos UC are “locally polynomial time”, in the sense required of UC, meaning they do not take more steps that some polynomial function T(N) of the net number of import tokens it has received.

In addition, our language has other benefits.
The Progress theorem is useful because it gives some evidence that ideal functionalities and protocols encoded in Nomos UC cannot get stuck. Together helps confirm that the process halts in polynomial time.
TODO: Give an example of a bad machine ruled out by progress guarantee.

\ignore{
Carries forward the same metatheory guarantees as ILC. Namely: if a process terminates, then it depends only on the random coins (unlike Pi calculus, including Session-type pi calculus). Thus simulating the execution of a Nomos UC experiment can be carried out by a probabilistic polynomial time Turing machine (PPT). This is essential in UC for reduction to computationally hard problems.
}

\ignore{
The Universal Composability Framework~\cite{uc} is the popular and widely-used framework for modelling the security of cryptographic and distributed protocols.
Its novel contribution compared to other frameworks is that it provides a very strong notion of security: a UC-secure protocol is proved to be secure even when composed with arbitrary other protocols running concurrently.
This constrasts with other, property-based notions of security~\todo{need to get some citations here}.

Analyzing large and complex protocols is a difficult task made easier by UC's ideal functionality abstraction. 
However, despite this additional modularity, UC proofs and models still tend to be very complex, unwieldy, and difficult to understand.
These issues are exacerbated when new communication models are added on top of UC~\cite{katz, etc}.
Therefore, we propose a two-fold solution: a new construction for modelling different communication models that removes all model-specific code from protocols and functionalities, and an implementation of the UC framework in the Nomos language. 
}



\section{Background} \label{sec:background}
\subsection{Universal Composability}
The universal composability framework~\cite{uc} proposes a new framework for proving the security of cryptographic and distributed protocol.
Compared to previous works, the UC framework provides a stronger notion of security where protocols that are UC-secure are secure even when composed with arbitrary other protocols running concurrently. 

Such a strong notion of security is achieved through the real-ideal world paradigm.
The ideal world encompasses an ideal implementation of a protocol, called the \textit{ideal functionality} $\mathcal{F}$, which acts as a trusted third party that caputures all the desired security properties.
The ideal functionality is usually a simple definition making it trivial to prove its security properties.
The real world, on the other hand, consists of parties running an actual protocol, $\pi$, against a real adversary.

Security proofs in UC involve creating a simulator $\mathcal{S}$ in the ideal world that can simulate every potential attack on a real protocol in the real world.
If $\mathcal{S}$ can make the two worlds indistinguishable for any real world adversary $\mathcal{A}$ for all distinguishing environments $\mathcal{Z}$, then we say the protocol $\pi$ UC-emulates the ideal functionality $\mathcal{F}$.
Indistinguishability of the two worlds to any $\mathcal{Z}$ implies that the protocol $\pi$ must exhibit the same security properties as the ideal functionality $\mathcal{F}$ otherwise there should be sobe distinguishing environment. 
More formally, indistinguishability is stated:

$$ \text{EXEC}_{\mathcal{F},\mathcal{S},\Environment} \approx \text{EXEC}_{\pi,\mathcal{A},\Environment} $$

\paragraph{GUC-Framework}


\subsection{The Import Mechanism}
A notion of resource-bound computation is necessary for the UC framework to reason about computationally efficient algorithms as well as the capabilities of ITIs under a particular resource constraint.
Often we would like to reason about adversarial capabilities under such constraints and perform efficient transformations (transforming an adversary into a simulator).

Previous definitions of polynomial-time computation have taken the form of bounding the computation of an ITI by some polynomial $T$:
given an input of length $n$ the machine $\mu$ halts within $T(n)$ steps.
However, using the length of the inputs to the machine as $n$, in this case leads to an infinite runs problems identified by Canetti~\cite{uc}.
Machines that are locally $T(n)$-bounded are able to spawn other machines to the point that an infinite chain of such machines can be spawed where each is locally $T$-bounded, but the whole system of machines can not be bounded by any polynomial $T$.

Therefore, a new notion of $n$ was needed. The UC paper defines an import mechanism where the first ITI, the environment, is spawned with a polynomially amount of import which can be thought of as tokens or coins.
The environment can then activate other ITIs with some import tokens allowing them to run for $T(n')$ computationsl steps for some $T$ and some amount of import $n'$.
In this new definition, an ITI that is $T$-bounded takes at most $T(n')$ steps where $n'$ is the difference between the import it has received from incoming messages and outgoing import it's given to other machines.
This definition therefore suffices to ensure that every machine is locally bounded by some polynomial but also guarantees that the system of ITMs is bounded by a polynomial number of import tokens. 


\section{Related Works} \label{sec:related}
There are many works that attempt to formalize the UC framework with an implementation for protocol analysis and proof generation.

One of the most relevant works to our own is EasyUC~\cite{easyuc}. 
EasyUC uses the existing EasyCrypy~\cite{easycrypt} toolset to model UC protocols and mechanize proof generation. 
It departs from EasyCrypt's limtations to game-based security definitions (lacking simulation-based composition).
However, it still lacks a notion of polynomial time. The authors, themselves, mentions that it can't detect deviant behavior like the adversary and functionality passing messages between each other indefinitely. 
Our use of the import mechainsm and session types let us reason about polynomial time in the sytem of ITMs encompassed by \msf{execUC} but also locally for \textit{open} terms. 
Furthermore, import in NomosUC lets us have guarantees of termination as well by the polyomial import constraints added to UC by Canetti et al.

Liao et al. introduce executable UC through a new process calculus called ILC~\ref{ilc}.
This work adds some notion of polynomial time although it proves to be too restrictive. 
It results from the fact that poly-time can only be reasoned about for \textit{closed} terms like a full UC execution.
In order to reason about polynomial time for a particular protocol $\pi$ we must reason over all possible other terms that connect to $\pi$ and require that it is polynomial in all such cases.
A simple ping-response server can not be proven to by poly-time in this definition for a deviant other ITM that connects to $\pi$. 
In Nomos, however, as mentioned above, open terms are limited to polytime regardless of the connected other terms because of the import mechanism and the NomosUC type system that guarantees termination. 

Other works that rely in symbolic modelling of cryptography, for example, SymbolicUC~\cite{symbolicuc}, are subsumed by the above ILC work and similarly lack any polynomial time notion. 
\todo{Say something about $\pi$-calculus with probabilistic polynomial time extensions}.


To the best of our knowledge, this is the first work to deal with the new import notion of polynomial time introduced to the UC framework in 2018.
A few other works refer to the import mechanism, but it is restricted to simply defining the import a protocol is given.
	
%easyUC:
%* can not dynamially create new instances of parties/functionalities must statically determine the number of functionalities/parties spawned
%* 
%
%
%The work of Liao et al.~\ref{ilc} is the closest to our own
%It proposes a new process calculus called ILC and a concrete implementation of the UC framework.
%The type system it introduces ensures that correctly types programs can be represented as ITMs.
%However, one drawback of the ILC work is that its polynomial time representation 
%
%
%The EasyUC approach uses the existing EasyCrypt toolset to implement model UC protocols and mechanize the generate of UC-security proofs and proofs of secure composition.
%This work aim considerably higher than our work in actually attempting to generate proofs for their protocols. 
%However, this work falls short in being able to capture any notion of resource bound computation whereas we are able to make guarnatees about polynomial bounds on our system of ITMs and even guarantee termination of programs through our realization of the import mechanism.
%The EasyUC work accepts that not even infinite loops of communication can be caught and, therefore, termination of protocols can't be guarnateed either whereas the import mechanism in Nomos ensures that such infinite loops can not stall protocol progress.

%Another work similar to our own is the Symbolic UC by B\"{o}hl and Unruh.
%This works uses an applied $\pi$-calculus to symbolically model UC protocols and analyze them.
%Similar to the EasyUC work, the goals of this work are somewhat orthogonal to the our own goals.
%However, Symbolic UC does attempt to create an implementatio of UC using the $\pi$-calculus however neglects to address any issues of polynomial runtime.
%
%Perhaps the closes work to our own is that of Liao et al.~\cite{ilc} that builds an executable version of the UC framework by introducing a new process calculus called ILC.
%ILC introduces a type system that guarantees that ILC programs (i.e. functionalities, protocols, etc) can be expressed as ITMs as in the UC framework.
%However, one drawback of ILC is that it's notion of polynomial time ends up being too restrictive.
%In ILC only closed terms without any unbonded variables, i.e. and entire UC exection of a system of ITMs, can be shown to be polynomial in their definition of polynomial time.
%Proving polynomial time for open terms, such as a protocol $\pi$, requires reasoning over all possible contexts in which the protocol could exist however such a definition of polynomial time becomes too restrictive where even a simple ping-responde server protocol wouldn't be considered polynomial time.


\section{Discussion and Future Work}
\input{sections/discussion}

\bibliographystyle{ACM-Reference-Format}
\bibliography{nomosuc}

\newpage

\appendix

\pagebreak

\end{document}
