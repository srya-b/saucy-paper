\documentclass[acmsmall, screen, review, anonymous]{acmart}
\settopmatter{printfolios=true,printccs=false,printacmref=false}
%\IEEEoverridecommandlockouts
% The preceding line is only needed to identify funding in the first footnote. If that is unneeded, please comment it out.
%\usepackage[dvipsnames]{xcolor}
%\usepackage{cite}
\usepackage[most]{tcolorbox}
%\usepackage{amsmath,amssymb,amsfonts,amsthm}
\usepackage{algorithmic}
\usepackage{graphicx}
\usepackage{subcaption}
\usepackage{textcomp}
\usepackage{mathtools}
\usepackage[shortlabels]{enumitem}
\usepackage[T1]{fontenc}
\usepackage{listings}
\usepackage{tabularx}
\usepackage{bbm}
%\usepackage{unicode-math}
\usepackage[utf8]{inputenc}
\usepackage{newunicodechar}
\usepackage{multirow}
\usepackage{booktabs}
\usepackage{adjustbox}

%% PL packages
\usepackage{stmaryrd} 
\usepackage{proof}
\usepackage{mathpartir}
%\usepackage{color}
\usepackage{xstring}
\usepackage{xspace}
\usepackage{turnstile}

\AtBeginDocument{%
  \providecommand\BibTeX{{%
    \normalfont B\kern-0.5em{\scshape i\kern-0.25em b}\kern-0.8em\TeX}}}

%% Rights management information.  This information is sent to you
%% when you complete the rights form.  These commands have SAMPLE
%% values in them; it is your responsibility as an author to replace
%% the commands and values with those provided to you when you
%% complete the rights form.
\setcopyright{none}


%%
%% These commands are for a JOURNAL article.
\acmJournal{PACMPL}
\acmVolume{1}
\acmNumber{ICFP} 
\acmArticle{1}
\acmYear{2022}
\acmMonth{9}
\acmDOI{} % \acmDOI{10.1145/nnnnnnn.nnnnnnn}
\startPage{1}

\citestyle{acmauthoryear}


\begin{document}
\usetikzlibrary{matrix, arrows.meta, calc, positioning}
\tikzset{myarrow/.style={-Latex, rounded corners},}

\newcommand*\emptycirc[1][1ex]{\tikz\draw (0,0) circle (#1);} 
\newcommand*\halfcircleft[1][1ex]{%
  \begin{tikzpicture}
  \draw[fill] (0,0)-- (90:#1) arc (90:270:#1) -- cycle ;
  \draw (0,0) circle (#1);
  \end{tikzpicture}}
\newcommand*\halfcircright[1][1ex]{%
  \begin{tikzpicture}
  \draw[fill] (0,0)-- (0:#1) arc (0:90:#1) -- cycle ;
  \draw[fill] (0,0)-- (270:#1) arc (270:360:#1) -- cycle;
  \draw (0,0) circle (#1);
  \end{tikzpicture}}
\newcommand*\fullcirc[1][1ex]{\tikz\fill (0,0) circle (#1);} 

\newcolumntype{R}[2]{
	>{\adjustbox{angle=#1, lap=\width-(#2)}\bgroup}
	c
	<{\egroup}
}
\newcommand*\rot{\multicolumn{1}{R{30}{1.5em}}}


\definecolor{vert}{RGB}{0,181,0}
\definecolor{oran}{RGB}{223,74,0}
\definecolor{viol}{RGB}{134,0,175}
\definecolor{roug}{RGB}{215,15,0}
\definecolor{bb}{RGB}{0,0,0}
\definecolor{gg}{RGB}{220,220,220}
\definecolor{royalblue}{rgb}{0.25, 0.41, 0.88}
\definecolor{forestgreen}{rgb}{0.13, 0.55, 0.13}
\definecolor{YellowOrange}{rgb}{0.98, 0.6, 0.01}
\definecolor{Red}{rgb}{0.89, 0.0, 0.13}
\definecolor{Black}{rgb}{0.0, 0.0, 0.0}
\definecolor{Purple}{rgb}{0.63, 0.36, 0.94}
\definecolor{purp}{rgb}{0.59, 0.48, 0.71}

\newcommand{\anote}[1]{{\color{magenta}{AM: {{#1}}}}}
\newcommand{\snote}[1]{{\color{green}{SB: {{#1}}}}}

\newtcolorbox[auto counter]{bbox}[2][]{%
    colback=white,
    colframe=bb,
    %colbacktitle=white!90!roug,
	colbacktitle=white!40!gg,
    coltitle=black,
    fonttitle=\small\bfseries, 
	fontupper=\small,
	fontlower=\small,
    enhanced,
    attach boxed title to top left={yshift=-2mm, xshift=0.5cm},%
    #1,% For possible options
}

\mathchardef\hyp="2D
\mathchardef\car="5E

\makeatletter
\newcommand\BeraMonottFamily{%
	\def\fvm@scale{0.85}%
	\fontfamily{fvm}\selectfont
}
\makeatother

\title{Eventually
}

\newcommand{\mc}[1]{\ensuremath{\mathcal{#1}}}
\newcommand{\msf}[1]{\ensuremath{{\mathsf {#1}}}}
\newcommand{\mathc}[1]{\ensuremath{\mathcal{#1}}}
\newcommand{\tsc}[1]{\textsc{#1}}
\newcommand{\f}[1]{\ensuremath{\mathcal{#1}}\xspace}
\newcommand{\F}{\f{F}}
\newcommand{\PI}{\ensuremath{\pi}\xspace}
\newcommand{\RHO}{\ensuremath{\rho}\xspace}
\newcommand{\achan}{\ensuremath{\F_{\msf{achan}}^{p_r,p_s}}}
%\newcommand{\C}{\mathcal{C}}
\newcommand{\con}[1]{\msf{Contract_{#1}}}
%\newcommand{\Fsync}[2]{\ensuremath{\F_{\msf{sync},#1,#2}}}
\newcommand{\Fsync}[2]{\ensuremath{\F_{\msf{BD-SEC}}(#1,#2)}}
\newcommand{\Fchan}[2]{\ensuremath{\F_{\msf{chan}}(#1,#2)}}
\newcommand{\Fbdsec}{\ensuremath{\F_{\msf{BD-SEC}}^{\delta,\ell}}}
\newcommand{\Fbc}{\ensuremath{\F_{\msf{broadcast}}}}
\newcommand{\Fsfe}{\ensuremath{\F_{\msf{SFE}}}}
\newcommand{\Fstate}{\ensuremath{\F_{\msf{state}}}}
\newcommand{\Fclock}{\ensuremath{\F_{\msf{clock}}}}
\newcommand{\Frbc}{\ensuremath{\F_{\msf{rbc}}}}
\newcommand{\Fpay}{\ensuremath{\F_{\msf{pay}}}}
\newcommand{\Fcom}{\ensuremath{\F_{\msf{com}}}\xspace}
\newcommand{\Fauth}{\ensuremath{\F_{\msf{auth}}}\xspace}
\newcommand{\Fflip}{\ensuremath{\F_{\msf{coinflip}}}\xspace}
\newcommand{\Fro}{\ensuremath{\F_{\msf{RO}}}\xspace}
\newcommand{\Fsmc}{\ensuremath{\F_{\msf{SMC}}}\xspace}
\newcommand{\Fropp}{\ensuremath{\F_{\msf{P2P\hyp RO}}}\xspace}
\newcommand{\Gledger}{\ensuremath{\f{G}_{\msf{ledger}}}}
\newcommand{\Wsync}{\ensuremath{\mathcal{W}_{\msf{sync}}}}
\newcommand{\Wasync}{\ensuremath{\mathcal{W}_{\msf{async}}}}
\newcommand{\Ssyncbracha}{\ensuremath{\mathc{S}_{\msf{sbracha}}}}
\newcommand{\Fbracha}{\ensuremath{\mathcal{F}_{\msf{bracha}}}}
\newcommand{\Schedule}{\tsc{Schedule}}
\newcommand{\Delay}{\tsc{Delay}}
\newcommand{\Advance}{\tsc{Advance}}
\newcommand{\Exec}{\tsc{Exec}}
%\newcommand{\Adversary}{\ensuremath{\mathcal{A}}\xspace}
\newcommand{\A}{\ensuremath{\mathcal{A}}\xspace}
\newcommand{\DummyAdv}{\ensuremath{\mathcal{A}_\mathcal{D}}\xspace}
\newcommand{\DA}{\ensuremath{\A_\mathcal{D}}\xspace}
\newcommand{\Sim}{\ensuremath{\mathcal{S}}\xspace}
\newcommand{\SIM}[1]{\ensuremath{\mathcal{S}_{#1}}\xspace}
\newcommand{\simcom}{\SIM{\msf{com}}}
\newcommand{\cf}{\ensuremath{\mathcal{C}}\xspace}
\newcommand{\ID}[1]{\ensuremath{\mathcal{I}(#1)}\xspace}
%\newcommand{\Sim}[1][]{\ifthenelse{\equal{#1}{}}{\ensuremath{\Simulator}}{\ensuremath{\Simulator_{#1}}}}
\newcommand{\DS}{\SIM{D}\xspace}
%\newcommand{\Environment}{\ensuremath{\mathcal{Z}}\xspace}
\newcommand{\Z}{\ensuremath{\mathcal{Z}}\xspace}
\newcommand{\Partyi}{\ensuremath{P_i}}
\newcommand{\Partyj}{\ensuremath{P_j}}
\newcommand{\partywrapper}{multiplexer\xspace}
\newcommand{\pw}{\PI}
\newcommand{\fwrapper}{\todo{fwrappername}\xspace}

\newcommand{\dealer}{\ensuremath{\mathcal{D}}}
\newcommand{\globalf}[1]{\ensuremath{{\overline{\mathcal{#1}}}}}
\newcommand{\todo}[1]{\textcolor{Red}{todo: #1}}
\newcommand{\edict}{\{\}}
\newcommand{\lar}{\leftarrow}
\newcommand{\rar}{\rightarrow}
\newcommand{\Init}{{\bf \color{NavyBlue} Init}~}
\newcommand{\OnInput}{{\bf \textcolor{Black} On input}~}
\newcommand{\Allinputs}{{\bf \color{Cerulean} All other input~}}
\newcommand{\OnAdvInput}{{\bf \color{BrickRed} On input}~}
\newcommand{\heading}[1]{\textbf{#1}}
\newcommand{\Type}{\ensuremath{\yo{type}}}
\newcommand{\Stype}{\ensuremath{\yo{stype}}}
\newcommand{\bangf}{\ensuremath{!\F}}
\newcommand{\execuc}{\ensuremath{\msf{execUC}}}
\newcommand{\iexecuc}{\inline{execUC}}
\newcommand{\UC}[4]{\ensuremath{\execuc #1  #2  #3  #4}}
\newcommand{\idealP}{\ensuremath{\mathbbm{1}_d}\xspace}
%\newcommand{\prot}[1][]{\ifthenelse{\equal{\ensuremath{#1}}{}}{\ensuremath{\Pi}}{\ensuremath{\Pi_{X #1}}}}
\newcommand{\prot}[1]{\ensuremath{\pi_{\msf{#1}}}}
\newcommand{\lla}{\leftarrow}
\newcommand{\lvd}{\vdash}
\newcommand{\tb}[1]{\text{\color{royalblue}{#1}}}
\newcommand{\tgr}[1]{\text{\color{forestgreen}{#1}}}
\newcommand{\tm}[1]{\text{\color{magenta}{#1}}}
\newcommand{\tg}[1]{\text{\color{gray}{#1}}}
\newcommand{\tp}[1]{\text{\color{purp}{#1}}}
\newcommand{\nparam}[1]{\tp{#1}}
\newcommand{\tr}[1]{\text{\color{Red}{#1}}}
\newcommand{\yo}[1]{\text{\color{YellowOrange}{#1}}}
\newcommand{\inline}[1]{\lstinline[basicstyle=\footnotesize\BeraMonottFamily, mathescape]!#1!}
\newcommand{\nrecv}{\tb{recv}}
\newcommand{\nsend}{\tb{send}}
\newcommand{\nget}{\tb{get}}
\newcommand{\npay}{\tb{pay}}
\newcommand{\nsimget}{\tm{simget}}
\newcommand{\nsimpay}{\tm{simpay}}
\newcommand{\ncase}{\tm{case}}
\newcommand{\nproc}{\tb{proc}}
\newcommand{\nwithdraw}{\tm{withdrawTokens}}
\newcommand{\nif}{\yo{if}}
\newcommand{\nthen}{\yo{then}}
\newcommand{\nend}{\yo{end}}
\newcommand{\nwhile}{\yo{while}}


%\newcommand{\pluseq}{\mathrel{+}=}
%\newcommand{\minuseq}{\mathrel{-}=}
\newcommand{\Assert}{{\bf \color{BrickRed} Assert }}
\newcommand{\Require}{{\bf \color{BrickRed} Require }}

%\theoremstyle{acmdefinition}
%\newtheorem{definition}{Definition}[section]
\newtheorem{ddef}{Definition}
%\newtheorem{theorem}{Theorem}
\newtheorem{claim}{Claim}
%\newtheorem{lemma}{Lemma}

%\newlist{renumerate}{enumerate}{1}
%\setlist[renumerate]{before=\setlength{\baselineskip}{20pt}, itemsep=-2ex, topsep=-2ex}
%\newenvironment{renumerate}{\begin{enumerate}[before=\setlength{\baselineskip}{20pt},itemsep=-2ex,topsep=0pt]}{\end{enumerate}}
\newenvironment{renumerate}{\begin{enumerate}[nosep]}{\end{enumerate}}
%\newenvironment{ritemize}{\begin{itemize}[before=\setlength{\baselineskip}{20pt},itemsep=-2ex,topsep=0pt]}{\end{itemize}}
\newenvironment{ritemize}{\begin{itemize}[nosep] \renewcommand\labelitemi{--}}{\end{itemize}}

\newenvironment{mylst}{\begin{lstlisting}[basicstyle=\small\BeraMonottFamily, frame=single, mathescape]}{\end{lstlisting}}

\makeatletter
\newcommand{\inmsg}[1]{%
(#1\checknextarg}
\newcommand{\checknextarg}{\@ifnextchar\bgroup{\gobblenextarg}{)~}}
\newcommand{\gobblenextarg}[1]{, #1\@ifnextchar\bgroup{\gobblenextarg}{)~}}
\makeatother


\newcommand{\transfermsg}{\inmsg{transfer}{to}{val}{data}{from}}
\newcommand{\createmsg}{\inmsg{contract \ create}{addr}{val}{data}{private}{from}}
\newcommand{\reject}{\textbf{reject}~}
\newcommand{\ignore}{\textbf{ignore}~}
%\newcommand{\For}{\textbf{For}~}
\newcommand{\Env}{\ensuremath{\mathcal{Z}}}
%\newcommand{\While}{\textbf{While}~}
\newcommand{\Buffer}{\textbf{Buffer}~}
\newcommand{\Send}{\textbf{Send}~}
\newcommand{\Output}{\emph{Output}~}
\newcommand{\Leak}{\textbf{Leak}}
\newcommand{\Eventually}{\textbf{Eventually}~}
\newcommand{\In}{\textbf{in}~}
\newcommand{\If}{\textbf{If}~}
\newcommand{\Else}{\textbf{Else}~}
%\newcommand{\Return}{\textbf{Return}~}

\newcommand{\pluseq}{\ensuremath{\mathrel{+}=}}
\newcommand{\minuseq}{\ensuremath{\mathrel{-}=}}
\newcommand{\Adv}{\ensuremath{\mathcal{A}}}
%\newcommand{\Partyi}{\ensuremath{\mathbf{P_i=(sid,pid)}}}
\newcommand{\sid}{\ensuremath{\msf{sid}}\xspace}
\newcommand{\pid}{\ensuremath{\msf{pid}}\xspace}
\newcommand{\dquad}{\quad \quad}
\newcommand{\qqquad}{\qquad \quad}
\newcommand{\qqqquad}{\qqquad \quad}
\newcommand{\qqqqquad}{\qqqquad \quad}

\newcommand*\circled[1]{\tikz[baseline=(char.base)]{
            \node[shape=circle,draw,inner sep=1pt] (char) {#1};}}

\newcommand*\token{~\circled{t}}

\DeclarePairedDelimiter{\ceil}{\lceil}{\rceil}


\newcommand{\spheading}[1]{ %
	\rotatebox{60}{\parbox{2.5cm}{\raggedright #1}}}




%Potential annotations
\newlength{\rWidth}

\newcommand{\funtype}[1]{%
    {\settowidth{\rWidth}{\ensuremath{#1}}%
        \;\ensuremath{{\xrightarrow{\hspace{\rWidth}}\hspace{-0.84\rWidth}}\!\!\!^%
         {#1}%{\BehindSubString{,}{#1} / \BeforeSubString{,}{#1}}%
         \hspace{0.2\rWidth}\;\;}}}


%% Notation
\newcommand{\m}[1]{\mathsf{#1}}
\newcommand{\mb}[1]{\mathbf{#1}}
\newenvironment{sill}{\begin{tabbing}}{\end{tabbing}}


%% Configuration
\newcommand{\conftree}[3]{\left[#1\right] \; \proc{#2}{#3}}
\newcommand{\confprovider}[2]{(#1)^{#2}}
\newcommand{\confset}[1]{\overline{#1}}
\newcommand{\esync}{\; \m{esync}}
\newcommand{\measure}{energy}
\newcommand{\measures}{energies}
% \newcommand{\mc}[1]{\mathcal{#1}}
\newcommand{\CC}{\mathcal{C}}
\newcommand{\DD}{\mathcal{D}}
\newcommand{\EE}{\mathcal{E}}
\newcommand{\FF}{\mathcal{F}}

%% Modes
\newcommand{\s}{\m{S}}
\newcommand{\li}{\m{L}}
\newcommand{\cl}{\m{C}}
\newcommand{\p}{\m{P}}

\newcommand{\lang}[1]{\mathbf{L}(#1)}

%% Contexts and Typing Judgment
\newcommand{\W}{\Omega}
\newcommand{\Sg}{\Sigma}
\newcommand{\xvdash}[2]{\sststile{#2}{#1}}
\newcommand{\xVdash}[1]{%
  \Vdash^{\mkern-8mu\scriptstyle\rule[-.9ex]{0pt}{0pt}#1}%
}
\newcommand{\confpot}[2]{\overset{#1}{\underset{#2}{\vDash}}}
\newcommand{\potconf}[1]{\overset{#1}{\vDash}}
\newcommand{\spanconf}{\vDash}
\newcommand{\confspan}[1]{\overset{(#1)}{\vDash}}
\newcommand{\confspanlocal}[1]{\overset{\langle #1 \rangle}{\vDash}}
\newcommand{\D}{\Delta}
%\newcommand{\G}{\Gamma}
\newcommand{\T}{\Theta}
\newcommand{\proves}{\vDash}
\newcommand{\w}{\omega}
\renewcommand{\C}{\mathcal{C}}
\newcommand{\set}[1]{\lvert\lvert#1\rvert\rvert}

\newcommand{\lin}[1]{\m{lin}(\overline{#1})}
\newcommand{\shd}[1]{\m{shd}(\overline{#1})}
\newcommand{\slin}[1]{\m{slin}(\overline{#1})}
\newcommand{\plin}{\; \m{purelin}}

%% Operational Semantics Predicates
\newcommand{\proc}[2]{\m{proc}(#1, #2)}
\newcommand{\msg}[2]{\m{msg}(#1, #2)}
\newcommand{\ichan}[3]{\m{ichan}(#1, #2, #3)}
\newcommand{\ochan}[3]{\m{ochan}(#1, #2, #3)}
\newcommand{\unavail}[1]{\m{unavail}(#1)}

%% Semantics
\newcommand{\step}{\; \mapsto \;}
\newcommand{\zerostep}{\step^{0}}
\newcommand{\timed}[2]{\{#1\}_{#2}}
\newcommand{\unit}{M}
\newcommand{\Step}{\Longrightarrow}
\newcommand{\info}{\mapsto}
\newcommand{\andin}{\; \m{and} \;}
\newcommand{\minus}{\setminus}
\newcommand{\fresh}[1]{(#1 \text{ fresh})}
\newcommand{\eval}[1]{\Downarrow_{#1}}

%% Expressions Semantics
\newcommand{\val}{\; \m{val}}

%% Expressions
\newcommand{\lam}[3]{\lambda #1 : #2 . M_x}
\newcommand{\inl}[1]{l \cdot #1}
\newcommand{\inr}[1]{r \cdot #1}
\newcommand{\case}[3]{\m{case} \; #1 \; (l \hookrightarrow #2, r \hookrightarrow #3)}
\newcommand{\pair}[2]{\left\langle #1, #2 \right\rangle}
\newcommand{\projl}[1]{#1 \cdot l}
\newcommand{\projr}[1]{#1 \cdot r}
\newcommand{\match}[4]{\m{match} \; #1 \; ([] \rightarrow #2, #3 \rightarrow #4)}
\newcommand{\eproc}[3]{\{#1 \leftarrow #2 \leftarrow #3\}}


%% Proof Terms
\newcommand{\ecase}[3]{\m{case} \; #1 \; (#2 \Rightarrow #3)}
\newcommand{\ecasecf}[3]{\m{case^{cf}} \; #1 \; (#2 \Rightarrow #3)}
\newcommand{\erecvch}[2]{#2 \leftarrow \m{recv} \; #1}
\newcommand{\erecvchcf}[2]{#2 \leftarrow \m{recv^{cf}} \; #1}
\newcommand{\erecvshift}[1]{\m{shift} \leftarrow \m{recv} \; #1}
\newcommand{\esendch}[2]{\m{send} \; #1 \; #2}
\newcommand{\esendchcf}[2]{\m{send^{cf}} \; #1 \; #2}
\newcommand{\esendshift}[1]{\m{send} \; #1 \; \m{shift}}
\newcommand{\ewait}[1]{\m{wait} \; #1}
\newcommand{\ewaitcf}[1]{\m{wait^{cf}} \; #1}
\newcommand{\eclose}[1]{\m{close} \; #1}
\newcommand{\eclosecf}[1]{\m{close^{cf}} \; #1}
\newcommand{\fwd}[2]{#1 \leftarrow #2}
\newcommand{\fwdp}[2]{#1 \overset{+}{\leftarrow} #2}
\newcommand{\fwdn}[2]{#1 \overset{-}{\leftarrow} #2}
\newcommand{\esendl}[2]{#1.#2}
\newcommand{\esendlcf}[2]{(#1.#2)^{\m{cf}}}
\newcommand{\ecut}[4]{#1 \leftarrow #2 \leftarrow #3 \semi #4}
\newcommand{\ecutna}[3]{#1 \leftarrow #2 \semi #3}
\newcommand{\espawn}[4]{#1 \leftarrow #2 \leftarrow #3 = #4}
\newcommand{\procg}[3]{\m{proc}(#1, #2, \overline{#3})}
\newcommand{\edelay}[1]{\m{delay} \; (#1)}
\newcommand{\ewhen}[2]{\m{when?} \; (#1) ; #2}
\newcommand{\enow}[2]{\m{now!} \; (#1) ; #2}
\newcommand{\etick}[1]{\m{tick} \; (#1)}
\newcommand{\ework}[1]{\m{work} \; \{#1\}}
\newcommand{\eget}[2]{\m{get} \; #1 \; \{#2\}}
\newcommand{\epay}[2]{\m{pay} \; #1 \; \{#2\}}
\newcommand{\procdef}[3]{#3 \leftarrow #1 \; #2}
\newcommand{\procdefna}[2]{#2 \leftarrow #1}
\newcommand{\casedef}[1]{\m{case} \; #1}
\newcommand{\labdef}[1]{#1 \Rightarrow}
\newcommand{\wk}[1]{\m{work}(#1)}
\newcommand{\eassume}[2]{\m{assume} \; #1 \; \{#2\}}
\newcommand{\eassert}[2]{\m{assert} \; #1 \; \{#2\}}
\newcommand{\eimpos}[2]{\m{impossible} \; #1 \; \{#2\}}
\newcommand{\eif}[1]{\m{if} \; (#1)}
\newcommand{\ethen}{\; \m{then} \; }
\newcommand{\eelse}{\m{else} \; }

%% Type Constructors
\newcommand{\lolli}{\multimap}
\newcommand{\tensor}{\otimes}
\newcommand{\with}{\mathbin{\binampersand}}
\newcommand{\paar}{\mathbin{\bindnasrepma}}
\newcommand{\one}{\mathbf{1}}
\newcommand{\zero}{\mathbf{0}}
\newcommand{\bang}{{!}}
\newcommand{\whynot}{{?}}
\newcommand{\semi}{\; ; \;}
\newcommand{\ichoiceop}{\oplus}
\newcommand{\echoiceop}{\with}
\newcommand{\ichoice}[1]{\ichoiceop \{ #1 \}}
\newcommand{\echoice}[1]{\echoiceop \{ #1 \}}
\newcommand{\fuse}{\bullet}
\newcommand{\mi}[1]{\mbox{\it #1}}
\newcommand{\lunder}{\mathbin{\backslash}}
\newcommand{\tassertop}{?}
\newcommand{\tassumeop}{!}
\newcommand{\tassert}[1]{\; \tassertop\{#1\}. \;}
\newcommand{\tassume}[1]{\; \tassumeop\{#1\}. \;}
\newcommand{\arrow}{\rightarrow}
\newcommand{\product}{\times}

%% Functional Types
\newcommand{\tproc}[2]{\{#1 \leftarrow #2\}}

%% Types with Potential
\newcommand{\pot}[2]{#1^{#2}}
\newcommand{\lollipot}[1]{\overset{#1}{\lolli}}
\newcommand{\tensorpot}[1]{\overset{#1}{\tensor}}
\newcommand{\potfop}{\phi}
\newcommand{\potf}[1]{\potfop(#1)}
\newcommand{\mlab}{M^{\textsf{label}}}
\newcommand{\mchan}{M^{\textsf{channel}}}
\newcommand{\mcl}{M^{\textsf{close}}}
\newcommand{\mall}{M}
\newcommand{\mint}{M^{\textsf{internal}}}
\newcommand{\mval}{M^{\textsf{value}}}
\newcommand{\mshd}{M^{\textsf{share}}}
\newcommand{\ms}{M_s}
\newcommand{\mr}{M_r}
\newcommand{\entailpot}[2]{\xvdash{#1}{#2}}
\newcommand{\exppot}[1]{\xVdash{#1}}
\newcommand{\texp}{\Vdash}
\newcommand{\pexp}{\vdash}
\newcommand{\paypot}{\triangleright}
\newcommand{\getpot}{\triangleleft}
\newcommand{\tgetpot}[2]{\getpot^{\{#2\}} #1}
\newcommand{\tpaypot}[2]{\paypot^{\{#2\}} #1}
\newcommand{\bigeval}[3]{#1 \Downarrow #2 \mid #3}
\newcommand{\share}{\curlyveedownarrow}
\newcommand{\zpot}{\overline{0}}


%% Temporal Types
\newcommand{\entailpotcf}[1]{\underset{\m{cf}}{\entailpot{#1}}}
\newcommand{\entailspan}{\vdash}
\newcommand{\entailtype}{\vdash}
\newcommand{\fpot}{\; @ \;}
\newcommand{\pay}[1]{#1^{1}}
\newcommand{\sync}[1]{#1^{2}}
\newcommand{\spanpot}[1]{\langle \pay{#1}, \sync{#1} \rangle}
\newcommand{\ichoicepot}[2]{\overset{#1}{\ichoiceop} \{ #2 \}}
\newcommand{\echoicepot}[2]{\overset{#1}{\echoiceop} \{ #2 \}}
\newcommand{\tlist}[1]{\m{list}_{#1}}
\newcommand{\plist}[2]{\m{list}_{#1}^{#2}}
\newcommand{\tdia}[1]{\Diamond #1}
\newcommand{\tbox}[1]{\Box #1}
\newcommand{\tforall}[1]{\forall . #1}
\newcommand{\texists}[1]{\exists . #1}
\newcommand{\Dia}{\Diamond}
\newcommand{\Next}{\raisebox{0.3ex}{$\scriptstyle\bigcirc$}}
\newcommand{\tdelay}[2]{
    \IfEqCase{#2}{%
        {1}{\next{#1}}%
        % you can add more cases here as desired
    }[{\Next^{#2} (#1)}]%
}%
\newcommand{\sch}[1]{\tau(#1)}
\newcommand{\lforce}[2]{[#1]_L^{#2}}
\newcommand{\rforce}[2]{[#1]_R^{#2}}
\newcommand{\force}[2]{#1 \circ (#2)}

\setlength{\inferLineSkip}{4pt}
\newcommand{\blue}[1]{{\color{blue}#1}}
\newcommand{\red}[1]{{\color{red}#1}}
\newcommand{\green}[1]{{\color{green}#1}}
\newcommand{\tick}{\blue{\m{tick}}}
\newcommand{\delay}{\red{\m{delay}}}
\newcommand{\when}[1]{\red{\m{when?}\;#1}}
\newcommand{\now}[1]{\red{\m{now!}\;#1}}
\newcommand{\noww}{\red{\m{now!}}}
\newcommand{\whenn}{\red{\m{when?}}}
% \newcommand{\vdashi}{\vdash^{\!\!{}^i}}
\newcommand{\vdashi}{\vdash^{\!\!\scriptscriptstyle i}}
\newcommand{\tock}{`}


%% Indices
\newcommand{\indv}[1]{\overline{\{#1\}}}
\newcommand{\ind}[1]{\{#1\}}


%% Syntactic Sugar
\newcommand{\config}{\mathcal{C}}
\newcommand{\cost}[2]{\mathrm{cost}(\proc{#1}{#2})}
\newcommand{\tcost}[2]{\mathrm{cost}(#1 \mapsto #2)}
\newcommand{\ccost}[1]{\mathrm{cost}(#1)}
\newcommand{\dc}{\mathcal{D}}
\newcommand{\ec}{\mathcal{E}}
\newcommand{\ac}{\mathcal{A}}
\newcommand{\st}[1]{\m{store}_{#1}}
\newcommand{\stack}[1]{\m{stack}_{#1}}
\newcommand{\queue}[1]{\m{queue}_{#1}}
\newcommand{\mapper}[1]{\m{mapper}_{#1}}
\newcommand{\fdr}[1]{\m{folder}_{#1}}
\newcommand{\lt}[1]{\m{list}_{#1}}
\newcommand{\bits}{\m{bits}}
\newcommand{\ctr}{\m{ctr}}
\newcommand{\trans}[2]{#1 \Longrightarrow #2}
\newcommand{\typetrans}[1]{\left\lvert{#1}\right\rvert}
\newcommand{\tree}{\m{tree}}
\newcommand{\bool}{\m{bool}}
\newcommand{\delayedbox}[1]{#1 \; \m{delayed}^{\Box}}
\newcommand{\delayeddia}[1]{#1 \; \m{delayed}^{\Diamond}}
\newcommand{\dom}[1]{\m{dom}(#1)}
\newcommand{\valid}[1]{#1 \; \m{valid}}
\newcommand{\invalid}[1]{#1 \; \m{invalid}}

%% Smart Contracts
\newcommand{\addr}{\m{addr}}
\newcommand{\ether}{\m{ether}}
\newcommand{\players}{\m{players}}
\newcommand{\lottery}{\m{lottery}}
\newcommand{\tint}{\m{int}}
\newcommand{\ballot}{\m{ballot}}
\newcommand{\tbool}{\m{bool}}
\newcommand{\lc}{\tlist{\m{coin}}}
\newcommand{\auction}{\m{auction}}
\newcommand{\object}{\m{object}}

%% Typing Judgments for Servers and Clients
\newcommand{\sentailpot}[1]{\prescript{}{S}{\xvdash{#1}} \hspace{2pt}}
\newcommand{\centailpot}[1]{\prescript{}{C}{\xvdash{#1}} \hspace{2pt}}


%% Sharing
\newcommand{\down}{\downarrow^{\m{S}}_{\m{L}}}
\newcommand{\up}{\uparrow^{\m{S}}_{\m{L}}}
\newcommand{\eacquire}[2]{#1 \leftarrow \m{acquire} \; #2}
\newcommand{\eaccept}[2]{#1 \leftarrow \m{accept} \; #2}
\newcommand{\erelease}[2]{#1 \leftarrow \m{release} \; #2}
\newcommand{\edetach}[2]{#1 \leftarrow \m{detach} \; #2}


%% Subtyping
\newcommand{\subt}[2]{#1 \leq #2}
\newcommand{\wsubt}{ <: }
\newcommand{\qsubt}[1]{\overset{#1}{\leq}}


%% Latex
%\newtheorem{theorem}{Theorem}
%\newtheorem{definition}{Definition}
%\newtheorem{lemma}{Lemma}
%\newtheorem{cor}{Corollary}


%%Global Semantics
\newcommand{\sinfer}[3]
{\inferrule
{#3}
{#2}
#1}
\newcommand{\enq}[2]{\m{enq}(#1, #2)}
\newcommand{\deq}[1]{\m{deq}(#1)}
\newcommand{\nil}{[]}
\newcommand{\elem}[1]{[#1]}


%% Channel typing
\newcommand{\eqdef}{\cong}


%% Types to Processes
\newcommand{\typeProc}[2]{#1 \Longrightarrow #2}

%% AARA
\newcommand{\abs}[1]{\left\lvert #1 \right\rvert}
\newcommand{\bin}[1]{(#1)_2}
% \newcommand{\ceil}[1]{\left\lceil #1 \right\rceil}
\newcommand{\bigO}[1]{\mathcal{O}(#1)}
% \newcommand{\ignore}[1]{\textcolor{red}{#1}}
% new \oset macro
\makeatletter
\newcommand{\oset}[3][-0.7ex]{%
  \mathrel{\mathop{#3}\limits^{
    \vbox to#1{\kern-2\ex@
    \hbox{$\scriptstyle#2$}\vss}}}}
\makeatother
\newcommand{\monus}{\oset{.}{-}}

%% Indexed Types
\newcommand{\cons}{\mathcal{C}}
\newcommand{\vars}{\mathfrak{v}}
\newcommand{\Vars}{\mathcal{V}}
\newcommand{\Cons}{\mathcal{C}}
\newcommand{\Tokens}{\mathcal{T}}
\newcommand{\Tokentypes}{\mathcal{K}}
\newcommand{\VTokens}{\mathcal{V}}
\newcommand{\TokSig}{\mathcal{S}}
\newcommand{\exchange}[3]{#1 \overset{#2}{\longrightarrow} #3}
\newcommand{\GlobalF}{\ensuremath{\mathfrak{f}}\xspace}
\newcommand{\depth}{\mathfrak{d}}

%% Two Counter Machines
\newcommand{\ins}{\iota}
\newcommand{\tcm}{\mathcal{M}}
\newcommand{\inc}[1]{\m{inc}(#1)}
\newcommand{\dec}[1]{\m{dec}(#1)}
\newcommand{\goto}{\m{goto}}
\newcommand{\zeroc}[1]{\m{zero}(#1) ?}
\newcommand{\halt}{\m{halt}}

%% UC stuff
\newcommand{\fcomm}{\mathcal{F}_{\msf{comm}}}
\newcommand{\B}[1]{\textcolor{blue}{#1}}
\newcommand{\wt}{\circled{w}}


%%% Local Variables:
%%% mode: plain-tex
%%% TeX-master: "pldi19"
%%% End:


\begin{abstract}
Software tooling for UC programming. Strongl typed language like Haskell we can define contexts for UC protocols/proofs in a modular way.
Results in on-paper definitions that can use clean abstractions through shell processes, and makes code easy.
We present Eventually and Optionally constructions and realize an ABA protocol using it. 
Show fuzz testing as tool for protocol analysis to discover buggy / in-secure code.
Fuzzing on ACast to detect problems with the code (relates directly to the ABA fuzzing because it is also an issue of thresholds).
Future work / discussion: contexts for more complex things like MPC. Write code directly but it translates to MPC instructions with the appropriate
communication assumptions.

\end{abstract}

\maketitle


%\begin{IEEEkeywords}
%component, formatting, style, styling, insert
%\end{IEEEkeywords}

\section{Introduction}
% RESEARCH QUESTIONS:
%     (RQ1). Is UC suitable and practical as a development framework rather than only a theoretical framework?
%             (i). Can existing UC models/techniques be improved for an engineering purpose?
%             (ii). What are the advantages of using UC for development?
%     (RQ2). Is UC as a development framework compatible with existing informal analysis techniques?
%             (i).  Fuzz testing is widely used, is a successful analysis tool, and is itself an engineering undertaking. Can we successful apply fuzzing to UC?
%             (ii).  Does the ideal functionality model and out realization of import aid in analyzng liveness in distributed protocols?

Universal composability is the leadeding framework for defining the security of message-passing cryptographic protocols between mutually distrustful parties.
Though largely used for cryptography, it has seem a reemergence in asynchronous distributed systems literature, especially due to the rise of decentralized protocols where applications are a composition of many interacting layers of other asynchronous distributed systems with different fault models and properties.
This highly compositional nature, combined with a new setting where financial incentives make properties like fairness, output distribution, and adversarial influence more relevant, have led to increasing interest in UC for distributed protocols.
At its core the framework's appeal is that it allows protocols to be designed in isolation and rely on idealized versions of other subprotocols and assumptions that behave like trusted third parties--called \emph{ideal functionalities}. 
It security definition allows proving of protocols, also in isolation, while ensuring that the proof holds when ideal functionalities are replaced with real protocols and the protocol is composed with arbitrarhy other protocols. 
Far from only a theoretical framework, this form of design and security definition lends itself well to the this setting.
%The UC composition operator and security definition allows replacing ideal functionalities with protocols that realize them (in isolation) to realize a larger protocol. 

The real-ideal paradigm plays a big role in the success and usefulness of the framework, because it allows defining the properties and security of a protocol through a single relation of indistinguishability with a protocol.
Comparing to an idealized program allows for expressing arbitrarily complicated and intertwined properties that can be cumbersome and error-prone with property-based definitions, i.e. a laundry list of assertions that must hold. 
In the earliest formulation of the framework, a simple two party computation (2PC) is described where the properties of secrecy and correctness are closely related, especialy in the byzantine setting when attempting to quantify how adversaries choose inputs, their influence on outputs, output fairness, and adversarial knowledge.
Expressing and analyzing these properties is crucial to meaningfully realize UC security, especially in the aforementioned world of decentralized protocols.
We proposet the following concrete research questions:
\begin{enumerate}[label=RQ\arabic*.]
\item Is the real-ideal paradigm useful even in identifying implementation-level bugs, and performing test cases analysis for protcol security?
\item Can UC implementation express and analyze protocols that express such properties and whether they hold in isolation and across layered and parallel composition?
\end{enumerate}
The real-ideal relationship is about testing generation against all environemnts, therefore, we select fuzz testing as our method for examing these two research questions. 
Not only is fuzzing a natural choice for testing the real-ideal relationship, but showing that we can meaningfully use such an important widespread tool in UC lends credence to our claim of UC as a candidate as a development framework. 
We use this tool to analyze a range of asynchronous protocols and idenfity implementation-level bugs that manifest themselves as distinguishing environments.

\todo{also do coin flipping under composition}

\todo{state why we need to do this for async because existing practices seem to fall short of doing this kind of analysis for them}


We believe that these advantages and features of UC mdoelling are specifically advantageous for asynchronous distributed protocol, and, in this work, we explore whether they can be realized as a software development framework using informal analysis techniques.


%\item Is UC suitable and practical as a development framework rather than only a theoretical one?
%    \begin{itemize}
%        \item [(i)] Can existing UC models/techniques be improved for easy of development?
%        \item [(ii)] What are the advantages of using UC for development?
%    \end{itemize}
%\item Is UC compatible with existing software analysis techniques?
%    \begin{itemize}
%        \item [(i)] Does the ideal functionality model and UC's polynomial time notion aid in reasoning about liveness in implementations?
%    \end{itemize}
%\todo{Hypotheses: why UC for async protocols and distributed systems are where this matters the most}
%
%\todo{maybe something along these lines: good as a dev framework for analyzing security and better researcher tooling for analyzing definitions without cumbersome extra work}.

There is a large amount of exsisting literature proposing programming language (PL) and formal verification tools for expressing UC security in limited, but useful ways~\cite{.}.
Contrary to our goal, working with new process calculi, a new domain-specific language, or a proving framework do little to make UC more accessible to non-cryptography researchers because 
\begin{enumerate}
\item Niche programming languages aren't well-suited to production environments and add to the already high learning curve.
\item The added obligations of using PL machinery for mechanizing proofs can add significant work for researchers as well~\cite{ironfleet,easycryptuc} \plan{easycrypt requires a whole new method of communication that makes translating definitions annoying} \todo{saying ``can add burden'' is an unproven claim but maybe still okay?}
\end{enumerate}

For researchers, software artefacts are a crucial for establishing more precise definitions that a broader audience can interact with, resuse, and rigorously test. 
Engineering aside, opening research up to broader falsiability of definitions/proofs is important.
For software engineers, there is a clear advanage to implementing code that matches academic definitions, the ideal functionality models provides a clear interface (specification, adversarial capabilities, and security guarantees) for software modules, and a UC environment facilitates more sophisticated protocol analysis because of its exposure to both honest party input/output and adversarial action.\todo{the point is along the lines mentioned that both scheduling and inputs/outputs allow more intelligent choice of actions for testing more interesting conditions...}
Simply put, UC is an efficient \todo{efficient?} harness for developing and testing protocos under byzantine conditions.
Furthremore, maintaining a symmetry between paper definitions and the resulting implementation solidifies the validity of existing security proofs.

\todo{maybe mention mainstream language earlier}
In order to address our research questions, and intuition, about UC, we implement it in a mainstream language, Haskell, and apply fuzz testing as our candidate analysis technique to explore UC's suitability as a development framework for distributed protocols.
Fuzz testing is a method of property-based testing that involves generating random inputs and checking output against a spec, and prior work shows that it can be as successful, if not better, than formal approaches like symbolic execution. 
It is widely used in practice and considered to be a vital tool in software testing.
Furthermore, implementing fuzzing is itself an engineering excercise that tests the framework's flexibility. 
Additionally, we propose a new mechanism for capturing an asynchronous network that extends asynchronous message deliver to asynchronous code execution. 
Our mechanism, an asynchronous wrapper, allows ITMs to schedule code blocks whose execution is controlled by the adversary and uses the new import mechanism for polynomial time to achieve eventual delivery~\footnote{The import mechanism was devised to overcome problems with previous versions of polytime like the ``length-of-input'' notion, and, as far as we know, we are the first to use it in this way}.
Moving away from traditional definitions that only work with messages, \emph{eventually} executing code blocks massively simplifies UC definitions and is an abstraction familiar to software engineers in other programming languages.
\plan{In fact, we posit that there is significant room for innovation on similar conventional UC-isms.}



We validate this vision for UC by implementing it in a mainsteam language and studying its compatibility with existing development practices. 
\todo{sopped edit here, te rest is not edited:} This implementation realizes the ITM computation model, and provides type checking of channel interfaces between module though we envision more descriptive type systems can be applied to this task.
We push modular and programming-inspired UC designs further by providing a new abstraction for realizing asynchronous, and other arbitrary, networks that both greatly simplifies paper-and-pencil definitions/proofs and an \emph{asynchronous code} abstraction familiar to programmers.
Finally, we employ fuzz testing, a critically important and highly successful testing strategy in modern software engineering, to our own implementations of canonical and modern byzantine agreement protocols and showcase
\begin{itemize}
\item the UC framework especially lends itself to fuzz testing by reducing complex distributed systems to a set of simple protocol and adversarial interfaces that greatly reduce the input space to be searched
\item the real/ideal paradigm already provides a built-in specification, the ideal functionality, of the inteded protocol to test protocols properties against
\item our novel design of asynchronous computing/networking \todo{something something}
\end{itemize}

Rough notes for the paragrapph. 
We implement bracha, ben-or, aba and inject faults into them. show how simple fuzzers that don't target specific vulnerabilities can find bugs that violate agreement/safety/etf
the better approach might be to idenftify only the set of bugs that would induce failures in safety and then say that those can be identified
but what about simpler bugs? that would require more meaningful testing but not what UC is good fr
existing fuzz testing is already good for bugs in single compiled progra like a single protocol running in isolation, so we don't think of finding those
but finding those of a distributed nature


Rather than bridge the gap between cryptography reserachers and protocol implementers, these approaches aid validation but accept and embrace the complexity of the framework.
Ultimately, the advantages of the proving framework, and the paper proofs that rely on it, are lost because of code that completely departs from them. 




%  Universal Composability (UC)~\cite{canettiUC} is the leading framework for defining security properties of cryptographic protocols.
%  It is considered the strongest definitional model since it guarantees the security properties hold even when the protocol is arbitrarily composed with
%  multiple concurrently-executing sessions of other protocol.
%  UC has gained popularity for analyzing cryptographic protocols due to its \emph{ideal world/real world} simulation mechanism.
%  In contrast to game-based cryptography where security properties are defined via attack games,
%  UC defines the \emph{ideal functionality}, a trusted third party that serves as the \emph{protocol specification}.
%  A core feature of the frameowk is its modularity where complex protocols are defined in terms of simple, ideal functionality building blocks that are secure by definition. 
%  A major drawback of UC is that security proofs can be quite complicated and difficult to analyze. 
%  Many related works \cite{ilc, easyuc, ipdl, etc} attempt to formalize UC security through a new programming language or defining UC security in an existing formal verification lanuage, and,  
%  although useful, such tools frequently never make their way to software engineering practice because
%  1. niche programming languages aren't suited to large scale development and can be difficult to use, and 2. the added proof obligations on the programmer can be extensive~\cite{ironfleet}.
%  Utimately, UC-secure prtocols end up being implemented in software frameworks that do not replicate UC, and, therefore, my invalidate on-paper proofs without further security proofs of the code.
%  
%  We address this gap in UC-driven software develoment by exploring how inforal security analysis of UC definitons, in our implementation of UC, can identify, and aid in eliminating, security vulnerabilities.
%  A key controbution of our software development framework for UC is proposing a noval new abstraction for capturing network assumptions. Our abstraction makes use of the novel import mechanism 
%  for polynomial time and fits nicely as a software abstraction. Our implementation and network model are, to the best of our knowledge, the first concretiziations of the import mechanism in this way.
%  Prior attempts at modelling asynchrnous networks, for example, focus primarily on adversarial delay of messages between parites. Such notions can require
%  protocols to encode signficant model-specific behavior which clutters functionalitiy (and protocol) definitions, places unecessary restrictions on protocol design, and is counter-productive for modular and reusable code. 
%  Out abstraction, on the other hand, acts are a wrapper around ITMs and offers a notion of \emph{asynchronous computation} in a way that is UC-compatible and firs well within a software framework. 
%  Not only is it natural for software development, but it also reduces functionality and protocol code to be almost model-agnostic. 
%  We use our network model of computation to focus solely on modeling and analzing distributed protocols in this work.
%  UC is most notably a framework for cryptographic protocols, however, in recent years the emergence of decentralized systems has renewed focus on modelling the security of asynchronous byzantine networks in UC~\cite{many,cit,ations}. 
%  Decentralized, namely blockchain, systems are highly modular with many protocols sharing state in unexpected ways and relying on numerous shared distrbuted sub protocols.
%  Naturally, compositional security in UC is ideal for capturing such protocols. 
%  
%  We opt for fuzz testing, by generating environments, as our analysis tool for three important reasons. First, numerous prior work has demonstrated the success of fuzz testing at identifyin software bugs. 
%  Some work suggests fuzzing is comparable in success to even formal approach such as symbolic execution. 
%  Second, the UC framework lends itself to compact definitions that compose through ideal functionalities making the input space for protocol parties and adversaries much smaller. 
%  Combined with our simple network model, UC modularity makes it easier for generated environmenst to explore more of the state space for a particular protocol or simulator proof. \todo{is this setting us up with an obligation to prove this statement with some coverage testing?}
%  Third, the real/ideal paradigm, and the ideal functionality, provide a built-in specification against which protocols can be tested and a method for comparing the two (the UC experiment). \todo{this last one is the least good the point is that we don't have to define state machines and added spec on top of a protocol, they should already exist from on-paper definitions it isn't an additional obligation to the programmer when using haskell saucy fuzzing.}
%  
%  Among the few related works that apply fuzzing to distributed systems, the work by Jepsen goes so far as to apply their methdology to a decentralized byzantine consensus protocol called Tendermint.
%  Jepsen deploys compiled Tenderming binaries and tests that operations on a distributed database are linearizable under various network conditions and limited byzantine behavior. 
%  As they admit in their results, the byzantine behavior that they capture is limited to replicated simple scenarios signing keys for multiple nodes. Designing a ful byzantine node for Tendermint is a considerable engineering effort.
%  Part of the hurdle is that testing a monolothic application like the Tendermint binary requires instrumenting and implementing every sub-component in the whole protocol.
%  Conversely, it is not possible, within \us, to test secrity under clock skews or race conditions in multi-threaded handling of network messages like Jepsen is able todo.
%  Hoever, this isn't a limitation of \us, but highlights a key distinction between us and works like Jepsen. \us is a \emph{development framework} rather than only a testing framework.
%  An application like Tendermint, implemented in \us, ensures that sub-components like network handling and clock timing are implemented and tested in isolation.
%  As mentioned in the previous paragraph, this makes allows us to test race conditions, should they arise, and capture byzantine complicated byzantine behaviort that Jepsen does not. 
%  \todo{feels like the point is there but not stated clearly enough. I'm trying to relate to the point of reduced input space from the previous paragraph to talk about why byzantine modelling is easy in \us and that not being able to do clock-skew type testing is a consequencer of Jepsen not being a development framework and having to work with existing monolithic binaries.}

\subsection{Below this is just notes}

\us is a not only a testing framework but a \emph{development framework} where subcomponents and sub protocols are created and tested in isolation.

Combined with paper and pencil proofs, we can reply on the composition theorem to test more complex protocols like Tendermint without worrying about the full code stack.
For example, even though \us can doesn't allow testing of race conditions resulting from multi-threaded handling of network messages 

our framework ensures that such sub-components of Tenderming are created and tested
within UC and composed correctly. 

Jepsen fuzz test a compiled binary for Tendermint and check that updates to a distributed database are linearizable under various network conditions and some byzantine behavior.



fuzz testing is a proven technique even when compared to formal analysis tool. this shows promise as a simple informal tool for protocol analysis
compared to things like jepsen running and testing protocols is much easier without any of the engineering effort require because of the subtle ways in which UC is designed to mimic a realistic computing environment and network
fuzzing byzantine messages is easier with UC for a few reasons:
* the simple interface for adversarial behavior makes it easy to play with message ordering and deliver / dropping messages
* the ideal functionality abstraction removes sub-protocol detail making the space of byzantine messgaes much smaller / manageable for fuzz testing

this leads into another point about comparing with jepsen. they test existing compiled binaries and so end up testing the full stack of code as part of their testing
they can test race conditions in how messages are received, for exampe, and standalone UC fuzz testing can not replicate that
but this is precisely where UC shines: not just testing code but developing it within the UC framework all such sub-protocols are tested and designed individually and larger, composed protocols only testing teir behavior with the ideal functionality model 



We address the state of software development around UC by exploring the extent to which informal security analysis of UC definitions, in our UC implementation can identify, and aid in elminiating, security vulnerabilities such as safety, correctness, and liveness in distributed systems .\todo{this first line needs to be better}.  
A key contribution of our software development framework around UC begins with proposing a novel abstraction for capturing different network models.
Prior attempts to capture, for example, asynchronous communication focus primarily on adversarial delay of messages between parties. 
Such notions can require protocols to encode significant model-specifi behavior in their definition which lutters definitions, places unecessary restrictions on protocol design, and is counter-productive for modular and reusable code. 
Our abstraction, on the other hand, acts as a wrapper around ITMs and extends the abstraction to asynchronous \emph{computation}: adverasrially delayed code rather than messages.  
Not only is this abstraction more natural for a software setting, but it also reduces functionalities and protocols to become almost model-agnostic in their definition. 
For asynchronous networks we dub our construction the \emph{asynchronous wrapper}. 
It uses the import mechanism introduced in UC to provide \emph{eventual} deliver guarantees.
To the best of our knowledge, it is the first concretization of the import mechanism with eventual delivery. 



\todo{a statement wrapping all this up in a key takeaway and positive result of this work in the context of the state of UC}.


\section{Background} \label{sec:background}
The UC framework~\cite{canettiUC} defines security using
a real-world / ideal-world paradigm:
the real world features a protocol $\pi$ and an arbitrary adversary \A that controls corrupt parties;
To achieve the ideal/real security relation, we roughly need to show that ``any possible attack in the real world can also be exhibited
in the ideal world''.
This means constructing a simulator \Sim that plays the role of an adversary in the ideal world.
What makes UC the strongest definitional model is that the inputs and number of honest parties are chosen \emph{adaptively}
and \emph{dynamically}, by a distinguishing environment \Z, which can also communicate with the adversary and interleave its interactions with honest parties.

The UC framework is defined on top of a communication model called interactive turing machines (ITMs), in which multiple Turing-complete processes run concurrently in a system and communicate by exchanging messages over channels~\cite{canettiUC}.
%Although the Turing-complete computations can be instantiated in any reasonable core calculus, the approach to message passing in ITMs has some essential but subtle restrictions.
%In order to do cryptographic analysis, we need to make reductions to (ordinary sequential) probabilistic polynomial time computations (PPTs).
%This rules out, for example, the ordinary semantics of $\pi$-calculus, which introduces unbounded non-determinism with the possible scheduler choices.
%This adversary model leads to a flexible approach to composition, which we'll say more about shortly.
The UC security definition is more formally defined by the following relationship between two UC exections. 
Here, \m{execUC} defines a UC experiment that takes four parameters: an environment \Z, a protocol $\pi$, a functionality \F, and an adversary \A.
Then, $\m{execUC}$ creates the main ITMs for each of these parameters and the internal parties that execute the protocol $\pi$ are created by
activation messages that are sent by these main ITMs.

We say that a protocol $\pi$, with access to a functionality $\F_1$, realizes an ideal functionality $\F_2$
if for every adversary $\A$, we can construct a simulator $\Sim$, such that the real and ideal worlds are indistinguishable to any environment \Z:
\begin{equation}
  \label{eqn:emulation}
  \forall \A \; \exists \Sim \; \forall \Z, \; \; \msf{execUC} \; \Z \, \idealP \, \F_2 \, \Sim \sim \msf{execUC} \; \Z \, \pi \, \F_1 \, \A
\end{equation}
We denote indistinguishability with the $\sim$ symbol, and it means that the ensembles of distributions resulting from the output of \Z in each execution, over all possible randomness and security parameters, have \emph{statistical difference} that is negligible in $k$. This relationship between two \m{execUC} instances is called \emph{emulation}. 
Generally, emulation deals with arbitrary protocols in both worlds, however, for UC realization (above) specifically the protocol in the real world is \idealP, or the \emph{dummy protocol}, that forwards messages between $\F_2$, \Z,  an \A. 
Similarly, the real world can also contain an ideal functionality $\F_1$ that may captures network assumptions or assumed primitives the protocol relies on. 
This real world is also called the $\F_1$-hybrid world.
Defining \msf{execUC} in our core calculus, and especially reconciling its dynamic nature with our static type system will be the central technical challenge we tackle later.

\paragraph*{\textbf{Composition Notation}}
The UC framework is designed to encourage a highly compositional and modular design approach, where we analyze single-instance protocols in isolation, then apply generic composition operators to build more complex systems.
Encoding the standard theory of UC composition in our core calculus is one of the main ways we validate the expressiveness of our language design.
Here we summarize the main composition theorems using category notation, where objects are functionalities and arrows are protocols.
First, if $\pi$ realizes a single instance of $\F_2$ in the $\F_1$-hybrid world corresponding exactly to the definition of \m{execUC} above, we use the notation:
\[
	\F_1 \xrightarrow{\pi} \F_2
\]
%This means in the real world $\pi$ makes use of a single instance of $\F_1$, and the ideal world consists of a single instance of $\F_2$.
The first composition theorem \ref{thm:singlecomp} states that this relation is transitive, where $\rho \circ \pi$ is a generic composition operator that combines two protocols by connecting the $\rho$ to $\pi$ where $\rho$'s communication with $\F_2$ is relayed as input to $\pi$. 
%$\rho \Leftarrow \F$ channels in $\rho$ to the $\pi \Leftarrow \Z$ channel of $F$. That is, when protocol $\rho$ communicates with its ideal functionality, it is is relayed as subroutine input to $\pi$.
\begin{theorem}[Composition]\label{thm:singlecomp}
\begin{mathpar}
\inferrule*[right=single-compose]
{
	\F_1 \xrightarrow{\pi} \F_2 \semi \F_2 \xrightarrow{\rho} \F_3 \\
}
{
	\F_1 \xrightarrow{\rho \circ \pi} \F_3
}
\end{mathpar}
\end{theorem}
To prove this theorem requires constructing a straightforward simulator that combines the underlying simulators of $\pi$ and $\rho$, and the complete security reduction involves translating a distinguisher $Z$ for the combined protocol $\rho \circ \pi$ to a distinguisher $Z^*$ for either $\pi$ or $\rho$ individually.
Although the proof is straightforward, the precise statement of it in our framework serves as good validation for the expressiveness of our framework: our theorem and proof are parametric in the session type of the protocol, i.e., the theorem places no restrictions on the communication pattern used by the underlying protocol and functionalities.
We further validate our approach by expressing the multi-session extension, $!\F$, of \F that let's \Z spawn arbitrary parallel sessions of \F and its corresponding composition theorem: 
$!\F_1 \xrightarrow{\pi} \F_2$, $!\F_2 \xrightarrow{\phi} \F_3 \implies !\F_1 \xrightarrow{\phi \circ !\pi} F_3$.

%The flexibility of the UC framework extends to allow \Z to spawn an arbitrary number of sessions of some \F, denoted $!\F$, in parallel, and the theorem 
%$!\F_1 \xrightarrow{\pi} \F_2$, $!\F_2 \xrightarrow{\phi} \F_3 \implies !\F_1 \xrightarrow{\phi \circ !\pi} F_3$ ensures that composition holds even in this setting.
%This theorem allows analysis of a single protocol in isolation to guarantee security under parallel composition, and it is crucial in proving full UC composition that ensures security under composition 
%with parellel sessions of arbitrary protocols.

%The next generic operation is the multi-session extension of $\F$, denoted $!\F$, which provides $\pi$ with an arbitrary number of instances of $\F$, each tagged with a separate \textsf{sid} (for \emph{session identifier}).
%Here is a central aspect of UC's flexibility, that the environment gets to determine at runtime the exact number and values of \textsf{sid}'s, with no static bound required.
%The Universal Composition theorem says that composition even holds in this setting, which we state as
%$!\F_1 \xrightarrow{\pi} \F_2$, $!\F_2 \xrightarrow{\phi} \F_3 \implies !\F_1 \xrightarrow{\phi \circ !\pi} F_3$.
%This theorem is essential to the appeal of UC as a framework because it encourages simple analysis of a single protocol in isolation (the proof that $\pi$ realizes one instance of $\F_2$), which is then safe to use in protocols like $\phi$ that rely on multiple concurrently-running subsessions of it.

\paragraph{Universal Composability and ITMs}

%Our approach, following ILC~\cite{ilc}, is to encode the ITMs framework as faithfully as possible.
%%This is because the final step in a UC proof is to show that a distinguishing environment Z can be leveraged to construct a polynomial time solution to a hard problem like Discrete Log.
%The basic rule that ITMs follow is that only one process is active at a given time. 
%Specifically whenever a process writes to one of its outgoing channels, the unique process that holds the read end of that channel is immediately activated next.
%In this way the message scheduling is essentially deterministic so it can be easily simulated by a sequential computation.
%This discipline means that modeling inherently non-deterministic phenomena, like network schedulers, requires us to explicitly offer choices to an adversary process defined in our model.
The ITM model has the advantage that only one process is active at a given time. It results in message scheduling that is essentially deterministic and easily simulated by sequential computation. 
Despite having an established message pattern, it is still not straightforward to define a notion of polynomial runtime for ITM systems.
We need a way to make local judgements about each of \A, \Z, $\pi$, and \F and conclude the entire $\msf{execUC}$ overall is polynomial time. 
%Looking at the order of quantifiers in the UC emulation definition from Relation~\ref{eqn:emulation}, we need a way to make local judgments about each $\A$, $\Z$, $\pi$, and $\F$ individual, and conclude that the entire $\msf{execUC} \; \Z \; \pi \; \A \; \F$ experiment overall is polynomial time as a result.
%We follow Canetti's approach~\cite{canettiUC}, which is to keep track at runtime of quantity called ``import tokens'' and assign a runtime budget based on these.
We use the ``import tokens''~\cite{canettiUC} notion of polynomial time and track runtime through tokens that are capped by a total and that can be passed between processes. 
%These tokens can be passed among the processes along with the messages sent on each tape, but the total amount of tokens must be conserved (neither created nor destroyed), and locally each process cannot take more steps than (a polynomial of) the amount of import tokens it has stored.
In this model, we say $P$ is \emph{locally polynomial time} if for \emph{any} evaluation context $e[\_]$, at any step $t$ during its execution,
\[
\#\textsf{stepsTaken}(e[P])_{t} \le T(n_{\textsf{in}} - n_{\textsf{out}})
\]
where $n_{\textsf{in}} - n_{\textsf{out}}$ is the net number of import tokens that $P$ possesses, and $T$ is an arbitrary polynomial.
This immediately ensures that if the system starts out with a total number of tokens bounded $n \in poly(k)$, then the overall number of steps taken by any of the processes in the system is also $poly(k)$.
The arbitrary polynomial $T$ serves as a slack parameter that allows, for example, the emulation of a universal turing machine program (which may incur up to quadratic overhead).
Our approach, described in Section~\ref{sec:basic}, is to encode this notion directly into the type system, by tracking import tokens and statically enforcing this polynomial runtime constraint.



\section{Related Works} \label{sec:related}
There is a large volume of work that introduces tools for modeling UC protocols and reasoning about UC security.
Some even mechanize UC proof-generation~\cite{certicrypt, easycrypt, cryptoverif, cryptol, fstar}.
However, unlike our work, many of them only do so in the standalone setting without any composition guarantees.
They also do not completely capture a robust polynomial-time notion or realize generalized composition. 

Canetti et al.~\cite{easyuc} and Barbosa et al.~\cite{barbosa} both build off EasyCrypt's game-based security definitions to reach UC-like simulation-based definitions.
They both mechanize security proof generation and allow more complicated reasoning for indistinguishability given a formal term capping the environments capabilities. Out work, like ILC defines indistinguishability in terms of a security parameter $k$, but without any program logic for security. 
Barbosa introduces a limited polynomial time notion for its definitions, whereas EasyUC~\cite{easyuc} does not encode any runtime guarantees in its programs. 
For example, it cannot detect indefinite message passing between \A and \F. Neither, however, capture full UC composition.
Barbosa realizes simplified UC~\cite{suc}, and, like EasyUC, does not capture dynamic party creation or the multisession operator.
In contrast, NomosUC derives a polynomial-time bound on ITMs statically with a proof that the computed bounds are correct.

%EasyUC~\cite{easyuc} introduces a toolset built on top of the existing EasyCrypt~\cite{easycrypt} toolset to model UC protocols and generate proofs of security.
%It moves past the game-based security limitation of EasyCrypt and achieves the broader simulation-based definitions of UC, but does not encode any runtime guarantees.
%%EasyUC mechanizes UC's notion of simulation-based security and formally verifies UC realization--something Nomos doesn't attempt to do.
%%The major limitation of EasyUC is that it does not encode any guarantees on the runtime of any process.
%For example it can not detect a functionality and the adversary can exchange messages \emph{indefinitely} in their key exchange example. It also can not capture full UC composition because it is limited to a statically determined number of parties in any execution.
%%NomosUC, on the other hand, proposes the full UC composition theorem, a robust polynomial time notion that relies on the import mechanism introduced in UC, and a full expressive language that can support arbitrary UC executions.
%
%Barbosa et al.~\cite{barbosa} builds off EasyCrypt as well, but introduce a polynomial time notion to their handling of UC. 
%However, it is limited by the procedure call commnication method of EasyCrypt, limiting expressiveness, and only realizes the simplidied SUC~\cite{suc} without dynamic party creation.
%%They're able to relate the guarantees provided by EasyCrypt to the execution time of an adversary that can break the security of the protocol. 
%%Barbarosa also must contend with the procedure call communication of EasyCrypt limiting the expressiveness of the framework. 
%%Furthermore it suffers similar drawbacks to EasyUC, and all works mentioned in this section, in that does not fully capture dynamic party creation in UC, and they realize only a simplified version of UC~\cite{suc}.

ILC proposed by Liao et al.~\cite{ilc} is another work closely related to ours which introduced the idea of using a write token
to resolve read and write non-determinism.
But it suffers some of the same drawbacks as EasyUC and the work of Barbosa et al.~\cite{barbosa},
because it does not support full composition and is also limited to a static number of parties in its UC definition.
It improves on EasyUC and provides a polynomial time notion, but ends up requiring simulation in both directions to prove emulation. 
This means that even simple protocols like a ping-response server cannot be judged secure.

The work on IPDL~\cite{ipdl} by Morrisett et al. also aims to mechanize proofs of security and improves upon EasyUC by providing a better notion of emulation,
more akin to the UC framework, and symbolically tracks the run time of straight-line programs (and those with statically upper-bounded loops).
IPDL further implements a unique communication mechanism that imposes a static dependency between channels ``firing''.
%It symbolically tracks the run time of programs but can only do so for straight-line programs or those with statically upper-bounded loops.
However, it precludes expressing constructs like the multisession operator, presented later in this work. 
%The operator allows creation of an arbitrary number of subsessions of a functionality and is critical to realizing the full UC composition theorem.

We summarize the imporant features of the most closely related work to NomosUC in Figure~\ref{fig:relatedworks}.
%The columns are broken up according to the features mentioned in the above discussion.
The first row discusses whether the environment can create an arbitrary number of parties at runtime. %There is no partial-support in this category, either it is supported or it isn't.
The second row broadly covers whether there is any restricted or full support for runtime or polynomial-time analysis or dynamic tracking. %restricted poly time (partiais performed by the project
%Notice that there is partial support where ILC provides a restricted notion of polynomial time.
The third row determines whether full UC-style composition is supported, i.e. replacement of an arbitrary number of functionalities with realizing protocols (the existence of a multisession operator is essential). 
The last row highlights a design choice in communication through channels or through programming-style procedure calls.
The distinction plays an important role in how communication is captured in each work, and whether arbitrary communication patterns from UC are not fully supported.

% Colums: blank | parties at runtime | notion of UC | polynomial time | generalized composition operator | channels | procedural 
\begin{figure}[H]
\centering 
\begin{table}[H]
	\vspace{-2em}
	%\scalebox{0.7}{\begin{tabular}{l | c  c  c  c  c}
	%& \rot{Dynamic \# Parties} & \rot{Polytime Notion} & \rot{General Composition} & \rot{\parbox{3cm}{Channels/Procedures}} \\
	%\hline
	%IPDL~\cite{ipdl} & \emptycirc[0.75ex] & \fullcirc[0.75ex] & \fullcirc[0.75ex] & Channels \\
	%%\hline
	%EasyUC~\cite{easyuc} & \emptycirc[0.75ex] & \emptycirc[0.75ex] & \emptycirc[0.75ex] & Procedure Calls \\
	%%\hline
	%Barbosa et al.~\cite{barbarosa} & \emptycirc[0.75ex] & \fullcirc[0.75ex] & \fullcirc[0.75ex] & Procedure  Calls \\
	%%\hline
	%ILC~\cite{ilc} & \emptycirc[0.75ex] & \halfcircleft[0.75ex] & \emptycirc[0.75ex]  & Channels    \\
	%%\hline
	%NomosUC (this work) & \fullcirc[0.75ex]  & \fullcirc[0.75ex]  & \fullcirc[0.75ex]  & Channels  \\
	%%\hline
	%\end{tabular}}
	\scalebox{0.7}{\begin{tabular}{l | c  c  c  c  c}
	& IPDL & EasyUC & Barbosa et al. & ILC & \textbf{NomosUC}\\
	\hline
	Dynamic \# of Parties & \emptycirc[0.75ex] & \emptycirc[0.75ex] & \emptycirc[0.75ex] & \emptycirc[0.75ex] & \fullcirc[0.75ex] \\
	%\hline
	Polytime Notion & \fullcirc[0.75ex] & \emptycirc[0.75ex] & \fullcirc[0.75ex] & \halfcircleft[0.75ex] & \fullcirc[0.75ex] \\
	%\hline
	General Composition & \fullcirc[0.75ex] & \emptycirc[0.75ex] & \fullcirc[0.75ex] & \emptycirc[0.75ex] & \fullcirc[0.75ex] \\
	Security Proofs & \fullcirc[0.75ex] & \fullcirc[0.75ex] & \fullcirc[0.75ex] & \emptycirc[0.75ex] & \emptycirc[0.75ex] \\
	%\hline
	Channels/Procedures & Channels & Procedure Calls & Procedure Calls  & Channels & Channels 
	%\hline
	\end{tabular}}
\end{table}
\vspace{-1em}
\caption{Aspects covered by related works.
Empty circle \emptycirc[0.5ex] indicates no support, half circle \halfcircleft[0.5ex] indicates partial support,
and full circle \fullcirc[0.5ex] indicates full support.}
\vspace{-1em}
\label{fig:relatedworks}
\end{figure}

\paragraph*{\textbf{Session Types}}
On a different thread, the core calculus of NomosUC is inspired from resource-aware session types~\cite{das2018work} which combines
session types~\cite{HondaCONCUR1993, HondaESOP1998, HondaPOPL2008,caires2010session, ToninhoESOP2013, PfenningFOSSACS2015,
WadlerICFP2012} and automatic amortized resource analysis~\cite{Hofmann03AARA,HoffmannW15}.
We build off binary session types in this work, however other formuations such as multi-party session types exist that allow describing a protocol between many processes rather than bi-directional~\cite{Capecchi10CONCUR}.
Though appearing as a better fit, multiparty session types are not well-suited for dynamic creation of parties since they require statically specifying a global communication protocol between all processes.
This would not allow for the possibility of spawning a dynamic (or even static) number of processes in the middle of a communication, which is central to UC.
Moreover, binary session types also provide support for cost analysis that is fundamental to the import mechanism, a core contribution of our work.

%Session types were introduced by Honda et al.~\cite{HondaCONCUR1993} to describe and enforce bi-directional communication protocols
%in message-passing systems.
%Resource-aware session types~\cite{das2018work} add potential annotations to session types for execution cost analysis
%of distributed protocols.
%Another formulation of session types that aren't binary~\cite{Derakhshan21LICS} are multiparty-session types~\cite{Capecchi10CONCUR} that
%More related to security, recent work has integrated session types with information flow type systems, both
%in the binary~\cite{Derakhshan21LICS} and multiparty~\cite{Capecchi10CONCUR} setting.
%The former uses logical relations to define non-interference while the latter
%guarantees a form of access control and secure information flow.

%There is large body of similar work that introduces process calculi, some extensions of $\pi$-calculus, like ILC.
%Mateus et al.~\cite{mateus} for example introduces process calulus for simpler, sequential composition but is constraints to a schedular-based construction where probabilistic state transitions follow unifor distribution at every step.
%SymbolicUC~\cite{symbolicuc} \todo{finish oter symbolic logic and state their weaknesses and that ther are subsumed by ILC}.
%
%NomosUC adds to the body of prior work by using resource-aware session types~\cite{das} to describe protocols, functionalities, and their behavior. 
%Session types express the steps and communication in a protocol at the type level, and offer greater tooling for creating large protocols with smaller, modular pieces. 
%Resource-aware session types add a mechanism called potential, that we use to implment the import mechanism described in the UC framework.
%Import provides a more precise notion of polynomial time in UC (refer to the UC framework~\cite{UC}), and, to the best of our knowledge, this is the first such work to implement and create tooling around it. \todo{surely there is better wording than ``create tooling around it'' to get across the point I'm trying to make}.

%One of the most relevant works to our own is EasyUC~\cite{easyuc}. 
%EasyUC uses the existing EasyCrypy~\cite{easycrypt} toolset to model UC protocols and mechanize proof generation. 
%It departs from EasyCrypt's limtations to game-based security definitions (lacking simulation-based composition).
%However, it still lacks a notion of polynomial time. The authors, themselves, mentions that it can't detect deviant behavior like the adversary and functionality passing messages between each other indefinitely. 
%Our use of the import mechainsm and session types let us reason about polynomial time in the sytem of ITMs encompassed by \msf{execUC} but also locally for \textit{open} terms. 
%Furthermore, import in NomosUC lets us have guarantees of termination as well by the polyomial import constraints added to UC by Canetti et al.
%
%Liao et al. introduce executable UC through a new process calculus called ILC~\ref{ilc}.
%This work adds some notion of polynomial time although it proves to be too restrictive. 
%It results from the fact that poly-time can only be reasoned about for \textit{closed} terms like a full UC execution.
%In order to reason about polynomial time for a particular protocol $\pi$ we must reason over all possible other terms that connect to $\pi$ and require that it is polynomial in all such cases.
%A simple ping-response server can not be proven to by poly-time in this definition for a deviant other ITM that connects to $\pi$. 
%In Nomos, however, as mentioned above, open terms are limited to polytime regardless of the connected other terms because of the import mechanism and the NomosUC type system that guarantees termination. 
%
%Other works that rely in symbolic modelling of cryptography, for example, SymbolicUC~\cite{symbolicuc}, are subsumed by the above ILC work and similarly lack any polynomial time notion. 
%\todo{Say something about $\pi$-calculus with probabilistic polynomial time extensions}.
%
%To the best of our knowledge, this is the first work to deal with the new import notion of polynomial time introduced to the UC framework in 2018.
%A few other works refer to the import mechanism, but it is restricted to simply defining the import a protocol is given.

%\begin{figure*}
%\begin{center}
%\begin{tabularx}{\textwidth}{|p{0.2\textwidth}|p{0.2\textwidth}|p{0.2\textwidth}|p{0.2\textwidth}|}
%\hline \\
%        & Polynomial-time termination                                  & simulator type mismatch                             & emulation fully proven \\ 
%\hline \\
%NomosUC & Guaranteed termination, local polynomial time for open terms & PPT Simulator under import constraints of adversary &  \\
%\hline \\
%IPDL    & Statically typed loops and straightline programs only        & 													 & proof generation and full emulation \\
%\hline \\
%EasyUC  & No polynomial time guarantee or guaranteed termination       & 													 & machine checked \\ 
%\end{tabularx}
%\end{center}
%\end{figure*}

%easyUC:
%* can not dynamially create new instances of parties/functionalities must statically determine the number of functionalities/parties spawned
%* 
%
%
%The work of Liao et al.~\ref{ilc} is the closest to our own
%It proposes a new process calculus called ILC and a concrete implementation of the UC framework.
%The type system it introduces ensures that correctly types programs can be represented as ITMs.
%However, one drawback of the ILC work is that its polynomial time representation 
%
%
%The EasyUC approach uses the existing EasyCrypt toolset to implement model UC protocols and mechanize the generate of UC-security proofs and proofs of secure composition.
%This work aim considerably higher than our work in actually attempting to generate proofs for their protocols. 
%However, this work falls short in being able to capture any notion of resource bound computation whereas we are able to make guarnatees about polynomial bounds on our system of ITMs and even guarantee termination of programs through our realization of the import mechanism.
%The EasyUC work accepts that not even infinite loops of communication can be caught and, therefore, termination of protocols can't be guarnateed either whereas the import mechanism in Nomos ensures that such infinite loops can not stall protocol progress.

%Another work similar to our own is the Symbolic UC by B\"{o}hl and Unruh.
%This works uses an applied $\pi$-calculus to symbolically model UC protocols and analyze them.
%Similar to the EasyUC work, the goals of this work are somewhat orthogonal to the our own goals.
%However, Symbolic UC does attempt to create an implementatio of UC using the $\pi$-calculus however neglects to address any issues of polynomial runtime.
%
%Perhaps the closes work to our own is that of Liao et al.~\cite{ilc} that builds an executable version of the UC framework by introducing a new process calculus called ILC.
%ILC introduces a type system that guarantees that ILC programs (i.e. functionalities, protocols, etc) can be expressed as ITMs as in the UC framework.
%However, one drawback of ILC is that it's notion of polynomial time ends up being too restrictive.
%In ILC only closed terms without any unbonded variables, i.e. and entire UC exection of a system of ITMs, can be shown to be polynomial in their definition of polynomial time.
%Proving polynomial time for open terms, such as a protocol $\pi$, requires reasoning over all possible contexts in which the protocol could exist however such a definition of polynomial time becomes too restrictive where even a simple ping-responde server protocol wouldn't be considered polynomial time.


\section{Discussion and Future Work}
\input{sections/discussion}

\bibliographystyle{ACM-Reference-Format}
\bibliography{nomosuc}

\newpage

\appendix

\pagebreak

\end{document}
