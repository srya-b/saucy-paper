To check programs with our UC framework in practice, we extend the type checker of the
PReST language~\cite{prest} with additional primitives for modeling UC. The PReST
language, through probabilistic refinement session types, already allows users to define
probabilistic concurrent protocols and \emph{symbolically} reason about the programs that
realize said protocols. The following $\textsf{pcoins}[p]$ type is an illustrative example of 
specifying probabilistic types in PReST:
\begin{align*}
  \textsf{pcoins}[p \mid 0 \leq p \leq 1] \triangleq 
    \oplusP\{ \Textsf{true}^{p} : \textsf{pcoins}[p]
           ,\ \Textsf{false}^{1 - p} : \textsf{pcoins}[p] \}
\end{align*}
The $\textsf{pcoins}[p]$ is parameterized by a symbolic variable $p$ from the
interval $[0, 1]$. This type represents a stream of biased coins, which through
the $\oplusP$ connective, one could either receive a $\Textsf{true}$ message
with symbolic probability $p$ or receive a $\Textsf{false}$ message with
probability $1 - p$.  During type checking, the checker traverses the syntax
tree of processes and synthesizes logical constraints to ensure the operations
performed by processes are always in strict coherence with the types of
channels. The constraints are then checked by an SMT solver backend
(z3~\cite{z3} or cvc5~\cite{cvc5}). If these constraints are solvable, then the
meta-theory of PReST ensures that the distribution of communicated messages over
a probabilistically typed channel at runtime is always \emph{consistent} with
the probability of messages specified by the type. For this reason, the PReST
type checker conservatively accepts processes that generate constraints solvable
by the SMT backend as well-typed.

Importantly, PReST features a cost analysis mechanism that allows users to
statically check the cost of programs. For example, one could define a 
\lst{debias1} program for converting a biased coin into a fair coin.
\begin{lstlisting}
  proc debias1[$p$ | $0 < p < 1$] (x : coin1[$p$])
$\vdash^{\color{red} (2 \cdot p^2 - 2 \cdot p + 1) / (2 \cdot p \cdot (1- p)) }$ 
  (y : $\oplusP${ true$^{0.5}$ : $\unit$, false$^{0.5}$ : $\unit$ }) = ...
\end{lstlisting}
In the elided body of \lst{debias1}, it is constructed such that it uses 1 unit 
of potential to recurse. The expression $(2 \cdot p^2 - 2 \cdot p + 1) / (2 \cdot p \cdot (1- p))$
which annotates the turnstile is a bound on the \emph{expected cost} of running
\lst{debias1} until termination. Similarly to checking probabilistic refinement 
session types, the PReST type checker synthesizes logical constraints regarding
the potential used by process invocations and attempts to solve these
constraints using its SMT backend. While the cost analysis feature of PReST is
related to the UC framework we propose here, there are subtle differences that
prevent its direct application for UC. For instance, in PReST, potential is
directly given to programs and programs that are running in parallel may
exchange potential to perform work. This is different from the UC setting which
requires programs to exchange import instead of potential. It is then through
import that processes may derive the potential they need for computation.
So unlike PReST processes whose potential is irreversibly lost upon usage, the
import mechanism of UC allows for processes to continue generating potential so
long as the amount generated is upper bounded by a polynomial of import.
To adapt PReST for usage with UC, we first modified its potential exchanging types to
exchange import instead. In order for processes to properly utilize import, we 
extended PReST with new constructs that derive potential from import. Finally,  
we adjust the cost checking algorithm of PReST to account for these changes and
make sure that the derived potential are always polynomially bounded.
