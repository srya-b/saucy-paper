\section{The first attempt to re-write after CSF}
A major contribution of NomosUC is its ability to ensure resource-bounded computation
as dictated by UC guidelines.
To this end, we introduce a static notion of \emph{import tokens} that provide a static upper
bound on the execution cost of NomosUC programs.
We use the exampe of a simple database functionality to explain and motivate the key design decisions around import in NomosUC.

The database functionality \Fdb (below) maintains a simple list with two calls from parites: \msf{add} which appends an item to the list and \msf{get} which returns the whole list. 
\begin{bbox}[title={Functionality $\F_{\msf{db}}$}]

Initialize list $l := []$

\OnInput \inmsg{add}{$x$} form $P_i$:
    \begin{ritemize}
        \item Append $x$ to $\ell$
        \item \Send $ok \rightarrow P_i$
    \end{ritemize}

\OnInput \inmsg{get}from $P_i$:
    \begin{ritemize}
        \item \Send $\ell \rightarrow P_i$
    \end{ritemize}
\end{bbox}

The session type can be intuitively defines follows.
\begin{tabbing}
    $\mi{type} \; \m{db[k][v]} = \ichoice{$\=$\textcolor{red}{\paypot{1}}$\=$ \; \mb{add}:\m{PID} \arrow \m{k} \arrow$ \\
    \>\>$\echoice{ \mb{OK}: \m{PID} \arrow \m{db[k][v]}},$ \\
    \>$\textcolor{red}{\paypot{1}}$\=$ \; \mb{get}: \m{PID} \arrow \m{k} \arrow$ \\
    \>\>$\echoice{$\=$\mb{yes}: \m{v} \arrow \m{db[k][v]},$ \\
    \>\>\>$\mb{no}: \m{db[k][v]}}}$
\end{tabbing}
The annotations of $\paypot{n}$ and $\getpot{n}$ define how much import is sent or received, respectively, between the two processes. The precise typing rules for the annotations is given later in the section. 
Formally, two new type constructors are introduced for sending import (we elide details on the types $\K$ of tokens $r$ for now):
\begin{center}
\begin{minipage}{0cm}
\begin{tabbing}
$A ::= \ldots \mid \tpaypot{A}{r : \K} \mid \tgetpot{A}{r : \K}$
\end{tabbing}
\end{minipage}
\end{center}
As expected, operations over a dynamic list are non-constant and therefore require some positive maount of import to compute over.
The code for communication over a channel adds only a new command to \msf{pay} import to the receiver, and for the receiver to \msf{get} the import.
\begin{lstlisting}[basicstyle=\footnotesize\BeraMonottFamily, mathescape, frame=single]
$\tg{// \$dbchan :: db[k][v]}$
$\$$dbchan.add 
$\npay$ $\$$dbchan K {1}  $\tr{(* new instructio *)}$
$\nsend$ $\$$dbchan key
$\nsend$ $\$$dbchan val
\end{lstlisting}
The \msf{pay} operation above contains an extra unknown parameter \inline{K}. 
This is the \emph{token type} of the tokens being sent to the process. 
The provider of $x : \tgetpot{A}{r : \K}$ is required to receive
$r$ import tokens 
%of type $\K$ 
from the client using the construct
$\eget{x}{r : \K}$. Dually, the client needs to pay this import
using the construct $\epay{x}{r : \K}$.

%Considering only real tokens a typing rule for the $\getpot$ operator looks as expected
Below we show the typing rule for the $\getpot$ operator without the use of token types.
We give the full typing rule later in this section after we introduce token types. 
\begin{mathpar} \small
  \infer[\getpot R]
  {(t, t') \semi \Psi \semi \D \vdash \eget{x}{r} \semi P ::
  (x : \tgetpot{A}{r})}
  {(t, t'+r) \semi \Psi \semi \wt, \D \vdash P :: (x : A)}
\end{mathpar}
In this rule, process $P$ storing $(t, t')$ import tokens receives $r$ additional tokens adding it to the current token counter, thus the continuation executes with $(t, t'+r)$ tokens. 

In NomosUC, we introduce the concept of token types to distinguish between ``real'' and ``virtual'' import tokens.
Token types are captured by the token context $\Gamma$ and consists of user-defined types
\vspace{-0.5em}
\begin{mathpar}
  \mi{token types\;}\;\K_0 \to \K_1 \to ... \to \K_m
  \vspace{-0.5em}
\end{mathpar}
Real tokens, type $\K_0$, are the standard UC tokens that permit all the computation in the system. 
Unlike, real tokens, virtual tokens are used as a tool in the type system to allow processes \emph{sandboxing} processes (i.e. run them internall) and satisfy the import requiremens of their types.
Instead of just $(t,t')$ each process has a token context $\Gamma$ which tracks the number of every token type for that process. 

We frist motivate additional tokens types through an example of sandboxxing and then present the relevant typing rules.
Sandboxing is an important and frequently occurring proof technique. 
In particular, ``overriding'' the random coin flips for an ITM is useful in rewinding proofs, since the random bits used by the sandboxed process can be recorded and replayed.
Consider a UC experiment with \Fdb and the dummy adversary in the real world. 
Naturally, the dummy adversary does nothing but forward whatever messages and import it receives to \Fdb. 
\begin{lstlisting}[basicstyle=\footnotesize\BeraMonottFamily, mathescape, frame=single]
$\nget$ K {x} $\$$z2a 
m $\leftarrow$ $\nrecv$ $\$$z2a
$\$$a2f.msg ; $\nsend$ $\$$a2f m
$\npay$ K {x} $\$$a2f
\end{lstlisting}

The simulator \Sim receives the identical messages and import for any given \Z, and it runs a copy of \Fdb in order to simulate the real world, internally.
\Sim needs to give \Fdb the same import it's given by \DA in the real world to ensure the same outputs. 
According to the $\paypot$ rule above, \Sim ends up with tokens $(t, t') = (\_, 0)$. Regardless of the polynomial used to judge computation, this ensures \Sim can only perform a constant amount of work.
Such a limitation on \Sim is unreasonable. We want that a sandboxed process uses \emph{the same} computation resources awarded to \S rather than using \emph{all} of it by consuming real import.
Therefore we allow \Sim to create ``virtual'' import tokens to give to \Fdb and ensure the new tokens are bounded, polynomially, by the ``real'' import tokens \S posseses. 

\paragraph{The Token Context $\Gamma$}
The token context of a process, $\Gamma$, maintains the number of each token types stored in it. 
Instead of $(t,t')$ as in the $\getpot$ rule above, $\Gamma$ is a set of $\gamma_i = (t_i, t_i') \in \Gamma$ to express the total and current number of each token type. 
The total $t_i$ is the total import created of that type and the current $t_i'$ are those currently owned by the process, so whenever tokens are exchanged they only affect $t_i'$ and not $t_i$.
We define a side condition for a process to be well-typed, and it is that its token context must always be valid \emph{w.r.t the security parameter $k$}.
To this end we define a globally known polynomial $\GlobalF : (\msf{nat}, \msf{nat}) \rightarrow \msf{nat}$ as bound between two successive token types. UC requires this function to be \emph{super additive}, i.e. $\GlobalF(x+y,k) \geq \GlobalF(x,k) + \GlobalF(y,k)$.
The polynomial constrains the created tokens $t_{i+1} \leq \GlobalF(t_i, k)$.
We express this condition with the following inductive rules.
\begin{mathpar}
  \infer
  {\K_0 \hookrightarrow (t_0, t_0') \;\; \m{valid}(k)}
  {}
  \and
  \infer
  {\Tokens, \K_{i+1} \hookrightarrow (t_{i+1}, t_{i+1}')\;\; \m{valid}(k)}
  {\Tokens\;\; \m{valid}(k) \and
  \K_{i} \hookrightarrow (t_i, t_i') \in \Tokens \and
  t_{i+1} \leq \GlobalF(t_i',k)}
\end{mathpar}
We mandate that this condition is satisfied by all the process typing rules presented in this paper.

\paragraph{Creating Virtual Tokens}
The first typing rule that affects the token context creates new virtual tokens.
We call this construct $\m{withdrawToken} \; K_i \; n \; K_{i+1}$.
\begin{mathpar} \small
  \inferrule*[right=$\m{tok}$]
  {\textcolor{blue}{\Tokens, \K_{i+1} \hookrightarrow} (t_{\textcolor{blue}{i+1}} + n, t_{\textcolor{blue}{i+1}}' + n) \semi
  \Psi \semi\wt, \D \entailpot{\B{q}}{\B{q'}} P :: (x : A)}
  {\textcolor{blue}{\Tokens, \K_{i+1} \hookrightarrow} (t_{\textcolor{blue}{i+1}}, t_{\textcolor{blue}{i+1}}') \semi \Psi \semi \wt, \D \entailpot{\B{q}}{\B{q'}} \hspace{4em} \\
    \hspace{5em}\m{withdrawToken} \; \K_i \; n\; \K_{i+1}  \semi P :: (x : A)}
% \vspace{-0.5em}
\end{mathpar}
We highlight the changes in the process definitions in \m{withdrawTokens}, from the simplified $\getpot \; R$ rule, in blue.
Namely, the process possesses a token context and adds the $n$ newly created virtual tokens to $(t_{i+1}, t_{i+1}')$ of type $K_{i+1}$. 
Implicitly, all processes have a ``real'' token type that they can't create more of. For \Sim in our above example, it's ``real token type'' is, in fact, the actual real tokens of type $\gamma_0$.
It is further restricted to creating tokens only one level deeper, i.e. of type $\gamma_1$. This it the token \Fdb receives and considers its ``real token type''. 
The side condition of token validity ensures that $t_{i+1} + n \leq \GlobalF(t_i, k)$ where $K_i \hookrightarrow (t_i, t_i') \in \Gamma$. 

\paragraph{Exchanging Import Tokens}
Here we present the full typing rule for $\getpot$ for each side of the channel.
We highlight in blue the additions when taking token types into account.
\begin{mathpar} \small
  \infer[\getpot R]
  {\textcolor{blue}{\Tokens, \K_i \hookrightarrow} (t_i, t_i') \semi \Psi \semi \D \entailpot{\B{q}}{\B{q'}} \eget{x}{r \textcolor{blue}{: \K_i}} \semi P ::
  (x : \tgetpot{A}{r \textcolor{blue}{: \K_i}})}
  {\textcolor{blue}{\Tokens, \K_i \hookrightarrow} (t_i, t_i'+r) \semi \Psi \semi \wt, \D \entailpot{\B{q}}{\B{q'}} P :: (x : A)}
  %
  \and
  %
  \inferrule*[Right=$\getpot L$]
  {\textcolor{blue}{\Tokens, \K_i \hookrightarrow} (t_i, t_i') \semi \Psi \semi \D, (x : A) \entailpot{\B{q}}{\B{q'}} P :: (z : C)}
  {\textcolor{blue}{\Tokens, \K_i \hookrightarrow} (t_i, t_i'+r) \semi \Psi \semi \wt, \D, (x : \tgetpot{A}{r \textcolor{blue}{: \K_i}}) \\\ \entailpot{\B{q}}{\B{q'}} 
  \epay{x}{r \textcolor{blue}{: \K_i}} \semi P :: (z : C)}
\end{mathpar}
In the rule $\getpot R$, process $P$ storing $(t_i, t_i')$ import tokens of type $\K_i$
receives $r$ additional $\K_i$ tokens adding it to the current token counter, thus
the continuation executes with $(t_i, t_i'+r)$ tokens of type $\K_i$.
Note that validity of token context is trivially satisfied in this case since the
process is gaining import tokens.
%
In the dual rule $\getpot L$, a process containing $(t_i, t_i'+r)$ tokens of type $\K_i$
pays $r$ units along channel $x$ leaving $(t_i, t_i')$ import tokens of type $\K_i$ with
the continuation.
In this case, the validity of the token context establishes that $t_{i+1} \leq \GlobalF(t_i',k)$,
a condition that is necessary for successful typechecking.
The typing rules for the dual constructor $\tpaypot{A}{r : \K}$
are the exact inverse and omitted for brevity.
Similar to prior rules, the sender of the import tokens transfers the write token $\wt$
along with the import to the receiver.

The corresponding NomosUC process for \Sim, below, is spawned as expected, with a type parameter $K$ as its token type. 
Every process is given a type parameter that encodes its ``real'' token type. 
Subsequently, \Sim spawns \Fdb with type parameter \inline{K1} (\Sim's virtual token type).
\begin{lstlisting}[basicstyle=\footnotesize\BeraMonottFamily, mathescape, frame=single]
$\nproc$ simulator[K][$\tg{other type params}$]:
  $\tg{(* simulator params, create channels or Fdb *)}$
  $\$$dbchan $\leftarrow$ F_db[K1] $\leftarrow$ $\tg{(*Fdb params*)}$
  $\tg{...}$
\end{lstlisting}
When sending a message for \Fdb, \Sim creates the tokens it needs to sastify the type $\m{db}[k][v]$ and then sends the message:
\begin{lstlisting}[basicstyle=\footnotesize\BeraMonottFamily, mathescape, frame=single]
$\tg{...}$
$\nwithdraw$ K 1 K1
$\$$dbchan.add ; $\npay$ K1 {1}
$\nsend$ $\$$dbchan key
$\nsend$ $\$$dbchan val
$\tg{...}$
\end{lstlisting}
In contrast to the previous code snippet showing how tokens are sent to \Fdb, here we only add a new \nwithdraw instruction, making virtualization, and code re-use, simple.

\paragraph*{\textbf{Process Definitions and Sandboxing}}
Process definitions in NomosUC have the form
$\Psi \semi \D \vdash f\{t : \K\} :: (x : A) = P$ where $f$
is the name of the process and $P$ its definition.
In addition, $\Psi$ and $\D$ denote the functional variables and session-typed channels
used by $f$ respectively while offering type $A$ on channel $x$.
In addition, $\K$ is the real token type for $f$ and we need $t$ tokens of type $\K$
to spawn $f$.
All definitions are collected in a fixed global process signature $\Sg$.
Also, since process definitions are mutually recursive, it is required that
for every process in the signature is well-typed w.r.t. $\Sg$.
This, in effect, requires us checking that each process definition obeys the type
specified for it in its signature.
Formally, for every definition of the form $\Psi \semi \D \vdash f\{t : \K\} :: (x : A) = P$ in $\Sg$,
we are required to check $\K \hookrightarrow (t, t) \semi \Psi \semi \D \entailpot{0}{0} P :: (x : A)$
Note that every process initiates with only one token type, i.e., its real token type,
and hence, its token context is trivially valid.
A process always starts with no potential, the reasoning for which is explained later.

\todo{the way the code is written doesn't look exactly like this, unclear how to give import at spawn time in Nomos code.}
\todo{Also below here it's back to the unmodified section from the original section}
But how is a new process spawned and how is the real token type for a newly spawned
process determined?
A new instance of a defined process $f$ can be spawned with
the expression $\procdef{f\{r : \K\}}{\overline{y}}{x} \semi Q$
where $\overline{y}$ is a sequence of argument expressions and channels matching the
antecedents $\Psi$ and $\D$ from the signature $\Sg$, and the real token type for $f$ is $\K$ with quantity $r$.
Sometimes a process invocation is a \emph{tail call}, written without
a continuation as $\procdef{f\{r : \K\}}{\overline{y}}{x}$.
This is a short-hand for
$\procdef{f\{r : \K\}}{\overline{y}}{x'} \semi \fwd{x}{x'}$ for a
fresh variable $x'$, that is, a fresh channel is created and
immediately identified with $x$.
A process spawn is typed as follows.
\begin{mathpar}
\inferrule*[right = $\m{spawn}$]
  {\Psi_1 \semi \D_1 \vdash f\{r : K\} :: (x : A) = P \in \Sg \and
  \\\ (\Psi_1 \semi \D_1) = \overline{y} \and
  \B{\Psi \share (\Psi_1, \Psi_2)} \\
  \Tokens, K \hookrightarrow (t, t') \semi \Psi_2 \semi \D_2, (x : A) \entailpot{\B{q}}{\B{q'}} Q :: (z : C)}
  {\Tokens, K \hookrightarrow (t, t'+r) \semi \Psi \semi \D_1, \D_2 \\\ \entailpot{\B{q}}{\B{q'}} \procdef{f\{r : K\}}{\overline{y}}{x} \semi Q ::
  (z : C)}
\end{mathpar}
We first look up the definition of $f$ in the signature $\Sg$
and match the arguments $\Psi$ and $\D$ with $\overline{y}$.
Next, we deduct $r$ token units of type $K$ from the current value of $K$
in the token context.
Finally, spawning the new process also creates channel $x : A$
which appears in the context for the typing of the continuation $Q$.

Remarkably, we can use the same syntactic construct for spawning
processes, whether it is a regular or sandboxed spawn.
Semantically, they are distinguished based on the token parameter.
For a regular spawn, the parent process passes in the real token type $\K_0$,
while for a sandboxed call, a virtual token type $\K_i (i > 0)$ is passed in.
We have a similar distinction for $\m{pay}$ and $\m{get}$ expressions:
if a real token is passed into these terms, it's a regular token
exchange; if a virtual token is passed in, it's a sandboxed $\m{pay}$
and $\m{get}$.

\paragraph*{\textbf{Potential}}
The main purpose of introducing import tokens was to bound the execution cost of ITMs.
But how do we connect import to the execution cost?
To this end, we introduce the notion of \emph{potential} in NomosUC to establish this connection.
Potential is an abstract quantity represented by a natural number stored
within each process.
To take an execution step, a process consumes \emph{one} unit of potential.
Therefore, the total potential stored in a process provides an upper bound on the total
number of execution steps that will ever be taken by the process.
To obtain a static upper bound, we integrate potential into the process typing judgment.
The symbol $\entailpot{q}{q'}$ denotes that the process stores $q$ and $q'$ units of
total and current potential (similar to total and current import).

Finally, the source of potential is the import tokens, completing the connection between
import and execution cost.
To this end, we introduce a novel construct $\m{genPot} \; r$ to generate potential
based on how much import is stored in the process.
\begin{mathpar}
  \inferrule*[right=$\m{pot}$]
  {q+r \leq \GlobalF(t_{m}',k) \and K_{m} \hookrightarrow (t_{m}, t_{m}') \in \Tokens \\\
  k \semi \Tokens \semi \Psi \semi \wt, \D \entailpot{q+r}{q'+r} P :: (x : A)}
  {k \semi \Tokens \semi \Psi \semi \wt, \D \entailpot{q}{q'} \m{genPot} \; r \semi P :: (x : A)}
\end{mathpar}
A process initially storing $(q, q')$ potential units generates $r$ potential so that
the continuation contains $(q+r, q'+r)$ potential units.
Note, however, that the maximum potential that can be stored in a process is bounded by $\GlobalF(t_{m}',k)$
where $\GlobalF$ is the connection rate, $m$ is the simulation depth, and $k$ is the security parameter.
This restricts us from generating an unbounded amount of potential which could have violated the
polynomial execution cost bound.

The necessity of potential as the main accounting mechanism, rather than just a single linear quantity like import, comes from the motivation of the import mechanism for polynomial time from Canetti~\ref{uc}.
At a high-level directly letting ITMs determine the number of steps another ITM can take provides information to the calling ITM that it should not have.
With such a mechanism, imagine two executions (say an ideal and real world) where the environment can control exactly the number of steps the executions are allowed to take. 
Such a mechanism allows \Z to artificially learn global information that it should not have~\footnote{The example given by Canetti~\cite{UC}: ``For instance, the initial ITI $I$ can start in a rejecting state, and then pass control to another ITI $M$. If $I$ ever gets activated again, it moves to an accepting state. Whether $I$ is activated again depends on the running time of $M$. If it exceeds the computation given to it (known by $I$) then $I$ accepts depending on information that should not be ``legitimately know'' to $I$.}

We assign a cost of 1 per syntactic construct introduced so far.
But this would cause potential to creep in all our typing rules, essentially deducting $1$ potential unit
from the process in each rule.
Instead, we simplify matters by introducing a $\etick{r}$ construct that consumes $r$
potential from the current stored process potential, as described below.
\begin{mathpar}
  \infer[\m{tick}]
  {\Tokens \semi \Psi \semi \wt, \D \entailpot{q}{q'+r} \etick{r} \semi P :: (x : A)}
  {\Tokens \semi \Psi \semi \wt, \D \entailpot{q}{q'} P :: (x : A)}
\end{mathpar}
Note how the process starts with $q'+r$ potential units, and executing $\etick{r}$
consumes $r$ units leaving $q'$ potential for the continuation.

Semantically, a process executing a $\etick{r}$ operation increments its work counter by $r$.
\begin{tabbing}
  $(\m{tick}) : \proc{c}{w, \etick{r} \semi P} \step \proc{c}{w+r, P}$
\end{tabbing}
Note that this is the only rule that increments the work counter of a process.
And since this operation consumes $r$ units of potential, we can infer
that the total sum of potential and work of a closed set of processes will always be bounded.

