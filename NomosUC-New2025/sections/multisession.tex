In Nomos, we use potetial to impement the import mechanism.
One of the sonequences of this is that the potential exchanged over a channel between any two parties must be statically defined.
Consider an ideal functionality that requires different amount of import for different messages from protocol parties. 
This may happen in a case where a certain activation requires the functionality to send import to another ITM.
In this, despite the different requirements, the single shared channel (hence, communicator) from the parties to the functionality must be statically typed to receive the greatest of the required imports.
The type of the communicator's channel is given in Figure~\ref{fig:communicator:potential}.
\begin{figure}
\begin{sill}
$\m{comm}[a]\{n\} = \m{<}\{n\}~ \& \{ \m{SEND}: \m{a} \rightarrow |\{n\} \vee \m{comm}[a]\{n\},$ \\
\qquad \qquad \qquad \qquad $\m{RECV}: \oplus\{ \m{no}: |\{0\}\m{<} \vee \m{comm}[a]\{n\}$ \\
\qquad \qquad \qquad \qquad \qquad \qquad \quad $\m{yes}: \m{a} \* |\{n\}\m{>} \m{comm}[a]\{n\}\}$
\end{sill}
\caption{The type of the communicator accepting a single parameter for the import sent over this communicator.\todo{This is in Nomos code. It should be like the types in the nomosuc section.}}
\label{fig:nomos:communicator:potential}
\end{figure}
Despite the parameterization, we can only affix integer values to the type parameter $n$.
A potential future work may extend potential in Nomos to be more expressive allowing types for potential. \todo{ not sure what the right terminology is here }

We do not lose all from of expressibility, as we can see from a process definition, for the forthcoming $\F_{\msf{com}}$, in Figure~\ref{fig:fcom:proc}
In this example the protocol specific code can specify what message types are sent over each of the channes and the import sent with the messages (\texttt{receiver2f} and \texttt{r2fn}, respectively)

\begin{figure*}[!ht]
\begin{lstlisting}[basicstyle=\small\ttfamily,frame=single]
proc asset ideal_functionality :
    (#s: comm[sender2f]{s2fn}), (#r: comm[receiver2f]{r2fn}), 
    (#f2sender: comm[f2sender]{f2sn}, (#f2receiver: comm[f2receiver]{f2rn})  |- ($ch : 1)
\end{lstlisting}
\caption{The process definition of the $\F_{\msf{com}}$ functionality. The channels are all parmaeterized with user-defined message types for each channel and the corresponding import sent.}
\label{fig:fcom:proc}
\end{figure*}
\subsection{Database}

\subsection{Multisession}
The multisession operator is an extension in UC that allows multiple sessions of an ideal functionality to be spawned and interacted with.
It is referred to as the $!$ operator, and it acts as wrapper around some functionality, $\F$, and routes messages between copies of $\F$ and the protocol parties/adversary/environment.
The full pseudo-code of the operator can be seen in Figure~\ref{fig:bangf}.
When \bangf is activated with a new message it expects the caller to enode the subsession ID, or \msf{ssid}, of the instance of \F~it wants to communicate with in the messages.
If $\F^{\msf{ssid}}$ doesn't exist, a new one is instantiated and appended to the list of $\F$ copies.

\begin{figure}
In Nomos, we use potetial to impement the import mechanism.
One of the sonequences of this is that the potential exchanged over a channel between any two parties must be statically defined.
Consider an ideal functionality that requires different amount of import for different messages from protocol parties. 
This may happen in a case where a certain activation requires the functionality to send import to another ITM.
In this, despite the different requirements, the single shared channel (hence, communicator) from the parties to the functionality must be statically typed to receive the greatest of the required imports.
The type of the communicator's channel is given in Figure~\ref{fig:communicator:potential}.
\begin{figure}
\begin{sill}
$\m{comm}[a]\{n\} = \m{<}\{n\}~ \& \{ \m{SEND}: \m{a} \rightarrow |\{n\} \vee \m{comm}[a]\{n\},$ \\
\qquad \qquad \qquad \qquad $\m{RECV}: \oplus\{ \m{no}: |\{0\}\m{<} \vee \m{comm}[a]\{n\}$ \\
\qquad \qquad \qquad \qquad \qquad \qquad \quad $\m{yes}: \m{a} \* |\{n\}\m{>} \m{comm}[a]\{n\}\}$
\end{sill}
\caption{The type of the communicator accepting a single parameter for the import sent over this communicator.\todo{This is in Nomos code. It should be like the types in the nomosuc section.}}
\label{fig:nomos:communicator:potential}
\end{figure}
Despite the parameterization, we can only affix integer values to the type parameter $n$.
A potential future work may extend potential in Nomos to be more expressive allowing types for potential. \todo{ not sure what the right terminology is here }

We do not lose all from of expressibility, as we can see from a process definition, for the forthcoming $\F_{\msf{com}}$, in Figure~\ref{fig:fcom:proc}
In this example the protocol specific code can specify what message types are sent over each of the channes and the import sent with the messages (\texttt{receiver2f} and \texttt{r2fn}, respectively)

\begin{figure*}[!ht]
\begin{lstlisting}[basicstyle=\small\ttfamily,frame=single]
proc asset ideal_functionality :
    (#s: comm[sender2f]{s2fn}), (#r: comm[receiver2f]{r2fn}), 
    (#f2sender: comm[f2sender]{f2sn}, (#f2receiver: comm[f2receiver]{f2rn})  |- ($ch : 1)
\end{lstlisting}
\caption{The process definition of the $\F_{\msf{com}}$ functionality. The channels are all parmaeterized with user-defined message types for each channel and the corresponding import sent.}
\label{fig:fcom:proc}
\end{figure*}
\subsection{Database}

\subsection{Multisession}
The multisession operator is an extension in UC that allows multiple sessions of an ideal functionality to be spawned and interacted with.
It is referred to as the $!$ operator, and it acts as wrapper around some functionality, $\F$, and routes messages between copies of $\F$ and the protocol parties/adversary/environment.
The full pseudo-code of the operator can be seen in Figure~\ref{fig:bangf}.
When \bangf is activated with a new message it expects the caller to enode the subsession ID, or \msf{ssid}, of the instance of \F~it wants to communicate with in the messages.
If $\F^{\msf{ssid}}$ doesn't exist, a new one is instantiated and appended to the list of $\F$ copies.

\begin{figure}
In Nomos, we use potetial to impement the import mechanism.
One of the sonequences of this is that the potential exchanged over a channel between any two parties must be statically defined.
Consider an ideal functionality that requires different amount of import for different messages from protocol parties. 
This may happen in a case where a certain activation requires the functionality to send import to another ITM.
In this, despite the different requirements, the single shared channel (hence, communicator) from the parties to the functionality must be statically typed to receive the greatest of the required imports.
The type of the communicator's channel is given in Figure~\ref{fig:communicator:potential}.
\begin{figure}
\begin{sill}
$\m{comm}[a]\{n\} = \m{<}\{n\}~ \& \{ \m{SEND}: \m{a} \rightarrow |\{n\} \vee \m{comm}[a]\{n\},$ \\
\qquad \qquad \qquad \qquad $\m{RECV}: \oplus\{ \m{no}: |\{0\}\m{<} \vee \m{comm}[a]\{n\}$ \\
\qquad \qquad \qquad \qquad \qquad \qquad \quad $\m{yes}: \m{a} \* |\{n\}\m{>} \m{comm}[a]\{n\}\}$
\end{sill}
\caption{The type of the communicator accepting a single parameter for the import sent over this communicator.\todo{This is in Nomos code. It should be like the types in the nomosuc section.}}
\label{fig:nomos:communicator:potential}
\end{figure}
Despite the parameterization, we can only affix integer values to the type parameter $n$.
A potential future work may extend potential in Nomos to be more expressive allowing types for potential. \todo{ not sure what the right terminology is here }

We do not lose all from of expressibility, as we can see from a process definition, for the forthcoming $\F_{\msf{com}}$, in Figure~\ref{fig:fcom:proc}
In this example the protocol specific code can specify what message types are sent over each of the channes and the import sent with the messages (\texttt{receiver2f} and \texttt{r2fn}, respectively)

\begin{figure*}[!ht]
\begin{lstlisting}[basicstyle=\small\ttfamily,frame=single]
proc asset ideal_functionality :
    (#s: comm[sender2f]{s2fn}), (#r: comm[receiver2f]{r2fn}), 
    (#f2sender: comm[f2sender]{f2sn}, (#f2receiver: comm[f2receiver]{f2rn})  |- ($ch : 1)
\end{lstlisting}
\caption{The process definition of the $\F_{\msf{com}}$ functionality. The channels are all parmaeterized with user-defined message types for each channel and the corresponding import sent.}
\label{fig:fcom:proc}
\end{figure*}
\subsection{Database}

\subsection{Multisession}
The multisession operator is an extension in UC that allows multiple sessions of an ideal functionality to be spawned and interacted with.
It is referred to as the $!$ operator, and it acts as wrapper around some functionality, $\F$, and routes messages between copies of $\F$ and the protocol parties/adversary/environment.
The full pseudo-code of the operator can be seen in Figure~\ref{fig:bangf}.
When \bangf is activated with a new message it expects the caller to enode the subsession ID, or \msf{ssid}, of the instance of \F~it wants to communicate with in the messages.
If $\F^{\msf{ssid}}$ doesn't exist, a new one is instantiated and appended to the list of $\F$ copies.

\begin{figure}
In Nomos, we use potetial to impement the import mechanism.
One of the sonequences of this is that the potential exchanged over a channel between any two parties must be statically defined.
Consider an ideal functionality that requires different amount of import for different messages from protocol parties. 
This may happen in a case where a certain activation requires the functionality to send import to another ITM.
In this, despite the different requirements, the single shared channel (hence, communicator) from the parties to the functionality must be statically typed to receive the greatest of the required imports.
The type of the communicator's channel is given in Figure~\ref{fig:communicator:potential}.
\begin{figure}
\begin{sill}
$\m{comm}[a]\{n\} = \m{<}\{n\}~ \& \{ \m{SEND}: \m{a} \rightarrow |\{n\} \vee \m{comm}[a]\{n\},$ \\
\qquad \qquad \qquad \qquad $\m{RECV}: \oplus\{ \m{no}: |\{0\}\m{<} \vee \m{comm}[a]\{n\}$ \\
\qquad \qquad \qquad \qquad \qquad \qquad \quad $\m{yes}: \m{a} \* |\{n\}\m{>} \m{comm}[a]\{n\}\}$
\end{sill}
\caption{The type of the communicator accepting a single parameter for the import sent over this communicator.\todo{This is in Nomos code. It should be like the types in the nomosuc section.}}
\label{fig:nomos:communicator:potential}
\end{figure}
Despite the parameterization, we can only affix integer values to the type parameter $n$.
A potential future work may extend potential in Nomos to be more expressive allowing types for potential. \todo{ not sure what the right terminology is here }

We do not lose all from of expressibility, as we can see from a process definition, for the forthcoming $\F_{\msf{com}}$, in Figure~\ref{fig:fcom:proc}
In this example the protocol specific code can specify what message types are sent over each of the channes and the import sent with the messages (\texttt{receiver2f} and \texttt{r2fn}, respectively)

\begin{figure*}[!ht]
\begin{lstlisting}[basicstyle=\small\ttfamily,frame=single]
proc asset ideal_functionality :
    (#s: comm[sender2f]{s2fn}), (#r: comm[receiver2f]{r2fn}), 
    (#f2sender: comm[f2sender]{f2sn}, (#f2receiver: comm[f2receiver]{f2rn})  |- ($ch : 1)
\end{lstlisting}
\caption{The process definition of the $\F_{\msf{com}}$ functionality. The channels are all parmaeterized with user-defined message types for each channel and the corresponding import sent.}
\label{fig:fcom:proc}
\end{figure*}
\subsection{Database}

\subsection{Multisession}
The multisession operator is an extension in UC that allows multiple sessions of an ideal functionality to be spawned and interacted with.
It is referred to as the $!$ operator, and it acts as wrapper around some functionality, $\F$, and routes messages between copies of $\F$ and the protocol parties/adversary/environment.
The full pseudo-code of the operator can be seen in Figure~\ref{fig:bangf}.
When \bangf is activated with a new message it expects the caller to enode the subsession ID, or \msf{ssid}, of the instance of \F~it wants to communicate with in the messages.
If $\F^{\msf{ssid}}$ doesn't exist, a new one is instantiated and appended to the list of $\F$ copies.

\begin{figure}
\input{figures/multisession}
\caption{The multisession operator, referred to as $!\F$ creates multiple copies of the input functionality $\F$ with different session IDs. It routes messages to and from the copies of the functionality. The enlarging data structure \msf{Fs} requires consideration of polynomial bounds with import token requirements indicated in \color{red} red\color{black}.}
\label{fig:bangf}
\end{figure}

We implement \bangf in Nomos and a UC execution of it parameterized by the $\F_{\msf{com}}$ functionality. 
An important consideration when working with the multisession operator is the import mechanism.
The underlying functionality, $\F_{\msf{com}}$ does not explicitly require any import because it is a one-shot constant-work functionality: it works for commitment to a single bit, performs a constant amount of work and terminates.
The multisession operator on the other hand, maintains an append-only list of instances of $\F_{\msf{com}}$.
It is important to enable $!\F$ to read over a list of polynomial size in the future.
Notice that the import mechanism here also constrains the size.

To ensure that even simple ITMs don't need to specify import consraints for all their messages, in Nomos we default the amount of import (i.e. potential) for every message to 1.
This ensures that simple database 

In general, we ensure that every message send in Nomos is accompanied by a default of 1 import token unless specified otherwise.
This ensures all there will always exist a polynomil for which any polynomial-time computation an be completed by an ITM.
However, leaving a default of 1 import limits the expressiveness of the system as ITMs receiving 1 import have no import to give other ITMs.\footnote{A party $p_i$ that receives 1 import from the environment can not perform computation and send 1 import to $!\F_{\msf{com}}$. In such a scenario, $p_i$ requires at least 2 units of import.}

\todo{below is not to be considered.}

Implementing the multisession wrapper in Nomos is straightforward because the only amount of import it ever accepts is 1 unit.
Recall that the use of communicators requires that the maximum amount import must be sent between two ITMs.
More specifically, if messages types sent from $M_1$ to $M_2$ require different amounts of import in the pseudo-code, they must all send the highest of these imports.

Recall the type of communicator messages between two parties in Figure~\ref{fig:nomos:communicatortype}. 

%\begin{figure}
%\begin{sill}
%$\m{comm}[a]\{n\} = \m{<}\{n\}~ \& \{ \m{SEND}: \m{a} \rightarrow |\{n\} \vee \m{comm}[a]\{n\},$ \\
%\qquad \qquad \qquad \qquad $\m{RECV}: \oplus\{ \m{no}: |\{0\}\m{<} \vee \m{comm}[a]\{n\}$ \\
%\qquad \qquad \qquad \qquad \qquad \qquad \quad $\m{yes}: \m{a} \* |\{n\}\m{>} \m{comm}[a]\{n\}\}$
%\end{sill}
%\caption{The type of the communicator accepting a single parameter for the import sent over this communicator.}
%\label{fig:nomos:communicatortype}
%\end{figure}
%\todo{What's with the $\wedge$?}

%The multisession communicates with protocol parties and the adversary.
%Hence, $!\F$ is parameterized with an amount of potential for each of these communicator.
%When a new 

\caption{The multisession operator, referred to as $!\F$ creates multiple copies of the input functionality $\F$ with different session IDs. It routes messages to and from the copies of the functionality. The enlarging data structure \msf{Fs} requires consideration of polynomial bounds with import token requirements indicated in \color{red} red\color{black}.}
\label{fig:bangf}
\end{figure}

We implement \bangf in Nomos and a UC execution of it parameterized by the $\F_{\msf{com}}$ functionality. 
An important consideration when working with the multisession operator is the import mechanism.
The underlying functionality, $\F_{\msf{com}}$ does not explicitly require any import because it is a one-shot constant-work functionality: it works for commitment to a single bit, performs a constant amount of work and terminates.
The multisession operator on the other hand, maintains an append-only list of instances of $\F_{\msf{com}}$.
It is important to enable $!\F$ to read over a list of polynomial size in the future.
Notice that the import mechanism here also constrains the size.

To ensure that even simple ITMs don't need to specify import consraints for all their messages, in Nomos we default the amount of import (i.e. potential) for every message to 1.
This ensures that simple database 

In general, we ensure that every message send in Nomos is accompanied by a default of 1 import token unless specified otherwise.
This ensures all there will always exist a polynomil for which any polynomial-time computation an be completed by an ITM.
However, leaving a default of 1 import limits the expressiveness of the system as ITMs receiving 1 import have no import to give other ITMs.\footnote{A party $p_i$ that receives 1 import from the environment can not perform computation and send 1 import to $!\F_{\msf{com}}$. In such a scenario, $p_i$ requires at least 2 units of import.}

\todo{below is not to be considered.}

Implementing the multisession wrapper in Nomos is straightforward because the only amount of import it ever accepts is 1 unit.
Recall that the use of communicators requires that the maximum amount import must be sent between two ITMs.
More specifically, if messages types sent from $M_1$ to $M_2$ require different amounts of import in the pseudo-code, they must all send the highest of these imports.

Recall the type of communicator messages between two parties in Figure~\ref{fig:nomos:communicatortype}. 

%\begin{figure}
%\begin{sill}
%$\m{comm}[a]\{n\} = \m{<}\{n\}~ \& \{ \m{SEND}: \m{a} \rightarrow |\{n\} \vee \m{comm}[a]\{n\},$ \\
%\qquad \qquad \qquad \qquad $\m{RECV}: \oplus\{ \m{no}: |\{0\}\m{<} \vee \m{comm}[a]\{n\}$ \\
%\qquad \qquad \qquad \qquad \qquad \qquad \quad $\m{yes}: \m{a} \* |\{n\}\m{>} \m{comm}[a]\{n\}\}$
%\end{sill}
%\caption{The type of the communicator accepting a single parameter for the import sent over this communicator.}
%\label{fig:nomos:communicatortype}
%\end{figure}
%\todo{What's with the $\wedge$?}

%The multisession communicates with protocol parties and the adversary.
%Hence, $!\F$ is parameterized with an amount of potential for each of these communicator.
%When a new 

\caption{The multisession operator, referred to as $!\F$ creates multiple copies of the input functionality $\F$ with different session IDs. It routes messages to and from the copies of the functionality. The enlarging data structure \msf{Fs} requires consideration of polynomial bounds with import token requirements indicated in \color{red} red\color{black}.}
\label{fig:bangf}
\end{figure}

We implement \bangf in Nomos and a UC execution of it parameterized by the $\F_{\msf{com}}$ functionality. 
An important consideration when working with the multisession operator is the import mechanism.
The underlying functionality, $\F_{\msf{com}}$ does not explicitly require any import because it is a one-shot constant-work functionality: it works for commitment to a single bit, performs a constant amount of work and terminates.
The multisession operator on the other hand, maintains an append-only list of instances of $\F_{\msf{com}}$.
It is important to enable $!\F$ to read over a list of polynomial size in the future.
Notice that the import mechanism here also constrains the size.

To ensure that even simple ITMs don't need to specify import consraints for all their messages, in Nomos we default the amount of import (i.e. potential) for every message to 1.
This ensures that simple database 

In general, we ensure that every message send in Nomos is accompanied by a default of 1 import token unless specified otherwise.
This ensures all there will always exist a polynomil for which any polynomial-time computation an be completed by an ITM.
However, leaving a default of 1 import limits the expressiveness of the system as ITMs receiving 1 import have no import to give other ITMs.\footnote{A party $p_i$ that receives 1 import from the environment can not perform computation and send 1 import to $!\F_{\msf{com}}$. In such a scenario, $p_i$ requires at least 2 units of import.}

\todo{below is not to be considered.}

Implementing the multisession wrapper in Nomos is straightforward because the only amount of import it ever accepts is 1 unit.
Recall that the use of communicators requires that the maximum amount import must be sent between two ITMs.
More specifically, if messages types sent from $M_1$ to $M_2$ require different amounts of import in the pseudo-code, they must all send the highest of these imports.

Recall the type of communicator messages between two parties in Figure~\ref{fig:nomos:communicatortype}. 

%\begin{figure}
%\begin{sill}
%$\m{comm}[a]\{n\} = \m{<}\{n\}~ \& \{ \m{SEND}: \m{a} \rightarrow |\{n\} \vee \m{comm}[a]\{n\},$ \\
%\qquad \qquad \qquad \qquad $\m{RECV}: \oplus\{ \m{no}: |\{0\}\m{<} \vee \m{comm}[a]\{n\}$ \\
%\qquad \qquad \qquad \qquad \qquad \qquad \quad $\m{yes}: \m{a} \* |\{n\}\m{>} \m{comm}[a]\{n\}\}$
%\end{sill}
%\caption{The type of the communicator accepting a single parameter for the import sent over this communicator.}
%\label{fig:nomos:communicatortype}
%\end{figure}
%\todo{What's with the $\wedge$?}

%The multisession communicates with protocol parties and the adversary.
%Hence, $!\F$ is parameterized with an amount of potential for each of these communicator.
%When a new 

\caption{The multisession operator, referred to as $!\F$ creates multiple copies of the input functionality $\F$ with different session IDs. It routes messages to and from the copies of the functionality. The enlarging data structure \msf{Fs} requires consideration of polynomial bounds with import token requirements indicated in \color{red} red\color{black}.}
\label{fig:bangf}
\end{figure}

We implement \bangf in Nomos and a UC execution of it parameterized by the $\F_{\msf{com}}$ functionality. 
An important consideration when working with the multisession operator is the import mechanism.
The underlying functionality, $\F_{\msf{com}}$ does not explicitly require any import because it is a one-shot constant-work functionality: it works for commitment to a single bit, performs a constant amount of work and terminates.
The multisession operator on the other hand, maintains an append-only list of instances of $\F_{\msf{com}}$.
It is important to enable $!\F$ to read over a list of polynomial size in the future.
Notice that the import mechanism here also constrains the size.

To ensure that even simple ITMs don't need to specify import consraints for all their messages, in Nomos we default the amount of import (i.e. potential) for every message to 1.
This ensures that simple database 

In general, we ensure that every message send in Nomos is accompanied by a default of 1 import token unless specified otherwise.
This ensures all there will always exist a polynomil for which any polynomial-time computation an be completed by an ITM.
However, leaving a default of 1 import limits the expressiveness of the system as ITMs receiving 1 import have no import to give other ITMs.\footnote{A party $p_i$ that receives 1 import from the environment can not perform computation and send 1 import to $!\F_{\msf{com}}$. In such a scenario, $p_i$ requires at least 2 units of import.}

\todo{below is not to be considered.}

Implementing the multisession wrapper in Nomos is straightforward because the only amount of import it ever accepts is 1 unit.
Recall that the use of communicators requires that the maximum amount import must be sent between two ITMs.
More specifically, if messages types sent from $M_1$ to $M_2$ require different amounts of import in the pseudo-code, they must all send the highest of these imports.

Recall the type of communicator messages between two parties in Figure~\ref{fig:nomos:communicatortype}. 

%\begin{figure}
%\begin{sill}
%$\m{comm}[a]\{n\} = \m{<}\{n\}~ \& \{ \m{SEND}: \m{a} \rightarrow |\{n\} \vee \m{comm}[a]\{n\},$ \\
%\qquad \qquad \qquad \qquad $\m{RECV}: \oplus\{ \m{no}: |\{0\}\m{<} \vee \m{comm}[a]\{n\}$ \\
%\qquad \qquad \qquad \qquad \qquad \qquad \quad $\m{yes}: \m{a} \* |\{n\}\m{>} \m{comm}[a]\{n\}\}$
%\end{sill}
%\caption{The type of the communicator accepting a single parameter for the import sent over this communicator.}
%\label{fig:nomos:communicatortype}
%\end{figure}
%\todo{What's with the $\wedge$?}

%The multisession communicates with protocol parties and the adversary.
%Hence, $!\F$ is parameterized with an amount of potential for each of these communicator.
%When a new 
