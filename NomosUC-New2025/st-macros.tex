


%Potential annotations
\newlength{\rWidth}

\newcommand{\funtype}[1]{%
    {\settowidth{\rWidth}{\ensuremath{#1}}%
        \;\ensuremath{{\xrightarrow{\hspace{\rWidth}}\hspace{-0.84\rWidth}}\!\!\!^%
         {#1}%{\BehindSubString{,}{#1} / \BeforeSubString{,}{#1}}%
         \hspace{0.2\rWidth}\;\;}}}


%% Notation
\newcommand{\m}[1]{\ensuremath{\mathsf{#1}}}
\newcommand{\mb}[1]{\ensuremath{\mathbf{#1}}}
\newenvironment{sill}{\begin{tabbing}}{\end{tabbing}}


%% Configuration
\newcommand{\conftree}[3]{\left[#1\right] \; \proc{#2}{#3}}
\newcommand{\confprovider}[2]{(#1)^{#2}}
\newcommand{\confset}[1]{\overline{#1}}
\newcommand{\esync}{\; \m{esync}}
\newcommand{\measure}{energy}
\newcommand{\measures}{energies}
% \newcommand{\mc}[1]{\mathcal{#1}}
\newcommand{\CC}{\mathcal{C}}
\newcommand{\DD}{\mathcal{D}}
\newcommand{\EE}{\mathcal{E}}
\newcommand{\FF}{\mathcal{F}}

%% Modes
\newcommand{\s}{\m{S}}
\newcommand{\li}{\m{L}}
\newcommand{\cl}{\m{C}}
\newcommand{\p}{\m{P}}

\newcommand{\lang}[1]{\mathbf{L}(#1)}

%% Contexts and Typing Judgment
\newcommand{\W}{\Omega}
\newcommand{\Sg}{\Sigma}
\newcommand{\xvdash}[2]{\sststile{#2}{#1}}
\newcommand{\xVdash}[1]{%
  \Vdash^{\mkern-8mu\scriptstyle\rule[-.9ex]{0pt}{0pt}#1}%
}
%\newcommand{\confpot}[2]{\overset{#1}{\underset{#2}{\vDash}}}
%\newcommand{\potconf}[1]{\overset{#1}{\vDash}}
%\newcommand{\spanconf}{\vDash}
%\newcommand{\confspan}[1]{\overset{(#1)}{\vDash}}
%\newcommand{\confspanlocal}[1]{\overset{\langle #1 \rangle}{\vDash}}
\newcommand{\confpot}[2]{\overset{#1}{\underset{#2}{\vdash}}}
\newcommand{\potconf}[1]{\overset{#1}{\vdash}}
\newcommand{\spanconf}{\vdash}
\newcommand{\confspan}[1]{\overset{(#1)}{\vdash}}
\newcommand{\confspanlocal}[1]{\overset{\langle #1 \rangle}{\vdash}}
\newcommand{\D}{\Delta}
%\newcommand{\G}{\Gamma}
\newcommand{\T}{\Theta}
%\newcommand{\proves}{\vDash}
\newcommand{\proves}{\vdash}
\newcommand{\w}{\omega}
%\newcommand{\Co}{\mathcal{C}}
%\renewcommand{\C}{\mathcal{C}}
\newcommand{\set}[1]{\lvert\lvert#1\rvert\rvert}

\newcommand{\lin}[1]{\m{lin}(\overline{#1})}
\newcommand{\shd}[1]{\m{shd}(\overline{#1})}
\newcommand{\slin}[1]{\m{slin}(\overline{#1})}
\newcommand{\plin}{\; \m{purelin}}

%% Operational Semantics Predicates
\newcommand{\proc}[2]{\m{proc}(#1, #2)}
\newcommand{\msg}[2]{\m{msg}(#1, #2)}
\newcommand{\ichan}[3]{\m{ichan}(#1, #2, #3)}
\newcommand{\ochan}[3]{\m{ochan}(#1, #2, #3)}
\newcommand{\unavail}[1]{\m{unavail}(#1)}

%% Semantics
\newcommand{\step}{\; \mapsto \;}
\newcommand{\zerostep}{\step^{0}}
\newcommand{\timed}[2]{\{#1\}_{#2}}
\newcommand{\unit}{M}
\newcommand{\Step}{\Longrightarrow}
\newcommand{\info}{\mapsto}
\newcommand{\andin}{\; \m{and} \;}
\newcommand{\minus}{\setminus}
\newcommand{\fresh}[1]{(#1 \text{ fresh})}
\newcommand{\eval}[1]{\Downarrow_{#1}}

%% Expressions Semantics
\newcommand{\val}{\; \m{val}}

%% Expressions
\newcommand{\lam}[3]{\lambda #1 : #2 . M_x}
\newcommand{\inl}[1]{l \cdot #1}
\newcommand{\inr}[1]{r \cdot #1}
\newcommand{\case}[3]{\m{case} \; #1 \; (l \hookrightarrow #2, r \hookrightarrow #3)}
\newcommand{\pair}[2]{\left\langle #1, #2 \right\rangle}
\newcommand{\projl}[1]{#1 \cdot l}
\newcommand{\projr}[1]{#1 \cdot r}
\newcommand{\match}[4]{\m{match} \; #1 \; ([] \rightarrow #2, #3 \rightarrow #4)}
\newcommand{\eproc}[3]{\{#1 \leftarrow #2 \leftarrow #3\}}


%% Proof Terms
\newcommand{\ecase}[3]{\m{case} \; #1 \; (#2 \Rightarrow #3)}
\newcommand{\ecasecf}[3]{\m{case^{cf}} \; #1 \; (#2 \Rightarrow #3)}
\newcommand{\erecvch}[2]{#2 \leftarrow \m{recv} \; #1}
\newcommand{\erecvchcf}[2]{#2 \leftarrow \m{recv^{cf}} \; #1}
\newcommand{\erecvshift}[1]{\m{shift} \leftarrow \m{recv} \; #1}
\newcommand{\esendch}[2]{\m{send} \; #1 \; #2}
\newcommand{\esendchcf}[2]{\m{send^{cf}} \; #1 \; #2}
\newcommand{\esendshift}[1]{\m{send} \; #1 \; \m{shift}}
\newcommand{\ewait}[1]{\m{wait} \; #1}
\newcommand{\ewaitcf}[1]{\m{wait^{cf}} \; #1}
\newcommand{\eclose}[1]{\m{close} \; #1}
\newcommand{\eclosecf}[1]{\m{close^{cf}} \; #1}
\newcommand{\fwd}[2]{#1 \leftarrow #2}
\newcommand{\fwdp}[2]{#1 \overset{+}{\leftarrow} #2}
\newcommand{\fwdn}[2]{#1 \overset{-}{\leftarrow} #2}
\newcommand{\esendl}[2]{#1.#2}
\newcommand{\esendlcf}[2]{(#1.#2)^{\m{cf}}}
\newcommand{\ecut}[4]{#1 \leftarrow #2 \leftarrow #3 \semi #4}
\newcommand{\ecutna}[3]{#1 \leftarrow #2 \semi #3}
\newcommand{\espawn}[4]{#1 \leftarrow #2 \leftarrow #3 = #4}
\newcommand{\procg}[3]{\m{proc}(#1, #2, \overline{#3})}
\newcommand{\edelay}[1]{\m{delay} \; (#1)}
\newcommand{\ewhen}[2]{\m{when?} \; (#1) ; #2}
\newcommand{\enow}[2]{\m{now!} \; (#1) ; #2}
\newcommand{\etick}[1]{\m{tick} \; (#1)}
\newcommand{\ework}[1]{\m{work} \; \{#1\}}
\newcommand{\eget}[2]{\m{get} \; #1 \; \{#2\}}
\newcommand{\epay}[2]{\m{pay} \; #1 \; \{#2\}}
\newcommand{\procdef}[3]{#3 \leftarrow #1 \; #2}
\newcommand{\procdefna}[2]{#2 \leftarrow #1}
\newcommand{\casedef}[1]{\m{case} \; #1}
\newcommand{\labdef}[1]{#1 \Rightarrow}
\newcommand{\wk}[1]{\m{work}(#1)}
\newcommand{\eassume}[2]{\m{assume} \; #1 \; \{#2\}}
\newcommand{\eassert}[2]{\m{assert} \; #1 \; \{#2\}}
\newcommand{\eimpos}[2]{\m{impossible} \; #1 \; \{#2\}}
\newcommand{\eif}[1]{\m{if} \; (#1)}
\newcommand{\ethen}{\; \m{then} \; }
\newcommand{\eelse}{\m{else} \; }

%% Type Constructors
\newcommand{\lolli}{\multimap}
\newcommand{\tensor}{\otimes}
\newcommand{\with}{\mathbin{\binampersand}}
\newcommand{\paar}{\mathbin{\bindnasrepma}}
\newcommand{\one}{\mathbf{1}}
\newcommand{\zero}{\mathbf{0}}
\newcommand{\bang}{{!}}
\newcommand{\whynot}{{?}}
\newcommand{\semi}{\, ; \,}
\newcommand{\ichoiceop}{\ensuremath{\oplus}}
\newcommand{\echoiceop}{\ensuremath{\with}}
\newcommand{\ichoice}[1]{\ichoiceop \{ #1 \}}
\newcommand{\echoice}[1]{\echoiceop \{ #1 \}}
\newcommand{\fuse}{\bullet}
\newcommand{\mi}[1]{\mbox{\it #1}}
\newcommand{\lunder}{\mathbin{\backslash}}
\newcommand{\tassertop}{?}
\newcommand{\tassumeop}{!}
\newcommand{\tassert}[1]{\; \tassertop\{#1\}. \;}
\newcommand{\tassume}[1]{\; \tassumeop\{#1\}. \;}
\newcommand{\arrow}{\rightarrow}
\newcommand{\product}{\times}

%% Functional Types
\newcommand{\tproc}[2]{\{#1 \leftarrow #2\}}

%% Types with Potential
\newcommand{\pot}[2]{#1^{#2}}
\newcommand{\lollipot}[1]{\overset{#1}{\lolli}}
\newcommand{\tensorpot}[1]{\overset{#1}{\tensor}}
\newcommand{\potfop}{\phi}
\newcommand{\potf}[1]{\potfop(#1)}
\newcommand{\mlab}{M^{\textsf{label}}}
\newcommand{\mchan}{M^{\textsf{channel}}}
\newcommand{\mcl}{M^{\textsf{close}}}
\newcommand{\mall}{M}
\newcommand{\mint}{M^{\textsf{internal}}}
\newcommand{\mval}{M^{\textsf{value}}}
\newcommand{\mshd}{M^{\textsf{share}}}
\newcommand{\ms}{M_s}
\newcommand{\mr}{M_r}
\newcommand{\entailpot}[2]{\xvdash{#1}{#2}}
\newcommand{\exppot}[1]{\xVdash{#1}}
\newcommand{\texp}{\Vdash}
\newcommand{\pexp}{\vdash}
\newcommand{\paypot}{\triangleright}
\newcommand{\getpot}{\triangleleft}
\newcommand{\tgetpot}[2]{\getpot^{\{#2\}} #1}
\newcommand{\tpaypot}[2]{\paypot^{\{#2\}} #1}
\newcommand{\bigeval}[3]{#1 \Downarrow #2 \mid #3}
\newcommand{\share}{\curlyveedownarrow}
\newcommand{\zpot}{\overline{0}}
\newcommand{\rgetpot}[1]{\textcolor{red}{\getpot^{#1}}}
\newcommand{\rpaypot}[1]{\textcolor{red}{\paypot^{#1}}}


%% Temporal Types
\newcommand{\entailpotcf}[1]{\underset{\m{cf}}{\entailpot{#1}}}
\newcommand{\entailspan}{\vdash}
\newcommand{\entailtype}{\vdash}
\newcommand{\fpot}{\; @ \;}
\newcommand{\pay}[1]{#1^{1}}
\newcommand{\sync}[1]{#1^{2}}
\newcommand{\spanpot}[1]{\langle \pay{#1}, \sync{#1} \rangle}
\newcommand{\ichoicepot}[2]{\overset{#1}{\ichoiceop} \{ #2 \}}
\newcommand{\echoicepot}[2]{\overset{#1}{\echoiceop} \{ #2 \}}
\newcommand{\tlist}[1]{\m{list}_{#1}}
\newcommand{\plist}[2]{\m{list}_{#1}^{#2}}
\newcommand{\tdia}[1]{\Diamond #1}
\newcommand{\tbox}[1]{\Box #1}
\newcommand{\tforall}[1]{\forall . #1}
\newcommand{\texists}[1]{\exists . #1}
\newcommand{\Dia}{\Diamond}
\newcommand{\Next}{\raisebox{0.3ex}{$\scriptstyle\bigcirc$}}
\newcommand{\tdelay}[2]{
    \IfEqCase{#2}{%
        {1}{\next{#1}}%
        % you can add more cases here as desired
    }[{\Next^{#2} (#1)}]%
}%
\newcommand{\sch}[1]{\tau(#1)}
\newcommand{\lforce}[2]{[#1]_L^{#2}}
\newcommand{\rforce}[2]{[#1]_R^{#2}}
\newcommand{\force}[2]{#1 \circ (#2)}

\setlength{\inferLineSkip}{4pt}
\newcommand{\blue}[1]{{\color{blue}#1}}
\newcommand{\red}[1]{{\color{red}#1}}
\newcommand{\green}[1]{{\color{green}#1}}
\newcommand{\tick}{\blue{\m{tick}}}
\newcommand{\delay}{\red{\m{delay}}}
\newcommand{\when}[1]{\red{\m{when?}\;#1}}
\newcommand{\now}[1]{\red{\m{now!}\;#1}}
\newcommand{\noww}{\red{\m{now!}}}
\newcommand{\whenn}{\red{\m{when?}}}
% \newcommand{\vdashi}{\vdash^{\!\!{}^i}}
\newcommand{\vdashi}{\vdash^{\!\!\scriptscriptstyle i}}
\newcommand{\tock}{`}


%% Indices
\newcommand{\indv}[1]{\overline{\{#1\}}}
\newcommand{\ind}[1]{\{#1\}}


%% Syntactic Sugar
\newcommand{\config}{\mathcal{C}}
\newcommand{\cost}[2]{\mathrm{cost}(\proc{#1}{#2})}
\newcommand{\tcost}[2]{\mathrm{cost}(#1 \mapsto #2)}
\newcommand{\ccost}[1]{\mathrm{cost}(#1)}
\newcommand{\dc}{\mathcal{D}}
\newcommand{\ec}{\mathcal{E}}
\newcommand{\ac}{\mathcal{A}}
\newcommand{\st}[1]{\m{store}_{#1}}
\newcommand{\stack}[1]{\m{stack}_{#1}}
\newcommand{\queue}[1]{\m{queue}_{#1}}
\newcommand{\mapper}[1]{\m{mapper}_{#1}}
\newcommand{\fdr}[1]{\m{folder}_{#1}}
\newcommand{\lt}[1]{\m{list}_{#1}}
%\newcommand{\bits}{\m{bits}}
\newcommand{\ctr}{\m{ctr}}
\newcommand{\trans}[2]{#1 \Longrightarrow #2}
\newcommand{\typetrans}[1]{\left\lvert{#1}\right\rvert}
\newcommand{\tree}{\m{tree}}
\newcommand{\bool}{\m{bool}}
\newcommand{\delayedbox}[1]{#1 \; \m{delayed}^{\Box}}
\newcommand{\delayeddia}[1]{#1 \; \m{delayed}^{\Diamond}}
\newcommand{\dom}[1]{\m{dom}(#1)}
\newcommand{\valid}[1]{#1 \; \m{valid}}
\newcommand{\invalid}[1]{#1 \; \m{invalid}}

%% Smart Contracts
\newcommand{\addr}{\m{addr}}
\newcommand{\ether}{\m{ether}}
\newcommand{\players}{\m{players}}
\newcommand{\lottery}{\m{lottery}}
\newcommand{\tint}{\m{int}}
\newcommand{\ballot}{\m{ballot}}
\newcommand{\tbool}{\m{bool}}
\newcommand{\lc}{\tlist{\m{coin}}}
\newcommand{\auction}{\m{auction}}
\newcommand{\object}{\m{object}}

%% Typing Judgments for Servers and Clients
\newcommand{\sentailpot}[1]{\prescript{}{S}{\xvdash{#1}} \hspace{2pt}}
\newcommand{\centailpot}[1]{\prescript{}{C}{\xvdash{#1}} \hspace{2pt}}


%% Sharing
\newcommand{\down}{\downarrow^{\m{S}}_{\m{L}}}
\newcommand{\up}{\uparrow^{\m{S}}_{\m{L}}}
\newcommand{\eacquire}[2]{#1 \leftarrow \m{acquire} \; #2}
\newcommand{\eaccept}[2]{#1 \leftarrow \m{accept} \; #2}
\newcommand{\erelease}[2]{#1 \leftarrow \m{release} \; #2}
\newcommand{\edetach}[2]{#1 \leftarrow \m{detach} \; #2}


%% Subtyping
\newcommand{\subt}[2]{#1 \leq #2}
\newcommand{\wsubt}{ <: }
\newcommand{\qsubt}[1]{\overset{#1}{\leq}}


%% Latex
%\newtheorem{theorem}{Theorem}
%\newtheorem{definition}{Definition}
%\newtheorem{lemma}{Lemma}
%\newtheorem{cor}{Corollary}


%%Global Semantics
\newcommand{\sinfer}[3]
{\inferrule
{#3}
{#2}
#1}
\newcommand{\enq}[2]{\m{enq}(#1, #2)}
\newcommand{\deq}[1]{\m{deq}(#1)}
\newcommand{\nil}{[]}
\newcommand{\elem}[1]{[#1]}


%% Channel typing
\newcommand{\eqdef}{\cong}


%% Types to Processes
\newcommand{\typeProc}[2]{#1 \Longrightarrow #2}

%% AARA
\newcommand{\abs}[1]{\left\lvert #1 \right\rvert}
\newcommand{\bin}[1]{(#1)_2}
% \newcommand{\ceil}[1]{\left\lceil #1 \right\rceil}
\newcommand{\bigO}[1]{\mathcal{O}(#1)}
% \newcommand{\ignore}[1]{\textcolor{red}{#1}}
% new \oset macro
\makeatletter
\newcommand{\oset}[3][-0.7ex]{%
  \mathrel{\mathop{#3}\limits^{
    \vbox to#1{\kern-2\ex@
    \hbox{$\scriptstyle#2$}\vss}}}}
\makeatother
\newcommand{\monus}{\oset{.}{-}}

%% Indexed Types
\newcommand{\cons}{\mathcal{C}}
\newcommand{\vars}{\mathfrak{v}}
\newcommand{\Vars}{\mathcal{V}}
\newcommand{\Cons}{\mathcal{C}}
\newcommand{\Tokens}{\Gamma}
\newcommand{\K}{\gamma}
\newcommand{\Tokentypes}{\mathcal{K}}
\newcommand{\VTokens}{\mathcal{V}}
\newcommand{\TokSig}{\mathcal{S}}
\newcommand{\exchange}[3]{#1 \overset{#2}{\longrightarrow} #3}
\newcommand{\GlobalF}{\ensuremath{\mathfrak{f}}\xspace}
\newcommand{\GlobalP}{\mathfrak{p}}
\newcommand{\depth}{\mathfrak{d}}
\newcommand{\poly}{\mathcal{P}}
\newenvironment{proofsketch}{%
  \renewcommand{\proofname}{Proof Sketch}\proof}{\endproof}

%% Two Counter Machines
%\newcommand{\ins}{\iota}
\newcommand{\tcm}{\mathcal{M}}
\newcommand{\inc}[1]{\m{inc}(#1)}
\newcommand{\dec}[1]{\m{dec}(#1)}
\newcommand{\goto}{\m{goto}}
\newcommand{\zeroc}[1]{\m{zero}(#1) ?}
\newcommand{\halt}{\m{halt}}

%% UC stuff
\newcommand{\fcomm}{\mathcal{F}_{\msf{comm}}}
%\newcommand{\B}[1]{\colorbox{gray}{#1}}
%\newcommand{\hlc}[2][yellow]{{%
%    \colorlet{foo}{#1}%
%        \sethlcolor{foo}\hl{#2}}%
%        }
%\newcommand{\hlcyan}[1]{{\sethlcolor{cyan}\hl{#1}}}
%\newcommand{\B}[1]{\hlc[pink]{#1}}
\definecolor{airforceblue}{rgb}{0.36, 0.54, 0.66}
\newcommand{\B}[1]{{\color{airforceblue}{#1}}}
\newcommand{\wt}{\circled{w}}

%% TODO
\newcommand{\ankush}[1]{\textcolor{red}{\textbf{Ankush: #1}}}


%%% Local Variables:
%%% mode: plain-tex
%%% TeX-master: "pldi19"
%%% End:

\newcommand{\oplusP}{{\oplus_{\textsf{P}}}}
\newcommand{\TUnit}{\mathbf{1}}

% PReSt syntax highlight

\definecolor{dkgreen}{rgb}{0,0.4,0}
\definecolor{ltblue}{rgb}{0,0.4,0.4}
\definecolor{dkblue}{rgb}{0,0,0.6}
\definecolor{dkviolet}{rgb}{0.3,0,0.5}
\definecolor{codegreen}{rgb}{0,0.6,0}
\definecolor{eminence}{RGB}{108,48,130}

\lstdefinelanguage{prest}{
  morekeywords=[1]{
    define, proc, yield, type,
    with_tokens
  },
  morekeywords=[2]{
    flip, match, case, pcase, send, recv, assert, assume, wait, close, work, pay, get,
    generate
  },
  morekeywords=[3]{},
  % Size of tabulations
  tabsize=2,
  % Case sensitivity
  sensitive=true,
  % Disable automatic breaking of long lines
  breaklines=false,
  % Additional characters
  extendedchars=true,
  alsoletter=*,
  % Position of captions is bottom
  captionpos=b,
  % comments
  morecomment=[l]{//},
  % unicode characters
  mathescape=true,
  % Basic Style
  basicstyle=\footnotesize\BeraMonottFamily, 
  frame=single,
  % Style for (listings') identifiers
  identifierstyle={\footnotesize\BeraMonottFamily\color{black}},
  % Style for toplevel keywords
  keywordstyle=[1]{\footnotesize\BeraMonottFamily\color{dkblue}},
  % Style for expression keywords
  keywordstyle=[2]{\footnotesize\BeraMonottFamily\color{dkviolet}},
  % Style for declaration keywords
  keywordstyle=[3]{\footnotesize\BeraMonottFamily\color{ltblue}},
  % Style for strings
  stringstyle=\BeraMonottFamily,
  % Style for comments
  commentstyle={\BeraMonottFamily\color{dkgreen}},
  literate=
  {|-}{$\vdash\ $}{2}
  {<-}{$\ \leftarrow\ $}{2}
  {=>}{$\ \Rightarrow\ $}{3}
  {<->}{$\ \leftrightarrow\ $}{2}
  {|\{*\}-}{$\mathlarger{\mathlarger{\vdash^*}}$}{3}
}

\newcommand{\lst}[1]{\lstinline[mathescape,language=prest]!#1!}


