The Universal Composability Framework~\cite{uc} is the popular and widely-used framework for modelling the security of cryptographic and distributed protocols.
Its novel contrbution compared to other frameworks is that it provides a very strong notion of securiy: a UC-secure protocol is proved to be secure even when composed with arbitrary other protocols running concurrently.
This constrasts with other, property-based notions of security~\todo{need to get some citations here}.

In this work we present a programmatic approach to the UC framework that takes advantage of the modularity the framework provides. 
Analyzing large and complex protocols is a difficult task made easier by UC's ideal functionality abstraction. 
However, despite this additional modularity, UC proofs and models still tend to be very complex, unweildy, and difficult to understand.
These issues are exacerbated when new communication models are added on top of UC~\cite{katz, etc}.
Therefore, we propose a two-fold solution: a new construction for modelling different communication models that removes all model-specific code from protocols and functionalities, and an implementation of the UC framework in the Nomos language. 


