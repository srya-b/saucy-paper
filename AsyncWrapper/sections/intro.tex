In cryptography, the Universal Composability (UC) framework is the leading framework for defining the security of protocols.
Security definitions in UC are considered the strongest among other approaches, because it guarantees security under concurrent composition with arbitrary protocols.




UC models communication between ITMs as beinf delivered by the adversary.
A more common model is a functionality that implements synchronous or asynchronous through delivery counters waiting to run to zero before messages are delivered.
Such notions rely on the traditioa length of input polynomial time notion to provide guarantees of eventual delivery.
These mechanisms, however, complicate ideal functionality definitions and don't fit well into a programming approach to UC.
Instead in this work we introduce our own model for communication which makes use of the import mechanism to ensure eventual delivery.
In our mechanism the adversary is in charge of a global scheduler that can be accesseed by arbitrary sessions in a model that realized GUC within the standard UC framework. 
(do we use existing one of formulate our own?)

We provide an implementation of UC, in Haskell, where we implement the framework and our modelling of asynchrony.
An implementation is key to our contribution because
