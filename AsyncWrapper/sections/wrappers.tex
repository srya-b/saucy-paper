\todo{I realized that I should introduce the concept of environment forcing progress before all the details here}

%\todo{Add leak keyword in there as well and explain why we buffer all the leaks in the wrapper instead of activating the adversary.}

\todo{change dealer to be $p_d$ instead of $\mathcal{D}$}

\todo{expand out pluseq and minuseq in figures}

\todo{expand other protocol sessions in simulator to be (sid', pid) instead of $\mathcal{F}$}.

%\todo{change Send macro to be bold}

\todo{simpartyoutput return OK and return leaks is not the same so it should be cleared up}

\todo{input parameters msf or regular math mode?}

\todo{should make clear that the simulator is always activated when a real world party would output something because it only outputs when poll, exec, or adverasry exec and simulator is active in all those cases and so can react and execute the corresponding codeblock in the ideal world.}

%Despite using sessions types and the Nomos language to make it easier to write complete UC protocols, it remains a challenge to use existing UC models, such as those for async/sync communication, as they greatly complicate the design of protocols and ideal functionalities.


Tools like session types and a concrete implementation of the UC framework in Nomos make working with the framework easier, however, on-paper definitions still suffer from over-complexity.
One of the main causes of the difficulty is in how different communication models are modelled in UC.
Even for trivial functionalities like a simple atmomic broadcast, the most common communication construction, the feth-delivery model shown in Figure~\ref{fig:fatomic:old}, greatly complicates the ideal functionality definition.
Having to work with these models is cubersome and errorprone, but they also make security proofs difficult to verify or build on top of.
Therefore, in this work we propose a new construction for synchronous and asynchronous communication.
Our construction allowes ITMs to execute whole blocks of code with adversarial delay rather than just messages, and it offloads all model-specific behavior to a separate idela functionality.
We additionally present generic code that all ITMs must be spawned with that wish to interact with our functionality and execute code with adversarial delay.
Finally, we present our consturction in the GUC~\ref{guc} framework as a shared functionality and use the new import mechanism to achieve delivery guarantees.

%While using tools like session types and the Nomos language improve the experience working with UC, many on-paper UC models still suffer from completxity making them difficult to understand and work with.
%Most of the complexity often arises, even for trivial functionalities and protocols, from the requirements of the underlying communication model and the additional messages/bookeeeping they require.
%Not only do they make writing and switching between models cumbersome and error-prone, but they also make security proofs difficult to understand and even harder to verify.
%In this work we also propose a new construction for the synchronous and asynchronous communication models that greatly improves upon existing approaches.
%Our construction removes all model-specific code from protocols and ideal functionalities, defines code that each ITI must be spawned with, and uses the new import mechanism to provide communiation guarantees.
%One of the key driving forces behind this work is an implementation of UC that simplifies working withe UC through simpler communication models and a domain-specific language. (\todo{$\leftarrow$ this is a real hot pile})

%In this section we define our synchronous and asynchronous communication models which enable delaying execution of entire code blocks and offload all communication-specific code from the protocols and functionalities to our new shared functionality. 
%We construct our models in the GUC-framework~\ref{guc} instead of the standard UC framework so that the environment can play a special role in ensuring delivery and termination.

At a high level, the goal of out ideal functionality (which we dub the \textit{wrapper}) is to enable ITIs to schedule code blocks for future execution.
An ITI that wants to execute some code with adversarial delay informs the wrapper of the code and asks it for activation when it's time to execute.
The adversary can control when a codeblock is executed, of course, by calling the wrapper to delay its exeution or force execution at any time.
The environment can also force a code block to execute.
This ensures that an adversary can not stall protocol progress indefinitely.
Therefore, the main idea in our construction is that, although the adverasry can control delay, it can never halt protocol progress entirely by indefinitely delaying a code block, because the environment can always step in and ensure protocol progress.
We describe in greater detail, later in this section, how the import mechanism is used to achieve this property.

%Before we divulge the details of how our model is implemented, we first give an example illustrating the disadvantage of current asynchronous (w.l.o.g synchronous) models compared to ours.
Before diving into the details of how the wrapper is implementated, and its consequences on protocol design, we first give an example illustrating the simplicity of our model compared to traditional UC communication.
Take, for exampe, a simple atomic broadcast functionality in Figures \ref{fig:atomic:old} and \ref{fig:atomic:new}. 
The top figure is the atomic broadcast written in the traditional fetch-based asynchronous model.
Parties submit messages to be broadcast and must continually be activated by the environment to check for check for new results and ultimately move the functionality forward.
The adversary, meanwhile, can add delay and force more calls to \textsc{Fetch} be made.
Our version on the other hand, Figure \ref{fig:atomic:new}, simplifies the pseudo-code by offloading model-specific behavior to the wrapper.
It makes use of the keyword \Eventually, which is used only for clarity in pseudo-code but is useful abstraction in our implementation.
Despite the abstraction, however, \Eventually only substitutes a single message write and receive from the point of view of the functionality.
The exact pseudo-code is shown in the final figure, Figure~\ref{fig:atomic:real}.

A natural question that arises is whether the same \Eventually syntactic sugar can be used to encapsulate the fetch-delay version of the functionality as well. 
The unique nested \Eventually behavior in our version of the atomic broadcast turns out to be difficult to replicate in the fetch-deliver model.
With the constraints of the per-party variables tracking activations, careful engineering would be required in order to achieve the same nested message delay. 
n ther other hande such a translation to the wrapper-based construction is trivial and any such intuitive and convenient layering of delayed code execution can easily be constructed.

\begin{figure}
\begin{subfigure}{\columnwidth}
    \begin{bbox}[title={$\mathcal{F}_\msf{Atomic-Fetch-Delay}$}]

Initialize: $\msf{BC} \leftarrow \emptyset$, $\msf{inputs} \leftarrow \emptyset$, $\msf{D}_1, \dots, \msf{D}_n \leftarrow 0$, 
$\msf{pos}_1 , \dots, \msf{pos}_n \leftarrow 0$

\OnInput \inmsg{tx} from \Partyi:
    
    \quad Append \msf{tx} to \msf{inputs}
    
    \quad Send \msf{tx} to \Adv
    
\OnInput \inmsg{\tsc{Fetch}} from \Partyi:

    \quad If $D_i > 0$: $D_i \leftarrow D_i - 1$
    
    \quad Else if $\msf{pos}_i < |\msf{BC}|$: $\msf{pos}_i \leftarrow \msf{pos}_i + 1$
        
    \quad Else if $\msf{pos}_i == |\msf{BC}|$:
        
        \qquad Pop $\msf{tx}$ from $\msf{inputs}$
        
        \qquad If $\msf{tx} \neq \bot$, append $\msf{tx}$ to $\msf{BC}$ and set $\msf{pos}_i \leftarrow \msf{pos}_i + 1$
        
    \quad Send $\msf{BC}[1:\msf{pos}_i]$ to \Partyi
    
\OnInput \inmsg{\Delay}{\Partyi}{\msf{T}} from \Adv:

    \quad If \msf{T} is in unary notation: $\msf{D}_i \leftarrow \msf{D}_i + \msf{T}$
    
    \quad Send \msf{(OK,)} to \Adv

\OnInput \inmsg{\tsc{Append}}{\msf{idx}} from \Adv:

    \quad Remove \msf{inputs[idx]} as \msf{tx}
    
    \quad Append \msf{tx} to \msf{BC}
    
    \quad Send \msf{(OK,)} to \Adv

\end{bbox}

    \caption{An atomic broadcast functionality in the fetch-delay model for the asynchronous world.}
    \label{fig:atomic:old}
\end{subfigure}
\begin{subfigure}{\columnwidth}
    \begin{bbox}[title={$\mathcal{F}_\msf{Atomic-Eventually}$}]

Initialize: $\msf{BC} \leftarrow \emptyset$

\OnInput \inmsg{tx} from $P_i$:

    \quad \Leak (\msf{tx}, \Partyi) and \Eventually:
    
    	\qquad {\color{Blue} Append \msf{tx} to \msf{BC}, $\msf{pos} \leftarrow |\msf{BC}|$}
    	
    	\qquad {\color{Blue} For each $P_j$:}
    	    
    	    \qqquad {\color{Blue} \Leak (\msf{pos}, $P_j$) and \Eventually:}

			\qqqquad {\color{Red} send $\msf{BC}[1:\msf{pos}]$ to $P_j$}
	
	\quad \Send Ok $\rightarrow P_i$
    
        
\end{bbox}

    \caption{The same atomic broadcast functionality written with our async wrapper construction using the \Eventually~syntax sugar.}
    \label{fig:atomic:new}
\end{subfigure}
\begin{subfigure}{\columnwidth}
    \begin{bbox}[title={$\mathcal{F}_\msf{Atomic-Final}$}]

Initialize: $\msf{BC} \leftarrow \emptyset$

\OnInput \inmsg{tx} from $P_i$:
	\begin{enumerate}
		%\item \Leak (\msf{tx}, \Partyi)
		\item \Send (\Leak, (\msf{tx}, \Partyi)) $\rightarrow \Wasync$

		\item \Send (\Schedule, \msf{Broadcast}, \msf{tx}) $\rightarrow \Wasync$

		\item Receive Ok $\leftarrow \Wasync$, \Send Ok $\rightarrow P_i$

	\end{enumerate}
   
{\color{Blue} {\bf Code} \tsc{Broadcast} $(msg)$:}

	\begin{enumerate}
		\item {\color{Blue} Append $msg$ to \msf{BC}, $\msf{pos} \leftarrow |\msf{BC}|$}

		\item {\color{Blue} For each $P_j$:}

			\quad {\color{Blue} \Leak (\msf{pos}, $P_j$)}

			\quad {\color{Blue} \Send (\Schedule, \tsc{Send}, (\msf{BC}[1:\msf{pos}], $P_j$)) $\rightarrow \Wasync$}
		
    \end{enumerate}

{\color{Red} {\bf Code} \tsc{Send} $(msg, P_i)$: \Send $msg \rightarrow P_i$ }


\end{bbox}

    \caption{The final version of atomic broadcast when \Leak~and \Eventually~sugar is removed.}
    \label{fig:atomic:real}
\end{subfigure}
\caption{Three representations of the same atomic broadcast protocol. The top is in the traditional feth-based model. The second is using our simplified wrapper construction with keywords for clarity, and the bottom the actual pseudo-code for the functionality.}
\label{fig:fatomic}
\end{figure}

\subsection{Asynchronous Wrapper}
In Figure~\ref{fig:asyncwrapper:short}, the async wrapper, \Wasync, accepts two messages from parties and functionalities: leaked messages and code blocks in the form of a \msf{schedule} message (alternatively, when code uses the \Eventually~keyword).
The \msf{schedule} message is nothing more than information that the caller wants to the be activated with in the future by \Wasync.
Generally, it's an identifier for what code to execute (an ideal functionalit may have several codeblocks it schedules) and some parameters to execute it with.
In our running atomic broadcast example, there are two codeblocks to be scheduled.
The first is a \textsc{Broadcast} codeblock which triggers an update to the functionality's internal state after input from a party and then schedules the updated state to be sent to all of the parties at some point in the future.
Therefore, the functionality schedules codeblocks with \Wasync~using identifiers to differentiate the two different codeblocks.

We implement all of the features of the wrapper discussed in Nomos as well.
In the Nomos code we do not use any syntax sugar such as \Eventually~and \Leak.
Instead, the Nomos implementation directly writes these messages to the wrapper a shown in Figure~\ref{fig:nomos:scheduleleak}.

\begin{figure}
\begin{lstlisting}[basicstyle=\small\ttfamily, frame=single]
(* A schedule message to the wrapper *)
$f_to_wraper.SEND ;
send Schedule(codeid, args) $f_to_wrapper ;
...
(* A leak message to the wrapper *)
$f_to_wrapper.SEND ;
send Leak($listofleaks) $f_to_wrapper ;
\end{lstlisting}
\caption{The Nomos code for the implementation of the \Eventually~and \Leak~keywords as writes to the wrapper of \texttt{Schedule(f, args)} message and \texttt{Leak(leaks)} messages.}
\label{fig:nomosscheduleleak}
\end{figure}

%We concurrently present the Nomos interprestation of \Wasync to make our later presentation of a protocol realization example in Nomos more clear.
%Illustrated below we see the Nomos message type defining \msf{schedule} messages sent to the wrapper in Nomos.
%\begin{align*}
%\mi{stype} \; \m{p2w} = \m{Schedule} \text{ of } \m{sid} \text{ \^{} } \m{pid} \text{ \^{} } \m{int} \text{ \^{} } \m{list[ARGS]} \\
%| \; \m{Leak} \text{ of } \m{list[ARGS]}  \\
%\mi{stype} \; \m{f2w} = \m{Schedule} \text{ of } \m{int} \text{ \^{} } \m{list[ARGS]} \\ 
%| \; \m{Leak} \text{ of } \m{list[ARGS]}
%\end{align*}
The \msf{schedule} messages from protocol parties are slightly different from those of the funtionality, because they need to include a party identifier whereas there is only \textit{one} ideal functionality in the execution.

On receipt of a \texttt{Schedule(f, args)} message by the wrapper, the codeblock is added to the \msf{runqueue} until it is executed (we will elaborate on the conditions for execution later). 
Figure~\ref{fig:wrapper:schedule} illustrates the steps taken when an ideal functionality schedules a new code block.
It is important to note that at the end of the activation, the wrapper gives control back to the caller wit an \texttt{Ok} message.
In fact, an important consequene of our wrapper construction is the requirment of functionalities and protocol parties to give control back to the caller when scheduling a codeblocks.

For functionalities, like $\F_\msf{atomic}$ in Figure~\ref{fig:fatomic:new}, after they schedule the code blocks in their current activation, they return contorl back to the party that called them.
Similarly, this requirement forces real world protocol parties to also output an \texttt{Ok} message when at the end of an activation by \Environment~(or $\mathcal{A}$ if corrupt).

The reason behind this design requirement comes down to removing any requirement for adversaries to receive \textit{enough} activation by the environment.
In the case of simulators, when it needs to give input to a corrupt party it needs to receive activation back from that party so that it can proceed and execute codeblocks required to maintain indistinguishability with the real world.\footnote{This becomes more apparent in Section~\ref{sec:rbc} when the simulator needs to react to honest party output in the real world by triggering a corrupt dealer to give input to the ideal functionality which schedules code that sends output to the corresponding ideal world party at the same time.}

When it's time to execute (triggered either by the adversary or the environment), \Wasync~writes an \Exec message back to $\F_{\msf{atomic}}$ instructing it to execute the \textsc{Broadcast} which then schedules sending the message to the different parties. 

\begin{figure}[!ht]
\begin{lstlisting}[basicstyle=\small\ttfamily, frame=single]
case msg (
  Schedule(f, args) => delay = delay + 1 ;
    $todonew <- tcons (f, args, False, 0) 
	    $todo ;
    $leaksnew <- lcons ("schedule", 
	    $todo.length-1) $leaks ;
    $wrapper_to_f.SEND ;
    send $wrapper_to_f Ok ;
\end{lstlisting}

\caption{Wrapper code that accepts a code idenfitifier, here \texttt{f} and some \texttt{args} to be sent back to the calling functionality. The codeblock increments the delay in the wrapper, adds the information to the current queue of codeblocks, \texttt{\$todo}, and finally adds some leak information to tell the adversary a new codeblock has been added to the wrapper. Finally, the wrapper returns control back to the caller with an \texttt{Ok} message.}
\label{fig:wrapper:schedule}
\end{figure}

The \Exec interface that $\F_{\msf{atomic}}$ accepts is not illustrated in Figure~\ref{fig:atomic_real}, however it is an interface that all ITIs that wish to schedule codeblocks must implement.
The interface accepts a \texttt{Exec} message of the type specified in Figure~\ref{fig:nomos:frbcexec} with an identifier in the form of an integer rather than a ``name'' as in the pseudo-code examples of $\F_{\msf{atomic}}$.

\paragraph{Synchronous Communication}
One of the advantages of our construction is that both the synchronous and asynchronous models arise from nearly identical ideal functionalities.
The only difference in the two is that the synchronous wrapper, \Wsync, includes a round number and code blocks must be executed by an upper round number.\footnote{Recalls synchronous communication impies an upper bound on the number of rounds a message send, or, in our case, code block execution can be delayed.}
The environment and adversary interact with the wrapper in the same way, through \Advance~and \Exec, however, the wrapper will never update is round past roung $r$, if there are code blocks that must be executed by round $r$.
A consequence of our design is that protocols and funtionalities become, effectively, model-independent---i.e. the same protocol has nearly identical code in both the synchronous and asynchronous settings.

Another closely related work to our own, that of Katz et al.~\cite{katzuc}, proposes a new model for synchronous communication consisting of two ideal functionalities: a clock $\F_{\msf{clock}}$ (shown in Figure~\ref{fig:katzclock} augmented with the import mechanism) and bounded-delay communication channels $\F_{\msf{bd-smt}}$.
However, the bounded delay channels used by Katz still uses the fetch-delay model described previously, and their design still suffers from non-modularity as protocols need to be carefully engineering to work for just the \textit{synchronous} world.
\todo{come back to this with some better language}

%A more closely related attempt to propose a new, modular approach to communication models comes from Katz et al.~\cite{katzuc}.
%In their work they propose a new formulation of synchronous communication 
%
%Notes to talk about fclock:
%\begin{itemize}
%\item \todo{haven't relaly introduced synchronous communication in this paper, is it worth addressing the last bullet point in these notes?}
%\item out modular approach to writing protocols in sync/async versus there's is completely different for katz model
%\item kaz real world requies a clock and bounded delay channels whereas there's one wrapper for both the real and ideal worlds
%\item guaranteed termination: less constrained environment not having to activate parties every round, parties can simply schedule activation of themselves with the synchronous wrapper every round
%\end{itemize}

%\begin{figure}
%\begin{bbox}[title={Functionality $\F_{\msf{clock}}(\mathcal{P})$}]

Initialize for each $p_i \in \mathcal{P}$ a bit $d_i := 0$

\OnInput \inmsg{RoundOK} \color{red} $1 \token$ \color{black} from $P_i$:
	\begin{renumerate}
	\item Set $d_i = 1$
	\item If for all $p_j \in \mathcal{H} : d_j = 1$ then reset $d_i := 0$ for all $p_j \in \mathcal{P}$
	\item \Send (switch, $p_i$) $\rightarrow \Adversary$
	\end{renumerate}

\OnInput \inmsg{RequestRound} \color{red} $0 \token$ \color{black} from $P_i$:
	\begin{renumerate}
	\item \Send $d_i \rightarrow P_i$
	\end{renumerate}

\OnInput \inmsg{corrupt}{$p_i$} \color{red} $0 \token$ \color{black} from \Adversary:
	\begin{renumerate}
	\item Set $\mathcal{H} := \mathcal{H} - p_i$
	\end{renumerate}
\end{bbox}

%\caption{The clock functionality by Katz et al.~\cite{katzclock}. The functionality waits for a message from all honest parties before moving to the next round.}
%\label{fig:katzclock}
%\end{figure}



\subsection{Import}
Before describing the details of \Wasync's construction, we first give a small primer on the import mechanism and how it is used in \Wasync.
The import mechanism was introduced by Canetti et al.~\ref{uc} as an alternatie formulation for polynomial time in ITMs.
The problem that import is meant to solve is an infinite machines problem where, despite being locallt polynomially bound by the size of its input, an ITM can spawn additional ITMs which, in turn, spawn more ITMs.
Through this mechanism even though ITMs are locally polynomially bound, the system of ITMs can consists of a potentially unbounded number of ITMs. 
This breaks the polynomial runtime guarantee of the system of ITMs. 
Refer to the original UC work~\ref{uc} for a more in depth explanation of the motivating problem.

At a high-level the import mechanism seeds the UC experiment with a polynomial amount of \textit{import}.
The environment, the first ITM in the execution is spawned with all of the import and, in the process of spawning other ITMs, can send them some of its import balance with messages.
A good interpretation of import is to treat them like tokens or coins that are passed around between ITMs.
The polynomial bound for ITMs is defined as follows: an ITM $M$ is said to be $T$-bounded by a polynomial $T$ if it takes at most $T(n)$ computation steps where $n$ is the difference between the amount of import $M$ received and sent to other machines.

The wrapper requires import be sent with some of the calls that other ITMs make. 
For exampe, in Figure~\ref{fig:asyncwrapper:short}, the adversary must spend import tokens in order to add to the internal delay and execute a code block.
The environment must alternatively spend $1 \token$ to try to \Advance~the wrapper.

\subsection{Executing Codeblocks}
Codeblocks can be executed in two ways in both the synchronous and asynchronous wrappers: by the adversary and by the environment \footnote{Recall that the environment can force the wrappers to make progress by sending {\em enough} \Advance~messages to it.}.
The adversary, of course, has control over the delay before a code block executes but is limited by the available import which prevents it from halting protocol progress.
To this effect the adversary can also execute a code block at a certain index at any point in execution through the \Exec interface.

The environment can can also execute code blocks by exhausting the import available to the adversary and forcing progress to be made in \Wasync.
We detail the {\em eventual delivery} property in a later section, however, at a high level: the environment has a greater balance of import than the adversary so it can always send an \Advance~message (1 unit of import) to \Wasync more than the adversary can \Delay (1 unit of import).
The cost of each of the operations is detailed in Figure \ref{fig:asyncwrapper:short}.

\subsection{Leaking}
In traditional UC definitions, when an ITM leaks a message to the adversary, it directly writes it tothe backdoor tape of the adversary, giving it activation.
In our construction leaking information to the adversary is also performed through the wrapper, hence \Leak~keyword.
The \Leak~keyword just writes the leaked message to the wrapper which buffers it until the adversary requests it.
We motivate this method of leaking for the same reason that we don't want to cede activation to the adversary during a leak operation and either wait indefinitely for the adversary or require continuous activation by the environmnt to complete all required computation.

\subsection{Balanced Environments}
In the previous subsection we state that the environment can force progress to be made in the wrapper by exhausting the amount of import that the adversary can use to delay.
However, from our previous discussion regarding \textit{balanced environments} in Definition~\ref{def:balancedenvironments} we arrive at a problem: \textit{for every unit of import that \Environment~sends to an ITI, it must also send to \Adversary}.
This would suggest that the adversary will \textit{always} have enough import to prevent the environment from advancing the wrapper, and, therefore \Adversary~can delay codeblocks indefinitely.
We address this problem by relaxing the balanced environments constraint in the GUC framework for our shared functionality.

We motivate our relaxing of the balanced environments constraint, first, by addressing the original reason behind Definition~\ref{def:balancedenvironments}.
In summary, the the argument states that when a protocol is extended trivially to instruct $\pi_i$ to send a message to \Adversary~proportional to it's import, the environment can ensure the $\pi_i$ always has more import than the \Adversary~and, therefore, the adversary is no longer simulatable. 
We can conclude that the balanced environments constraint exists to make the framework more expressive rather than as a technical requirement to achieve simulation or composition. 
Therefore, we relax the balanced environments constraint specifically for calls of \Advance~by the environment to our wrapper.
The reaso behind this is that we require the adversary not be able to indefinitely delay codeblock execution and it does not limit the simulatability of protocols.
We propose that any pair of protocols that were UC-emulatable under the balanced environments definition remain UC-emulatable under our weakening.

\begin{lemma}\label{lem:enoughimport}

The \textit{weakly-balanced} relaxation guarantees a simulator \mathc{S} will always have sufficient import to delay ideal world codeblocks \textit{at least} as long as the corresponding real world code blocks.

\end{lemma}

\textit{Proof of Lemma \ref{lem:enoughimport}.}

The crux of the argument lies in ensuring that, even when minimal import is provided to the adversary in the ideal world, it has sufficient import to meaningfully delay ideal world codeblocks to ensure simulatability. 
Therefore we set up an extreme scenario:
\begin{itemize}
\item A protocol $\pi$ which schedules $n$ codeblocks and outputs to the environment in the last codeblock. An ideal protocol $\phi$ schedules a single codeblock that outputs the same message to the environment.
\item The simulator must be able to ensure that the ideal world codeblock can execute at the same time as the final codeblock in $\pi$.
\end{itemize}

In the real world with dummy adversary $\mathcal{D}$, $\mathcal{Z}$ needs to \Advance \Wasync $n+1$ times in order to force execute all codeblocks.
Scheduling $n$ codeblocks requires at least $n$ unites of import on the part of the parties, and the adversary in both worlds also receives at least $n$ units of import.
Therefore, in order to ensure indistinguishability, the simulator must be able to delay the ideal world \Wasync $n$, which it can plainly do \footnote{The adverasry is also given $k$, the security parameter, amount of import at the start of execution ensuring that a polynomial time simulator is still polynomial under this new definition.}.



%\subsection{An Atomic Broadcast Protocol with the Wrapper}
%In this section we provide a construction of a simple broadcast protocol with our wrapper and demonstrate a UC-emulation proof at a high level (the full proof is left to the appendix).



%\todo{below this line is no longer up to date}
%
%
%\subsection{Balanced Environments}
%Some rough ideas on the writeup for this section.
%In the GUC setting the environment’s interaction with the shared functionality is intended to represent parties from other sessions of potentially different protocols. 
%Therefore, intuitively, it doesn’t make sense that participants of the challenge protocol have a different interface to those of other sessions. 
%Therefore, in order to fit within the GUC framework, we require that protocol parties also be able to call “poll” to the wrapper. 
%Although this seems counter intuitive, it has no impact on the delivery properties or functionality of the wrapper. 
%The reason is the balanced environments definition presented above. 
%As expected in UC, an environment that gives import to parties in order to call “poll” must give that import to the adversary as well. 
%Therefore, the adversary will always have enough import to effectively neutralize any calls to “poll” that parties of the challenge protocol make. 
%As the wrapper is a shared functionality, a simulator need only observe which protocol session’s party calls “poll” and add “delay” to the wrapper if it’s the challenge protocol session.  
%
%Specific to this wrapper construction is the idea that the environment doesn’t have to give the simulator the same construction as the ideal wrapper in this case. 
%This leads to a place where we explicitly reject the balanced environment condition for our wrapper construction as it further relaxes the constraints on the environment without sacrificing simulatability. 
%We show that the simulator always has enough import to at least delay codeblocks in the ideal world enough such that they can be delivered at the same time. 
%Therefore, the simulatability requirements are already satisfied by the wrapper delay incrementing on every scheduled code block and, hence, the simulator receiving as much import as the real world parties can always delay.
%
%It’s not enough that constant sized output shared functionalities be exempt from the balanced environment definition. Because you can still have a shared functionality that requires the adversary to react a \# of times proportional to the import of the shared functionality and then the adversary runs out of import because the environment can always give more import to the shared functionality than it does the adversary.
%In the case of the wrapper, however, we specifically require that the adversary run out of import eventually but not so quickly that it is unable to delay the codeblocks in the ideal world.
%
%
%
%Distinction in features and side effects
%\begin{itemize}
%\item Features: explicit design goals that the wrapper achieves. This includes:
%   \begin{itemize}
%   \item Simplifying through the removal of model-specific code in the pseudo-code. Improves using UC from people verifying proofs and creating there own. Towards the programmability we're aiming for?? but what exactly is the programmability message here
%   \item Uses import to achieve eventual delivery in asynchronous networks and input completeness in synchronous network.
%   \item Offloading work from protocols/functionalities to the wrapper and some work to the environment but without useless fetches.
%   \end{itemize}
%
%\item side effects:
%   \begin{itemize}
%   \item doesn't lose activation
%   \item all functionalities and must give control back to caller. p2p messages go through a functionality and must be delayed in the wrapper so functionalities can only write to other functionalities after being activated. Recursively, at some ponint either a code block is schedules or some local computation is completed and the functionality halts. In both cases, wrie back to caller instead --> makes riting code better. Not necessary but reomves conditions from environment.
%   \end{itemize}
%\end{itemize}
%
%
%
%The asynchronous wrapper, presented in Figure~\ref{fig:asyncwrapper:short}, is only a snippet of the important parts of the full wrapper code shown in ~\ref{fig:asyncwrapper:complete}.
%We first cover the basic operations of the wrapper.
%
%
%\subsection{Delayed Code Execution}
%In traditional synchronous and asynchronous communication models, adversrial delay usually takes the form of only message delay where parties require continuous activation in order to request new messgaes from a message passing functionality or cedeing control to the adversary mid-activation.
%An example of whot this can lead to more complicated environents and protocols is evident in the synchronous model of Katz et al.~\cite{katzclock}. 
%Even the original formulation of aysnchronous communication had issues where adversaries were able to indefinitely stall protocol progress.
%
%Our construction, on the other hand, offloads much of the communication model-specific code, to our shared wrapper.
%Further, it expands message delay to the delay of entire code blocks, and it also also simplifies
%example of this in figure~\ref{fig:rbc:both}.
%
%
%\paragraph{\textsc{Eventually} Keyword.}
%For the remainder of this work, we will make advantages of keywords that abstract the message passing between the sync/async wrappers and other ITIs.
%The most important codeword is the \textsc{Eventually} keyword which is used to indicate delayed execution of a code block within the code of some ITI.
%For example, the simple message passing functionality introduced in Figure~\ref{fig:fsmcasync}, is transformed into Figure~\ref{fig:fsmc:wrapper}.

\begin{figure}
\begin{bbox}[title={\textbf{Wrapper} $\Wasync$} ] 

Intialize $\msf{run\_queue} := \emptyset, \msf{leak\_buffer} := \emptyset, \msf{delay} := 0$

\vspace{2mm} \hrule \vspace{2mm}

\OnInput \inmsg{\Schedule}{\msf{func}}{\msf{args}} \color{red} $1 \token$ \color{black} from $\mathcal{F}/P_i$:
	\begin{enumerate}
		\item Append ($\mathcal{F}/P_i$, \msf{func}, \msf{args}) to \msf{run\_queue}
		\item Append ($\mathcal{F}/P_i$, \msf{func}, \msf{args}) to \msf{leak\_buffer}
		\item $\msf{delay} = \msf{delay} + 1$
		\item \Send $\msf{OK} \rightarrow \mathcal{F}/P_i$
	\end{enumerate}


\OnInput \inmsg{\Exec}{\msf{idx}} \color{red} $1 \token$ \color{black} from \Ag:
	\begin{enumerate}
		\item \F/\Partyi, \msf{func}, \msf{args} $\leftarrow$ pop \msf{run\_queue}
		\item \Send (\Exec, \msf{func}, \msf{args}) $\rightarrow$ \F/\Partyi
	\end{enumerate}

\OnInput \inmsg{\textsc{Advance}} \color{red} $1 \token$ \color{black} from $\mathcal{Z}$:
	\begin{enumerate}
		\item If $\msf{delay} > 0$:

			\quad  $\msf{delay} = \msf{delay} - 1$

			\quad  \Send \textsc{Advance} $\rightarrow \mathcal{A}$
		\item Else:
			
			\quad $(\mathcal{F}/P_i, \msf{func}, \msf{args}) \leftarrow$ pop $\msf{run\_queue}$

			\quad \Send $(\Exec, \msf{func}, \msf{args}) \rightarrow \mathcal{F}/P_i$
	\end{enumerate}

\OnInput \inmsg{\textsc{Delay}}{$d$} \color{red} $d \token$ \color{black} from \Ag:
	\begin{enumerate}
		\item If $\msf{rounds} > 0:$

			\quad $\msf{delay} = \msf{delay} + \msf{rounds}$
	\end{enumerate}

%\OnInput \inmsg{\textsc{Leak}}{\msf{msg}} from \F:
%	\begin{enumerate}
%		\item Append (\F, \msf{msg}) to \msf{leak\_buffer}
%	\end{enumerate}
%
%\OnInput \inmsg{\textsc{Get-Leaks}} from \Ag:
%	\begin{enumerate}
%		\item \msf{leaks} $\leftarrow$ pop all from \msf{leak\_buffer}
%		\item \Send \msf{leaks} to \Ag
%	\end{enumerate}
\end{bbox}

\caption{The wrapper mechanism for scheduling codeblocks and executing them.}
\label{fig:asyncwrapper:short}
\end{figure}

