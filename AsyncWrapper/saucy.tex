\documentclass[conference]{IEEEtran}
\IEEEoverridecommandlockouts
% The preceding line is only needed to identify funding in the first footnote. If that is unneeded, please comment it out.
\usepackage[dvipsnames]{xcolor}
\usepackage{cite}
\usepackage[most]{tcolorbox}
\usepackage{amsmath,amssymb,amsfonts,amsthm}
\usepackage{algorithmic}
\usepackage{graphicx}
\usepackage{subcaption}
\usepackage{textcomp}
\usepackage{mathtools}
\usepackage[shortlabels]{enumitem}

%% PL packages
\usepackage{stmaryrd} 
\usepackage{proof}
\usepackage{mathpartir}
\usepackage{color}
\usepackage{xstring}

\def\BibTeX{{\rm B\kern-.05em{\sc i\kern-.025em b}\kern-.08em
    T\kern-.1667em\lower.7ex\hbox{E}\kern-.125emX}}
\begin{document}

\usetikzlibrary{matrix, arrows.meta, calc, positioning}
\tikzset{myarrow/.style={-Latex, rounded corners},}

\definecolor{vert}{RGB}{0,181,0}
\definecolor{oran}{RGB}{223,74,0}
\definecolor{viol}{RGB}{134,0,175}
\definecolor{roug}{RGB}{215,15,0}
\definecolor{bb}{RGB}{0,0,0}
\definecolor{gg}{RGB}{220,220,220}

\newtcolorbox[auto counter]{bbox}[2][]{%
    colback=white,
    colframe=bb,
    %colbacktitle=white!90!roug,
	colbacktitle=white!40!gg,
    coltitle=black,
    fonttitle=\bfseries, 
    enhanced,
    attach boxed title to top left={yshift=-2mm, xshift=0.5cm},%
    #1,% For possible options
}

\title{Eventually Delivering Liveness and Composability}

\newcommand{\mc}[1]{\ensuremath{\mathcal{#1}}}
\newcommand{\msf}[1]{\ensuremath{{\mathsf {#1}}}}
\newcommand{\mathc}[1]{\ensuremath{\mathcal{#1}}}
\newcommand{\tsc}[1]{\textsc{#1}}
\newcommand{\f}[1]{\ensuremath{\mathcal{#1}}\xspace}
\newcommand{\F}{\f{F}}
\newcommand{\PI}{\ensuremath{\pi}\xspace}
\newcommand{\RHO}{\ensuremath{\rho}\xspace}
\newcommand{\achan}{\ensuremath{\F_{\msf{achan}}^{p_r,p_s}}}
%\newcommand{\C}{\mathcal{C}}
\newcommand{\con}[1]{\msf{Contract_{#1}}}
%\newcommand{\Fsync}[2]{\ensuremath{\F_{\msf{sync},#1,#2}}}
\newcommand{\Fsync}[2]{\ensuremath{\F_{\msf{BD-SEC}}(#1,#2)}}
\newcommand{\Fchan}[2]{\ensuremath{\F_{\msf{chan}}(#1,#2)}}
\newcommand{\Fbdsec}{\ensuremath{\F_{\msf{BD-SEC}}^{\delta,\ell}}}
\newcommand{\Fbc}{\ensuremath{\F_{\msf{broadcast}}}}
\newcommand{\Fsfe}{\ensuremath{\F_{\msf{SFE}}}}
\newcommand{\Fstate}{\ensuremath{\F_{\msf{state}}}}
\newcommand{\Fclock}{\ensuremath{\F_{\msf{clock}}}}
\newcommand{\Frbc}{\ensuremath{\F_{\msf{rbc}}}}
\newcommand{\Fpay}{\ensuremath{\F_{\msf{pay}}}}
\newcommand{\Fcom}{\ensuremath{\F_{\msf{com}}}\xspace}
\newcommand{\Fauth}{\ensuremath{\F_{\msf{auth}}}\xspace}
\newcommand{\Fflip}{\ensuremath{\F_{\msf{coinflip}}}\xspace}
\newcommand{\Fro}{\ensuremath{\F_{\msf{RO}}}\xspace}
\renewcommand{\O}[1]{\ensuremath{\mathcal{O}(#1)}\xspace}
\newcommand{\kbits}{\ensuremath{\{0,1\}^k}\xspace}
\newcommand{\samplek}{\ensuremath{\xleftarrow{\$} \kbits}\xspace}
\newcommand{\Fsmc}{\ensuremath{\F_{\msf{SMC}}}\xspace}
\newcommand{\Fropp}{\ensuremath{\F_{\msf{P2P\hyp RO}}}\xspace}
\newcommand{\Gledger}{\ensuremath{\f{G}_{\msf{ledger}}}}
\newcommand{\Wsync}{\ensuremath{\mathcal{W}_{\msf{sync}}}}
\newcommand{\Wasync}{\ensuremath{\mathcal{W}_{\msf{async}}}}
\newcommand{\Ssyncbracha}{\ensuremath{\mathc{S}_{\msf{sbracha}}}}
\newcommand{\Fbracha}{\ensuremath{\mathcal{F}_{\msf{bracha}}}}
\newcommand{\Schedule}{\tsc{Schedule}}
\newcommand{\Delay}{\tsc{Delay}}
\newcommand{\Advance}{\tsc{Advance}}
\newcommand{\Exec}{\tsc{Exec}}
%\newcommand{\Adversary}{\ensuremath{\mathcal{A}}\xspace}
\newcommand{\A}{\ensuremath{\mathcal{A}}\xspace}
\newcommand{\DummyAdv}{\ensuremath{\mathcal{A}_\mathcal{D}}\xspace}
\newcommand{\DA}{\ensuremath{\A_\mathcal{D}}\xspace}
\newcommand{\Sim}{\ensuremath{\mathcal{S}}\xspace}
\newcommand{\SIM}[1]{\ensuremath{\mathcal{S}_{#1}}\xspace}
\newcommand{\simcom}{\SIM{\msf{com}}}
\newcommand{\cf}{\ensuremath{\mathcal{C}}\xspace}
\newcommand{\ID}[1]{\ensuremath{\mathcal{I}(#1)}\xspace}
%\newcommand{\Sim}[1][]{\ifthenelse{\equal{#1}{}}{\ensuremath{\Simulator}}{\ensuremath{\Simulator_{#1}}}}
\newcommand{\DS}{\SIM{D}\xspace}
%\newcommand{\Environment}{\ensuremath{\mathcal{Z}}\xspace}
\newcommand{\Z}{\ensuremath{\mathcal{Z}}\xspace}
\newcommand{\Partyi}{\ensuremath{P_i}}
\newcommand{\Partyj}{\ensuremath{P_j}}
\newcommand{\partywrapper}{multiplexer\xspace}
\newcommand{\pw}{\PI}
\newcommand{\fwrapper}{\todo{fwrappername}\xspace}

\newcommand{\dealer}{\ensuremath{\mathcal{D}}}
\newcommand{\globalf}[1]{\ensuremath{{\overline{\mathcal{#1}}}}}
\newcommand{\todo}[1]{\textcolor{Red}{todo: #1}}
\newcommand{\edict}{\{\}}
\newcommand{\lar}{\leftarrow}
\newcommand{\rar}{\rightarrow}
\newcommand{\Init}{{\bf \color{NavyBlue} Init}~}
\newcommand{\OnInput}{{\bf \textcolor{Black} On input}~}
\newcommand{\Allinputs}{{\bf \color{Cerulean} All other input~}}
\newcommand{\OnAdvInput}{{\bf \color{BrickRed} On input}~}
\newcommand{\heading}[1]{\textbf{#1}}
\newcommand{\Type}{\ensuremath{\yo{type}}}
\newcommand{\Stype}{\ensuremath{\yo{stype}}}
\newcommand{\bangf}{\ensuremath{!\F}}
\newcommand{\execuc}{\ensuremath{\msf{execUC}}}
\newcommand{\iexecuc}{\inline{execUC}}
\newcommand{\UC}[4]{\ensuremath{\execuc #1  #2  #3  #4}}
\newcommand{\idealP}{\ensuremath{\mathbbm{1}_d}\xspace}
%\newcommand{\prot}[1][]{\ifthenelse{\equal{\ensuremath{#1}}{}}{\ensuremath{\Pi}}{\ensuremath{\Pi_{X #1}}}}
\newcommand{\prot}[1]{\ensuremath{\pi_{\msf{#1}}}}
\newcommand{\lla}{\leftarrow}
\newcommand{\lvd}{\vdash}
\newcommand{\tb}[1]{\text{\color{royalblue}{#1}}}
\newcommand{\tgr}[1]{\text{\color{forestgreen}{#1}}}
\newcommand{\tm}[1]{\text{\color{magenta}{#1}}}
\newcommand{\tg}[1]{\text{\color{gray}{#1}}}
\newcommand{\tp}[1]{\text{\color{purp}{#1}}}
\newcommand{\nparam}[1]{\tp{#1}}
\newcommand{\tr}[1]{\text{\color{Red}{#1}}}
\newcommand{\yo}[1]{\text{\color{YellowOrange}{#1}}}
\newcommand{\inline}[1]{\lstinline[basicstyle=\footnotesize\BeraMonottFamily, mathescape]!#1!}
\newcommand{\nrecv}{\tb{recv}}
\newcommand{\nsend}{\tb{send}}
\newcommand{\nget}{\tb{get}}
\newcommand{\npay}{\tb{pay}}
\newcommand{\nsimget}{\tm{simget}}
\newcommand{\nsimpay}{\tm{simpay}}
\newcommand{\ncase}{\tm{case}}
\newcommand{\nproc}{\tb{proc}}
\newcommand{\nwithdraw}{\tm{withdrawTokens}}
\newcommand{\nif}{\yo{if}}
\newcommand{\nthen}{\yo{then}}
\newcommand{\nend}{\yo{end}}
\newcommand{\nwhile}{\yo{while}}


%\newcommand{\pluseq}{\mathrel{+}=}
%\newcommand{\minuseq}{\mathrel{-}=}
\newcommand{\Assert}{{\bf \color{BrickRed} Assert }}
\newcommand{\Require}{{\bf \color{BrickRed} Require }}

%\theoremstyle{acmdefinition}
%\newtheorem{definition}{Definition}[section]
\newtheorem{ddef}{Definition}
%\newtheorem{theorem}{Theorem}
\newtheorem{claim}{Claim}
%\newtheorem{lemma}{Lemma}

%\newlist{renumerate}{enumerate}{1}
%\setlist[renumerate]{before=\setlength{\baselineskip}{20pt}, itemsep=-2ex, topsep=-2ex}
%\newenvironment{renumerate}{\begin{enumerate}[before=\setlength{\baselineskip}{20pt},itemsep=-2ex,topsep=0pt]}{\end{enumerate}}
\newenvironment{renumerate}{\begin{enumerate}[nosep]}{\end{enumerate}}
%\newenvironment{ritemize}{\begin{itemize}[before=\setlength{\baselineskip}{20pt},itemsep=-2ex,topsep=0pt]}{\end{itemize}}
\newenvironment{ritemize}{\begin{itemize}[nosep] \renewcommand\labelitemi{--}}{\end{itemize}}

\newenvironment{mylst}{\begin{lstlisting}[basicstyle=\small\BeraMonottFamily, frame=single, mathescape]}{\end{lstlisting}}

\makeatletter
\newcommand{\inmsg}[1]{%
(#1\checknextarg}
\newcommand{\checknextarg}{\@ifnextchar\bgroup{\gobblenextarg}{)~}}
\newcommand{\gobblenextarg}[1]{, #1\@ifnextchar\bgroup{\gobblenextarg}{)~}}
\makeatother


\newcommand{\transfermsg}{\inmsg{transfer}{to}{val}{data}{from}}
\newcommand{\createmsg}{\inmsg{contract \ create}{addr}{val}{data}{private}{from}}
\newcommand{\reject}{\textbf{reject}~}
\newcommand{\ignore}{\textbf{ignore}~}
%\newcommand{\For}{\textbf{For}~}
\newcommand{\Env}{\ensuremath{\mathcal{Z}}}
%\newcommand{\While}{\textbf{While}~}
\newcommand{\Buffer}{\textbf{Buffer}~}
\newcommand{\Send}{\textbf{Send}~}
\newcommand{\Output}{\emph{Output}~}
\newcommand{\Leak}{\textbf{Leak}}
\newcommand{\Eventually}{\textbf{Eventually}~}
\newcommand{\In}{\textbf{in}~}
\newcommand{\If}{\textbf{If}~}
\newcommand{\Else}{\textbf{Else}~}
%\newcommand{\Return}{\textbf{Return}~}

\newcommand{\pluseq}{\ensuremath{\mathrel{+}=}}
\newcommand{\minuseq}{\ensuremath{\mathrel{-}=}}
\newcommand{\Adv}{\ensuremath{\mathcal{A}}}
%\newcommand{\Partyi}{\ensuremath{\mathbf{P_i=(sid,pid)}}}
\newcommand{\sid}{\ensuremath{\msf{sid}}\xspace}
\newcommand{\pid}{\ensuremath{\msf{pid}}\xspace}
\newcommand{\dquad}{\quad \quad}
\newcommand{\qqquad}{\qquad \quad}
\newcommand{\qqqquad}{\qqquad \quad}
\newcommand{\qqqqquad}{\qqqquad \quad}

\newcommand*\circled[1]{\tikz[baseline=(char.base)]{
            \node[shape=circle,draw,inner sep=1pt] (char) {#1};}}

\newcommand*\token{~\circled{t}}

\DeclarePairedDelimiter{\ceil}{\lceil}{\rceil}


\newcommand{\spheading}[1]{ %
	\rotatebox{60}{\parbox{2.5cm}{\raggedright #1}}}




%Potential annotations
\newlength{\rWidth}

\newcommand{\funtype}[1]{%
    {\settowidth{\rWidth}{\ensuremath{#1}}%
        \;\ensuremath{{\xrightarrow{\hspace{\rWidth}}\hspace{-0.84\rWidth}}\!\!\!^%
         {#1}%{\BehindSubString{,}{#1} / \BeforeSubString{,}{#1}}%
         \hspace{0.2\rWidth}\;\;}}}


%% Notation
\newcommand{\m}[1]{\ensuremath{\mathsf{#1}}}
\newcommand{\mb}[1]{\ensuremath{\mathbf{#1}}}
\newenvironment{sill}{\begin{tabbing}}{\end{tabbing}}


%% Configuration
\newcommand{\conftree}[3]{\left[#1\right] \; \proc{#2}{#3}}
\newcommand{\confprovider}[2]{(#1)^{#2}}
\newcommand{\confset}[1]{\overline{#1}}
\newcommand{\esync}{\; \m{esync}}
\newcommand{\measure}{energy}
\newcommand{\measures}{energies}
% \newcommand{\mc}[1]{\mathcal{#1}}
\newcommand{\CC}{\mathcal{C}}
\newcommand{\DD}{\mathcal{D}}
\newcommand{\EE}{\mathcal{E}}
\newcommand{\FF}{\mathcal{F}}

%% Modes
\newcommand{\s}{\m{S}}
\newcommand{\li}{\m{L}}
\newcommand{\cl}{\m{C}}
\newcommand{\p}{\m{P}}

\newcommand{\lang}[1]{\mathbf{L}(#1)}

%% Contexts and Typing Judgment
\newcommand{\W}{\Omega}
\newcommand{\Sg}{\Sigma}
\newcommand{\xvdash}[2]{\sststile{#2}{#1}}
\newcommand{\xVdash}[1]{%
  \Vdash^{\mkern-8mu\scriptstyle\rule[-.9ex]{0pt}{0pt}#1}%
}
\newcommand{\confpot}[2]{\overset{#1}{\underset{#2}{\vDash}}}
\newcommand{\potconf}[1]{\overset{#1}{\vDash}}
\newcommand{\spanconf}{\vDash}
\newcommand{\confspan}[1]{\overset{(#1)}{\vDash}}
\newcommand{\confspanlocal}[1]{\overset{\langle #1 \rangle}{\vDash}}
\newcommand{\D}{\Delta}
%\newcommand{\G}{\Gamma}
\newcommand{\T}{\Theta}
\newcommand{\proves}{\vDash}
\newcommand{\w}{\omega}
%\newcommand{\Co}{\mathcal{C}}
%\renewcommand{\C}{\mathcal{C}}
\newcommand{\set}[1]{\lvert\lvert#1\rvert\rvert}

\newcommand{\lin}[1]{\m{lin}(\overline{#1})}
\newcommand{\shd}[1]{\m{shd}(\overline{#1})}
\newcommand{\slin}[1]{\m{slin}(\overline{#1})}
\newcommand{\plin}{\; \m{purelin}}

%% Operational Semantics Predicates
\newcommand{\proc}[2]{\m{proc}(#1, #2)}
\newcommand{\msg}[2]{\m{msg}(#1, #2)}
\newcommand{\ichan}[3]{\m{ichan}(#1, #2, #3)}
\newcommand{\ochan}[3]{\m{ochan}(#1, #2, #3)}
\newcommand{\unavail}[1]{\m{unavail}(#1)}

%% Semantics
\newcommand{\step}{\; \mapsto \;}
\newcommand{\zerostep}{\step^{0}}
\newcommand{\timed}[2]{\{#1\}_{#2}}
\newcommand{\unit}{M}
\newcommand{\Step}{\Longrightarrow}
\newcommand{\info}{\mapsto}
\newcommand{\andin}{\; \m{and} \;}
\newcommand{\minus}{\setminus}
\newcommand{\fresh}[1]{(#1 \text{ fresh})}
\newcommand{\eval}[1]{\Downarrow_{#1}}

%% Expressions Semantics
\newcommand{\val}{\; \m{val}}

%% Expressions
\newcommand{\lam}[3]{\lambda #1 : #2 . M_x}
\newcommand{\inl}[1]{l \cdot #1}
\newcommand{\inr}[1]{r \cdot #1}
\newcommand{\case}[3]{\m{case} \; #1 \; (l \hookrightarrow #2, r \hookrightarrow #3)}
\newcommand{\pair}[2]{\left\langle #1, #2 \right\rangle}
\newcommand{\projl}[1]{#1 \cdot l}
\newcommand{\projr}[1]{#1 \cdot r}
\newcommand{\match}[4]{\m{match} \; #1 \; ([] \rightarrow #2, #3 \rightarrow #4)}
\newcommand{\eproc}[3]{\{#1 \leftarrow #2 \leftarrow #3\}}


%% Proof Terms
\newcommand{\ecase}[3]{\m{case} \; #1 \; (#2 \Rightarrow #3)}
\newcommand{\ecasecf}[3]{\m{case^{cf}} \; #1 \; (#2 \Rightarrow #3)}
\newcommand{\erecvch}[2]{#2 \leftarrow \m{recv} \; #1}
\newcommand{\erecvchcf}[2]{#2 \leftarrow \m{recv^{cf}} \; #1}
\newcommand{\erecvshift}[1]{\m{shift} \leftarrow \m{recv} \; #1}
\newcommand{\esendch}[2]{\m{send} \; #1 \; #2}
\newcommand{\esendchcf}[2]{\m{send^{cf}} \; #1 \; #2}
\newcommand{\esendshift}[1]{\m{send} \; #1 \; \m{shift}}
\newcommand{\ewait}[1]{\m{wait} \; #1}
\newcommand{\ewaitcf}[1]{\m{wait^{cf}} \; #1}
\newcommand{\eclose}[1]{\m{close} \; #1}
\newcommand{\eclosecf}[1]{\m{close^{cf}} \; #1}
\newcommand{\fwd}[2]{#1 \leftarrow #2}
\newcommand{\fwdp}[2]{#1 \overset{+}{\leftarrow} #2}
\newcommand{\fwdn}[2]{#1 \overset{-}{\leftarrow} #2}
\newcommand{\esendl}[2]{#1.#2}
\newcommand{\esendlcf}[2]{(#1.#2)^{\m{cf}}}
\newcommand{\ecut}[4]{#1 \leftarrow #2 \leftarrow #3 \semi #4}
\newcommand{\ecutna}[3]{#1 \leftarrow #2 \semi #3}
\newcommand{\espawn}[4]{#1 \leftarrow #2 \leftarrow #3 = #4}
\newcommand{\procg}[3]{\m{proc}(#1, #2, \overline{#3})}
\newcommand{\edelay}[1]{\m{delay} \; (#1)}
\newcommand{\ewhen}[2]{\m{when?} \; (#1) ; #2}
\newcommand{\enow}[2]{\m{now!} \; (#1) ; #2}
\newcommand{\etick}[1]{\m{tick} \; (#1)}
\newcommand{\ework}[1]{\m{work} \; \{#1\}}
\newcommand{\eget}[2]{\m{get} \; #1 \; \{#2\}}
\newcommand{\epay}[2]{\m{pay} \; #1 \; \{#2\}}
\newcommand{\procdef}[3]{#3 \leftarrow #1 \; #2}
\newcommand{\procdefna}[2]{#2 \leftarrow #1}
\newcommand{\casedef}[1]{\m{case} \; #1}
\newcommand{\labdef}[1]{#1 \Rightarrow}
\newcommand{\wk}[1]{\m{work}(#1)}
\newcommand{\eassume}[2]{\m{assume} \; #1 \; \{#2\}}
\newcommand{\eassert}[2]{\m{assert} \; #1 \; \{#2\}}
\newcommand{\eimpos}[2]{\m{impossible} \; #1 \; \{#2\}}
\newcommand{\eif}[1]{\m{if} \; (#1)}
\newcommand{\ethen}{\; \m{then} \; }
\newcommand{\eelse}{\m{else} \; }

%% Type Constructors
\newcommand{\lolli}{\multimap}
\newcommand{\tensor}{\otimes}
\newcommand{\with}{\mathbin{\binampersand}}
\newcommand{\paar}{\mathbin{\bindnasrepma}}
\newcommand{\one}{\mathbf{1}}
\newcommand{\zero}{\mathbf{0}}
\newcommand{\bang}{{!}}
\newcommand{\whynot}{{?}}
\newcommand{\semi}{\, ; \,}
\newcommand{\ichoiceop}{\ensuremath{\oplus}}
\newcommand{\echoiceop}{\ensuremath{\with}}
\newcommand{\ichoice}[1]{\ichoiceop \{ #1 \}}
\newcommand{\echoice}[1]{\echoiceop \{ #1 \}}
\newcommand{\fuse}{\bullet}
\newcommand{\mi}[1]{\mbox{\it #1}}
\newcommand{\lunder}{\mathbin{\backslash}}
\newcommand{\tassertop}{?}
\newcommand{\tassumeop}{!}
\newcommand{\tassert}[1]{\; \tassertop\{#1\}. \;}
\newcommand{\tassume}[1]{\; \tassumeop\{#1\}. \;}
\newcommand{\arrow}{\rightarrow}
\newcommand{\product}{\times}

%% Functional Types
\newcommand{\tproc}[2]{\{#1 \leftarrow #2\}}

%% Types with Potential
\newcommand{\pot}[2]{#1^{#2}}
\newcommand{\lollipot}[1]{\overset{#1}{\lolli}}
\newcommand{\tensorpot}[1]{\overset{#1}{\tensor}}
\newcommand{\potfop}{\phi}
\newcommand{\potf}[1]{\potfop(#1)}
\newcommand{\mlab}{M^{\textsf{label}}}
\newcommand{\mchan}{M^{\textsf{channel}}}
\newcommand{\mcl}{M^{\textsf{close}}}
\newcommand{\mall}{M}
\newcommand{\mint}{M^{\textsf{internal}}}
\newcommand{\mval}{M^{\textsf{value}}}
\newcommand{\mshd}{M^{\textsf{share}}}
\newcommand{\ms}{M_s}
\newcommand{\mr}{M_r}
\newcommand{\entailpot}[2]{\xvdash{#1}{#2}}
\newcommand{\exppot}[1]{\xVdash{#1}}
\newcommand{\texp}{\Vdash}
\newcommand{\pexp}{\vdash}
\newcommand{\paypot}{\triangleright}
\newcommand{\getpot}{\triangleleft}
\newcommand{\tgetpot}[2]{\getpot^{\{#2\}} #1}
\newcommand{\tpaypot}[2]{\paypot^{\{#2\}} #1}
\newcommand{\bigeval}[3]{#1 \Downarrow #2 \mid #3}
\newcommand{\share}{\curlyveedownarrow}
\newcommand{\zpot}{\overline{0}}


%% Temporal Types
\newcommand{\entailpotcf}[1]{\underset{\m{cf}}{\entailpot{#1}}}
\newcommand{\entailspan}{\vdash}
\newcommand{\entailtype}{\vdash}
\newcommand{\fpot}{\; @ \;}
\newcommand{\pay}[1]{#1^{1}}
\newcommand{\sync}[1]{#1^{2}}
\newcommand{\spanpot}[1]{\langle \pay{#1}, \sync{#1} \rangle}
\newcommand{\ichoicepot}[2]{\overset{#1}{\ichoiceop} \{ #2 \}}
\newcommand{\echoicepot}[2]{\overset{#1}{\echoiceop} \{ #2 \}}
\newcommand{\tlist}[1]{\m{list}_{#1}}
\newcommand{\plist}[2]{\m{list}_{#1}^{#2}}
\newcommand{\tdia}[1]{\Diamond #1}
\newcommand{\tbox}[1]{\Box #1}
\newcommand{\tforall}[1]{\forall . #1}
\newcommand{\texists}[1]{\exists . #1}
\newcommand{\Dia}{\Diamond}
\newcommand{\Next}{\raisebox{0.3ex}{$\scriptstyle\bigcirc$}}
\newcommand{\tdelay}[2]{
    \IfEqCase{#2}{%
        {1}{\next{#1}}%
        % you can add more cases here as desired
    }[{\Next^{#2} (#1)}]%
}%
\newcommand{\sch}[1]{\tau(#1)}
\newcommand{\lforce}[2]{[#1]_L^{#2}}
\newcommand{\rforce}[2]{[#1]_R^{#2}}
\newcommand{\force}[2]{#1 \circ (#2)}

\setlength{\inferLineSkip}{4pt}
\newcommand{\blue}[1]{{\color{blue}#1}}
\newcommand{\red}[1]{{\color{red}#1}}
\newcommand{\green}[1]{{\color{green}#1}}
\newcommand{\tick}{\blue{\m{tick}}}
\newcommand{\delay}{\red{\m{delay}}}
\newcommand{\when}[1]{\red{\m{when?}\;#1}}
\newcommand{\now}[1]{\red{\m{now!}\;#1}}
\newcommand{\noww}{\red{\m{now!}}}
\newcommand{\whenn}{\red{\m{when?}}}
% \newcommand{\vdashi}{\vdash^{\!\!{}^i}}
\newcommand{\vdashi}{\vdash^{\!\!\scriptscriptstyle i}}
\newcommand{\tock}{`}


%% Indices
\newcommand{\indv}[1]{\overline{\{#1\}}}
\newcommand{\ind}[1]{\{#1\}}


%% Syntactic Sugar
\newcommand{\config}{\mathcal{C}}
\newcommand{\cost}[2]{\mathrm{cost}(\proc{#1}{#2})}
\newcommand{\tcost}[2]{\mathrm{cost}(#1 \mapsto #2)}
\newcommand{\ccost}[1]{\mathrm{cost}(#1)}
\newcommand{\dc}{\mathcal{D}}
\newcommand{\ec}{\mathcal{E}}
\newcommand{\ac}{\mathcal{A}}
\newcommand{\st}[1]{\m{store}_{#1}}
\newcommand{\stack}[1]{\m{stack}_{#1}}
\newcommand{\queue}[1]{\m{queue}_{#1}}
\newcommand{\mapper}[1]{\m{mapper}_{#1}}
\newcommand{\fdr}[1]{\m{folder}_{#1}}
\newcommand{\lt}[1]{\m{list}_{#1}}
%\newcommand{\bits}{\m{bits}}
\newcommand{\ctr}{\m{ctr}}
\newcommand{\trans}[2]{#1 \Longrightarrow #2}
\newcommand{\typetrans}[1]{\left\lvert{#1}\right\rvert}
\newcommand{\tree}{\m{tree}}
\newcommand{\bool}{\m{bool}}
\newcommand{\delayedbox}[1]{#1 \; \m{delayed}^{\Box}}
\newcommand{\delayeddia}[1]{#1 \; \m{delayed}^{\Diamond}}
\newcommand{\dom}[1]{\m{dom}(#1)}
\newcommand{\valid}[1]{#1 \; \m{valid}}
\newcommand{\invalid}[1]{#1 \; \m{invalid}}

%% Smart Contracts
\newcommand{\addr}{\m{addr}}
\newcommand{\ether}{\m{ether}}
\newcommand{\players}{\m{players}}
\newcommand{\lottery}{\m{lottery}}
\newcommand{\tint}{\m{int}}
\newcommand{\ballot}{\m{ballot}}
\newcommand{\tbool}{\m{bool}}
\newcommand{\lc}{\tlist{\m{coin}}}
\newcommand{\auction}{\m{auction}}
\newcommand{\object}{\m{object}}

%% Typing Judgments for Servers and Clients
\newcommand{\sentailpot}[1]{\prescript{}{S}{\xvdash{#1}} \hspace{2pt}}
\newcommand{\centailpot}[1]{\prescript{}{C}{\xvdash{#1}} \hspace{2pt}}


%% Sharing
\newcommand{\down}{\downarrow^{\m{S}}_{\m{L}}}
\newcommand{\up}{\uparrow^{\m{S}}_{\m{L}}}
\newcommand{\eacquire}[2]{#1 \leftarrow \m{acquire} \; #2}
\newcommand{\eaccept}[2]{#1 \leftarrow \m{accept} \; #2}
\newcommand{\erelease}[2]{#1 \leftarrow \m{release} \; #2}
\newcommand{\edetach}[2]{#1 \leftarrow \m{detach} \; #2}


%% Subtyping
\newcommand{\subt}[2]{#1 \leq #2}
\newcommand{\wsubt}{ <: }
\newcommand{\qsubt}[1]{\overset{#1}{\leq}}


%% Latex
%\newtheorem{theorem}{Theorem}
%\newtheorem{definition}{Definition}
%\newtheorem{lemma}{Lemma}
%\newtheorem{cor}{Corollary}


%%Global Semantics
\newcommand{\sinfer}[3]
{\inferrule
{#3}
{#2}
#1}
\newcommand{\enq}[2]{\m{enq}(#1, #2)}
\newcommand{\deq}[1]{\m{deq}(#1)}
\newcommand{\nil}{[]}
\newcommand{\elem}[1]{[#1]}


%% Channel typing
\newcommand{\eqdef}{\cong}


%% Types to Processes
\newcommand{\typeProc}[2]{#1 \Longrightarrow #2}

%% AARA
\newcommand{\abs}[1]{\left\lvert #1 \right\rvert}
\newcommand{\bin}[1]{(#1)_2}
% \newcommand{\ceil}[1]{\left\lceil #1 \right\rceil}
\newcommand{\bigO}[1]{\mathcal{O}(#1)}
% \newcommand{\ignore}[1]{\textcolor{red}{#1}}
% new \oset macro
\makeatletter
\newcommand{\oset}[3][-0.7ex]{%
  \mathrel{\mathop{#3}\limits^{
    \vbox to#1{\kern-2\ex@
    \hbox{$\scriptstyle#2$}\vss}}}}
\makeatother
\newcommand{\monus}{\oset{.}{-}}

%% Indexed Types
\newcommand{\cons}{\mathcal{C}}
\newcommand{\vars}{\mathfrak{v}}
\newcommand{\Vars}{\mathcal{V}}
\newcommand{\Cons}{\mathcal{C}}
\newcommand{\Tokens}{\Gamma}
\newcommand{\K}{\gamma}
\newcommand{\Tokentypes}{\mathcal{K}}
\newcommand{\VTokens}{\mathcal{V}}
\newcommand{\TokSig}{\mathcal{S}}
\newcommand{\exchange}[3]{#1 \overset{#2}{\longrightarrow} #3}
\newcommand{\GlobalF}{\ensuremath{\mathfrak{f}}\xspace}
\newcommand{\GlobalP}{\mathfrak{p}}
\newcommand{\depth}{\mathfrak{d}}

%% Two Counter Machines
\newcommand{\ins}{\iota}
\newcommand{\tcm}{\mathcal{M}}
\newcommand{\inc}[1]{\m{inc}(#1)}
\newcommand{\dec}[1]{\m{dec}(#1)}
\newcommand{\goto}{\m{goto}}
\newcommand{\zeroc}[1]{\m{zero}(#1) ?}
\newcommand{\halt}{\m{halt}}

%% UC stuff
\newcommand{\fcomm}{\mathcal{F}_{\msf{comm}}}
%\newcommand{\B}[1]{\colorbox{gray}{#1}}
%\newcommand{\hlc}[2][yellow]{{%
%    \colorlet{foo}{#1}%
%        \sethlcolor{foo}\hl{#2}}%
%        }
%\newcommand{\hlcyan}[1]{{\sethlcolor{cyan}\hl{#1}}}
%\newcommand{\B}[1]{\hlc[pink]{#1}}
\definecolor{airforceblue}{rgb}{0.36, 0.54, 0.66}
\newcommand{\B}[1]{{\color{airforceblue}{#1}}}
\newcommand{\wt}{\circled{w}}

%% TODO
\newcommand{\ankush}[1]{\textcolor{red}{\textbf{Ankush: #1}}}


%%% Local Variables:
%%% mode: plain-tex
%%% TeX-master: "pldi19"
%%% End:


\author{\IEEEauthorblockN{Anon}
\IEEEauthorblockA{\textit{Nowhere}}
}

\maketitle

\begin{abstract}
The goals of this work are to introduce liveness and eventual delivery into the asynchronous communication model of UC that takes advantage of its polynomial time notion.
Our construction wraps around individual ITMs and provides a common interface for protocols and functioalities to execute code ``eventually''.
The design allows global access to the scheduler without the cumbersome GUC model. Instead we use a simplified version of GUC that fits within the standard UC definition (either we rely on the global subroutines or we design our own?).
Finally we showcase our modelling through an case study of a common-coin binary agreement protocol in an implementation of UC and our model using the Haskell Type system. 
\end{abstract}

\begin{IEEEkeywords}
component, formatting, style, styling, insert
\end{IEEEkeywords}

\section{Introduction}
In this work we present a programming language design based on the Universal (UC) Composability framework from cryptography.

We build on existing work, ILC, which also aims to be a programming language for UC, but start from a more powerful language, called Nomos, which incorpoates session types and work aware resource types and has been previously used for  smart contracts and distributed applications.
Both of the features of Nomos turn out to have 

Second, Nomos features a notion of Work-aware types. This is useful for capturing the notion of “locally polynomial runtime.” This allows us to model UC more faithfully than any prior work to date

As a starting point, we build a language that merges types rules from ILC into Nomos. The main design idea of ILC is that it is uses static typing rules to encode the requirements of the Interacting Turing Machines (ITMs) model, a model that is uniquely associated with UC. The ILC rules roughly ensure that simulations of the language can be carried out by probabilistic Turing machines, which is necessary for reduction to computationally hard problems, required for cryptographic security proofs. The rules from ILC are compatible with session types, so it turns out to be straightforward to merge these into nomos. The result provides benefits associated with session types, namely that it avoids potential errors from internally-inconsistent programs.

   Beyond just session types, the Work-aware component of Nomos allows us to tackle a fundamental challenge in defining a programming language for UC that ILC (and all other related work) left unfulfilled, which is to express the notion of polynomially runtime.
   
   The challenge of “polynomial runtime” in UC is that individual processes must be judged as polynomial, but when eveluated in context with other concurrently running process it is difficult to assign blame.
       The current best way to define polynomial runtime, found in the 2019 and later version of UC, is based a concept of ``import tokens.''
   We identify how to relate the “Potential” concept from Nomos, to the import tokens from UC. 
	The result is a deep connection between session type semantics and the formal foundation of UC.
	The Preservation theorem we prove associated with our type system and operational semantics proves the following: 
well-typed terms in Nomos UC are “locally polynomial time”, in the sense required of UC, meaning they do not take more steps that some polynomial function T(N) of the net number of import tokens it has received.

In addition, our language has other benefits.
The Progress theorem is useful because it gives some evidence that ideal functionalities and protocols encoded in Nomos UC cannot get stuck. Together helps confirm that the process halts in polynomial time.
TODO: Give an example of a bad machine ruled out by progress guarantee.

\ignore{
Carries forward the same metatheory guarantees as ILC. Namely: if a process terminates, then it depends only on the random coins (unlike Pi calculus, including Session-type pi calculus). Thus simulating the execution of a Nomos UC experiment can be carried out by a probabilistic polynomial time Turing machine (PPT). This is essential in UC for reduction to computationally hard problems.
}

\ignore{
The Universal Composability Framework~\cite{uc} is the popular and widely-used framework for modelling the security of cryptographic and distributed protocols.
Its novel contribution compared to other frameworks is that it provides a very strong notion of security: a UC-secure protocol is proved to be secure even when composed with arbitrary other protocols running concurrently.
This constrasts with other, property-based notions of security~\todo{need to get some citations here}.

Analyzing large and complex protocols is a difficult task made easier by UC's ideal functionality abstraction. 
However, despite this additional modularity, UC proofs and models still tend to be very complex, unwieldy, and difficult to understand.
These issues are exacerbated when new communication models are added on top of UC~\cite{katz, etc}.
Therefore, we propose a two-fold solution: a new construction for modelling different communication models that removes all model-specific code from protocols and functionalities, and an implementation of the UC framework in the Nomos language. 
}



\section{Related Works}
There are many works that attempt to formalize the UC framework with an implementation for protocol analysis and proof generation.

One of the most relevant works to our own is EasyUC~\cite{easyuc}. 
EasyUC uses the existing EasyCrypy~\cite{easycrypt} toolset to model UC protocols and mechanize proof generation. 
It departs from EasyCrypt's limtations to game-based security definitions (lacking simulation-based composition).
However, it still lacks a notion of polynomial time. The authors, themselves, mentions that it can't detect deviant behavior like the adversary and functionality passing messages between each other indefinitely. 
Our use of the import mechainsm and session types let us reason about polynomial time in the sytem of ITMs encompassed by \msf{execUC} but also locally for \textit{open} terms. 
Furthermore, import in NomosUC lets us have guarantees of termination as well by the polyomial import constraints added to UC by Canetti et al.

Liao et al. introduce executable UC through a new process calculus called ILC~\ref{ilc}.
This work adds some notion of polynomial time although it proves to be too restrictive. 
It results from the fact that poly-time can only be reasoned about for \textit{closed} terms like a full UC execution.
In order to reason about polynomial time for a particular protocol $\pi$ we must reason over all possible other terms that connect to $\pi$ and require that it is polynomial in all such cases.
A simple ping-response server can not be proven to by poly-time in this definition for a deviant other ITM that connects to $\pi$. 
In Nomos, however, as mentioned above, open terms are limited to polytime regardless of the connected other terms because of the import mechanism and the NomosUC type system that guarantees termination. 

Other works that rely in symbolic modelling of cryptography, for example, SymbolicUC~\cite{symbolicuc}, are subsumed by the above ILC work and similarly lack any polynomial time notion. 
\todo{Say something about $\pi$-calculus with probabilistic polynomial time extensions}.


To the best of our knowledge, this is the first work to deal with the new import notion of polynomial time introduced to the UC framework in 2018.
A few other works refer to the import mechanism, but it is restricted to simply defining the import a protocol is given.
	
%easyUC:
%* can not dynamially create new instances of parties/functionalities must statically determine the number of functionalities/parties spawned
%* 
%
%
%The work of Liao et al.~\ref{ilc} is the closest to our own
%It proposes a new process calculus called ILC and a concrete implementation of the UC framework.
%The type system it introduces ensures that correctly types programs can be represented as ITMs.
%However, one drawback of the ILC work is that its polynomial time representation 
%
%
%The EasyUC approach uses the existing EasyCrypt toolset to implement model UC protocols and mechanize the generate of UC-security proofs and proofs of secure composition.
%This work aim considerably higher than our work in actually attempting to generate proofs for their protocols. 
%However, this work falls short in being able to capture any notion of resource bound computation whereas we are able to make guarnatees about polynomial bounds on our system of ITMs and even guarantee termination of programs through our realization of the import mechanism.
%The EasyUC work accepts that not even infinite loops of communication can be caught and, therefore, termination of protocols can't be guarnateed either whereas the import mechanism in Nomos ensures that such infinite loops can not stall protocol progress.

%Another work similar to our own is the Symbolic UC by B\"{o}hl and Unruh.
%This works uses an applied $\pi$-calculus to symbolically model UC protocols and analyze them.
%Similar to the EasyUC work, the goals of this work are somewhat orthogonal to the our own goals.
%However, Symbolic UC does attempt to create an implementatio of UC using the $\pi$-calculus however neglects to address any issues of polynomial runtime.
%
%Perhaps the closes work to our own is that of Liao et al.~\cite{ilc} that builds an executable version of the UC framework by introducing a new process calculus called ILC.
%ILC introduces a type system that guarantees that ILC programs (i.e. functionalities, protocols, etc) can be expressed as ITMs as in the UC framework.
%However, one drawback of ILC is that it's notion of polynomial time ends up being too restrictive.
%In ILC only closed terms without any unbonded variables, i.e. and entire UC exection of a system of ITMs, can be shown to be polynomial in their definition of polynomial time.
%Proving polynomial time for open terms, such as a protocol $\pi$, requires reasoning over all possible contexts in which the protocol could exist however such a definition of polynomial time becomes too restrictive where even a simple ping-responde server protocol wouldn't be considered polynomial time.


\section{Background} \label{sec:background}
\subsection{Universal Composability}
The universal composability framework~\cite{uc} proposes a new framework for proving the security of cryptographic and distributed protocol.
Compared to previous works, the UC framework provides a stronger notion of security where protocols that are UC-secure are secure even when composed with arbitrary other protocols running concurrently. 

Such a strong notion of security is achieved through the real-ideal world paradigm.
The ideal world encompasses an ideal implementation of a protocol, called the \textit{ideal functionality} $\mathcal{F}$, which acts as a trusted third party that caputures all the desired security properties.
The ideal functionality is usually a simple definition making it trivial to prove its security properties.
The real world, on the other hand, consists of parties running an actual protocol, $\pi$, against a real adversary.

Security proofs in UC involve creating a simulator $\mathcal{S}$ in the ideal world that can simulate every potential attack on a real protocol in the real world.
If $\mathcal{S}$ can make the two worlds indistinguishable for any real world adversary $\mathcal{A}$ for all distinguishing environments $\mathcal{Z}$, then we say the protocol $\pi$ UC-emulates the ideal functionality $\mathcal{F}$.
Indistinguishability of the two worlds to any $\mathcal{Z}$ implies that the protocol $\pi$ must exhibit the same security properties as the ideal functionality $\mathcal{F}$ otherwise there should be sobe distinguishing environment. 
More formally, indistinguishability is stated:

$$ \text{EXEC}_{\mathcal{F},\mathcal{S},\Environment} \approx \text{EXEC}_{\pi,\mathcal{A},\Environment} $$

\paragraph{GUC-Framework}


\subsection{The Import Mechanism}
A notion of resource-bound computation is necessary for the UC framework to reason about computationally efficient algorithms as well as the capabilities of ITIs under a particular resource constraint.
Often we would like to reason about adversarial capabilities under such constraints and perform efficient transformations (transforming an adversary into a simulator).

Previous definitions of polynomial-time computation have taken the form of bounding the computation of an ITI by some polynomial $T$:
given an input of length $n$ the machine $\mu$ halts within $T(n)$ steps.
However, using the length of the inputs to the machine as $n$, in this case leads to an infinite runs problems identified by Canetti~\cite{uc}.
Machines that are locally $T(n)$-bounded are able to spawn other machines to the point that an infinite chain of such machines can be spawed where each is locally $T$-bounded, but the whole system of machines can not be bounded by any polynomial $T$.

Therefore, a new notion of $n$ was needed. The UC paper defines an import mechanism where the first ITI, the environment, is spawned with a polynomially amount of import which can be thought of as tokens or coins.
The environment can then activate other ITIs with some import tokens allowing them to run for $T(n')$ computationsl steps for some $T$ and some amount of import $n'$.
In this new definition, an ITI that is $T$-bounded takes at most $T(n')$ steps where $n'$ is the difference between the import it has received from incoming messages and outgoing import it's given to other machines.
This definition therefore suffices to ensure that every machine is locally bounded by some polynomial but also guarantees that the system of ITMs is bounded by a polynomial number of import tokens. 


\section{Nomos UC}
Cryptographic protocols are, well, protocols and therefore, follow
a predefined communication pattern.
Our key innovation is to represent such communication protocols using
\emph{session types}.

\paragraph*{\textbf{Example Protocol}}
As an example, consider the two-phase commitment protocol.
The ideal functionality of this protocol consists of a \emph{sender} $S$
and \emph{receiver} $R$ connected to a trusted third-party, which we
name $\fcomm$.
The protocol initiates with $S$ sending a $\mb{commit}$ message to $\fcomm$
indicating its intent to \emph{commit} to a bit.
Next, $S$ sends this committed bit to $\fcomm$.
After receiving the committed bit, $\fcomm$ sends a $\mb{commit}$ message
to $R$ indicating that a bit has been committed to, but does not reveal
this bit to $R$.
At a later time, $S$ sends an $\mb{open}$ message to $\fcomm$ expressing
that $S$ wishes to reveal the secret bit to $R$.
Receiving this message, $\fcomm$ in turn sends an $\mb{open}$ message
to $R$ followed by this bit.
The protocol concludes with each party (process) terminating.

In the session-typed setting, we use typed channels to connect two
parties. For instance, the channel connecting $S$ to $\fcomm$
has the session type $\m{sender}$ defined as
\[
  \mi{stype} \; \m{sender} = \ichoice{\mb{commit} : \m{bit} \product
  \ichoice{\mb{open} : \one}}
\]
The type constructor $\ichoice$ denotes an \emph{internal choice}
(here with only one choice) dictating that $S$ must send the
$\mb{commit}$ message to $\fcomm$.
Next, we use the type constructor $\product$ to denote that $S$
sends a value of type $\m{bit}$ ($\m{bit} \product \ldots$).
We then use the $\ichoiceop$ constructor again enforcing
that $S$ sends $\mb{open}$ to $\fcomm$.
Finally, the type $\one$ denotes termination, indicating that
$S$ will send $\m{close}$ message to $\fcomm$.

Analogously, the channel connecting $R$ and $\fcomm$ has
type $\m{receiver}$ defined as
\[
  \mi{stype} \; \m{receiver} = \echoice{\mb{commit} : 
  \echoice{\mb{open} : \m{bit} \arrow \one}}
\]
Type constructor $\echoice$ represents \emph{external choice}
which is the dual to internal choice.
It prescribes that $R$ must receive a $\mb{commit}$ message from $\fcomm$,
followed by an $\mb{open}$ message (using another $\echoiceop$ constructor).
$R$ must then receive a bit using the $\arrow$ constructor (dual to
$\product$).
Finally, the session terminates as indicated by type $\one$.

Protocols expressed via session types are strictly enforced by
process definitions.
As an illustration, consider the $\fcomm$ process that is connected
to both $S$ and $R$.
The process declaration is written as
\begin{lstlisting}[basicstyle=\small\ttfamily]
decl F_comm :
  (S : sender), (R : receiver) |- (F : 1)
\end{lstlisting}
Here, $\fcomm$ is the name of the process, and $S$ and $R$ are the names
of channels \emph{used} by $\fcomm$, while $F$ is the channel \emph{provided}
by $\fcomm$.
Every session-typed process provides a unique channel while acting as a client
of a non-negative number of channels.
The used channels with their types are written to the left of the turnstile
($\vdash$) while the offered channel and type are written on the right.
This is analogous to function definitions where used channels correspond to
arguments, while offered channel corresponds to the result.

The $\fcomm$ process is defined as follows:
\begin{lstlisting}[basicstyle=\small\ttfamily, numbers=left,xleftmargin=2em]
proc F <- F_comm S R =
  case S (
    commit => b = recv S ;
              R.commit ;
              case S (
                open => R.open ;
                        send R b ;
                        wait S ; wait R ;
                        close F ) )
\end{lstlisting}
The process first case analyzes on channel $S$ branching on the
message received.
Since there is only one choice $\mb{commit}$, we only have one
branch in the definition.
$\fcomm$ then receives the bit $b$ (line 3) on $S$, followed by sending the
commit message on channel $R$ (line 4).
Once $\fcomm$ receives the $\mb{open}$ message on $S$, it sends the
$\mb{open}$ message on $R$ (line 6), followed by the bit $b$ (line 7).
Finally, the process waits for channels $S$ and $R$ to close (line 8),
followed by ultimately closing the channel $F$ (line 9).


\begin{figure*}[!ht]
  \begin{lstlisting}[basicstyle=\small\ttfamily,frame=single]
stype sender = +{ commit : bit ^ +{ open : 1 } } 

stype receiver = &{ commit : &{ open : bit -> 1 } }

decl F_comm : (S : sender), (R : receiver) |- (F : 1)

def F <- F_comm S R =
  case s (
    commit => b = recv S ;
              R.commit ;
              case S (
                open => R.open ;
                        send R b ;
                        wait S ; wait R ; close F ) )
\end{lstlisting}

  \caption{The $\mathcal{F}_{\msf{comm}}$
  commitment ideal functionality in Nomos.
  The types for the sender and receiver channel define what inputs they
  can give to the functionality and what messsages are sent from the
  functionality back to the receiver.}
  \label{fig:nomos:commitment}
  \end{figure*}

\subsection{Formal Description of Nomos}
Nomos internally relies on session types to express and enforce protocols
of interaction between different parties.
Session types are a type discipline for communication-centric programming
based on message passing via channels.
The underlying base system of session types is derived from a Curry-Howard
interpretation~\cite{Caires10concur,Caires16mscs} of intuitionistic linear logic
\cite{Girard87tapsoft}. The key idea is that an intuitionistic linear sequent
$A_1, A_2, \ldots, A_n \vdash A$
is interpreted as the interface to a process expression $P$. We label each of the
antecedents with a channel name $x_i$ and the succedent with channel name $z$.
The $x_i$'s are \emph{channels used by} $P$ and $z$ is the \emph{channel provided by} $P$.
\begin{center}
  \begin{minipage}{0cm}
  \begin{tabbing}
  $x_1 : A_1, x_2 : A_2, \ldots, x_n : A_n \vdash P :: (z : C)$
  \end{tabbing}
  \end{minipage}
\end{center}
The resulting judgment formally states that process $P$ provides a service of
session type $C$ along channel $z$, while using the services of session types $A_1,
\ldots,A_n$ provided along channels $x_1, \ldots, x_n$ respectively.
We abbreviate the antecedent of the sequent by $\Delta$.

The standard connectives of intuitionistic linear logic give rise
to the type constructors in session types.
In this work, we restrict ourselves 
In addition to the type constructors arising
from the connectives of intuitionistic linear logic, we have type names,
indexed by a
sequence of arithmetic expressions $V \indv{e}$, existential and
universal quantification over natural numbers ($\texists{n} A$,
$\tforall{n} A$) and existential and universal constraints
($\tassert{\phi} A$, $\tassume{\phi} A$).  We write $i$ for constant
and $n$ for variable natural numbers.
\[
  \begin{array}{lrcl}
    \mbox{Types} & A, B & ::= & \ichoice{\ell : A_\ell}_{\ell \in L}
    \mid \echoice{\ell : A_\ell}_{\ell \in L} \\
                 & & \mid & A \tensor B \mid A \lolli B \mid \one \mid V \indv{e} \\
                 & & \mid & \tassert{\phi} A \mid \tassume{\phi} A
                            \mid \texists{n} A \mid \tforall{n} A \\[0.5em]
    \mbox{Arith. Exps.} & e & ::= & i \mid e + e \mid e - e \mid i \times e \mid n \\[0.5em]
    \mbox{Arith. Props.} &
    \phi & ::= & e = e \mid e > e \mid \top \mid \bot
                 \mid \phi \land \phi \\
    & & \mid &  \phi \lor \phi \mid \lnot \phi \mid \texists{n}\phi \mid \tforall{n} \phi \\[0.5em]
    \mbox{Procs} & P, Q & ::= & \esendl{x}{k} \semi P \mid \ecase{x}{l}{P}_{l \in L} \\
                & & \mid & \esendch{x}{y} \semi P \mid \erecvch{x}{y} \semi P \\
                & & \mid & \eclose{x} \mid \ewait{x} \semi P \\
                & & \mid & \fwd{x}{y} \mid \ecut{x}{f}{y}{P} \\
                & & \mid & \eassert{x}{\phi} \semi P \mid \eassume{x}{\phi} {\semi} P
  \end{array}
\]
Our implementation does not support type polymorphism but it is convenient in
some of the examples.  We therefore allow definitions such as
$\queue{A}[n] = \ldots$ and interpret them as a family of definitions,
one for each possible type $A$.

The typing judgment has the form of a sequent
\begin{center}
  \begin{minipage}{0cm}
  \begin{tabbing}
  $\vars \semi \cons \semi \D \vdash_\Sg P :: (x : A)$
  \end{tabbing}
  \end{minipage}
\end{center}
where $\vars$ are index variables $n$, $\cons$ are constraints over
these variables expressed as a single proposition,
% $\psi$\footnote{\it should we adopt maybe $\psi$ for the constraints?}
$\D$ are the linear antecedents $x_i : A_i$, $P$ is a process
expression, and $x : A$ is the linear succedent. We propose and maintain
that the $x_i$'s and $x$ are all distinct, and that all free index
variables in $\cons$, $\D$, $P$, and $A$ are contained among $\vars$.
Finally, $\Sigma$ is a fixed signature containing type and process
definitions (explained in Section~\ref{subsec:base}) Because it is
fixed, we elide it from the presentation of the rules.  In addition we
write $\vars \semi \cons \proves \phi$ for semantic entailment
(proving $\phi$ assuming $\cons$) in the
constraint domain where $\vars$ contains all arithmetic variables in
$\cons$ and $\phi$.
Table~\ref{tab:language} overviews the session types their associated
process terms, their continuation (both in types and terms) and operational description.
% Figure~\ref{fig:basic-typing} describes selected
% typing rules (ignore the premises and annotation on the turnstile
% marked in blue, introduced in Section~\ref{subsec:ergo}) leaving the complete
% set of rules to Appendix~\ref{app:formal}.

We formalize the operational semantics as a system of \emph{multiset rewriting
rules}~\cite{Cervesato09SEM}. We introduce semantic objects $\proc{c}{P}$
and $\msg{c}{M}$ which mean that process $P$ or message $M$ provide
along channel $c$.
A process configuration is a multiset of such objects, where any two
channels provided are distinct
(formally described in Section~\ref{subsec:soundness}).

\subsection{Basic Session Types}\label{subsec:base}
In this subsection, we review the syntax and semantics for the basic
session type operators ($\with$, $\oplus$, $\tensor$, $\lolli$ and $\one$).
A summary of the corresponding process terms and intuitive explanation
for semantics is provided in Table~\ref{tab:language}.

\paragraph{\textbf{External Choice}}
The \emph{external choice} type constructor
$\echoice{\ell : A_{\ell}}_{\ell \in L}$ is an $n$-ary labeled
generalization of the additive conjunction $A \with B$. Operationally,
it requires the provider of
$x : \echoice{\ell : A_{\ell}}_{\ell \in L}$ to branch based on the
label $k \in L$ it receives from the client and continue to provide
type $A_{k}$. The corresponding process term is written as
$\ecase{x}{\ell}{P}_{\ell \in L}$.  Dually, the client must send one
of the labels $k \in L$ using the process term
$(\esendl{x}{k} \semi Q)$ where $Q$ is the continuation.
\begin{mathpar}
\infer[{\with}R]
{\vars \semi \cons \semi \D \vdash \ecase{x}{\ell}{P_\ell}_{\ell \in L} ::
(x : \echoice{\ell : A_\ell}_{\ell \in L})}
{(\forall \ell \in L)
 & \vars \semi \cons \semi \D \vdash P_\ell :: (x : A_\ell)}
\and
\infer[{\with}L]
{\vars \semi \cons \semi \D, (x : \echoice{\ell : A_\ell}_{\ell \in L}) \vdash
(\esendl{x}{k} \semi Q) :: (z : C)}
{(k \in L) & \vars \semi \cons \semi \D, (x : A_k) \vdash Q :: (z : C)}
\end{mathpar}
Communication is asynchronous, so that the client
$\esendl{c}{k} \semi Q$ sends a message $k$ along $c$ and continues as $Q$
without waiting for it to be received. As a technical device to ensure that
consecutive messages on a channel arrive in order, the sender also creates a
fresh continuation channel $c'$ so that the message $k$ is actually represented
as $(\esendl{c}{k} \semi \fwd{c}{c'})$ (read: send $k$ along $c$ and continue along
$c'$). When the message $k$ is received along $c$, we select branch $k$ and
also substitute the continuation channel $c'$ for $c$.
Rules ${\with}S$ and ${\with}C$ below describe the operational behavior of the
provider and client respectively $\fresh{c'}$.

\begin{tabbing}
$({\with}S) : \m{proc}(d, c.k \semi Q) \;\mapsto\; \m{msg}(c', c.k \semi c' \leftarrow c),
\m{proc}(d, Q[c'/c])$ \\
$({\with}C):$ \= $\m{proc}(c, \m{case}\;c\;(\ell \Rightarrow Q_\ell)_{\ell \in L}),$\\
\> $\m{msg}(c', c.k \semi c' \leftarrow c)
\;\mapsto\; \m{proc}(c', Q_k[c'/c])$
\end{tabbing}

The \emph{internal choice} constructor
$\ichoice{\ell : A_{\ell}}_{\ell \in L}$ is the dual of external
choice requiring the provider to send one of the labels $k \in L$ that
the client must branch on.
\begin{mathpar}
  \infer[{\oplus}R]
    {\vars \semi \cons \semi \D \vdash (\esendl{x}{k} \semi P) :: (x : \ichoice{\ell : A_\ell}_{\ell \in L})}
    {(k \in L) & \vars \semi \cons \semi \D \vdash P :: (x : A_k)}
  \and
  \infer[{\oplus}L]
    {\vars \semi \cons \semi \D, (x : \ichoice{\ell : A_\ell}_{\ell \in L}) \vdash
    \ecase{x}{\ell}{Q_\ell}_{\ell \in L} :: (z : C)}
    {(\forall \ell \in L) &
      \vars \semi \cons \semi \D, (x : A_\ell) \vdash Q_\ell :: (z : C)}
\end{mathpar}
This dual constructor reverses the role of the provider and client.
The provider $(\esendl{x}{k} \semi P)$ of $x : \ichoice{\ell : A_\ell}_{\ell \in L})$
sends the label $k$ along $x$ and continues to provide $x : A_k$.
Correspondingly, the client branches on the label received using channel
$x : A_\ell$ in branch $\ell$ with process term $Q_\ell$.
The rules of operational semantics (${\oplus}S, {\oplus}C$) are exact dual
of ${\with}S$ and ${\with}C$ and omitted for brevity.

\paragraph{\textbf{Channel Passing}}
The \emph{tensor} operator $A \tensor B$ prescribes that the provider of
$x : A \tensor B$
sends a channel $y$ of type $A$ and continues to provide type $B$. The
corresponding process term is $\esendch{x}{y} \semi P$ where $P$ is
the continuation.  Correspondingly, its client must receives a channel
using the term $\erecvch{x}{y} \semi Q$, binding it to variable $y$
and continuing to execute $Q$.
\begin{mathpar}
  \infer[{\tensor}R]
    {\vars \semi \cons \semi \D, (y : A) \vdash (\esendch{x}{y} \semi P) :: (x : A \tensor B)}
    {\vars \semi \cons \semi \D \vdash P :: (x : B)}
  \and
  \infer[{\tensor}L]
    {\vars \semi \cons \semi \D, (x : A \tensor B) \vdash (\erecvch{x}{y} \semi Q) :: (z : C)}
    {\vars \semi \cons \semi \D, (y : A), (x : B) \vdash Q :: (z : C)}
\end{mathpar}
Operationally, the provider $\esendch{c}{d} \semi P$ sends the
channel $d$ and the continuation channel $c'$ along $c$ as a message and
continues with executing $P$. The client receives the channel $d$ and continuation
channel $c'$ and substitutes $d$ for $x$ and $c'$ for $c$.
\begin{tabbing}
$({\tensor}S) : \m{proc}(c, \m{send}\; c\; d \semi P) \;\mapsto\; $\=
$\m{proc}(c', P[c'/c]),$\\
\>$\m{msg}(c, \m{send}\; c\; d \semi c \leftarrow c')$ \\
$({\tensor}C) : $ \= $\m{msg}(c, \m{send}\; c\; d \semi c \leftarrow c'),$\\
\>$\m{proc}(e, x \leftarrow \m{recv}\; c \semi Q)
\;\mapsto\; \m{proc}(e, Q[c', d/c, x])$

\end{tabbing}
The dual operator $A \lolli B$ allows the provider to receive a
channel of type $A$ and continue to provide type $B$. The client
of $A \lolli B$, on the other hand, sends the channel of type $A$
and continues to use $B$.
\begin{mathpar}
  \infer[{\lolli}R]
    {\vars \semi \cons \semi \D \vdash (\erecvch{x}{y} \semi P) :: (x : A \lolli B)}
    {\vars \semi \cons \semi \D, (y : A) \vdash P :: (x : B)}
  \and
  \infer[{\lolli}L]
    {\vars \semi \cons \semi \D, (x : A \lolli B), (y : A) \vdash (\esendch{x}{y} \semi Q) :: (z : C)}
    {\vars \semi \cons \semi \D, (x : B) \vdash Q :: (z : C)}
\end{mathpar}

\paragraph{\textbf{Termination}}
The type $\one$, the multiplicative unit of linear logic,
indicates \emph{termination} requiring that the provider send a
\emph{close} message followed by terminating the communication.
Linearity enforces that the provider not use any channels.
\begin{mathpar}
  \infer[{\one}R]
    {\vars \semi \cons \semi \cdot \vdash (\eclose{x}) :: (x : \one)}
    {}
  \and
  \infer[{\one}L]
    {\vars \semi \cons \semi \D, (x : \one) \vdash (\ewait{x} \semi Q) :: (z : C)}
    {\vars \semi \cons \semi \D \vdash Q :: (z : C)}
\end{mathpar}
Operationally, the provider waits for the closing message, which
has no continuation channel since the provider terminates.
\begin{tabbing}
$({\one}S) : \m{proc}(c, \m{close}\; c) \;\mapsto\; \m{msg}(c, \m{close}\; c)$ \\
$({\one}C) : $ \= $\m{msg}(c, \m{close}\; c),
\m{proc}(d, \m{wait}\; c \semi Q) \;\mapsto\; \m{proc}(d, Q)$
\end{tabbing}

\paragraph{\textbf{Forwarding}}
A process $\fwd{x}{y}$ identifies the channels $x$ and $y$ so that any
further communication along either $x$ or $y$ will be along the unified
channel. Its typing rule corresponds to the logical rule of identity.
\begin{mathpar}
  \infer[\m{id}]
    {\vars \semi \cons \semi y : A \vdash (\fwd{x}{y}) :: (x : A)}
    {}
\end{mathpar}
Operationally, a process $\fwd{c}{d}$ \emph{forwards} any message M
that arrives on $d$ to $c$ and vice-versa. Since channels are used
linearly, the forwarding process can then terminate, ensuring proper
renaming, as exemplified in the rules below.
\begin{tabbing}
$(\m{id}^+C) : $ \= $\m{msg}(d, M),
\m{proc}(c, c \leftarrow d) \;\mapsto\; \m{msg}(c, [c/d]M)$ \\
$(\m{id}^-C) : $ \= $\m{proc}(c, c \leftarrow d),
\m{msg}(e, M(c)) \;\mapsto\; \m{msg}(e, [d/c]M(c))$
\end{tabbing}
We write $M(c)$ to indicate that $c$ must occur in message $M$ ensuring that $M$ is the
sole client of $c$.

\paragraph{\textbf{Process Definitions}}
Process definitions have the form
$\D \vdash f \indv{n} = P :: (x : A)$ where $f$ is the name of the
process and $P$ its definition. In addition, $\overline{n}$ is a
sequence of arithmetic variables that $\D$, $P$ and $A$ can refer to.
All definitions are collected in a fixed global signature $\Sg$.
For a \emph{well-formed signature}, we
require that $\overline{n} \semi \top \semi \D \vdash P :: (x : A)$
for every definition, thereby allowing
definitions to be mutually recursive. A new instance of a defined
process $f$ can be spawned with the expression
$\ecut{x}{f \indv{e}}{\overline{y}}{Q}$ where $\overline{y}$ is a
sequence of channels matching the antecedents $\D$ and $\indv{e}$ is a
sequence of arithmetic expression matching the variables
$\indv{n}$. The newly spawned process will use all variables in
$\overline{y}$ and provide $x$ to the continuation $Q$.
\begin{mathpar}
  \inferrule*[right=$\m{def}$]
  {\overline{y':B} \vdash f \indv{n} = P_f :: (x' : A) \in \Sg \\
  \D' = \overline{(y:B)}[\overline{e}/\overline{n}] \and
  \vars {\semi} \cons {\semi} \D, (x : A[\overline{e}/\overline{n}]) \vdash Q :: (z : C)}
  {\vars \semi \cons \semi \D, \D' \vdash (\ecut{x}{f \indv{e}}{\overline{y}}{Q}) :: (z : C)}
\end{mathpar}
The declaration of $f$ is looked up in the signature $\Sg$ (first premise), and $\overline{e}$
is substituted for $\overline{n}$ while matching the types in $\D'$ and $\overline{y}$
(second premise). Similarly, the freshly created channel $x$ has type $A$ from the signature
with $\overline{e}$ substituted for $\overline{n}$.
The corresponding semantics rule also performs a similar substitution
$\fresh{a}$.
\begin{tabbing}
$(\m{def}C) : $ \= $\m{proc}(c, \ecut{x}{f \indv{e}}{\overline{d}}{Q}) \; \mapsto \;$ \\
\> $\m{proc}(a, P_f[a/x, \overline{d}/\overline{y'}, \overline{e}/\overline{n}]), \;
   \m{proc}(c, Q[a/x])$
\end{tabbing}
where $\overline{y' : B} \vdash f \indv{n} = P_f :: (x' : A) \in \Sg$.

Sometimes a process invocation is a tail call,
written without a continuation as $\procdef{f \indv{e}}{\overline{y}}{x}$. This is a
short-hand for $\procdef{f \indv{e}}{\overline{y}}{x'} \semi \fwd{x}{x'}$ for a fresh
variable $x'$, that is, we create a fresh channel
and immediately identify it with x.

\paragraph{\textbf{Type Definitions}}
As our queue example already showed, session types can be defined
recursively, departing from a strict Curry-Howard interpretation of
linear logic, analogous to the way pure ML or Haskell depart from
a pure interpretation of intuitionistic logic.  For this purpose we
allow (possibly mutually recursive) type definitions $V \indv{n} = A$
in the signature $\Sg$. Here, $\overline{n}$ denotes a sequence of
arithmetic variables. Again, for a well-formed signature, we require $A$ to be
\emph{contractive}~\cite{Gay2005} meaning $A$ should not itself be a
type name. Our type definitions are \emph{equirecursive} so we can
silently replace type names $V \indv{e}$ indexed with arithmetic
refinements by $A [\overline{e}/\overline{n}]$ during type checking,
and no explicit rules for recursive types are needed.

All types in a signature must be \emph{valid}, formally denoted with the judgment
$\vars \semi \cons \vdash A\; \mi{valid}$, which requires that all
free arithmetic variables of $\cons$ and $A$ are contained in $\vars$,
and that for each arithmetic expression $e$ in $A$ we can prove
$\vars' \semi \cons' \vdash e : \m{nat}$ for the constraints $\cons'$
known at the occurrence of $e$ (implicitly proving that $e \geq 0$).


\subsection{Preservation and Progress}
The main theorems that exhibit the deep connection between our type
system and the operational semantics are the usual \emph{type
  preservation} and \emph{progress}, sometimes called \emph{session
  fidelity} and \emph{deadlock freedom}, respectively.

So far, I have only described individual processes. However, processes
exist in a \emph{configuration}. A process configuration is a multiset
of semantic objects, $\proc{c}{P}$ and $\msg{c}{M}$, where any
two offered channels are distinct. A key question is how to type these
configurations. Since they consist of both processes and messages, they
both \emph{use} and \emph{provide} a collection of channels.
And even though a configuration is treated as a multiset, typing imposes
a partial order on the processes and messages where a provider of a
channel appears to the left of its client.

A configuration is typed w.r.t. a signature providing the type declaration
of each process.
A signature $\Sg$ is \emph{well formed} if
(a) every type definition $V = A_V$ is \emph{contractive},
and (b) every process definition
$\D \vdash f = P :: (x : A)$ in $\Sg$
is well typed according to the process typing judgment, i.e.
$\Sg \semi \D \vdash P :: (x : A)$.

I use the following judgment to type a configuration.
\[
\Sg \semi \D_1 \vDash \config :: \D_2
\]
It states that $\Sg$ is well-formed
and that the configuration $\config$
uses the channels in the context $\D_1$ and provides
the channels in the context $\D_2$.
\begin{figure}[t]
\begin{mathpar}
\infer[\m{empty}]
{\Sg \semi \D \vDash (\cdot) :: \D}
{}
\and
\infer[\m{compose}]
{\Sg \semi \D_0 \vDash (\config_1 \; \config_2) :: \D_2}
{\Sg \semi \D_0 \vDash \config_1 :: \D_1 \qquad
\Sg \semi \D_1 \vDash \config_2 :: \D_2}
\and
\infer[\m{proc}]
{\Sg \semi \D, \D_1 \vDash \proc{c}{P} :: (\D, (c : A) )}
{\Sg \semi \D_1 \vdash P :: (c : A)}
\and
\infer[\m{msg}]
{\Sg \semi \D, \D_1 \vDash \msg{c}{P} :: (\D, (c : A) )}
{\Sg \semi \D_1 \vDash P :: (c : A)}
\end{mathpar}
\caption{Typing rules for a configuration}
\label{fig:config_typing}
\end{figure}
The configuration typing judgment is defined using
the rules presented in Figure~\ref{fig:config_typing}.
%
The rule $\m{empty}$ defines that an empty configuration
is well-typed. The rule $\m{compose}$
composes two
configurations $\config_1$ and $\config_2$; $\config_1$ provides
service on the channels in $\D_1$ while $\config_2$ uses
the channels in $\D_2$. The $\m{proc}$ rule creates a configuration
out of a single process. Similarly, the $\m{msg}$ rule creates a
configuration out of a single message.

\begin{theorem}[Type Preservation]
\label{thm:preservation}
If $\Sg \semi \D' \vDash \config :: \D$ and $\config \step \dc$,
then $\Sg \semi \D' \vDash \dc :: \D$.
\end{theorem}
\begin{proof}
  By case analysis on the transition rule, applying inversion to the
  given typing derivation, and then assembling a new derivation of
  $\dc$.
\end{proof}

A process or message is said to be \emph{poised} if it is trying to
communicate along the channel that it provides.  A poised process is
comparable to a value in a sequential language. A configuration is
poised if every process or message in the configuration is poised.
Conceptually, this implies that the configuration is trying to communicate
externally, i.e. along one of the channel it provides.
The progress theorem then shows that either a configuration can take a
step or it is poised.  To prove this I show first that the typing
derivation can be rearranged to go strictly from right to left and
then proceed by induction over this particular derivation.

\begin{theorem}[Global Progress]
\label{thm:progress}
\mbox{}
If $\cdot \vDash \config :: \D$ then either
\begin{enumerate}
\item[(i)] $\config \mapsto \dc$ for some $\dc$, or
\item[(ii)] $\config$ is poised.
\end{enumerate}
\end{theorem}
\begin{proof}
By induction on the right-to-left typing of $\config$ so that either
$\config$ is empty (and therefore poised) or
$\config = (\dc\; \proc{c}{P})$ or
$\config = (\dc\; \msg{c}{M})$. By induction hypothesis, $\dc$ can
either take a step (and then so can $\config$), or $\dc$ is poised.  In
the latter case, I
analyze the cases for $P$ and $M$, applying multiple steps of
inversion to show that in each
case either $\config$ can take a step or is poised.
\end{proof}


\subsection{UC Communicators}
The linear aspect of session types imposes an important restriction on programs.
The only provision to spawn new processes is when a parent process creates a new
child process, and uses an exclusive linear channel to communicate with the child.
Thus, any two processes connected by a channel inherently maintain this parent-child
relationship.
Intuitively, this leads to a linear tree-like hierarchy among the processes,
thus preventing a cycle in the process graph.

Unfortunately, this restriction precludes practical programming scenarios
where process topologies indeed have a cyclic dependency (e.g. ring networks,
dining philosophers, etc.).
Recognizing this limitation, Balzer et al.~\cite{Balzer17icfp} proposed
a \emph{shared} extension of session types that allows arbitrary process topologies.
We have found this extension exceedingly helpful in the design and implementation
of cryptographic protocols.

One illustration of such a use of shared session types is a \emph{communicator}.
We use communicators as message buffers between two arbitrary processes: a
\emph{sender} and a \emph{receiver}.
The communicator is connected to both the sender and the receiver using a shared
channel.
Intuitively, the communicator receives \emph{push} requests from the sender followed
by receiving a message and stores them internally.
Analogously, the communicator receives \emph{pop} requests from the receiver,
and responds appropriately with the message if one is stored inside the communicator.
Formally, a communicator has the following polymorphic session type
\begin{tabbing}
  $\mi{stype} \; \m{comm[msg]} =$\\
  \quad $\up \echoice{$\=$\mb{push} : \m{msg} \arrow
  \down \m{comm[msg]},$\\
  \>$\mb{pop} : \ichoice{$\=$\mb{yesmsg} : \m{msg} \product \down \m{comm[msg]},$\\
  \>\>$\mb{nomsg} : \down \m{comm[msg]} }}$
\end{tabbing}
The type $\m{comm}$ is parameterized by the type $\m{msg}$, i.e., the type of
messages in the buffer.
The type initiates with an $\up$ denoting that $\m{comm}$ is a shared session type.
The type prescribes that the communicator needs to be acquired by the sender (or receiver)
for further interaction.
Such an acquire-release discipline is automatically enforced by the shared session type.
Once acquired, the communicator can either receive $\mb{push}$ (from sender) or
$\mb{pop}$ requests (from receiver).
In the former case, the communicator receives a message of type $\m{msg}$, and
then detaches from the client using the dual $\down$ operator.
In the latter case, the communicator checks if it internally contains a message
for the receiver.
If yes, the communicator replies with the $\mb{yesmsg}$ label followed by sending
the message (the $\product$ constructor).
Otherwise, the communicator replies with the $\mb{nomsg}$ label.
In either case, the communicator then detaches from the client matching the $\down$
operator.
Internally, the communicator stores these messages in a first-in-first-out order.

The communicator is also the perfect opportunity to implement an unreliable
message buffer that can drop or reorder messages.
All we would need to do is change the internal implementation of the communicator
\emph{without} changing the offered session type.


%\section{Import}
%The most recent iteration of the Universal Composability framework introduces a new notion of polynomial time: \textit{import}.
The import mechanism allocates to the environment some integer $n$ of units of import which it can consume or send to other ITIs.
When writing to another ITI, $\mathcal{Z}$ can specify an amount of import to sent alongside the message, and the receiving ITI can now use this import accordingly.
A good analogy to make for import is to consider them as tokens that are exchanged between ITIs. 
The definition of a polynomially bounded syste of ITIs now becomes: 





\section{Programming-Friendly Communication Models for UC} \label{sec:wrappers}
One of the main hurdles to existing UC models is that asynchronous/synchronous communication models uncecessarily complicate the design of protocols and ideal functionalities.
It makes it difficult to understand them and more painstaking to check or falsify UC security proofs. 
In this section we introduce a new alternative for sync/async communication models in the form of two shared functionalities, which allows delayed execution of entire code blocks rather than just delayed message delivery.
Furthermore, the functionalities take advantage of the new import mechanism to achieve eventual delivery in the asynchronous world.




\section{Reliable Broadcast} \label{sec:rbc}
In this section we demonstrate what protocol and ideal functionality design looks like with our new wrapper construction for asynchronous communication.
We omit parts of the code that are generic to all ITM code as mentioned in the previous section such as the code to react to \Exec messages from the wrapper.

We also describe a generic simulator that works for all full-information protocols like reliable broadcast where all information is leaked to the adversary.
In this scenario, a full simulation of the real world reveals nothing about the protocol being proven secure. 
Therefore, we provide a rationale to argue that if a simulator exists for the given protocol and ideal functionality, then our generic simulator is one.

\subsection{The Ideal World Functionality \Frbc}
In this section we use a reliable broadcast as the running example for both the real and ideal worlds.
The reliable broadcast ideal functionality, \Frbc, can be seen in pseudo-code in Figure~\ref{fig:frbc}.
\Frbc~is quite simple.
It waits for an input value $m$ from a designated dealer. 
Upon receipt of $v$ from the dealer, it schedules several codeblocks to \Eventually output this value to the other participants indicating they have committed to $m$.

The pseudo-code in Figure~\ref{fig:frbc} also details an amount of import tokens to be given by the dealer with the input to \Frbc.
For the high-level description, the amount of import is dynamically determined based on the number of participants in the protocol.
In Nomos, however, the ideal functionality must be designed for a particulat number of participants.
Therefore, for the remainder of this section for the Nomos examples we fix $n=4$.

\begin{figure}
\begin{bbox}[title={Functionality $\mathcal{F}_\msf{RBC} (\mathcal{D}, \mathcal{P})$}]

\OnInput \inmsg{\textsc{input}}{$m$}{$n(4n+1) \token$} from $\mathcal{D}$:

	\begin{renumerate}

		\item {\bf Leak} \inmsg{\textsc{input}}{$m$}
		
		\item {\bf For} each $P_i \in \mathcal{P}$:
		\begin{renumerate}

			\item \Send \inmsg{\textsc{schedule}}{\textsc{send}}{($m$, $P_i$)} $\rightarrow \mathcal{W}_\msf{async}$		

		\end{renumerate}
	\end{renumerate}

{\bf Code} $\textsc{send}(m, P_i)$:
	\begin{renumerate}

		\item \Send $m \rightarrow P_i$
	\end{renumerate}
\end{bbox}

\caption{A reliable broadcast functionality for $n$ participants parameterized by a dealer \texttt{pid} and a set of parties. It accepts input from the dealer and a required import token balance of $n(4n + 1)$ tokens.}
\label{fig:frbc}
\end{figure}

The Nomos equivalent of \Frbc~can be seen in Figure~\ref{fig:nomos:frbc}.
The ideal functionality attempts to read from the communicator for new messages from the dealer (w.l.o.g the dealer is fixed to be \texttt{pid = 1}).
Upon receiving the \texttt{Commit} message from the dealer, the functionality sends a \texttt{Schedule} message to the wrapper with an identifier for the code that sends a message to a protocol party: in Nomos, programs identify code blocks with an integer. 
In the case of \Frbc, the the only code it executes with adversarial delay is sending a message to a protocol party, hence it also passes the arguments to the ``send'' operation to the wrapper: the receiver and the message \texttt{m}.

\begin{figure}
\begin{lstlisting}[basicstyle=\small\ttfamily, frame=single]
type p2f = Commit of pid ^ int ;
type f2p = Commit of pid ^ int | Ok ;
type f2w = Schedule of int ^ list[ARGS] 
             | Leak of list[ARGS] ;
type w2f = Exec of int ^ list[ARGS] | Ok ;
...
$pf <- acquire #p_to_f ;
$pf.RECV ;
case $pf (
  yes => msg = recv $pf ;
    case msg (
      Commit(pid, int) => 
        if pid == 1 then
          $fw <- acquire #f_to_w ;
    	  for p in parties
    	    $fw.SEND ;
    	    send $fw Schedule(1, [p v]) ;
    	    $wf <- acquire #w_to_f ;
    	    case $wf (
    	    	yes => msg = recv $wf ;
    	  	      case msg (
    	  	        Ok =>
    	  	      )
    	    )
    	    #w_to_f <- release $wf ;
    	  #f_to_w <- release $fw ;
    	end
    	$fp <- acquire #f_to_pw ;
    	$fp.SEND ;
    	send $fp Ok ;
    	...
    )
| no =>
)
\end{lstlisting}
\caption{\Frbc~definition in Nomos accepts an input from the dealer (pid=1) and \Eventually sends the input to all of the parties. The send operation is scheduled with the wrapper.}
\label{fig:nomos:frbc}
\end{figure}

Recall from the discussion in the previous section regarding the wrapper design, that the wrapper always passes control back to the caller with an \texttt{Ok} message.
It can be seen in the message type passed from the wrapper to the functionality, \texttt{type w2f}, in Figure~\ref{fig:nomos:frbc}.
Therefore, the functionality must wait to receive activation back from the wrapper for each scheduled codeblock.
Finally, we require that functionalities also pass control back to the party that called them with an \texttt{Ok} message.\footnote{Recall that the reason behind this is to ensure that simulators are able to receive enough activations to perform all required simulating.}

Finally when the wrapper executes one of the scheduled code blocks, it writes an \texttt{Exec} message with the code identifier and the arguments specified by the message \texttt{type w2f} in Figure~\ref{fig:nomos:frbc}.
In Figure~\ref{fig:nomos:frbcexec}, \Frbc~accepts \Exec message and sends the message \texttt{m} to the receiver \texttt{p}.

\begin{figure}
\begin{lstlisting}[basicstyle=\small\ttfamily, frame=single]
type w2f = Exec of int ^ list[ARGS] | Ok ; 

case msg (
  Exec(f, args) =>
    if f == 1 =>
	then
	  p = args[0] ;
	  v = args[1] ;
	  $fp <- acquire #f_to_pw ;
	  $fp.SEND ;
	  send $fp Commit(p,v) ;
	  #f_to_pw <- release $fp ;
	end
)
\end{lstlisting}
\caption{The functionality, like any other process that schedules codeblocks with the wrapper, presents an interface to be activate by the wrapper with an \texttt{Exec} message. We elide the case statement surrounding \texttt{case msg} that attempts to read from the communicator. This code can be seen in Figure~\ref{fig:nomos:frbc}.}
\label{fig:nomos:frbcexec}
\end{figure}

\subsection{Reliable Broadcast in the Real World}
We realize \Frbc~in the real world by adapting an existing broadcast protocol, the Bracha broadcast protocol~\cite{bracha}, to our UC model. 
The pseudo-code of the bracha broadcast is described in Figure~\ref{fig:protbracha} along with the import tokens sent between each of the parties.
Notice that the dealer requires the same amount of import in the real world as \Frbc~does in the ideal world.
Additionally, even the real world protocol must adhere to the convention established in Section~\ref{sec:wrappers} of returning control to the caller when the dealer is given input.
In the ideal world, the functionality returns control to an honest dealer which outputs an \texttt{Ok} message back to the environment. 
Therefore, the real world parties must do the same with an \texttt{Ok} message.

\begin{figure}
\input{figures/prot_bracha}
\caption{Some caption}
\label{fig:protbracha}
\end{figure}

The only interaction that the real world has with the wrapper is through the asynchronous communication channel that allows the parties to communicate.
The asynchronous channel is quite simple. 
In many ways it mimics the interaction of \Frbc~with the wrapper in the ideal world where it schedules message sends to different participants in the protocol.
Figure~\ref{fig:nomos:fchan} demonstrates the $\F_{\msf{chan}}$ functionality that ensures the sender and receiver are in the set of predefined parties, but besides that schedules message sends on request.

\begin{figure}
\begin{lstlisting}[basicstyle=\small\ttfamily, frame=single]
case msg (
  Send(from, to, msg) =>
    if isin from $parties && isin to $parties
	then
	  $fw <- acquire #fh_to_w ;
	  $fw.SEND ;
	  send $fw Schedule(1, [to, from, msg])
	  ...
	  $pw <- acquire #fh_to_pw ;
	  $pw.SEND ;
	  send $pw Ok ;
      ...
)
\end{lstlisting}
\caption{Nomos code for the hyrid $\F_{\msf{chan}}$ functionality that sends messages asynchronously beween any two parties in the protocol.}
\label{fig:nomos:fchan}
\end{figure}

The Bracha protocol adapted to UC is quite straightforward and follows the pseudo-code algorithm in Figure~\ref{fig:protbracha}, closely.
The protocol awaits input from the dealer and waits for a sufficient number of other parties to echo the same input value it received from the dealer.
Finally, it waits for enough other parties to have confirmed the same input value in the form of \texttt{READY} messages. 
Figure~\ref{fig:nomos:ready} shows some reference code for a party waiting to receive a threshold of \texttt{READY} messages from other parties before outputting a commit message to \Environment indiciating the protocol is over an it has committed to a value.

\begin{figure}
\begin{lstlisting}[basicstyle=\small\ttfamily, frame=single]
case msg (
  Send(from, to, msg) =>
    case msg (
      Ready(x) =>
        if x == input
        then
          numready = numready + 1 ;
          if numready == 2*(n/3) + 1
          then
            $pz <- acquire #p_to_z ;
            $pz.SEND ;
            send Commit(x) $pz ;
            #p_to_z <- release $pz ;
          else
            (* recurse and wait for messages *)
        else
          (* recurse and wait for messages *)
    )
)
\end{lstlisting}
\caption{After receiving $\frac{\ceil{ n + \frac{n}{3}}}{2}$ \texttt{ECHO} messages for a particular \texttt{input} a party waits to receive $2 \frac{n}{3} + 1$ \texttt{READY} messages for the same input before committing to the input and outputing the value to the environment.}
\label{fig:nomos:ready}
\end{figure}


\subsection{$\Pi_{\msf{bracha}}$ EUC-realizes \Frbc~in the $\F_{\msf{chan}}$-hybrid world.}
The protocol in question, and the corresponding ideal functionality, is a full-information protocol that leaks everything to the adversary.
In this scenario we argue that a generic simulator which runs a full simulation of the real world \textit{always} a sufficient simulator for the dummy adversary.

Given that the protocol is full-information, the simulator definition dosn't reveal much about the underlying protocol or proving it's indistinguishability as the simulator doesn't need to do any complicated work.
Therefore, it is sufficient to show that the simulator is always able to ensure ideal world parties output messages at the same time as the analogous message in the real world.
This requires ensuring:
\begin{enumerate}
\item \label{req1} The simulator can always delay ideal world messages at least as long as their analagous real world messages and
\item \label{req2} the simulator is always activated when a real world party outputs a message to \Environment (or \Adversary if it's corrupt).
\end{enumerate}

Requirement \ref{req1} is satisfied by Lemma~\ref{lem:enoughimport} which argues that any simulator in the weakened balanced environments constraint will always have enough import tokens to delay at least as much as the real world.

Requirement \ref{req2} is satisfied for the following reason.
In our design the only outputs we care about simulating are those that result from codeblocks executing.
The wrapper is only activated through two mechanisms, either a \texttt{poll} message from \Environment or an \texttt{Exec} request from the adversary.
The case for \texttt{Exec} is trivial as it's a message sent by the simulator.
The case for \texttt{Poll} message relies on the Lemma~\ref{lem:enoughimport}.
The lemma guarantees that the simulator will always be activated by \Wasync~in the ideal world when a codeblock is executed in the real world (and, hence, in the simulation).
Given that in both of the possible scenarios that a real world party could output a message, the simulator is guaranteed to always be activated, we conclude that the siulator can always react and ensure the ideal world parties output the correct message at the correct time.

%in such a full information protocol as the Bracha broadcast and \Frbc, we find that a generic simulator that runs a full simulation is sufficient in this case and, in fact, for all protocols that leak all information to the adversary.
%Therefore, for a generic simulator we must argue only that it is able to force output in the ideal world at the same time as the real world.
%More precisely we argue
%\begin{lemma}
%A generic simulator for full information protocols is \textit{always} activated when the real world outputs a value. Therefore, it is always able to ensure indistinguishable output.
%\end{lemma}
%
%All outputs that parties give to the environment in our wrapper world happen with the execution of a codeblock.
%For example, in \Frbc~output of a committed value happens with adversarial delay determining when each party commits to a value.
%Therefore, this implies that the only mechanisms by which a party outputs anything is through the \texttt{Exec} and \texttt{Poll} interfaces of \Wasync.
%It's obvious that when the simulator is told to execute a code block by \Environment through the \texttt{Exec} interface it is activated and can react to any real world output.
%In the case of \texttt{Poll}, the simulator is only activated when the internal \texttt{delay} in the ideal world \Wasync is non-zero.
%As we prove in Lemma~\ref{lem:enoughimport} the simulator always possesses enough import to delay codeblocks in the ideal world at least as long as the real world.
%We never have the case where the environment can force a codeblock in the ideal world and it's analogous code block in the real world to execute at the same time by exhausting the import available to the simulator.
%Therefore, whenever a code block in the real world executes due to a \texttt{poll} operation that outputs some value to the environment or adversary (for corrupt parties), the simulator is always guaranteed to be activated by the \texttt{poll} message in the ideal world allowing it to react to the output and force the same output to occurr in the ideal world.
%By examining the only mechanisms by which code blocks execute in through the wrapper, and, therefore, the only mechanisms by which parties output some message, we demonstrate that the simulator always has sufficient activation to react to the such outputs in the real world and ensure indistinguishability in the ideal world.
%
%
%If there exists a simulator, then the generic simulator is such a simulator $\leftarrow$ can just be a conjecture. Just explaining the simulator doesn't say anything about the protocol we need to say something else in the analysis about the simulator being sufficient.


\section*{Acknowledgment}

\section*{References}

\input{figures/asyncwrapper}

\appendix

\section{Proofs}
\begin{theorem}\label{thm:bracha:sync}
Let $\Pi_{\msf{bracha}}$ and $\mathcal{F}_{\msf{bracha}}$ be $\overline{\mathcal{W}}_{\msf{sync}}$-subroutine respecting protocols. 
Then $\Pi_{\msf{bracha}}$ EUC-realizes $\mathcal{F}_{\msf{bracha}}$ in the $\mathcal{F}^{n(n-1)}_{\msf{sync}}$-hybrid world in the static corruptions ($t < \frac{n}{3}$) model for any $\overline{\mathcal{W}}_{\msf{sync}}$-externally constrained, weakly-balanced environment $\mathcal{Z}$. We have:

$$\textsc{EXEC}^{\Wsync}_{\mathcal{F}_{\msf{bracha}}, \Ssyncbracha, \Environment} \approx \textsc{EXEC}^{\Wsync}_{\pi_{\msf{bracha}}, \Dadv, \Environment}$$
\end{theorem}

\textit{Proof of Theorem \ref{thm:bracha:sync}.}

We first construct the simulator \Ssyncbracha. 
The full pseudo-code of \Ssyncbracha~ is shown in Figure~\ref{fig:sim:bracha:sync} however, for brevity, we describe it here.

\Ssyncbracha~ runs a full simulation of the real world internally with its own copy of the the wrapper, \Wsync'. It also maintains a copy the \msf{runqueue} in \Wsync and its \msf{delay}. 
It also requests leaks from \Wsync every time is it is activted with input from the wrapper (\Advance), or the enviromnent (corrupt input, \Delay, \Exec).

\heading{Simulating Honest Input to \Fbracha.}
The honest input to \Fbracha is leaked to \Ssyncbracha~ when requesting leaks from \Wsync. 
It passes the input to it's internally simulated dealer $\mathcal{D}'$ and increments its internal \msf{delay}.

\heading{Simulating Other Codeblocks.}
For all other scheduled code blocks (from other protocol sessions), it adds those corresponding codeblocks to \Wsync' and updates its copy of \msf{runqueue} and \msf{delay}.

\heading{Simulating Corrupt Input.} When the simulator receives corrupt input for some party $\pi_i$ it simply passes the input to its internal copy of it, $\pi_i'$. In the case of Bracha, only the dealer submit input to the ideal functionality, though \Ssyncbracha~ waits to give input to the corrupted dealer until the simulation commits to an input.

\heading{Simulating a \msf{Poll} Call by the Environment.}
For every \msf{Poll} to \Wsync that does not execute a codeblock (it's delay is positive), \Ssyncbracha~ is activated with the \msf{poll} message.
The simulator updates its internal $\msf{delay}$ and adds 1 unit of delay to \Wsync if it is 0 (does not want the environment to control when the next codeblock is executed).
Then it simulates \msf{poll} to \Wsync' and stores any new leaks from it.

\heading{Handling Codeblocks in the Simulation.}
When a codeblock executes in the simulation \Ssyncbracha~ executes the corresponding codeblock in the ideal world if it was schedules by a protocol session other than the challenge protocol.
When a simulated honest challenge protocol party, $\pi_i'$, outputs a value $v$, i.e. commits to a value, \Ssyncbracha:
\begin{itemize}
\item If the dealer is corrupt, \Ssyncbracha~ gives input $v$ to the corrupt ideal world dealer if this is the first such output. It then executes the outer codeblock from \Fbracha. It then executes the codeblock that outputs $v$ to $\pi_i$ in the ideal world.
\item If the dealer is honest, the codeblocks already exist and it executes the one which delivers output to $\pi_i$.
\end{itemize}

\heading{Simulating \Exec Calls.}
When activated by \Environment with an \Exec call, it forwards the call to it's internal simulated adversary. If a codeblock is executed, it responds as specified above. 

\heading{Handling Corrupt Simulated Output.} 
\Ssyncbracha outputs all corrupt party output from the simulated adversary to the environment.


%\begin{lemma}\label{lem:enoughimport}

The \textit{weakly-balanced} relaxation guarantees a simulator \mathc{S} will always have sufficient import to delay ideal world codeblocks \textit{at least} as long as the corresponding real world code blocks.

\end{lemma}

\textit{Proof of Lemma \ref{lem:enoughimport}.}

The crux of the argument lies in ensuring that, even when minimal import is provided to the adversary in the ideal world, it has sufficient import to meaningfully delay ideal world codeblocks to ensure simulatability. 
Therefore we set up an extreme scenario:
\begin{itemize}
\item A protocol $\pi$ which schedules $n$ codeblocks and outputs to the environment in the last codeblock. An ideal protocol $\phi$ schedules a single codeblock that outputs the same message to the environment.
\item The simulator must be able to ensure that the ideal world codeblock can execute at the same time as the final codeblock in $\pi$.
\end{itemize}

In the real world with dummy adversary $\mathcal{D}$, $\mathcal{Z}$ needs to \Advance \Wasync $n+1$ times in order to force execute all codeblocks.
Scheduling $n$ codeblocks requires at least $n$ unites of import on the part of the parties, and the adversary in both worlds also receives at least $n$ units of import.
Therefore, in order to ensure indistinguishability, the simulator must be able to delay the ideal world \Wasync $n$, which it can plainly do \footnote{The adverasry is also given $k$, the security parameter, amount of import at the start of execution ensuring that a polynomial time simulator is still polynomial under this new definition.}.



\pagebreak

\begin{figure}
\begin{bbox}[title={\textbf{Functionality} $\F_\msf{Async} (P_i, P_j)$}]

\OnInput \inmsg{\textsc{send}}{\msf{msg}} from $P_i$:
	
	\begin{renumerate}

		\item {\bf Eventually} send \msf{msg} to $P_j$

	\end{renumerate}
	
\end{bbox}

\end{figure}

%\begin{figure}
%\begin{bbox}[title={Functionality $\mathcal{F}_\msf{RBC} (\mathcal{D}, \mathcal{P})$}]

\OnInput \inmsg{\textsc{input}}{$m$}{$n(4n+1) \token$} from $\mathcal{D}$:

	\begin{renumerate}

		\item {\bf Leak} \inmsg{\textsc{input}}{$m$}
		
		\item {\bf For} each $P_i \in \mathcal{P}$:
		\begin{renumerate}

			\item \Send \inmsg{\textsc{schedule}}{\textsc{send}}{($m$, $P_i$)} $\rightarrow \mathcal{W}_\msf{async}$		

		\end{renumerate}
	\end{renumerate}

{\bf Code} $\textsc{send}(m, P_i)$:
	\begin{renumerate}

		\item \Send $m \rightarrow P_i$
	\end{renumerate}
\end{bbox}

%\end{figure}

\begin{figure}
\begin{bbox}[title={Protocol $\Pi_\msf{Bracha} (\mathcal{D}, \mathcal{P})$}]

	Define $\msf{BQ} \leftarrow \frac{n+t}{2}$

	{\bf As Dealer $\mathcal{D}$}:

	\OnInput $m$ from $\mathcal{Z}$:

	\begin{renumerate}

		\item {\bf For} each $P_i \in \mathcal{P}$:
		\begin{renumerate}

			\item \Send \inmsg{\textsc{val}}{m} $\rightarrow P_i$		

		\end{renumerate}


	\end{renumerate}

	As any party $P_i$:

	\OnInput \inmsg{\textsc{val}}{$m$} from $\mathcal{D}$ or

	\parbox{\widthof{\OnInput}}~\inmsg{\textsc{echo}}{$m$} from \msf{BQ} parties or

	\parbox{\widthof{\OnInput}}~\inmsg{\textsc{ready}}{$m$} from $\msf{BQ}-\msf{t}$ parties:

	\begin{renumerate}
	
		\item {\bf For} each $P_i \in \mathcal{P}$:

			\begin{renumerate}

				\item \Send \inmsg{\textsc{echo}}{$m$} to $P_i$

			\end{renumerate}
	\end{renumerate}

	\OnInput \inmsg{\textsc{echo}}{$m$} from \msf{BQ} parties or

	\parbox{\widthof{\OnInput}}~\inmsg{\textsc{ready}}{$m$} from $\msf{BQ}-\msf{t}$ parties:

	\begin{renumerate}
	
		\item {\bf For} each $P_i \in \mathcal{P}$:

			\begin{renumerate}

				\item \Send \inmsg{\textsc{ready}}{$m$} to $P_i$

			\end{renumerate}
	\end{renumerate}

	\OnInput \inmsg{\textsc{ready}}{$m$} from \msf{BQ} parties:
		
	\begin{renumerate}

		\item Output $m$	

	\end{renumerate}
\end{bbox}

\end{figure}

\begin{figure}

\begin{bbox}[title={Simulator $\mathcal{S}_\msf{Bracha} (\mathcal{D}, \mathcal{P}, \Delta)$}]

Simulate real world parties $P_1',...,P_n'$ and the simulated dealer $\mathcal{D}'$.

Init $\msf{ideal\_queue}$ := $\emptyset$, $\msf{ideal\_delay}$ := $0$, $\msf{sim\_leaks}$ := $\emptyset$

$\msf{pid\_to\_idx} := \{\}$

\vspace{2mm} \hrule \vspace{2mm}

\underline{On every activation:} \vspace{2mm}

\begin{renumerate}
	\item $\msf{leaks} \leftarrow$ \{\Send (\textsc{get-leaks}) $\rightarrow \mathcal{W}_\msf{sync}$\}
	
	\item {\bf For} $\msf{leak} \in \msf{leaks}$:
	\begin{renumerate}
		\item {\bf Match} $\msf{leak}$:
			\begin{renumerate}
				
				\item {\bf Case} (\textsc{input}, $m$) from $\F_\msf{RBC}$:

				\quad Simulate (\textsc{input}, $m$, $n(4n+1) \token$) $\rightarrow \mathcal{D}'$ 

				\item {\bf Case} (\textsc{schedule}, $P_i$, \msf{rnd}, \msf{idx}) from $\F_\msf{RBC}$:

					\quad $\msf{pid\_to\_idx}[P_i] = \msf{(rnd, idx)}$

					\quad $\msf{ideal\_delay} \pluseq 1$

				\item {\bf Case} \msf{msg} from $\F$:
					
					\quad Simulate $\F'$ leaking \msf{msg}
			\end{renumerate}
		\end{renumerate}
	%\item \Send $leaks \rightarrow \mathcal{Z}$
	\end{renumerate}


\OnInput \inmsg{\textsc{get-leaks}} from $\mathcal{Z}$:
	\begin{renumerate}
	% \item \Send $simleaks \rightarrow \mathcal{Z}$
	\item $\msf{leaks} \leftarrow$ \{Simulate (\textsc{get-leaks}) $\rightarrow \mathcal{W}'_\msf{sync}$\}
	
	\item \Send \msf{leaks} $\rightarrow \mathcal{Z}$
	\end{renumerate}

\OnInput \inmsg{\textsc{poll}} from $\mathcal{W}_\msf{sync}$:
	\begin{renumerate}
	\item Execute \msf{Poll}
	\end{renumerate}

\OnInput \inmsg{\textsc{delay}}{$d \token$} from $\mathcal{Z}$:
	\begin{renumerate}
	\item Simulate $(\textsc{delay}, d \token) \rightarrow \mathcal{W}_\msf{sync}'$

%	\item \Send $(\textsc{delay}, d \token) \rightarrow \mathcal{W}_{sync}$

%	\item $idealdelay \pluseq d$

	\item \Send $\textsc{OK} \rightarrow \mathcal{Z}$
	\end{renumerate}

\OnInput \inmsg{\textsc{exec}}{\msf{rnd}}{\msf{idx}} from $\mathcal{W}_\msf{sync}$:
	\begin{renumerate}
	\item Simulate $(\textsc{exec}, \msf{rnd}, \msf{idx}) \rightarrow \mathcal{W}_\msf{sync}'$

	%\item $msg \leftarrow$ wait for output from some simulated ITM.

	\item If output $m$ from simulated party $P_i'$:

%		\quad Call $\msf{SimGeteaks}$

		\quad Call \msf{SimPartyOutput}($m$, $P_i'$)

	\item Else if output $m$ from simulated adversary $\mathcal{A}'$:

		\quad \Send $m \rightarrow \mathcal{Z}$

%	\item Execute $\msf{SimGetLeaks}$
%
%	\item Match $msg$ with:
%
%-- \OnInput (m) from $P_i$':
%  
%	\qquad call $\msf{SimPartyOutput}(m, P_i')$
%
%-- \OnInput (m) from $\mathcal{A}'$:
%
%	\qquad \Send $m \rightarrow \mathcal{Z}$
%
	\end{renumerate}

\end{bbox}

\end{figure}

\begin{figure}
\begin{subfigure}{\columnwidth}

\begin{bbox}[title={Algorithm $\msf{Poll}$}]

\begin{renumerate}

  	\item $\msf{ideal\_delay} \minuseq 1$
  	
  	\item If $\msf{ideal\_delay} = 0$:
  	 
  		\quad \Send $(\textsc{delay}, 1 \token) \rightarrow \mathcal{W}_\msf{sync}$

  		\quad $\msf{ideal\_delay} = 1$

  	\item \Send $(\textsc{poll},) \rightarrow \mathcal{W}_\msf{sync}'$
 
  	\item If output $m$ from simulated party $P_i'$:

%			\quad Call $\msf{SimGetLeaks}$

			\quad Call \msf{SimPartyOutput}($m$, $P_i'$)
		
		Else if output $m$ from simulated adversary $\mathcal{A}'$:

			\quad \Send $m \rightarrow \mathcal{Z}$

\end{renumerate}

\end{bbox}

\end{subfigure}
%\begin{subfigure}{\columnwidth}
%\input{figures/algosimgetleaks}
%\end{subfigure}
\begin{subfigure}{\columnwidth}

\begin{bbox}[title={Algorithm $\msf{SimPartyOutput}(m, P_i')$}]

	\begin{renumerate}
		\item If $P_i'$ simulates a dishonest $P_i$:
		\begin{renumerate}	

			\item Send $(m, P_i) \rightarrow \mathcal{Z}$
		\end{renumerate}

		\item Else, if no dealer input to $\mathcal{F}_\msf{RBC}$:
		\begin{renumerate}

			\item \Assert $\mathcal{D}$ is dishonest

			\item $\msf{OK} \leftarrow \{ \Send (\textsc{input}, m) \rightarrow \mathcal{D} \}$

			\item $\msf{leaks} \leftarrow \{ \Send (\textsc{getleaks}) \rightarrow \mathcal{W}_\msf{sync}\}$
		
			\item {\bf For} $\msf{leak} \in \msf{leaks}$, {\bf Match} $\msf{leak}$:
						\begin{renumerate}
							\item {\bf Case} (\textsc{schedule}, $P_i$, \msf{rnd}, \msf{idx}) from $\F_\msf{RBC}$:

								\quad $\msf{pid\_to\_idx}[P_i] = \msf{(rnd, idx)}$

								\quad $\msf{ideal\_delay} \pluseq 1$

							\item {\bf Case} \msf{msg} from $\F$:
					
								\quad Simulate $\F'$ leaking \msf{msg}

						\end{renumerate}
		\end{renumerate}
		\item $\msf{rnd}, \msf{idx} \leftarrow \text{pop } \msf{pid\_to\_idx}[P_i]$

		\item Update $\msf{pid\_to\_idx}$ indices like $\mathcal{W}_\msf{sync}$

		\item $\Send (\msf{exec}, \msf{rnd}, \msf{idx}) \rightarrow \mathcal{W}_\msf{sync}$
		
	\end{renumerate}

\end{bbox}

\end{subfigure}
\end{figure}


\begin{bbox}[title={Subroutine {\bf ExecuteWithTimeout} $(c, T)$}]

$t \leftarrow$ current round \# in \Wsync

\Send $(\Schedule, c, T-t) \color{red} $1 \token$ \color{black} \rightarrow \Wsync$

\OnInput \inmsg{exec} from $\F_{\msf{State}}$:

\quad $t \leftarrow$ current round \# from \Wsync
\begin{renumerate}
\item \If $t < T$:
	
	\qquad \Send $(\Schedule, c, T-t) \color{red} 1 \token \color{black} \rightarrow \Wsync$

\item \Else:
	
	\qquad Execute $c$

\end{renumerate}

\end{bbox}


\begin{bbox}[title={Functionality $\F_{\msf{State}} (\Delta, U, C, \mathcal{P} = \{P_1,...,P_n\}$}]

Initialize $\msf{state} = \emptyset, \msf{buf} = [], \msf{aux\_in} = [], \msf{ptr} = 0$

$r := 0$

\vspace{2mm} \hrule \vspace{2mm}


\OnInput \inmsg{input}{v} from $P_i$:

\quad \If first input received from $P_i$ in $r$:

\begin{renumerate}
	\item \If 
\end{renumerate}

\end{bbox}




\end{document}
