In this section we show how virtual tokens are used to construct a simulator for the running example, \Fcom, in this work.
The design pattern is identical to the simulator proofs of Theorems~\ref{thm:dummythicclemma}, \ref{thm:singlecomp}, and \ref{thm:functor}, and it highlights the simplicity of our encoding and token hierarchy.

\paragraph*{\textbf{Commitment Protocol}}
The commitment protocol realizes \Fcom in the \Fropp-hybrid world.
\Fropp is a combination of a random oracle \Fro (idealized hash function) and a one-way channel \Fauth.
We depart slightly from the standard formulation of \Fauth. Normally \Fauth receives a message to send and it leaks it to the adversary and then sends it to the receiving party.
This design, and similar design patterns in general, assumes additional behavior on the part of the adversary giving control back to \Fauth. 
We prefer to remove such assumptions and only rely on a process writing to one other process. In this case, \Fauth stores the message which the adversary can 
ask for, and we send the message to the receiving party only.
Let the output of the function $\O{x}$ be the hash reply of \Fropp on query $x$ with 1 import token.
On input \mb{commit(b)} and 1 import from \Z, the committer chooses a random blind $n \sample \bits{k-1}$ and sends $h = \O{b || n}$ to the receiver. 
On \mb{open} from \Z, the committer sends $b,n$ and the receiver checks $h \equiv \O{b || n}$ and outputs \mb{opened} and $b$ if true. 

% \paragraph*{\textbf{A Simulator for \Fcom}}
A common strategy for simulators is to run the real world in a sandbox and give the parties in it the same input as the ideal world. 
In this case, the simulator \simcom is even simpler and need only sandbox the random oracle \Fro to maintain the real world hash table, because there is no need to simulate message passing from one party to another. 
Sandboxing in NomosUC is achieved simply by passing a virtual token type as the token type of the process rather than the real one. 
\simcom spawns the wrapped \Fro simply with
\begin{lstlisting}[basicstyle=\footnotesize\BeraMonottFamily, mathescape, frame=single, numbers=left, xleftmargin=2em, xrightmargin=2em]
$\$$p_to_f $\leftarrow$ chan[G1][rop2f]{1} ;
$\$$f_to_p $\leftarrow$ chan[G1][rof2p]{1} ;
$\$$_ $\leftarrow$ Fro[G1] <- k rng sid $\$$p_to_f $\$$f_to_p ;
$\$$p2f' $\leftarrow$ ro_p2f[G1] $\$$p_to_f $\$$f_to_p;
\end{lstlisting}
where channel \inline{$\$$p2f'}, the linear endpoint of the providerless channel with \Fro, is typed as:

\begin{tabbing}
$\m{hash} = \textcolor{red}{\getpot^1} \ichoice{\mb{query}: $\=$\m{PID} \product \m{int} \product$ \\
\>$\echoice{\mb{hash}: \m{PID} \arrow \m{int} \arrow \m{hash}}}$
\end{tabbing}

Communication between \Z and \A is governed by the simple linear type \inline{comm[G][Z2A[2p][2f]} introduced in Section~\ref{sec:basic} where \inline{z2a} is defined as 
\begin{lstlisting}[basicstyle=\footnotesize\BeraMonottFamily, mathescape]
$\Type$ z2a[a][b] = Z2A2P of PID x a | Z2A2F of b 
$\Type$ a2z[a][b] = P2A2Z of PID x a | F2A2Z of b
\end{lstlisting}

\paragraph*{\textbf{Corrupt Committer}}
We only cover the case of the corrupt committer here, as it is the most interesting, with the incomplete case match below.
When \Z gives \simcom a hash query for the committer with 1 unit of import, \simcom creates 1 virtual token of type $G_1$ and queries the sandboxed \Fro. 
If a collision is found, then the commitment must fail and the simulator aborts:
\begin{lstlisting}[basicstyle=\footnotesize\BeraMonottFamily, mathescape, frame=single, numbers=left, xleftmargin=2em, xrightmargin=0.9em, firstnumber=8]
Z2A2F,*,* =>
  QHash(x) = $\nrecv$ $\$$z_to_a ;
  t = search l x ;
  $\nif$ t $\nthen$
    $\tg{(* collision no commitment so distinguish *)}$
    $\$$a_to_z.P2A2Z ; $\nsend$ $\$$a_to_z 1 ;
    $\$$ch $\leftarrow$ sim_com_sender[K][K1] $\leftarrow$ 
      $\tg{(* args *)}$ l first_input True ; 
  $\nelse$ 
    $\nwithdraw$ K K1 1 ;
    l' $\leftarrow$ append l x ;
    $\$$s_to_ro.hash ; $\nsend$ $\$$s_to_ro x ;
    $\npay$ {1} K1 $\$$s_to_ro ;
    $\ncase$ $\$$s_to_ro (
      hash => h = $\nrecv$ $\$$s_to_ro ;
        $\$$a_to_z.F2A2Z ; $\nsend$ $\$$a_to_z RHash(h) ; 
    )
    $\$$ch $\leftarrow$ sim_com_sender[K][K1] $\leftarrow$ 
      $\tg{(* args *)}$ l' first_input failed ;
\end{lstlisting}
The hash from \Fro is forwarded back to \Z as output.
The import tokens that \simcom receives ensures that it can make a polynomial number of virtual tokens for a potentially polynomial number of queries. 

When a message is sent to the receiver from \Z, \simcom checks whether a pre-image exists in the sandboxed \Fro. 
If so, it begins the commitment in \Fcom. 
For the open message, if the same bit is being opened too, then the commitment is correct and should open. 
Otherwise, \simcom aborts again.

More precisely, \simcom always checks whether collisions occur (as above). 
If such a case happens, then the commitment fails and \simcom outputs garbage.
If the bit opened is not one that was committed to, \simcom randomly selects a bit to commit to rendering such a UC exeriment distinguishable with non-negligible probability.
Finally, correct commitments proceed unhindered.

%When the commitment is opened by \Z and if no bit $b$ was found, the open is ignored and the \Fcom doesn't open.
%If a $b$ was found, \simcom simply instructs the sender to open (\inline{$\$$a2s.open}) and it, itself, terminates.  
%
%% \subsection{Sandboxing}
%Akin to how \Fro was used above in a sandbox, the main theorems in this work are proven through connecting simulators, and potentially adversaries, in a sandbox. 
%As we can see above, the providerless channel abstraction allows specification to be very general and straightforward. 
%The token heirarchy neatly abstracts away the notion of virtualization from process definitions. 

%The resulting type definition of \simcom below is straightdorward, and we omit the default parameters for the sake of space.
%\begin{lstlisting}[basicstyle=\footnotesize\BeraMonottFamily, mathescape, frame=single]
%$\nproc$ simcom[G][G1]: ($\$$z2a: comm[G][A2Z[rop2f]]{1}),
%($\$$a2s: sender[G][a]), ($\$$a2r: reciever[G][a]), ($\$$a2f: 1) 
%  |- ($\$$ch: 1)
%\end{lstlisting}
%It is parameterized by its own token type \inline{G} and a virtual token type \inline{G1} to be used for sandboxing other processes.
%Much like the environment and \Fcom, \simcom is parameterized  two channels for the sender and the receiver, and to the functionality with a channel typed \m{1}.

%\begin{figure}
%\begin{lstlisting}[basicstyle=\footnotesize\BeraMonottFamily, mathescape, frame=single, numbers=left, xleftmargin=2em, xrightmargin=0.9em, firstnumber=8]
%$\ncase$ m (
%  Z2A2P(pid, msg) =>
%    $\nget$ {1} G $\$$z2a
%    $\ncase$ msg (
%      QHash(x) => 
%        $\nwithdraw$ G G1 1 ;
%        $\$$l <- append $\$$l x ;
%        $\$$p2f'.hash; $\nsend$ $\$$p2f' pid ; 
%        $\nsend$ $\$$p2f' x ; $\npay$ {1} G1 $\$$p2f' ;
%        $\ncase$ $\$$p2f' (
%          shash => pid = $\nrecv$ $\$$p2f' ; 
%            h = $\nrecv$ $\$$p2f'; $\$$a2z.send ; 
%            $\nsend$ $\$$a2z P2A2Z(pid, RHash(h))
%        )
%        $\$$ch <- sim_com[G][G1] <- $\tg{(* args *)}$
%      Send(pid, d) =>
%        b = sample rng 1 ;
%        forSeq $\$$l \p ->
%          $\nwithdraw$ G G1 1 ;
%          $\$$p2f'.hash ; $\nsend$ $\$$p2f' 1 ; 
%          $\nsend$ $\$$p2f' p ; $\npay$ {1} G1 $\$$p2f' ;
%          $\ncase$ $\$$p2f'( 
%            shash => $\nrecv$ $\$$p2f'; h = $\nrecv$ $\$$p2f'
%              $\nif$ h == d $\nthen$
%                (b: n) = h 
%                $\$$a2s.commit ; $\nsend$ $\$$a2s b ;
%                break
%              $\nelse$ ())
%        $\$$ch <- sim_com_z2p[G][G1] <- $\tg{(* args *)}$)
%\end{lstlisting}
%\vspace{-0.5em}
%\label{fig:corrupt_committer}
%\caption{Code snippet for the corrupt committer}
%\vspace{-1em}
%\end{figure}


\subsection{ZK from \Fcom}
In order to show the modularity of our approach we discuss briefly here realizing a zero-knowledge functionality $\F_\msf{zk}$ with a protocol $\pi_\msf{zk}$ in the $!\Fcom$-hybrid world.
We then show composition for protocols where $!\Fropp \xrightarrow{\pi_\msf{zk} \circ !\pi_\msf{com}} \F_\msf{zk}$.
We specifically realize the zero-knowledge protocol for the Hamiltonian cycle in a graph from ~\cite{uccommitments}.
$\F_\msf{zk}$ is easily defined in our framework as shown below:

\begin{lstlisting}[basicstyle=\footnotesize\BeraMonottFamily, mathescape, frame=single, numbers=left, xleftmargin=2em, xrightmargin=0.9em, firstnumber=8]
$\nproc$ f_zk[a][b]{na,nb,nadv} :
  $\tg{(* std params *)}$
  ($\$$v: verifier[a][b]{na}), ($\$$p: prover[a][b]{nb}),
  ($\$$a: adv[a][b]{nadv}) $\vdash$ ($\$$ch: 1) =
{
  case $\$$v (
    commitment => x = $\nrecv$ $\$$v ;
      $\nget$ {nb} $\$$v ;
      $\$$a.commitment; $\nsend$ $\$$a x ;
      $\ncase$ $\$$p (
        witness => w = $\nget$ $\$$p; x' = $\nget$ $\$$p ;
          $\nif$ x == x' $\nthen$
            $\$$R $\leftarrow$ NP_rel $\leftarrow$ ;
            $\$$R.pair ; $\nsend$ $\$$R (x,w) ;
            $\ncase$ $\$$R (
              yes => $\$$v.yes ;
              no => $\$$v.no ;
            )
          $\nelse$ $\$$v.no ;
		  $\nend$
	  )
  )
}
\end{lstlisting}
We provide the full details of this construction, the protocol realizing it, and the composition in the appendix. 


