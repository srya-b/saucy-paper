In this section we explore some of the tradeoffs introduced earlier and the design space for ideal functionalities in NomosUC.
Specifically, we explore the polymorphism described in Section~\ref{sec:execuc} that mitigates the loss of expressiveness of session types that arises from splitting communication, between a party and a functionality, into two uni-directional channels.
We also share two functionalities, the random oracle \Fro as an example of a type that allows dynamic parties with only one channel, and we compare secure message communicatin (\Fsmc, a simpler version of \Fauth) to writing functionalities in a closely related work: easyUC~\cite{easyuc}.

\subsection{Polymorphism for Ideal Functionalities}
We mention in the previous section that despite splitting communication for \Fauth into two uni-directional channels per party, we use polymorphism to achieve message ordering between channels as well.
A contrived example of such a type below shows that with minimal extra variables in a type can achieve the same for arbitrarily complicated functionalities. 
The idea is that we can add an arbitrary number of variables to achieve the desired oredering by allowing the protocol to determine the concrete types for each type parameter. 
The tradeoff is extra session type clutter, from including multiple type variables that don't carry any data, but as we see with the type and functionality definition below the type is still clear upon inspection.

Imagine an ideal functionality where two parties take one of two roles with the following session types:
\begin{center}
	\parbox{0cm}{
	\begin{tabbing}
		$\m{role1\_p2f}[a] = \ichoice{\mb{do1}: \m{Int} \product \m{a} \product \m{role1\_p2f}}$ \\
		$\m{role1\_f2p}[a] = \echoice{\mb{do2}: \m{String} \arrow \m{a} \arrow 1}$ \\
		$\m{role2\_p2f}[a][b][c] = \ichoice{$\=$\mb{do3}: \m{a} \product \m{b} \product \m{role2\_p2f}$ \\
		\>$\mb{do4}: \m{Int} \product \m{c} \product \m{role2\_p2f}}$ \\
		$\m{role2\_f2p}[b][c] = \echoice{\mb{do5}: \arrow \m{b} \arrow \m{c} \arrow 1}$
	\end{tabbing}}
\end{center}
The session types define party 1 as giving 1 input \mb{do1} and party 2 giving two inputs \mb{do2}. 
When used with the following type definition for the a functionality
\begin{lstlisting}[basicstyle=\scriptsize\BeraMonottFamily, frame=single, mathescape]
$\nproc$ F[a, b, c]: $\tg{...}$
  ($\$$P12F: role1_p2f[a]), ($\$$F2P1: role1_f2p[a]), 
  ($\$$P22F: role2_p2f[a,b,c]), ($\$$F2P2: role2_f2p[b,c]) $\tg{...}$
\end{lstlisting}
we can achieve the following ordering of messages \emph{between} the multiple sessions:
\begin{itemize}
	\item Party 1 must give input \mb{do1} and determine the type \m{a} before \F can output \mb{do2} of that same type.
	\item Party 2 must give input \mb{do3} after party 1 gives \mb{do1}, and it must give inputs \mb{do3}, \mb{do4} (concretizing the types of \m{b} and \m{c}) before \F can output \mb{do5}.   
\end{itemize}

Functionalities whose code attempts to send concrete types over the sessions will fail to type check, and protocols that wish to realize them must also follow the same communication pattern
It is important to note here that the session type isn't enough to enforce order, but the functionality's type definition must also enforce the dependency.

\subsection{Secure Message Communication, \Fsmc}
Secure message communication, or \Fsmc, is the \Fauth functionality reduced to handle one-way communication only.
Therefore, unlike \Fauth the session type can be captured by a single channel for each party. 
The session types for \Fsmc are
{\centering
\parbox{0cm}{
\begin{tabbing}
$\m{sender}[a] = \ichoice{\mb{sendmsg}: PID \product a \product 1}$ \\
$\m{receiver}[a] = \echoice{\mb{recv}: PID \arrow a \arrow 1}$ 
\end{tabbing}}
}
The session type, in a sense, defines a state machine that is enforced by the type system.
The same functionality in a related work, EasyUC, only defines messages and function types, and requires the user code to define a state machine, perform runtime checks on inputs, reject inputs out of order, and manage communication through the EasyUC's addressing scheme. 
When compared to the entire code for \Fsmc below, the EasyUC code is at least three times as long.
Not only does that make UC definitions more complicated by introduces the potential for bugs in what should be a simple functionality\footnote{The full code of the EasyUC SMC example can be found at \url{https://github.com/easyuc/EasyUC/blob/master/smc/SMC.ec}}.
\begin{lstlisting}[basicstyle=\scriptsize\BeraMonottFamily, frame=single, mathescape]
$\nproc$ f_smc[K][a] : ($\$$S: sender[a]), ($\$$R: receiver[a]), 
	($\$$A: adv[a]) |- ($\$$ch: 1) = {
  let pidS :: PID, pidR :: PID = sid ;
  case $\$$S (
    sendmsg => pid = recv $\$$S ; m = recv $\$$S ;
               if pid == pidS then
                 $\$$A.leakmsg ; send $\$$A m ;
                 case $\$$A ( deliver => $\$$R.recvmsg ;
                              send $\$$R m ;
                 )
               else error "bad sender " ++ pid )
}
\end{lstlisting}
\todo{@andrew: i hesitate with this section because we can't afford to include code for \Fsmc in EasyUC because of space constraints so the point is slightly lost aside from my claims below but there is a URL}

We also note that we can realize \Fauth by applying the multisession operator to \Fsmc ($!\Fsmc$) and remove the need for splitting communication between two channels like we did in Section~\ref{sec:execuc}.
In general, it is better to write ideal functionalities to be as small and single-purpose as possible and combine them with generic operators, like $!$, to create more complex functionalities. 

%\subsection{Append-only Database}
%Prior notions of polynomial time in the UC framework require that ITMs execute in time polynomial in the length of their input.
%Other notions of polynomial time suggested in prior versions of the UC framework and other such notions~\ref{hofheinz} also run into problems when realizing functionalities where protocol parties 
