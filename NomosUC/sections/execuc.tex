In this section we introduce the UC experiment in Nomos and the resulting emulation definition.
We continue on to state the dummy lemma theorem as well as a composition theorem for Nomos UC.
An important part of our definition is using code-generation techniques~\cite{somecodegeneration} to constructs some processes in the UC experiment as well ass useful operators to achieve full composition in the sense of Canetti et al.~\cite{uc}.


We first introduce some convenient notation.
For the remainder of this section, when we refer to a protocol, we actually refer to a pair of ITMs as in Definition~\ref{def:protocol}.
\begin{definition}\label{def:protocol}
A \textit{protocol} is a pair of terms ($\pi$, $\mathcal{F}$) where $\pi$ is the local protocol code run by honest parties and \F is an ideal functionality parties can communcated with.
\end{definition}
In Definition~\ref{def:protocol}, \F is called a \textit{hybrid functionality}. 
The composability guarantees of UC allow protocols to rely on ideal functionalities as subroutines, and a composition theorem exists that allows replacing functionalities with protocols which realize them.
Importantly, a protocol may not use any hybrid functionalities, in which case \F is the dummy functionality which does nothing on activation. 


\subsection{The UC Experiment}
The UC experiment is an execution of a \textit{challenge protocol}, consisting of protocol parties and an ideal functionality, reacting to input by an adversary \Adversary and an environment providing inputs, \Environment. 
The experiment is created by an \msf{execUC} function which spawns the environment, a special construction called the \textit{protocol wrapper}, the adversary, and any functionalities (wrapped by the \textit{functionality wrapper}, a simplified version of the \textit{protocolwrapper}).

One constraint imposed by our virtual tokens definition is that all token types must be statically initialized in order to be used. 
What this means for \msf{execUC} and, in fact, the protocol wrapper is that we must rely on code generation to create unique process definitions for each protocol we wish to express.
For example, an ITM that internally simulates another ITM which in turn simulates a third ITM requires at least two virtual token types for its \textit{simulation depth} of two. 
A generic \msf{execUC} function could not express arbitrary simulation depth and therefore must be dynamically generated depending on the protocol and adversary at hand. 

In Figure~\ref{fig:execuc}, we illustrate \msf{execUC} for our running commitment example throughout the paper. 

\begin{figure*}
\begin{lstlisting}[basicstyle=\small\BeraMonottFamily, frame=single,  mathescape]
$\tb{type}$ sid[a] = SID of String ^ a ;

$\tb{proc}$ execUC[$\tm{K1}$, $\tm{K2}$]
           [p2z][z2p][p2f][f2p][p2a][a2p][f2a][a2f][a2z][z2a]
           {p2zn}{z2pn}{p2fn}{f2pn}{p2an}{a2pn}{f2an}{a2fn}{a2zn}{z2an} : 
    (k: int), (rng: [Bit]) |- ($\$$d : bit) = 
{
    $\$$z <- PS.env[$\tm{K1}$][z2p][p2z][z2a][a2z] <- k r ;
    sid = $\tb{recv}$ $\$$z ;
    clist = $\tb{recv}$ $\$$z ;
    ...
    #pw_to_f <- communicator_init[$\tm{K1}$][p2f]{p2fn} <- ;
    #f_to_pw <- communicator_init[$\tm{K1}$][f2p]{f2pn} <- ;
    ...
    send $\$$z #pw_to_z ;
    send $\$$z #z_to_pw ;
    $\$$pw <- protocol_wrapper[$\tm{K1}$][p2z][z2p][p2f][f2p][p2a][a2p]{p2zn}{z2pn}{p2fn}{f2pn}{p2an}{z2pn} 
             <- sid k rng clist #pw_to_f #f_to_pw #pw_to_a #a_to_pw #pw_to_z #z_to_pw ;
    $\$$fw <- functionality_wrapper[$\tm{K1}$][p2f][f2p][f2a][a2f] {p2fn}{f2pn}{f2an}{a2fn}
             <- sid k rng clist #pw_to_f #f_to_pw #f_to_a #a_to_f ;
    ...
    ...
    $\$$z.$\tm{start}$ ;
    $\$$d <- $\$$z ;

}
\end{lstlisting}
\caption{The \msf{execUC} function used for the two-party commitment example used througout this paper. Recall, the \msf{execUC} is customized insofar as it takes in some number of virtual token types (here, $K_1$) to support macines that simulate other machines. In the commitment exampe, there is no sch simulation happening at the protocol or functionality level, therefore only rhe real token type $K_1$ is used here. The funtion spawns all the necessary ITMs in the UC execution: the environment, the protocol wrapper, the functionalty (wrapped), and the adversary. Each is parameterized with the security parameter $k$ and a random bit sequence $\msf{rng} \in \{0,1\}^{poly(k)}$.
At the end, the environment is initiated and it returns a bit $b$ which is its guess for which world it is in. The full code can be found in the Appendix.}
\label{lst:execuc}
\end{figure*}

A blatant omission from the \msf{execUC} process definition is the protocol, functionality, adversary, and environments as input parameters.
The reason for this omission is that passing process definitions as parameter is not supported yet in the Nomos impementation.
Therefore, we rely on importing modules which define the relevant processes and define them in scope for \msf{execUC}.

A module representing the protocol must define a process called \msf{PS.prot} and \msf{PS.func} for the protocol and the hybrid functionality. 
In the ideal world, \msf{PS.prot} is the dummy protocol and \msf{PS.func} is the functionality being realized.
Similarly, an environment \msf{PS.env} and adversary \msf{PS.adv} must be defined as well.
The message types exchanged between the processes are provided directly to \msf{execUC} as type parameters of the form \msf{p2f}, \msf{f2p}, and so on. 

The environment is spawned first and selects the session id, or \msf{sid}, for the execution and determines the corrupted parties, \msf{clist}.
The rest of the ITMs are then spawned with this \msf{sid} and are given the list of corrupt parties.
Recall in the UC framework, corrupt parties accept input and give output to the adversary instead of the environment, the protocol wrapper runs dummy parties in their place that forward messages between \Adversary and \F.

Finally, the environment executes its own code when activated by \msf{\$z.start} and returns a bit that indicates its guess as to which world it is operatin in: real or ideal.
Over all possible environments, security parameters $k$, and random bit sequences $r$, the output of \msf{execUC} represents an ensemble of distributions. 



\subsection{The Protocol Wrapper}
The \msf{execUC} definition introduces a new construct called the \textit{protocol wrapper}. 
In the UC experiment, the environment can create protocol parties on the fly and none exist until the first message is written to them.
Thefeore the wrapper is intended to create new parties on demand.

The necessity of a protocol wrapper leads to an interesting problem in how channels and session typed can be used.
All communciation between protocol parties and \Environment, \F, and \Adversary is managed by the protocol wrapper, and the need to multiplex and de-multiplex communication between parties and other machines makes session types between them impossible.
It isn't possible to use multiple session-typed channels through a single communicator either, because parties can have different roles within a protocol so not even the same code is being executed (hence different session types governing the protocol) for each party.
Therefore, we define a new approach to creating protocol-specific party wrappers, but with a generic construction that can be used for any protocol.
We use the two-party commitment protocol as an example to demonstrate how the construction works and show that it is generic enough to allow code generation of a wrapper for any protocol.


Recall the session type governing the committer and receiver in the commitment protocol:
\begin{gather}
	\mi{stype} \; \m{sender} = \ichoice{\mb{commit} : \m{bit} \product \m{scommitted}} \\
	\mi{stype} \; \m{scommitted} = \ichoice{\mb{open} : 1} \\
	\mi{stype} \; \m{receiver} = \echoice{\mb{commit} : \m{rcommitted}} \\
	\mi{stype} \; \m{rcommitted} = \echoice{\mb{open} : \m{bit} \arrow \one}
\end{gather}

A protocol has multiple roles--for commitment they are committer and receiver.
The wrapper maintains a list for each possible session type which hold the channels of that type. 
For commitment the lists for the \msf{z2p} channels are the following:
\begin{gather}
	\m{R1L1}[\m{sender}] \\
	\m{R1L2}[\m{scommitted}] \\
	\m{R2L1}[\m{receiver}] \\
	\m{R2L2}[\m{rcommitted}] 
\end{gather}

The channels between the protocol wrapper and the rest of the machines are still limited to functional types through a communicator. The communicator is spawned with type \msf{z2p[z2pmsg]} and the real token type $K$.
\begin{gather}
\mi{type} \; \m{z2pmsg} = \m{Commit} \; \mi{of} \; \m{bit} \; | \; \m{Open}
\end{gather}
\[
	\mi{stype} \; \m{z2p}[a] = \m{Z2P} \; \mi{of} \; \msf{pid} \; \hat{ } \; a
\]

When the protocol wrapper receives a message for some \msf{pid}, if the party doesn't exist the protocolwrapper creates all the party's channel parameterized by the correct session types (the party's role and session types are determined by functional type of the incoming message).
The channels are stored in the appropriate lists corresponding to their type.
The session type of the message is determined by the functional type, and the session-typed message is sent along that channel.
If a message is sent out of order, \textit{the program fails to type check} due to the type mismatch between the channel at the message.
After delivering the message, the channel is moved to the next list corresponding to its new type.

For outgoing messages, a new process per party channel waits to read and does the inverse conversion: from session type to funtional type and attaches the party's \msf{pid} to it.
\todo{Does the wrapper have to use the simulation of parties? I don't think so it can just generate enough potential $T(0)$ to do constant work of routing}.

\paragraph{Functionality Wrapper}
For the same reasons as the protocol wrapper we also create a functionality wrapper around the ideal functionaltiy (in the real or the ideal world).
The key difference between the functionality wrapper and the protocol wrapper is that there is only one instance of the functionality running.
It still creates internal channels for the functionality, which are session-typed, converts incoming functional messages (from $\mathcal{P}$, the \Environment, or the \Adversary) to a session typed message and forwards it along the appropriate channel. It does the same for outgoing messages. 
There is no need to maintain lists like the protocol wrapper as there is only one functionality.
From this point on whenever we refer to a functionality in a UC experiment, it is wrapped by the \textit{functionality wrapper}.

\subsection{Polynomial Bound}
The UC import mechanism provides a way to define polynomial time computation and resource-bounds by ensuring that a single ITM's execution is upper-bounded by some value $T(n)$ where $T$ is a polynomial and $n$ the total units of import the ITM ever receives.
This is enough to reason about polynomial-time, however when reasoning about security we care specifically of ITMs whose computation is bounded by some poynomial in the security parameter $k$. 
In NomosUC, we take advantage of the import built into the type system to ensure ITMs are PPT in the security parameter. 

\begin{definition}[PPT Term]\label{def:pptterm}
A \textit{PPT term} is a \textit{well-typed} term $e(k, r)$ that is \textit{closed} except for security parameter $k$, random bit sequence $r$.
\end{definition}

We first-define terms that are well-typed in the traditional session-types-sense in Definition~\ref{def:pptterm}, i.e. without any resource constraints~\cite{sessiontypes}.
Such terms are closed except for the security parameter $k$ and some uniformly random bit sequence $r$.

However, we also want to reason about terms that are well-typed when connected to another Nomos terms.
We introduce the term \textit{well-matched} to mean a PPT term $e$ is well-typed when connected to another term $e'$.
Simply put, the types that $e$ uses to communicate on its outgoing channels match those expected by $e'$ and vice versa.
This new definition becomes important when we discuss UC emulation below as we want to reason about environments that are \textit{well-matched} for a protocol $\pi$ or a specific adversary \Adversary.

\begin{definition}[Well-Matched]\label{def:wellmatched}
\begin{mathpar}
\footnotesize
\inferrule*[right=Well-matched]
{\Tokens_1, K \semi \Delta_1 \vdash C_1 :: \Delta_1' \semi 
\Tokens_2, K' \semi \Delta_2 \vdash C_2 :: \Delta_2' \\ \\
 S \equiv \Delta_1 \bigcap \Delta_2 \neq \emptyset}
{\Delta_1 \equiv_{S} \Delta_2 \semi K \equiv K'} 
\end{mathpar}
\todo{whats the way to express this point that they are equivalent only over the pairs $(x, \tau)$ which they share?}
\end{definition}

Notice that in Definition~\ref{def:wellmatched} we are concerned with two terms that are \textit{open} even when connected. 
We only reason about being well-matched, when connected to another term, on the channels over which they are connected.

Next we introduce our definition of a polynomial-bound in the security parameter $k$.
Terms that obey PPT in $k$ are dubbed \textit{well-resource-typed}.
\begin{theorem}[PPT in $k$]\label{thm:ppt}
A \textit{PPT Term} $e(k, r)$ is well-resource-typed if, given initial import $n(k) \in poly(k)$, there exists a polynomial $T$ s.t. $\forall k, r, e(k, r) \{n(k)\}$ terminates in at most $T(n)$ steps. 
\end{theorem}

\begin{proof}
The Nomos type system guarantees that a satisfying assignment of $n$ and $T$ will correctly type-check.
Therefore, given an initial amount of import $n(k) \in poly(k)$, the existence of some $T$ ensures that any process, regardless of its randomized execution according to the bit sequence $r$, $e$ is guarantees to be upper-bounded by $poly(k)$ satisfying the definition of probabilistic polynomial time in $k$.
\end{proof}

\subsection{Emulation}
A proof of security in the UC framework relies upon emulation.

In general, we say that a protocol $\pi$ exhibits some desired security properties if no environment providing inputs to protocol parties (and the adversary) can distinguish between the $\pi$ and another pprotocol $\phi$ that possesses the desired properties.
In most cases we compare a real protocol $\pi$ with an idealized protocol $\phi$ which encapsulated by a single ideal functionality $\F$.
The ideal functionality is a trusted third party that executes the entire protocol on behalf of all of the participants.
Therefore, it is much simpler than a real protocol where mutually distrustful parties communicate with each other, and its security properties are easily proven.

It is clear from Figure~\ref{lst:execuc} that its output is a distribution induced by the random bit string $r$ (note that without any randonmess \msf{execUC} is entirely deterministic for any environment).
When we reason about emulation between two protocols what we really means is the indistinguishability between the ensembles that represent the outputs of \msf{execUC}.
We define indistinguishabiliy between ensembles in a standard way using \textit{statistical distance} in Definition~\ref{def:distance}.

\begin{definition}[Indisinguishability]\label{def:distance}
Two ensembles $\mathcal{D}_{1,k}, \mathcal{D}_{2,k}$ are indistinguishable, $\mathcal{D}_{1,k} \sim \mathcal{D}_{2,k}$, if their statistical distance is at most $negl(k), \forall k$.
\end{definition}

Before we introduce the emulation definition, we first define what valid protocols, valid functionalities, and what it means for protocols, functionalities, adversaries, and environments to be well-matched with each other.
We shorten the communicator type \msf{comm} to \msf{c} in the following definitions.

\todo{Ankush: The context of a valid functionality must contain channels typed with the type parameters given by \msf{execUC}. An the machine, parameterized with security parameter $k$ and random bit sequence $r$ are bounded by some polynomial $T_\F$. $\leftarrow$ the last part is meant to capture the well-resource-typed (from the well-matched definition), but maybe we can just say $\F$ is well-resource-typed given $k$,$r$}
\begin{definition}[Valid Functionality]\label{def:validfunc}
\begin{mathpar}
\footnotesize
\inferrule*[right=valid-F]
{\exists c_1:c[\msf{p2f}], c_2:c[\msf{f2p}], c_3: c[\msf{f2a}], c_4:c[\msf{a2f}] \in \Delta_1 \\
\Delta_1 \models (\F(k, r) : T_\F) :: \Delta_1'}
{\msf{validF}\ \F \rightarrow \Delta_1'}
\end{mathpar}
\end{definition}

\todo{The intent is the same as above here execpt for protocol having channels with the right types. Again here I could just say $\pi$ is well-resource-typed instead of the $\pi(k,r)$ that is there now.}
\begin{definition}[Valid Protocol]\label{def:validprot}
\begin{mathpar}
\footnotesize
\inferrule*[right=valid-P]
{\exists c_1: \msf{p2f}, c_2: \msf{f2p}, c_3: \msf{p2a}, c_4: \msf{a2p}, c_5: \msf{z2p}, c_6: \msf{p2z} \in \Delta_1 \\
\Delta_1 \models (\pi(k, r) : T_\pi) :: \Delta_1' }
{\msf{validP}\ \pi \rightarrow \Delta_1'}
\end{mathpar}
\end{definition}

\todo{Ankush: this defines what it means for a protocol and functionality to be well-matched. Namely, they shared channels typed according to parameters given by execUc (p2f, f2p, ...) and have the same type and import parameters on their communicators}
\begin{definition}[Well-Matched]
\begin{mathpar}
\footnotesize
\inferrule*[right=p2f match] 
{\msf{validP}\ \pi \rightarrow \D_1 \semi \msf{validF}\ \F \rightarrow \Delta_2 \\
\Delta_1:, (\msf{c}[K][\msf{f2p}]), (\msf{c}[K][\msf{p2f}]) \equiv \\
\Delta_2, (\msf{c}[K][\msf{pid \textasciicircum f2p}]), (\msf{c}[K][\msf{pid \textasciicircum p2f}])}
{\langle \pi \leftrightarrow \F \rangle}
\end{mathpar}
\end{definition}

\todo{Ankush: same for this one and the rest, as above}
\begin{definition}
\begin{mathpar}
\footnotesize
\inferrule*[right=p2a match] 
{\msf{validP}\ \pi \rightarrow \Delta_1 \semi \Adversary \rightarrow \Delta_2 \\
\Delta_1:, (\msf{c}[K][\msf{a2p}]), (\msf{c}[K][\msf{p2a}]) \equiv \\ 
\Delta_2, (\msf{c}[K][\msf{pid \textasciicircum a2p}]), (\msf{c}[K][\msf{pid \textasciicircum p2a}])}
{\langle \pi \leftrightarrow \Adversary \rangle}
\end{mathpar}
\end{definition}

\begin{definition}
\begin{mathpar}
\footnotesize
\inferrule*[right=f2a match] 
{\msf{validF}\ \F \rightarrow \Delta_1 \semi \Adversary \rightarrow \Delta_2 \\
\Delta_1:, (\msf{c}[K][\msf{a2f}]\{a2fn\}), ( \msf{c}[[K]\msf{f2a}]\{0\}) \equiv \\
 \Delta_2, (\msf{c}[K][\msf{a2f}]\{a2fn\}), ( \msf{c}[K][\msf{f2a}]\{0\})}
{\langle \F \leftrightarrow \Adversary \rangle}
\end{mathpar}
\end{definition}

\begin{definition}
\begin{mathpar}
\footnotesize
\inferrule*[right=p2z match] 
{\msf{validP}\ \pi \rightarrow \Delta_1 \semi \Environment \rightarrow \Delta_2}
{\Delta_1:, (\msf{c}[K][\msf{z2p}]), (\msf{c}[K][\msf{p2z}]) \equiv \\
 \Delta_2, (\msf{c}[K][\msf{pid \textasciicircum z2p]}), (\msf{c}[K][\msf{pid \textasciicircum p2z}])}
\end{mathpar}
\end{definition}

Indisintiguishability between two protocols is defined as follows (we shorten the communicator type \msf{comm} to \msf{c}):

\begin{definition}[Emulation]\label{def:emulation}
Given two protocols $(\pi, \F_1), (\phi, \F_2)$ that are well-resource-typed then if $\forall \Adversary$ well-matched with $(\pi, \F_1)$, $\exists \Simulator$ s.t. $\forall \Environment$ well-matched with \Adversary and $(\pi, \F_1)$: \Simulator is well-matched with $(\phi, \F_2)$, \Environment is well-matched with $(\phi, \Simulator)$, and $\msf{execUC}(\pi, \F_1, \Environment, \Adversary) \approx \msf{execUC}(\phi, \F_2, \Environment, \Simulator)$:

\begin{mathpar}
\footnotesize
	\inferrule*[right=emulate]
	{
		. \models \msf{execUC}[\Tokentypes][\alpha] :: \Delta[\Tokentypes][\alpha] \\ \\
		% Protocols that are well-matched with their functionalities
		\msf{validP}\ \pi \rightarrow \Delta_1' \semi
		\msf{validP} \phi \rightarrow \Delta_2' \semi
		\langle \pi \leftrightarrow \F_2 \rangle, \langle \phi \leftrightarrow \F_1 \rangle \\
		% Type of execUC[DELTA_pi] and execUC[DELTA_phi]
		\Delta_1'[\Tokentypes][\mathrm{T}_{\pi}] \equiv_{\Environment} \Delta_1\ 
		\semi \Delta_2'[\Tokentypes][\mathrm{T}_{\phi}] \equiv_\Environment \Delta_2 \\
		% For all A if exists well-typed A that is well-matched with real world
		\forall \Adversary, (\exists (\Delta_4, \Delta_4') | \Delta_4 \vdash \Adversary :: \Delta_4',\ \langle \Adversary \leftrightarrow \pi \rangle, \langle \Adversary \leftrightarrow \F_1 \rangle \\
		% implies simulator that is well-matched for ideal world
		\Rightarrow \exists (\Delta_3,\Delta_3') | \Delta_3 \vdash \Simulator_\Adversary :: \Delta_3', \langle \Simulator_\Adversary \leftrightarrow \phi \rangle, \langle \Simulator_\Adversary \leftrightarrow \F_2 \rangle \\
		% for all Z they that's well-matched for the real world => Z is well-matched with S and ideal world
		\forall \Environment (\langle \Environment \leftrightarrow \Adversary \rangle, \langle \Environment \leftrightarrow \pi \rangle \Rightarrow \langle \Environment \leftrightarrow \Simulator_\Adversary \rangle, \langle \Environment \leftrightarrow \phi \rangle \\
		% and emulation has to hold
		\msf{execUC} \ \pi\ \Environment\ \F_1\ \Adversary \approx\ \msf{execUC} \ \phi\ \Environment\ \F_2\ \Simulator_\Adversary))
	}
	{
		% EMULATION DEFINITION
		\lambda \Adversary . \Simulator_\Adversary \vdash (\pi, \F_1) \sim (\phi, \F_2)
	}
\end{mathpar}
\end{definition}


Particularly, we care about emulation with respect to an ideal protocol $\phi$ which is really just $(\idealP, \F)$ where \idealP is the ideal protocol which forwards all messages to/from \Environment and \F.
We say the protocol $\pi$ UC-realizes an ideal functionality $\F_2$ if Definition~\ref{def:emulation} holds for $(\pi, \F_1)$ and  $\phi = (\idealP, \F_2)$

\begin{definition}[UC-Realize]
A protocol $\pi$ UC-realized an ideal functionality $\F_1$ if $(\pi, \F_2) \sim (\idealP, \F_1)$ for some $\F_2$.
\end{definition}

This definition does not explicitly mention the import requirements of the types of the channels in the context in each of the two protocols. 
We elide this point in the definition because the type system already provides this guarantee given the UC execution model.
The same environment \Environment is run for both protocols (and both worlds), and, therefore, the UC execution type checks when \Environment is well-matched for both worlds (i.e. gives the same messages and the same import along all channels).

%\paragraph{Simulation Proofs}
%Simulation proofs in Nomos UC rely heavily on the type system to ensur that the constructed simlation is well-matched with the protocols in the ideal world and that it is locally polynomially-bound.
%Showing that a simulator is well-matched with $\phi$ and environments \Environment that are well-matched with \Adversary requires ensuring that \Simulator is well-typed under the type parameters \msf{z2a}, \msf{p2a}, \msf{f2a} and their analogues for the opposite direction of communication. 
%Well-typed, though, only captures the fact that the import sent with messages to \Simulator has \textit{enough} import to perform all of its computation on activation. 
%To show  polynomial in PPT it suffices to provide a bounding polynomial $T$ such that the import \Simulator receives, $n$, is sufficient to bound its run-time by $T(n)$. 
%
%Finally, indistinguishability of the distributions of \msf{execUC} requires standard UC reasoning about the message sent by \Simulator, when they are sent and their affect on the output received by \Environment.

\subsection{Dummy Lemma}
The dummy lemma makes use of the \msf{runInSandbox} and \msf{withdrawTokens} program definisions defined in Section~\ref{sec:nomosuc}.
These to programs enable process re-use to simplify complicated protocols, such as a simulator, which can refer to existing processes as black boxes.
Virtual tokens are required to satisfy the message and import types of the processes being simulated and exists only to enable this behavior.
The actual import usage required by a machine simulating another is no more than if all code was run natively within one process.

The Lemma states that if dummy simulator satisfies emulation with respect to the dummy adversary, then for any \Adversary a simulator can be constructed with the dummy simulator. 
The constructed simulator runs \Adversary and \Dummysim internally. It sends messages from \Environment to \Adversary and outputs of \Adversary to \Dummysim.
At a high level the proof asserts that the constructed simulator is the same as \Environment running \Adversary internally and sending its output to \Dummysim -- an environment for which dummy emulation is already guaranteed.


\begin{theorem}[Dummy Lemma]\label{thm:dummy}
If $\exists \Dummysim^{K}$ s.t. $ \DummyAdv^K, \Dummysim^K \vdash (\pi, \F_2) \sim (\phi, \F_1)$ then $\forall \Adversary \ \exists \Simulator_\Adversary^K$ s.t. $\Simulator_{\Adversary} \vdash  (\pi, \F_2) \sim (\phi, \F_1)$ 
%For protocols $\pi$ and $\phi$, $\msf{execUC}(\pi, \Environment, \DummyAdv) \sim \msf{execUC}(\phi, \Environment, \Simulator_{\mathcal{D}}) \forall \Environment \Rightarrow \exists$ \textit{well-resource-typed, PPT} $\Simulator_{\mathcal{D}}$ s.t. $\msf{execUC}(\pi, \Environment, \Adversary') \sim \msf{execUC}(\phi, \Environment, \Simulator')$ for any other \textit{well-resource-typed} $\Adversary'$.
\end{theorem}

\begin{proof}
The constructed simulator $\Simulator_\Adversary^K$ internally simulates \Dummysim and \Adversary with respected virtual token type $K_1$. 
We use the internal simulation pattern described above to simulat messages to \Dummysim and \Adversary through virtual tokens.
Recall that the virtual tokens consturction is a tool to make witing complex protocols easier.
In fact it has no impact on the amount of import that the simulating machine requires or receives.
The only difference in running all code natively is the potential usage is higher in managing the simulated processes and messages between them.

On input from \Environment on channel \msf{z2p}, \Simulator:
\begin{lstlisting}[basicstyle=\small\BeraMonottFamily, frame=single,  mathescape, label={lst:sim}]
msg = $\nrecv$ $\$$z2a ;
$\nget$ $\$$z2a {z2an : K} ;
$\tm{withdrawTokens}$ f K K1 z2an ;
$\nsend$ $\$$a_z2a msg ;
$\npay$ {z2an : K1} $\$$a_z2a ; 
\end{lstlisting}

Similarly, on output from \Adversary to a protocol party on channel \msf{a2p}
\begin{lstlisting}[basicstyle=\small\BeraMonottFamily, frame=single,  mathescape]
pid = $\tb{recv}$ $\$$a_a2p ;
msg = $\tb{recv}$ $\$$a_a2p ;
$\tb{get}$ K1 $\$$aa2p {a2pn} ;
$\tb{send}$ $\$$sd_z2a A2P(pid, msg) ;
$\npay$ $\$$sd_z2a {z2an : K1} ;
\end{lstlisting}

$\Simulator_\Adversary$ forwards input from \Environment and forwards it to the internal \Adversary. 
\Adversary output to either the protocol parties or the 

The construction provided is identical to the original dumy lemma in the UC framework~\cite{uc}, however, we must show that the constructed simulator $\Simulator_\Adversary$ is well-resource-typed for all well-resource-typed \Adversary.
By the assumption that \Adversary and \Dummysim are well-resource-typed, $\Simulator_\Adversary^K$ is well-resource-typed because it performs consatnt work routing messages internally. 
A simple and satisfactory runtime polynomial bound for $\Simulator_\Adversary^K$ can be given as:
\[
T(n) = T_{\Adversary,\Dummysim}(n) + T_{\Adversary,\Dummysim}(n) + O(n)
\]
where $T_{\Adversary,\Dummysim}(n)$ is the greater of the two bounding polynomials for \Dummysim and \Adversary evaluated at $n$, and $n$ is the import that \Environment sends to \Adversary. 
The constant factor accounts for the a constant overhead for routing messages between \Dummysim and \Adversary.

The same \textit{well-resource typed} reasoning extends to the token context where amount of virtual tokens created are polyomial in number and generate potential that is bounded by the above bounding polynomial for $\Simulator_\Adversary$.

\end{proof}

\subsection{Single Composition}
In this section we present a simplified composition theorem and another theorem, which we call the \textit{squash theorem}.
These two theorems combine to prove the full generalized composition theorem as it appears in the UC framework~\cite{uc}.

The composition operator defines a way for some protocol $\rho$ that uses a functionality $\F$ to swap $\F$ for a procol $(\pi, \F')$, which realizes $\F$, such that $(\rho, \F) \sim (\phi, \F'') \Rightarrow (\rho \circ \pi, \F') \sim (\phi, \F'')$.
The $\circ$ composition operator is defined in Nomos in Figure~\ref{lst:compose}.

Recall that the Nomos language currently does not support passing processes as arguments to other processes even though the theory allows it. 
In the $\circ$ code the protocols $\pi$ and $\phi$ exist globally.

\begin{figure*}
\begin{lstlisting}[basicstyle=\small\BeraMonottFamily, frame=single,  mathescape]
$\tb{proc}$ compose[K][z2r][r2z][f2r][r2f][p2f][f2p] : 
    (pid: Int), ($\$$z_to_p: c[K][z2p]), ($\$$p_to_z: c[K][r2z]), 
    ($\$$f_to_p: c[K][f2r]), ($\$$p_to_f: c[K][r2f])  |- ($\$$D : 1) =
{
	$\$$rho_to_pi <- $\tm{createchan}$[K][p2f];
	$\$$pi_to_rho <- $\tm{createchan}$[K][f2p];

	 <- pi  <-                 $\$$rho_to_pi $\$$pi_to_rho $\$$p_to_f $\$$f_to_p ;
	 <- phi <- $\$$z_to_p $\$$p_to_z $\$$rho_to_pi $\$$pi_to_rho ; 
}
\end{lstlisting}
\caption{Composition operator in Nomos that connects a protocol $\rho$ to a protocol $\pi$ that uses some functionality $\F$.}
\label{lst:compose} 
\end{figure*}

\todo{Include a graphical illustration of wtf is going on, and going on inside the party wrapper as}

\begin{theorem}[Composition]\label{thm:composition}
\begin{mathpar}
\inferrule*[right=single-compose]
{
	(\pi, \F_1) \sim (\idealP, \F_2) \semi (\rho, \F_2) \sim (\idealP, \F_3) \\
	\Rightarrow \exists \Simulator(\Adversary) \vdash (\rho^{\F_2 \rightarrow \pi}, \F_1) \sim (\idealP, \F_3)
}
{
	(\rho \circ \pi, \F_1) \sim (\idealP, \F_3)
}
\end{mathpar}

If \textit{well-typed} $(\pi, \F_1$) realizes $\F_2$ and ($\rho$, $\F_2$) realizes some $\F_3$, then $(\rho \circ \pi, \F_2)$ is \textit{well-typed} and realizes $\F_3$ when $\circ$ is defined as in Figure~\ref{lst:compose}.
\end{theorem}

\begin{proof}
The pre-condition ensures the existence of a \textit{well-resource-typed} simulator $\Simulator_\pi$ for $(\pi, \F_1) \sim (\idealP, \F_2)$. 
We construct a simulator $S$ which composes $\Simulator_\rho \circ \Simulator_\pi$ for:
\[
	\msf{execUC}\ (\rho \circ \pi)\ \F_1\ \Environment\ \Adversary \approx \msf{execUC}\ \idealP\ \F_3\ \Environment\ \Simulator
\]	

The constructed simulator is trivial as it only relies on \Sim{\pi}.
We don't need to perform simulation on any inputs by \Environment to the main parties of $\rho$ (it's the same protocol in both worlds).
The constructed simulator \Simulator simulates \Sim{\pi} internall and passes messages intended for the parties of $\pi$, or for $\F_2$, to \Sim{\pi} and simulates its computation.
Similariy, \Simulator sends any message from $\F_3$ to \Sim{\pi} for simulation.  
Input to any party of the main protocol $\rho$ from \Environment, or outout from them to \Simulator, are forwarded without any modification or simulation.

The argument follows directly by a series of inferences:
\begin{align}
& \msf{execUC} \: \Environment \, (\rho \circ \pi) \, \F_1 \, \DummyAdv \\
\equiv \; & \msf{execUC} \: (\Environment \circ \rho) \, \pi \, \F_1 \, \DummyAdv \\
\approx \; & \msf{execUC} \: (\Environment \circ \rho) \, \idealP \, \F_2 \, \Sim{\pi} \\
\equiv \; & \msf{execUC} \: \Environment \, \rho \, \F_2 \, \Sim{\pi} 
%\approx \; & \msf{execUC} \: (\Environment \circ \Sim{\pi}) \, \idealP \, \F_3 \, \Sim{\rho} \\
%\equiv \; & \msf{execUC} \: \Environment \, \idealP \, \F_3 \, (\Sim{\pi} \circ \Sim{\rho}) 
\end{align}

In line (13) above, $\rho$ is moved into the execution environment with an unchanged simulator as no additional simulation is required: the simulator allows unfettered communication between parties of $\rho$ and \Environment.
The constructed simulator performs constant overhead in routing messages to the simulated \Sim{\pi} and forwrading messages to/from parties of $\rho$/\Environment. 

Given that \Sim{\pi} is \textit{well-resource-typed}, with bounding polynomial $T_{\Sim{\pi}}$, it suffices to show that an additional linear term is sufficient to create a bounding polynomial for \Simulator.
\end{proof}

\subsection{Multisession}
The multi-session extension of a protocol or functionality, specified by the $!$ operator (such as $!\rho$ or $!\F$), allows multiple instances to be run within a sinlge ITM.
The ITM simulates multiple instances of the protocol/functionality intnerally and multiplexes input/output to/from them in same way as the party wrapper for protocol parties.
The channel from the protocol wrapper to the multisession operator can be typed as:
\begin{gather}
\mi{stype} \; \m{{P2MS}[a]\{n\}} = \echoice{\mb{push}: pid \textasciicircum ssid \textasciicircum a \arrow |\{n\}> \m{P2MS[a]\{n\}}}
\end{gather}
The operator accepts messages of the form $(\msf{ssid}, msg)$ from a particular \msf{pid}, where \msf{ssid} is a sub-session identifier.
If an instance of the functionality with $\msf{sid} := \msf{ssid}$ then $!\F$ creates one and forwards the message to it.
Additionally, $!\F$ listens for outgoing messages from each of the instances and forwards them to the outside execution.
The operator differs from the party wrapper in one crucial way: it only works with functional messages types and does not wrap around any session types like any other standalone functionality in Nomos UC.

The multisession behaves like the protocol wrapper in that we rely on code generation to create the operator for a particular functionality. 
The reason behind this is that the operator simulates many instances of a functionality and must use virtual tokens to communicate with them. 
For the commitment example we've used throughout this paper, the multisession needs only one virtual token type alongside the real token type.
The commitment functionality doesn't internally simulate any other machines and therefore does not need any virtual token type itself. 
The process definition for $!\F_\msf{com}$ is shown in Figure \ref{lst:bangf} accepting two token types: the real token type $K$ and the virtual token type $K_1$ for instances of $\F_\msf{com}$.

The communicators between \bangf and the other ITMs all use the real token type.
Only the internal channels that it creates use virtual token types.
The communication pattern between the operator and the simulated functionalities works in the same was as Listing \ref{lst:sim}.

\begin{figure*}
\begin{lstlisting}[basicstyle=\small\BeraMonottFamily, frame=single, mathescape]
type sid[a] = SID of String ^ a ;

proc bangF_1[K, K1][$p2f$][$f2p$][$a2f$][$f2a$]{$p2fn$}{$f2pn$}{$a2fn$} : 
    ($\$$pw_to_f: P2MS[K][p2f]), ($\$$f_to_pw: MS2P[K][f2p]), ($\$$f_to_a: MS2A[K][f2a]), ($\$$a_to_f: A2MS[K][a2f]),
	($\$\l1: list[sender] ), ($\$$l2: list[scommitted]), ($\$$l3: list[receiver]), ($\$$l4: list[rcommitted]) |- ($\$$ms: 1)
\end{lstlisting}
\caption{The type definition for the multisession operator for functionalities and the correspond message type and import parameters.}
\label{lst:bangf}
\end{figure*}

\begin{theorem}[PPT !]\label{thm:bangppt}
If a functionality $\F$ is well-resource-typed, then it's multisession extension $!\F$ is well-resource-typed.
\end{theorem}

\begin{proof}
A \textit{well-resource-typed} \F guarantees a polynomial $T_{\F}$ bounding its execution.
In the worse-case, the multisession operator must spawn a new instance of $\F$ an every activation. 
Let $N_{\F}$ denote the total number of instances (and, hence, number of activations) of $\F$ created by the operator.
Note that $N_{\F}$ is polynomial in the security parameter $k$ for all well-typed environments, protocols, and adversary.
Therefore, there always exists a bounding polynomial to bound a polynomial number of simulated instances of \F.
The polynomial can be given as:
$$ P_{!\F}(n) = N_{\F} P_{\F}(n) + \mathcal{O}(N_{\F}) $$
where the $\mathcal{O}(N_{\F})$ is due to the overhead of maintaining and accessing the set of all instances.

Similarly, \F being \textit{well-resource-typed} ensures a valid token context for all processes it may simulate. 
Therefore, it is clear that there exists a global connecting poltnomial $f$ that ensures a valid token context for $!\F$.
\end{proof}

\begin{theorem}[Squash Theorem]
%If a functionality \F is well-resource-typed, then $!\F$ and $!!\F$ are well-resource-typed (by Theorem~\ref{thm:bangppt}) and $(\idealP, !!\F) \sim (\msf{squash}, !\F)$.
\textit{Well-resource-typed} \F $\Rightarrow$ $(\idealP, !!\F) \sim (\msf{squash}, !\F)$
\end{theorem}

\begin{proof}
First we describe the \msf{squash} protocol in figure \ref{fig:squash}.
Note that $!!\F$ is nested $!$ operators. The top level process maintains multiple sessions of $!\F$ each with their own \msf{ssid}.
Functionalities in each $!\F[\msf{ssid}]$ have their own \msf{sid}. 

In $(\idealP, !!\F)$, \idealP~expects to receive messages of the form $(\msf{ssid}_1, (\msf{ssid}_2, m))$ where $\msf{ssid_2}$ is a sub-session of $\F$ (i.e. instance) inside some $!\F$ with sub-session id $\msf{ssid}_1$ inside of $!!\F$ (the message accesses functionality $!!\F[\msf{ssid}_1][\msf{ssid}_2]$).
The \msf{squash} protocol flattens the indexing of instances of \F and combines session ids $\msf{ssid}_1$ and $\msf{ssid}_2$ into a single \msf{ssid}: $\msf{ssid}_3 := \msf{ssid}_1 \cdot \msf{ssid}_2$.
If follows intuitively that the view for the environment remains the same. 

We construct a simulator such that:
\[
\msf{execUC} \, \Environment \, \idealP \, !!\F \, \Sim{\msf{squash}} \approx \msf{execUC} \, \Environment \, \msf{squash} \, !\F \DummyAdv 
\]
The simulator is very simple. 
Inputs to/from parties/\Environment for a corrupt party is forwarded unmodified.
Input intended for $!\F$ of the form $(\msf{ssid}_1 \cdot \msf{ssid}_2, msg)$ sends $(\msf{ssid}_1, (\msf{ssid}_2, msg))$ to $!!\F$. 
Output from $!!\F$ is modified inversely and sent to \Environment.

The simulator is clearly \textit{well-typed} 

\end{proof}

