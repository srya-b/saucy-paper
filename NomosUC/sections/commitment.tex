In this section we use our emulation definition to model a simple commitment protocol in the random oracle mode.
We present a complete emulation from the canonical $\F_{\msf{com}}$ definition with a simulator for the dummy adversary.

Recall the functionality \Fcom presented earlier in this paper, and that functionalities in NomosUC are wrapped by other code which converts between functional and session typed messages. 
It accepts messages from parties on two channels: one from the sender and one from the receiver.
The channels don't directly connect to the two parties (recall the functionality wrapper and the protocol wrapper exists between the two).
The types for sender and receiver messages to \Fcom are the following:
\begin{gather}
\mi{stype} \; \m{sender} = \echoice{\mb{commit} : \m{bit} \product \m{scommitted}} \\
\mi{stype} \; \m{scommitted} = \echoice{\mb{open} : 1} \\
\mi{stype} \; \m{receiver} = \ichoice{\mb{committed} : \m{rcommitted}} \\
\mu{stype} \; \m{rcommitted} = \ichoice{\mb{open} : \m{bit} \arrow \one}
\end{gather}

The ideal functionality \Fcom is the same one introduced in Figure~\ref{fig:fcom} in Section~\ref{sec:nomosuc}. 
In FIgure~\ref{lst:fcom} we present the Nomos definition of the same functionality and provide the channel types in FIgure~\ref{fig:fcomtypes}.
We elide some of the clutter of acquiring and releasing shared channels in the form of: \texttt{\$p2f $\leftarrow$ acquire \#p\_to\_f} for clarity. 
Wherever a linear channel like \texttt{\$p2f} is used it is in fact an acquired shared channel \texttt{\#p\_to\_f}.

The key difference to note between \Fcom and its Nomos version is that the functionality is split up into two processes rather than compressed into one.
This design decision is required because of how Nomos cycles between processes in a round-robin fashion and the communicator design.
Therefore, processes must recurse when there is no message to be read and move to the next processes after the first expected message is received--in this case the \msf{P2FCommit(b)} message.


\begin{figure}
\centering
\msf{type} \msf{Ip2f} = \msf{P2FCommit} of \msf{Bit} | \msf{P2FOpen}

\msf{type} \msf{If2p} = \msf{F2PCommit} | \msf{F2POpen} of \msf{Bit}

\msf{type} \msf{Ip2f} = \msf{SCommit} of Bit | \msf{SOpen}

\msf{type} \msf{If2p} = \msf{RCommit} | \msf{ROpen} of Bit

\msf{type} \msf{Rp2f} = \msf{SHash} of \msf{Int} | \msf{Send} of \msf{pid} \textasciicircum \msf{pid} \textasciicircum \msf{Int}

\msf{type} \msf{Rf2p} = \msf{Pre} of \msf{Int} | \msf{RHash} of \msf{Int} | \msf{MSG} of \msf{pid} \textasciicircum \msf{pid} \textasciicircum \msf{Int}

\caption{Types for the channels in the ideal world for \Fcom. Notice that Ip2f and If2p is the type of the channels \msf{z2p} and \msf{p2z} as they much match for both worlds and the ideal world parties simply forward messages to the functionality. The \msf{p2f} and \msf{f2p} channels are specific to the real and ideal world as the functionalities are not the same. Hence the real-world \msf{p2f} is typed with \msf{Rp2f} for the random oracle and the ideal world \msf{p2f} is typed with \msf{Ip2f} for \Fcom.}
\label{fig:fcomtypes}
\end{figure}

\begin{figure*}
\begin{lstlisting}[basicstyle=\small\BeraMonottFamily]
proc F_code:
  (s: sid), (k: Int), (rng: [Bit]), (clist: list[Int]),
  (#p_to_f: comm[pid ^ Ip2f]{Ip2fn}), (#f_to_p: comm[pid ^ If2p]{If2pn}),
  (#a_to_f: comm[Ia2f]{Ia2fn}), (#f_to_a: comm[If2a]{If2an})  |- ($ch: FtOE) =
{
  case $p2f (
    yes =>	
      pid, msg = recv $p2f ;
      get $pwf {Ip2fn} ;
      case msg (
        P2FCommit(b) =>	
          if pid == 1
          then
            send F2PCommit $f2p ;
            pay {If2pn} $f2p ;
            $ch <- F_com_open s k rng clist #p_to_f #f_to_p b ;
          end
      )
   | no =>  
       $ch <- F_code s k rng clist #p_to_f #f_to_p ;
  )
}

proc F_code_open:
{
  case $p2f (
    yes =>	
      pid, msg = recv $p2f ;
      get $p2f {0} ;
      case msg (
        P2FOpen =>	
          if pid == 1
          then
            send F2POpen(b) $f2p ;
            pay {0} $f2p ;
            $ch <- 1 ;
          end
      )
   | no =>
       $ch <- F_code_open s k rng clist #p_to_f #f_to_p b ;
  )
}
\end{lstlisting}
\end{figure*}

The real world protocol for commitment follows a simple communication patter:
\begin{enumerate}
\item On input bit $(\msf{P2FCommit}\ b)$ from the environment, the committer queries the random oracle with the message $(\msf{SHash}\ b | r)$ where $r \xleftarrow{\$} \{0,1\}^k$.
The returned ``hash value'' is sent to the receiver as the commitment.
\item On input \msf{P2FOpen} from \Environment, the committer sends $(\msf{Send}\ p_L\ b\ r)$ to the receiver.
\item The receiver checks that the commitment is correct be querying \Fro in the same way and asserting that the has returned $(\msf{RHash}\ h)$ is the same as the one sent by $p_C$.
\item \todo{the type of Rf2p is kind of wrong so need to correct it}
\end{enumerate}

We provide only a simulator for the dummy adversary as that guarantees a simulator for all adversaries required our emulation definition.
The simulator for commitment is relatively simple so we only provide a high-level description here and leave the full simulator code to the appendix.
The simulator internally simulats the random oracle by maintaining a table of key-value pairs that it can control entirely.
If the committer is corrupt:
\begin{enumerate}
\item The simulator can not determine the bit \Environment wants to commit to so selects a random bit when activatd by the environment and gives it as input to the corrupted committer.
\item When it's asked to open the commitment it simply forwards the request to the corrupt committer and stops.
\end{enumerate}
If the receiver is corrupt:
\begin{enumerate}
\item When activated by the receiver with (\msf{F2PCommit}), the simulator generates some random string $h$ to represent the commitment, stores it, and sends ($\msf{P2A}\ \msf{MSG}(p_C, p_R, h)$) to \Environment.
\item When it receives ($\msf{F2POpen}\ b$) from the receiver, it returns $(\msf{P2A}\ \msf{MSG}(p_C, p_R, b, r)$ to \Environment where $r$ is a randomly generated sequence keeping the pair $(b | r, h)$ as the corresponding entry in the table.
\item When activated by \Environment to check the commitment, \Simulator simply returns the commitment hash or creates a new one.
\end{enumerate}

\paragraph{Simulator Well-Matched}
It is immediately obvious that the constructed simulator is well-typed if the dummy adversary is well-typed with the given type parameters.
The simulator receives 1 import token per activation from \Environment which suffices to simulated \Fro internally. 
Subsequently, \Simulator keeps all of the import it receives, and, therefore when one of the partiesis corrupt a simple bounding polynomial can be given as:
\[
	T_{\Dummysim}(n) = T_{\Fro}(n) + O(1)
\]
where $T_{\Fro}$ is a satisfying polynomial for \Fro. The additional constant factor simply accounts for sending messages to the corrupt parties.
Therefore,
\begin{gather}
	\forall \Environment, \langle \Environment \leftrightarrow \DummyAdv \rangle \Rightarrow \langle \Environment \leftrightarrow \Dummysim \rangle
\end{gather}


\begin{figure*}
\begin{lstlisting}[basicstyle=\BeraMonottFamily]
(* Z2P interface *)
type Ip2f = P2FCommit of Bit | P2FOpen ;
type If2p = F2PCommit | F2POpen of Bit ;

(* Ideal World *)
type Ip2f = SCommit of Bit | SOpen ;
type If2p = RCommit | ROpen of Bit ;

type Ia2f = 0 ;
type If2a = 0 ;

type Ia2p = Ip2f ;	(* crupt input is same as z2p *)
type Ip2a = If2p ;

(* Real World *)
type Rp2f = SHash of Int | Send of pid ^ pid ^ Int ;
type Rf2p = Pre of Int | RHash of Int | MSG of pid ^ pid ^ Int ;

type Ra2f = A2Hash of Int ;
type Rf2a = Hash2A of Int ;

type Ra2p = Rp2f ;
type Rp2a = Rf2p ;

(* the import here is given as those for the dummy adversary in the real world *)
p2zn <- 0 ; z2pn <- 1 ;
a2zn <- 0 ; z2an <- 1 ; 

Rf2pn <- 0 ; Rp2fn <- 1 ;
Rp2an <- 0 ; Ra2pn <- 1 ;
Rf2an <- 0 ; Ra2fn <- 1 ;

If2pn <- 0 ; Ip2fn <- 0 ;
Ip2an <- 0 ; Ia2pn <- 0 ;
I

(* channels *)
#z_to_p <- comm[pid ^ Ip2f]
#p_to_z <- comm[pid ^ If2p]
#z_to_a <- comm[ z2d[Ra2p][Ra2f] ] ;
#z_to_z <- comm[ d2z[Rp2a][Rf2a] ] ;


(* Real World exec PI *)
execUC[Ip2f][If2p][Rp2f][Rf2p][Rp2a][Ra2p][Rf2a][Ra2f][a2z][z2a]
	  {p2zn}{z2pn}{f2pn}{p2fn}{p2an}{a2pn}{f2an}{a2fn}{a2zn}{z2an}

(* Ideal world exec PHI *)
execUC[If2p][Ip2f][Ip2f][If2p][Ip2a][Ia2p][If2a][Ia2f][a2z][z2a]
	  {p2zn}{z2pn}






\end{lstlisting}
\end{figure*}
