In this section we give more extensive code on how the multisession extension work. We also explore Theorems \ref{thm:squash} and \ref{thm:functor} in greater detail, and, specifically address the proof obligation for both.

\subsection{!\F}
The multisession operator presents the same interface to \inline{execUC} as any other functionality.
However, the shell code that we run it inside operates directly on functional messages from its communicators with \F and the \partywrapper instead of performing any conversion to session-types.
The reason for this choice is that a session type for the operator is not very meaningful as it only provides an interface of ``input'' and ``ouput'' to/from underlying instances of the functionality.

!\F communicates with the \partywrapper through the type
\begin{lstlisting}[basicstyle=\small\BeraMonottFamily, mathescape]
$\yo{type}$ p2ms[a] = P2MS of ssid ^ a ;
$\yo{type}$ ms2p[b] = MS2P of ssid ^ b ;
\end{lstlisting}
and with \A through the type
\begin{lstlisting}[basicstyle=\small\BeraMonottFamily, mathescape]
$\yo{type}$ a2ms[a] = A2MS of ssid ^ a ;
$\yo{type}$ ms2a[b] = MS2A of ssid ^ b ;
\end{lstlisting}
both of which are parameterized by the functionality message types \inline{a} and \inline{b}.


Recall that all functionalities are run inside some shell code, and their shell code communicates with communicators to other parities--we can call this a functionality wrapper as it wraps around the actual functionality code.
The multisession operator runs each instance of the functionality inside the wrapper, and provides each with ``virtual'' communicators for \A and the \partywrapper.

The !\F, on input from the \partywrapper, reads a message from the communicator and then does the following:
\begin{lstlisting}[basicstyle=\footnotesize\BeraMonottFamily, frame=single, mathescape, numbers=left]
$\nproc$ f_multisession_p2f_input[K][K1]{p2fn,f2pn} : 
  (k: Int), (rng: [Bit]), (sid: session[a]), (crupt: list[pid]),
  (#p2f: comm[p2fmsg[p2ms]][K]), (#f2p: comm[f2pmsg[ms2p]][K]), 
  (#a2f: comm[a2fmsg[a2ms]][K]), (#f2a: comm[f2amsg[ms2a]][K]) |- ($\$$ch: 1) =
{
  ...
  msg = $\nrecv$ #p2f ;
  $\ncase$ msg (
    P2F(pid, P2MS(ssid, msg)) =>
      $\nif$ not ssid in $\$$p2ssid
      $\nthen$
      	#new_p2ssid <- communicator_init[K1][p2f] <- ;
      	#new_ssid2p <- communicator_init[K1][f2p] <- ;
      	#new_a2ssid <- communicator_init[K1][a2f] <- ;
      	#new_ssid2a <- communicator_init[K1][f2a] <- ;
      
      	$\$$p2ssidnew <- pappend $\$$p2ssid #new_p2ssid ;
      	$\$$ssid2pnew <- pappend $\$$ssid2p #new_ssid2p ;
      	$\$$a2ssidnew <- pappend $\$$a2ssid #new_a2ssid ;
      	$\$$ssid2anew <- pappend $\$$ssid2a #new_ssid2a ;
      
      	$\$$chprime <- f_wrapper[K1][f2p,p2f][f2a,a2f] <- 
                           k rng sid clist $\#$p2ssid $\#$ssid2p $\#$a2ssid $\#$ssid2a $\#$z ;
      $\nend$
  
      #ch <- get_channel #p2ssid ssid ;
      $\nwithdraw$ K K1 {p2fn} ;
      $\$$ch.SEND ;
      $\nsend$ $\$$ch P2F(pid, msg) ;
      $\npay$ $\$$ch {p2fn : K1} ;
  	
    $\$$ch <- f_multisession_f2p_i[K][K1] <- .... $\$$p2ssidnew $\$$ssid2pnew $\$$a2ssidnew $\$$ssid2anew 0;	
}
\end{lstlisting}

First it checks whether the \m{ssid} exists. It not it spawns new communicators for the instance \F: two for the \partywrapper and two for \A.
It then grabs the \inline{#p2f} channel for $\F_{\m{ssid}}$, creates the appropriate virtual tokens and sends the message along that channel.
Finally, after handling the incoming message, it moves to the next step of checking for outgoing messages, to the \partywrapper, from each \F by calling \inline{f_multisession_f2p_i}.
The function accepts an integer at the end, in this example 0, which is the index to check in the communicator list. Eventually all communicators in the list are checked.

Recall that the Theorem~\ref{sec:squash} realizes !!\F, a functionality wrapped twice in the multisession operator an indexed by a pair of ssids $(\m{ssid}_1 \product \m{ssid}_2)$.
The real world that realizes it is comprised of a protocol that ``flattens'' the pair of ssids $(\m{ssid}_1, \m{ssid}_2)$ into a single ssid $\m{ssid}_3 = \m{ssid}_1 || \m{ssid}_2$, the concatenation (can be any other one-to-one and invertible function on the pair) of them.
The simulator for this construction is a direct simulation where $\Sim{\m{squash}}$ takes as input one of 
\begin{itemize}
	\item \inline{Z2A2P(pid, P2MS(ssid, msg))}: $\SIM{\m{squash}}$ un-concatenates \inline{ssid} into a pair (\inline{ssid1}, \inline{ssid2}) and forward the message \inline{A2P(pid, P2MS(ssid1, P2MS(ssid2, msg)))}.
	\item \inline{Z2A2F(A2MS(ssid, msg))}: Similar to above, $\SIM{\m{squash}}$ splits the given \inline{ssid} and sends \\ \inline{A2F(A2MS(ssid1, A2MS(ssid2, msg)))}. 
\end{itemize}

\SIM{\m{squash}} also keeps one import for every message it ges and forwards the remainder to either of \F or the \partywrapper.
On input from \Z, \SIM{\m{squash}} does the following; 

\begin{lstlisting}[basicstyle=\footnotesize\BeraMonottFamily, frame=single, mathescape]
proc sim_squash[K][p2f,f2p,a2f,f2a]{p2fn,f2pn,a2fn} :
  (k: Int), (rng: [Bit]), (sid: session[a]), (crupt: list[pid]),
  (#z_to_a: comm[z2amsg]), (#a_to_z: comm[a2zmsg]) ... |- ($\$$ch: 1) =
{
  ...
  $\ncase$ #z_to_a (
    Z2A2P(pid, P2MS(ssid, msg))) =>
      ssid1, ssid2 <- split ssid
      #a_to_p.SEND ;
      $\npay$ #a_to_p {a2pn-1} ;
      $\nsend$ #a_to_p A2P(pid, P2MS(ssid2, P2MS(ssid2, msg)) ;
    Z2A2F(A2MS(ssid, msg)) =>
      ssid1,ssid2 <- split ssid
      #a_to_f.SEND ; 
      $\npay$ #a_to_f {a2fn-1} ;
	  $\nsend$ #a_to_f A2P(A2MS(ssid1, A2MS(ssid2, msg))) ;
  ...
}	  
\end{lstlisting}

We revisit Theorem~\ref{thm:squash} and argue that we satisfy this theorem. 
It is clear that from a direct simulation, it is clear that emulation holds, and we have that 
\[
	!\F \xrightarrow{\m{squash}} !!\F
\] 
It remains to argue that $\SIM{\m{squash}}$ is PPT in it's construction. We refer to the trivial nature of \SIM{\m{squash}} that does constant work on any activation by \F, \Z, or by the \partywrapper. Therefore, we conclude that \SIM{\m{squash}} is obviously PPT.

We further require Theorem~\ref{thm:functor} in order to conclude full UC-style composition. We restate it here
\begin{theorem}[Something About Functors]
	\begin{mathpar}
		\inferrule*[right=somefunctor]
		{
			\F_1 \xrightarrow{\pi} \F_2
		}
		{
			!\F_1 \xrightarrow{!\pi} !\F_2
		}
	\end{mathpar}
\end{theorem}

The simulator for this theorem runs internal copies of $\Sim_\pi$ for each new \m{ssid} spawned.
It is clear to see why simulating in isolation works given that the instances of the protocol and functionality don't share any state with each other.
However, the proof of this theorem relies on the creating a reduction from an environment that can distinguish the two worlds in this theorem to an environment that can distinguish emulation of $pi$ realizing $\F$.

The proof proeeeds in... \anote{TODO}

\subsection{Squash Theorem}
\begin{theorem}[Squash Theorem]\label{thm:squash}
	\begin{mathpar}
		\inferrule*[right=squash]
		{
			\textit{well-resource-typed} \; \F
		}
		{
			!\F \xrightarrow{\msf{squash}} !!\F
		}
	\end{mathpar}
\end{theorem}
The Squash theorem, is more straightforward to prove than Theorem~\ref{thm:functor}.

\begin{proof} (Theorem~\ref{thm:squash})
On examination the \msf{squash} theorem does constant work per activation and concantenates two \inline{ssid}s into one for messages going to the functionality and does the inverse for messages incoming from the functionality.
In other words, it is a direct simulation and the same ideal functionality exists if both worlds, and it is clear to see that the this theorem holds with a trivial simulator.
%First we describe the \msf{squash} protocol where $!!\F$ are nested $!$ operators.
%The protocol accepts messages intended for $!!\F$ of type \inline{P2MS[p2ms[a]][ms2p[b]]}, i.e. of the form $(\msf{ssid}_1, (\msf{ssid}_2, msg))$, and ``flattens'' them into a single message of type $\inline{P2MS[a][b]}$, i.e. of the form $(\msf{ssid}_3, msg)$.
%
%In $(\idealP, !!\F)$, \idealP~expects to receive messages of the form $(\msf{ssid}_1, (\msf{ssid}_2, m))$ where $\msf{ssid_2}$ is a sub-session of $\F$ (i.e. instance) inside some $!\F$ with sub-session id $\msf{ssid}_1$ inside of $!!\F$ (the message accesses functionality $\F[\msf{ssid}_1][\msf{ssid}_2]$).
%The \msf{squash} protocol flattens the indexing of instances of \F and combines session ids $\msf{ssid}_1$ and $\msf{ssid}_2$ into a single \msf{ssid}: $\msf{ssid}_3 := \msf{ssid}_1 \cdot \msf{ssid}_2$.
%If follows intuitively that the view for the environment remains the same. 
%
%We construct a simulator such that:
%\[
%\msf{execUC} \, \Z \, \idealP \, !!\F \, \SIM{\msf{squash}} \approx \msf{execUC} \, \Z \, \msf{squash} \, !\F \DA 
%\]
%The simulator is very simple. 
%Inputs to/from parties/\Z for a corrupt party is forwarded unmodified.
%Input intended for $!\F$ of the form $(\msf{ssid}_1 \cdot \msf{ssid}_2, msg)$ is sent as $(\msf{ssid}_1, (\msf{ssid}_2, msg))$ to $!!\F$. 
%Output from $!!\F$ is modified inversely and sent to \Z.
%
%The proposed simulator is trivially analyzed to be \textit{well-resource-typed}.
%It performs constant work per activation and does ``real'' simulation other than message modification to/from $!!\F$.
\end{proof}
