In this section we give more extensive code on how themultisession extension work. We also exlore Theorems \ref{thm:squash} and \ref{thm:functor} in greater detail, and, specifically address the proof obligation for both.

\subsection{!\F}
The multisession operator presents the same interface to \inline{execUC} as any other functionality.
However, the shell code that we run it inside operates directly on functional messages from its communicators with \F and the \partywrapper instead of performing any conversion to session-types.
The reason for this choice is that a session type for the operator is not very meaningful as it only provides an interface of ``input'' and ``ouput'' to/from underlying instances of the functionality.

!\F communicates with the \partywrapper through the type
\begin{lstlisting}[basicstyle=\small\BeraMonottFamily, mathescape]
$\yo{type}$ p2ms[a] = P2MS of ssid ^ a ;
$\yo{type}$ ms2p[b] = MS2P of ssid ^ b ;
\end{lstlisting}
and with \A through the type
\begin{lstlisting}[basicstyle=\small\BeraMonottFamily, mathescape]
$\yo{type}$ a2ms[a] = A2MS of ssid ^ a ;
$\yo{type}$ ms2a[b] = MS2A of ssid ^ b ;
\end{lstlisting}
both of which are parameterized by the functionality message types \inline{a} and \inline{b}.


Recall that all functionalities are run inside some shell code, and their shell code communicates with communicators to other parities--we can call this a functionality wrapper as it wraps around the actual functionality code.
The multisession operator runs each instance of the functionaly inside the wrapper, and provides each with ``virtual'' communicators for \A and the \partywrapper.

For input from the \partywrapper the multisession extension reads a message from the communicator and then does the following:
\begin{lstlisting}[basicstyle=\footnotesize\BeraMonottFamily, frame=single, mathescape]
P2F(pid, P2MS(ssid, msg)) = $\nrecv$ $\$$p2f ;
$\nif$ not ssid in $\$$p2ssid
$\nthen$
	#new_p2ssid <- communicator_init[K1][p2f] <- ;
	#new_ssid2p <- communicator_init[K1][f2p] <- ;
	#new_a2ssid <- communicator_init[K1][a2f] <- ;
	#new_ssid2a <- communicator_init[K1][f2a] <- ;

	$\$$p2ssidnew <- pappend $\$$p2ssid #new_p2ssid ;
	$\$$ssid2pnew <- pappend $\$$ssid2p #new_ssid2p ;
	$\$$a2ssidnew <- pappend $\$$a2ssid #new_a2ssid ;
	$\$$ssid2anew <- pappend $\$$ssid2a #new_ssid2a ;

	$\$$chprime <- f_wrapper[K1][f2p,p2f][f2a,a2f] <- 
                     k rng sid clist $\#$p2ssid $\#$ssid2p $\#$a2ssid $\#$ssid2a $\#$z ;
$\nend$

#ch <- get_channel #p2ssid ssid ;
$\nwithdraw$ K K1 {p2fn} ;
$\$$ch.SEND ;
$\nsend$ $\$$ch P2F(pid, msg) ;
$\npay$ $\$$ch {p2fn : K1} ;
	
$\$$ch <- f_ms_p2f_i[K][K1] <- .... $\$$p2ssidnew $\$$ssid2pnew $\$$a2ssidnew $\$$ssid2anew 0;	
\end{lstlisting}
If an instance with the given $\m{ssid}$ does not exist, it creates one by calling the shell code for it, and creating two new virtual tokens.
Then it withdraws the appropriate number of virtual tokens \inline{K1} and sends it along with the message to the input communicator for the appropriate instance of \F.
Finally, !\F moves on to checking for new outgoing messages on the output communicators, to the \partywrapper, of each existing instance of \F, starting with index 0 in the communicator list.


Recall that the Theorem~\ref{sec:squash} realizes !!\F, a functionality wrapped twice in the multisession operator an indexed by a pair of ssids $(\m{ssid}_1 \product \m{ssid}_2)$.
The real world that realizes it is comprised of a protocol that ``flattens'' the pair of ssids $(\m{ssid}_1, \m{ssid}_2)$ into a single ssid $\m{ssid}_3 = \m{ssid}_1 || \m{ssid}_2$, the concatenation (can be any other one-to-one and invertible function on the pair) of them.
The simulator for this construction is a direct simulation where $\Sim{\m{squash}}$ takes as input one of 
\begin{itemize}
	\item \inline{Z2A2P(pid, P2MS(ssid, msg))}: $\SIM{\m{squash}}$ un-concatenates \inline{ssid} into a pair (\inline{ssid1}, \inline{ssid2}) and forward the message \inline{A2P(pid, P2MS(ssid1, P2MS(ssid2, msg)))}.
	\item \inline{Z2A2F(A2MS(ssid, msg))}: Similar to above, $\SIM{\m{squash}}$ splits the given \inline{ssid} and sends \\ \inline{A2F(A2MS(ssid1, A2MS(ssid2, msg)))}. 
\end{itemize}

\SIM{\m{squash}} also keeps one import for every message it ges and forwards the remainder to either of \F or the \partywrapper.
On input from \Z, \SIM{\m{squash}} does the following; 

\begin{lstlisting}[basicstyle=\footnotesize\BeraMonottFamily, frame=single, mathescape]
proc sim_squash[K][p2f,f2p,a2f,f2a]{p2fn,f2pn,a2fn} :
  (k: Int), (rng: [Bit]), (sid: session[a]), (crupt: list[pid]),
  (#z_to_a: comm[z2amsg]), (#a_to_z: comm[a2zmsg]) ... |- ($\$$ch: 1) =
{
  ...
  $\ncase$ #z_to_a (
    Z2A2P(pid, P2MS(ssid, msg))) =>
      ssid1, ssid2 <- split ssid
      #a_to_p.SEND ;
      $\npay$ #a_to_p {a2pn-1} ;
      $\nsend$ #a_to_p A2P(pid, P2MS(ssid2, P2MS(ssid2, msg)) ;
    Z2A2F(A2MS(ssid, msg)) =>
      ssid1,ssid2 <- split ssid
      #a_to_f.SEND ; 
      $\npay$ #a_to_f {a2fn-1} ;
	  $\nsend$ #a_to_f A2P(A2MS(ssid1, A2MS(ssid2, msg))) ;
  ...
}	  
\end{lstlisting}
