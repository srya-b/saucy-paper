\documentclass[acmsmall, screen, review, anonymous]{acmart}
%\IEEEoverridecommandlockouts
% The preceding line is only needed to identify funding in the first footnote. If that is unneeded, please comment it out.
%\usepackage[dvipsnames]{xcolor}
%\usepackage{cite}
\usepackage[most]{tcolorbox}
%\usepackage{amsmath,amssymb,amsfonts,amsthm}
\usepackage{algorithmic}
\usepackage{graphicx}
\usepackage{subcaption}
\usepackage{textcomp}
\usepackage{mathtools}
\usepackage[shortlabels]{enumitem}
\usepackage[T1]{fontenc}
\usepackage{listings}
\usepackage{tabularx}
\usepackage{bbm}
%\usepackage{unicode-math}
\usepackage[utf8]{inputenc}
\usepackage{newunicodechar}
\usepackage{multirow}
\usepackage{booktabs}
\usepackage{adjustbox}

%% PL packages
\usepackage{stmaryrd} 
\usepackage{proof}
\usepackage{mathpartir}
%\usepackage{color}
\usepackage{xstring}
\usepackage{xspace}
\usepackage{turnstile}

\def\BibTeX{{\rm B\kern-.05em{\sc i\kern-.025em b}\kern-.08em
    T\kern-.1667em\lower.7ex\hbox{E}\kern-.125emX}}
\begin{document}
\usetikzlibrary{matrix, arrows.meta, calc, positioning}
\tikzset{myarrow/.style={-Latex, rounded corners},}

\newcommand*\emptycirc[1][1ex]{\tikz\draw (0,0) circle (#1);} 
\newcommand*\halfcircleft[1][1ex]{%
  \begin{tikzpicture}
  \draw[fill] (0,0)-- (90:#1) arc (90:270:#1) -- cycle ;
  \draw (0,0) circle (#1);
  \end{tikzpicture}}
\newcommand*\halfcircright[1][1ex]{%
  \begin{tikzpicture}
  \draw[fill] (0,0)-- (0:#1) arc (0:90:#1) -- cycle ;
  \draw[fill] (0,0)-- (270:#1) arc (270:360:#1) -- cycle;
  \draw (0,0) circle (#1);
  \end{tikzpicture}}
\newcommand*\fullcirc[1][1ex]{\tikz\fill (0,0) circle (#1);} 

\newcolumntype{R}[2]{
	>{\adjustbox{angle=#1, lap=\width-(#2)}\bgroup}
	c
	<{\egroup}
}
\newcommand*\rot{\multicolumn{1}{R{30}{1.5em}}}


\definecolor{vert}{RGB}{0,181,0}
\definecolor{oran}{RGB}{223,74,0}
\definecolor{viol}{RGB}{134,0,175}
\definecolor{roug}{RGB}{215,15,0}
\definecolor{bb}{RGB}{0,0,0}
\definecolor{gg}{RGB}{220,220,220}
\definecolor{royalblue}{rgb}{0.25, 0.41, 0.88}
\definecolor{forestgreen}{rgb}{0.13, 0.55, 0.13}
\definecolor{YellowOrange}{rgb}{0.98, 0.6, 0.01}
\definecolor{Red}{rgb}{0.89, 0.0, 0.13}
\definecolor{Black}{rgb}{0.0, 0.0, 0.0}
\definecolor{Purple}{rgb}{0.63, 0.36, 0.94}
\definecolor{purp}{rgb}{0.59, 0.48, 0.71}

\newcommand{\anote}[1]{{\color{magenta}{AM: {{#1}}}}}
\newcommand{\snote}[1]{{\color{green}{SB: {{#1}}}}}

\newtcolorbox[auto counter]{bbox}[2][]{%
    colback=white,
    colframe=bb,
    %colbacktitle=white!90!roug,
	colbacktitle=white!40!gg,
    coltitle=black,
    fonttitle=\small\bfseries, 
	fontupper=\small,
	fontlower=\small,
    enhanced,
    attach boxed title to top left={yshift=-2mm, xshift=0.5cm},%
    #1,% For possible options
}

\mathchardef\hyp="2D
\mathchardef\car="5E

\makeatletter
\newcommand\BeraMonottFamily{%
	\def\fvm@scale{0.85}%
	\fontfamily{fvm}\selectfont
}
\makeatother

\title{Nomos-UC: a programming framework for cryptography based on resource-aware session types\\
%\thanks{Identify applicable funding agency here. If none, delete this.}
}

\newcommand{\mc}[1]{\ensuremath{\mathcal{#1}}}
\newcommand{\msf}[1]{\ensuremath{{\mathsf {#1}}}}
\newcommand{\mathc}[1]{\ensuremath{\mathcal{#1}}}
\newcommand{\tsc}[1]{\textsc{#1}}
\newcommand{\f}[1]{\ensuremath{\mathcal{#1}}\xspace}
\newcommand{\F}{\f{F}}
\newcommand{\PI}{\ensuremath{\pi}\xspace}
\newcommand{\RHO}{\ensuremath{\rho}\xspace}
\newcommand{\achan}{\ensuremath{\F_{\msf{achan}}^{p_r,p_s}}}
%\newcommand{\C}{\mathcal{C}}
\newcommand{\con}[1]{\msf{Contract_{#1}}}
%\newcommand{\Fsync}[2]{\ensuremath{\F_{\msf{sync},#1,#2}}}
\newcommand{\Fsync}[2]{\ensuremath{\F_{\msf{BD-SEC}}(#1,#2)}}
\newcommand{\Fchan}[2]{\ensuremath{\F_{\msf{chan}}(#1,#2)}}
\newcommand{\Fbdsec}{\ensuremath{\F_{\msf{BD-SEC}}^{\delta,\ell}}}
\newcommand{\Fbc}{\ensuremath{\F_{\msf{broadcast}}}}
\newcommand{\Fsfe}{\ensuremath{\F_{\msf{SFE}}}}
\newcommand{\Fstate}{\ensuremath{\F_{\msf{state}}}}
\newcommand{\Fclock}{\ensuremath{\F_{\msf{clock}}}}
\newcommand{\Frbc}{\ensuremath{\F_{\msf{rbc}}}}
\newcommand{\Fpay}{\ensuremath{\F_{\msf{pay}}}}
\newcommand{\Fcom}{\ensuremath{\F_{\msf{com}}}\xspace}
\newcommand{\Fauth}{\ensuremath{\F_{\msf{auth}}}\xspace}
\newcommand{\Fflip}{\ensuremath{\F_{\msf{coinflip}}}\xspace}
\newcommand{\Fro}{\ensuremath{\F_{\msf{RO}}}\xspace}
\renewcommand{\O}[1]{\ensuremath{\mathcal{O}(#1)}\xspace}
\newcommand{\kbits}{\ensuremath{\{0,1\}^k}\xspace}
\newcommand{\samplek}{\ensuremath{\xleftarrow{\$} \kbits}\xspace}
\newcommand{\Fsmc}{\ensuremath{\F_{\msf{SMC}}}\xspace}
\newcommand{\Fropp}{\ensuremath{\F_{\msf{P2P\hyp RO}}}\xspace}
\newcommand{\Gledger}{\ensuremath{\f{G}_{\msf{ledger}}}}
\newcommand{\Wsync}{\ensuremath{\mathcal{W}_{\msf{sync}}}}
\newcommand{\Wasync}{\ensuremath{\mathcal{W}_{\msf{async}}}}
\newcommand{\Ssyncbracha}{\ensuremath{\mathc{S}_{\msf{sbracha}}}}
\newcommand{\Fbracha}{\ensuremath{\mathcal{F}_{\msf{bracha}}}}
\newcommand{\Schedule}{\tsc{Schedule}}
\newcommand{\Delay}{\tsc{Delay}}
\newcommand{\Advance}{\tsc{Advance}}
\newcommand{\Exec}{\tsc{Exec}}
%\newcommand{\Adversary}{\ensuremath{\mathcal{A}}\xspace}
\newcommand{\A}{\ensuremath{\mathcal{A}}\xspace}
\newcommand{\DummyAdv}{\ensuremath{\mathcal{A}_\mathcal{D}}\xspace}
\newcommand{\DA}{\ensuremath{\A_\mathcal{D}}\xspace}
\newcommand{\Sim}{\ensuremath{\mathcal{S}}\xspace}
\newcommand{\SIM}[1]{\ensuremath{\mathcal{S}_{#1}}\xspace}
\newcommand{\simcom}{\SIM{\msf{com}}}
\newcommand{\cf}{\ensuremath{\mathcal{C}}\xspace}
\newcommand{\ID}[1]{\ensuremath{\mathcal{I}(#1)}\xspace}
%\newcommand{\Sim}[1][]{\ifthenelse{\equal{#1}{}}{\ensuremath{\Simulator}}{\ensuremath{\Simulator_{#1}}}}
\newcommand{\DS}{\SIM{D}\xspace}
%\newcommand{\Environment}{\ensuremath{\mathcal{Z}}\xspace}
\newcommand{\Z}{\ensuremath{\mathcal{Z}}\xspace}
\newcommand{\Partyi}{\ensuremath{P_i}}
\newcommand{\Partyj}{\ensuremath{P_j}}
\newcommand{\partywrapper}{multiplexer\xspace}
\newcommand{\pw}{\PI}
\newcommand{\fwrapper}{\todo{fwrappername}\xspace}

\newcommand{\dealer}{\ensuremath{\mathcal{D}}}
\newcommand{\globalf}[1]{\ensuremath{{\overline{\mathcal{#1}}}}}
\newcommand{\todo}[1]{\textcolor{Red}{todo: #1}}
\newcommand{\edict}{\{\}}
\newcommand{\lar}{\leftarrow}
\newcommand{\rar}{\rightarrow}
\newcommand{\Init}{{\bf \color{NavyBlue} Init}~}
\newcommand{\OnInput}{{\bf \textcolor{Black} On input}~}
\newcommand{\Allinputs}{{\bf \color{Cerulean} All other input~}}
\newcommand{\OnAdvInput}{{\bf \color{BrickRed} On input}~}
\newcommand{\heading}[1]{\textbf{#1}}
\newcommand{\Type}{\ensuremath{\yo{type}}}
\newcommand{\Stype}{\ensuremath{\yo{stype}}}
\newcommand{\bangf}{\ensuremath{!\F}}
\newcommand{\execuc}{\ensuremath{\msf{execUC}}}
\newcommand{\iexecuc}{\inline{execUC}}
\newcommand{\UC}[4]{\ensuremath{\execuc #1  #2  #3  #4}}
\newcommand{\idealP}{\ensuremath{\mathbbm{1}_d}\xspace}
%\newcommand{\prot}[1][]{\ifthenelse{\equal{\ensuremath{#1}}{}}{\ensuremath{\Pi}}{\ensuremath{\Pi_{X #1}}}}
\newcommand{\prot}[1]{\ensuremath{\pi_{\msf{#1}}}}
\newcommand{\lla}{\leftarrow}
\newcommand{\lvd}{\vdash}
\newcommand{\tb}[1]{\text{\color{royalblue}{#1}}}
\newcommand{\tgr}[1]{\text{\color{forestgreen}{#1}}}
\newcommand{\tm}[1]{\text{\color{magenta}{#1}}}
\newcommand{\tg}[1]{\text{\color{gray}{#1}}}
\newcommand{\tp}[1]{\text{\color{purp}{#1}}}
\newcommand{\nparam}[1]{\tp{#1}}
\newcommand{\tr}[1]{\text{\color{Red}{#1}}}
\newcommand{\yo}[1]{\text{\color{YellowOrange}{#1}}}
\newcommand{\inline}[1]{\lstinline[basicstyle=\footnotesize\BeraMonottFamily, mathescape]!#1!}
\newcommand{\nrecv}{\tb{recv}}
\newcommand{\nsend}{\tb{send}}
\newcommand{\nget}{\tb{get}}
\newcommand{\npay}{\tb{pay}}
\newcommand{\nsimget}{\tm{simget}}
\newcommand{\nsimpay}{\tm{simpay}}
\newcommand{\ncase}{\tm{case}}
\newcommand{\nproc}{\tb{proc}}
\newcommand{\nwithdraw}{\tm{withdrawTokens}}
\newcommand{\nif}{\yo{if}}
\newcommand{\nthen}{\yo{then}}
\newcommand{\nend}{\yo{end}}
\newcommand{\nwhile}{\yo{while}}


%\newcommand{\pluseq}{\mathrel{+}=}
%\newcommand{\minuseq}{\mathrel{-}=}
\newcommand{\Assert}{{\bf \color{BrickRed} Assert }}
\newcommand{\Require}{{\bf \color{BrickRed} Require }}

%\theoremstyle{acmdefinition}
%\newtheorem{definition}{Definition}[section]
\newtheorem{ddef}{Definition}
%\newtheorem{theorem}{Theorem}
\newtheorem{claim}{Claim}
%\newtheorem{lemma}{Lemma}

%\newlist{renumerate}{enumerate}{1}
%\setlist[renumerate]{before=\setlength{\baselineskip}{20pt}, itemsep=-2ex, topsep=-2ex}
%\newenvironment{renumerate}{\begin{enumerate}[before=\setlength{\baselineskip}{20pt},itemsep=-2ex,topsep=0pt]}{\end{enumerate}}
\newenvironment{renumerate}{\begin{enumerate}[nosep]}{\end{enumerate}}
%\newenvironment{ritemize}{\begin{itemize}[before=\setlength{\baselineskip}{20pt},itemsep=-2ex,topsep=0pt]}{\end{itemize}}
\newenvironment{ritemize}{\begin{itemize}[nosep] \renewcommand\labelitemi{--}}{\end{itemize}}

\newenvironment{mylst}{\begin{lstlisting}[basicstyle=\small\BeraMonottFamily, frame=single, mathescape]}{\end{lstlisting}}

\makeatletter
\newcommand{\inmsg}[1]{%
(#1\checknextarg}
\newcommand{\checknextarg}{\@ifnextchar\bgroup{\gobblenextarg}{)~}}
\newcommand{\gobblenextarg}[1]{, #1\@ifnextchar\bgroup{\gobblenextarg}{)~}}
\makeatother


\newcommand{\transfermsg}{\inmsg{transfer}{to}{val}{data}{from}}
\newcommand{\createmsg}{\inmsg{contract \ create}{addr}{val}{data}{private}{from}}
\newcommand{\reject}{\textbf{reject}~}
\newcommand{\ignore}{\textbf{ignore}~}
%\newcommand{\For}{\textbf{For}~}
\newcommand{\Env}{\ensuremath{\mathcal{Z}}}
%\newcommand{\While}{\textbf{While}~}
\newcommand{\Buffer}{\textbf{Buffer}~}
\newcommand{\Send}{\textbf{Send}~}
\newcommand{\Output}{\emph{Output}~}
\newcommand{\Leak}{\textbf{Leak}}
\newcommand{\Eventually}{\textbf{Eventually}~}
\newcommand{\In}{\textbf{in}~}
\newcommand{\If}{\textbf{If}~}
\newcommand{\Else}{\textbf{Else}~}
%\newcommand{\Return}{\textbf{Return}~}

\newcommand{\pluseq}{\ensuremath{\mathrel{+}=}}
\newcommand{\minuseq}{\ensuremath{\mathrel{-}=}}
\newcommand{\Adv}{\ensuremath{\mathcal{A}}}
%\newcommand{\Partyi}{\ensuremath{\mathbf{P_i=(sid,pid)}}}
\newcommand{\sid}{\ensuremath{\msf{sid}}\xspace}
\newcommand{\pid}{\ensuremath{\msf{pid}}\xspace}
\newcommand{\dquad}{\quad \quad}
\newcommand{\qqquad}{\qquad \quad}
\newcommand{\qqqquad}{\qqquad \quad}
\newcommand{\qqqqquad}{\qqqquad \quad}

\newcommand*\circled[1]{\tikz[baseline=(char.base)]{
            \node[shape=circle,draw,inner sep=1pt] (char) {#1};}}

\newcommand*\token{~\circled{t}}

\DeclarePairedDelimiter{\ceil}{\lceil}{\rceil}


\newcommand{\spheading}[1]{ %
	\rotatebox{60}{\parbox{2.5cm}{\raggedright #1}}}




%Potential annotations
\newlength{\rWidth}

\newcommand{\funtype}[1]{%
    {\settowidth{\rWidth}{\ensuremath{#1}}%
        \;\ensuremath{{\xrightarrow{\hspace{\rWidth}}\hspace{-0.84\rWidth}}\!\!\!^%
         {#1}%{\BehindSubString{,}{#1} / \BeforeSubString{,}{#1}}%
         \hspace{0.2\rWidth}\;\;}}}


%% Notation
\newcommand{\m}[1]{\ensuremath{\mathsf{#1}}}
\newcommand{\mb}[1]{\ensuremath{\mathbf{#1}}}
\newenvironment{sill}{\begin{tabbing}}{\end{tabbing}}


%% Configuration
\newcommand{\conftree}[3]{\left[#1\right] \; \proc{#2}{#3}}
\newcommand{\confprovider}[2]{(#1)^{#2}}
\newcommand{\confset}[1]{\overline{#1}}
\newcommand{\esync}{\; \m{esync}}
\newcommand{\measure}{energy}
\newcommand{\measures}{energies}
% \newcommand{\mc}[1]{\mathcal{#1}}
\newcommand{\CC}{\mathcal{C}}
\newcommand{\DD}{\mathcal{D}}
\newcommand{\EE}{\mathcal{E}}
\newcommand{\FF}{\mathcal{F}}

%% Modes
\newcommand{\s}{\m{S}}
\newcommand{\li}{\m{L}}
\newcommand{\cl}{\m{C}}
\newcommand{\p}{\m{P}}

\newcommand{\lang}[1]{\mathbf{L}(#1)}

%% Contexts and Typing Judgment
\newcommand{\W}{\Omega}
\newcommand{\Sg}{\Sigma}
\newcommand{\xvdash}[2]{\sststile{#2}{#1}}
\newcommand{\xVdash}[1]{%
  \Vdash^{\mkern-8mu\scriptstyle\rule[-.9ex]{0pt}{0pt}#1}%
}
\newcommand{\confpot}[2]{\overset{#1}{\underset{#2}{\vDash}}}
\newcommand{\potconf}[1]{\overset{#1}{\vDash}}
\newcommand{\spanconf}{\vDash}
\newcommand{\confspan}[1]{\overset{(#1)}{\vDash}}
\newcommand{\confspanlocal}[1]{\overset{\langle #1 \rangle}{\vDash}}
\newcommand{\D}{\Delta}
%\newcommand{\G}{\Gamma}
\newcommand{\T}{\Theta}
\newcommand{\proves}{\vDash}
\newcommand{\w}{\omega}
%\newcommand{\Co}{\mathcal{C}}
%\renewcommand{\C}{\mathcal{C}}
\newcommand{\set}[1]{\lvert\lvert#1\rvert\rvert}

\newcommand{\lin}[1]{\m{lin}(\overline{#1})}
\newcommand{\shd}[1]{\m{shd}(\overline{#1})}
\newcommand{\slin}[1]{\m{slin}(\overline{#1})}
\newcommand{\plin}{\; \m{purelin}}

%% Operational Semantics Predicates
\newcommand{\proc}[2]{\m{proc}(#1, #2)}
\newcommand{\msg}[2]{\m{msg}(#1, #2)}
\newcommand{\ichan}[3]{\m{ichan}(#1, #2, #3)}
\newcommand{\ochan}[3]{\m{ochan}(#1, #2, #3)}
\newcommand{\unavail}[1]{\m{unavail}(#1)}

%% Semantics
\newcommand{\step}{\; \mapsto \;}
\newcommand{\zerostep}{\step^{0}}
\newcommand{\timed}[2]{\{#1\}_{#2}}
\newcommand{\unit}{M}
\newcommand{\Step}{\Longrightarrow}
\newcommand{\info}{\mapsto}
\newcommand{\andin}{\; \m{and} \;}
\newcommand{\minus}{\setminus}
\newcommand{\fresh}[1]{(#1 \text{ fresh})}
\newcommand{\eval}[1]{\Downarrow_{#1}}

%% Expressions Semantics
\newcommand{\val}{\; \m{val}}

%% Expressions
\newcommand{\lam}[3]{\lambda #1 : #2 . M_x}
\newcommand{\inl}[1]{l \cdot #1}
\newcommand{\inr}[1]{r \cdot #1}
\newcommand{\case}[3]{\m{case} \; #1 \; (l \hookrightarrow #2, r \hookrightarrow #3)}
\newcommand{\pair}[2]{\left\langle #1, #2 \right\rangle}
\newcommand{\projl}[1]{#1 \cdot l}
\newcommand{\projr}[1]{#1 \cdot r}
\newcommand{\match}[4]{\m{match} \; #1 \; ([] \rightarrow #2, #3 \rightarrow #4)}
\newcommand{\eproc}[3]{\{#1 \leftarrow #2 \leftarrow #3\}}


%% Proof Terms
\newcommand{\ecase}[3]{\m{case} \; #1 \; (#2 \Rightarrow #3)}
\newcommand{\ecasecf}[3]{\m{case^{cf}} \; #1 \; (#2 \Rightarrow #3)}
\newcommand{\erecvch}[2]{#2 \leftarrow \m{recv} \; #1}
\newcommand{\erecvchcf}[2]{#2 \leftarrow \m{recv^{cf}} \; #1}
\newcommand{\erecvshift}[1]{\m{shift} \leftarrow \m{recv} \; #1}
\newcommand{\esendch}[2]{\m{send} \; #1 \; #2}
\newcommand{\esendchcf}[2]{\m{send^{cf}} \; #1 \; #2}
\newcommand{\esendshift}[1]{\m{send} \; #1 \; \m{shift}}
\newcommand{\ewait}[1]{\m{wait} \; #1}
\newcommand{\ewaitcf}[1]{\m{wait^{cf}} \; #1}
\newcommand{\eclose}[1]{\m{close} \; #1}
\newcommand{\eclosecf}[1]{\m{close^{cf}} \; #1}
\newcommand{\fwd}[2]{#1 \leftarrow #2}
\newcommand{\fwdp}[2]{#1 \overset{+}{\leftarrow} #2}
\newcommand{\fwdn}[2]{#1 \overset{-}{\leftarrow} #2}
\newcommand{\esendl}[2]{#1.#2}
\newcommand{\esendlcf}[2]{(#1.#2)^{\m{cf}}}
\newcommand{\ecut}[4]{#1 \leftarrow #2 \leftarrow #3 \semi #4}
\newcommand{\ecutna}[3]{#1 \leftarrow #2 \semi #3}
\newcommand{\espawn}[4]{#1 \leftarrow #2 \leftarrow #3 = #4}
\newcommand{\procg}[3]{\m{proc}(#1, #2, \overline{#3})}
\newcommand{\edelay}[1]{\m{delay} \; (#1)}
\newcommand{\ewhen}[2]{\m{when?} \; (#1) ; #2}
\newcommand{\enow}[2]{\m{now!} \; (#1) ; #2}
\newcommand{\etick}[1]{\m{tick} \; (#1)}
\newcommand{\ework}[1]{\m{work} \; \{#1\}}
\newcommand{\eget}[2]{\m{get} \; #1 \; \{#2\}}
\newcommand{\epay}[2]{\m{pay} \; #1 \; \{#2\}}
\newcommand{\procdef}[3]{#3 \leftarrow #1 \; #2}
\newcommand{\procdefna}[2]{#2 \leftarrow #1}
\newcommand{\casedef}[1]{\m{case} \; #1}
\newcommand{\labdef}[1]{#1 \Rightarrow}
\newcommand{\wk}[1]{\m{work}(#1)}
\newcommand{\eassume}[2]{\m{assume} \; #1 \; \{#2\}}
\newcommand{\eassert}[2]{\m{assert} \; #1 \; \{#2\}}
\newcommand{\eimpos}[2]{\m{impossible} \; #1 \; \{#2\}}
\newcommand{\eif}[1]{\m{if} \; (#1)}
\newcommand{\ethen}{\; \m{then} \; }
\newcommand{\eelse}{\m{else} \; }

%% Type Constructors
\newcommand{\lolli}{\multimap}
\newcommand{\tensor}{\otimes}
\newcommand{\with}{\mathbin{\binampersand}}
\newcommand{\paar}{\mathbin{\bindnasrepma}}
\newcommand{\one}{\mathbf{1}}
\newcommand{\zero}{\mathbf{0}}
\newcommand{\bang}{{!}}
\newcommand{\whynot}{{?}}
\newcommand{\semi}{\, ; \,}
\newcommand{\ichoiceop}{\ensuremath{\oplus}}
\newcommand{\echoiceop}{\ensuremath{\with}}
\newcommand{\ichoice}[1]{\ichoiceop \{ #1 \}}
\newcommand{\echoice}[1]{\echoiceop \{ #1 \}}
\newcommand{\fuse}{\bullet}
\newcommand{\mi}[1]{\mbox{\it #1}}
\newcommand{\lunder}{\mathbin{\backslash}}
\newcommand{\tassertop}{?}
\newcommand{\tassumeop}{!}
\newcommand{\tassert}[1]{\; \tassertop\{#1\}. \;}
\newcommand{\tassume}[1]{\; \tassumeop\{#1\}. \;}
\newcommand{\arrow}{\rightarrow}
\newcommand{\product}{\times}

%% Functional Types
\newcommand{\tproc}[2]{\{#1 \leftarrow #2\}}

%% Types with Potential
\newcommand{\pot}[2]{#1^{#2}}
\newcommand{\lollipot}[1]{\overset{#1}{\lolli}}
\newcommand{\tensorpot}[1]{\overset{#1}{\tensor}}
\newcommand{\potfop}{\phi}
\newcommand{\potf}[1]{\potfop(#1)}
\newcommand{\mlab}{M^{\textsf{label}}}
\newcommand{\mchan}{M^{\textsf{channel}}}
\newcommand{\mcl}{M^{\textsf{close}}}
\newcommand{\mall}{M}
\newcommand{\mint}{M^{\textsf{internal}}}
\newcommand{\mval}{M^{\textsf{value}}}
\newcommand{\mshd}{M^{\textsf{share}}}
\newcommand{\ms}{M_s}
\newcommand{\mr}{M_r}
\newcommand{\entailpot}[2]{\xvdash{#1}{#2}}
\newcommand{\exppot}[1]{\xVdash{#1}}
\newcommand{\texp}{\Vdash}
\newcommand{\pexp}{\vdash}
\newcommand{\paypot}{\triangleright}
\newcommand{\getpot}{\triangleleft}
\newcommand{\tgetpot}[2]{\getpot^{\{#2\}} #1}
\newcommand{\tpaypot}[2]{\paypot^{\{#2\}} #1}
\newcommand{\bigeval}[3]{#1 \Downarrow #2 \mid #3}
\newcommand{\share}{\curlyveedownarrow}
\newcommand{\zpot}{\overline{0}}


%% Temporal Types
\newcommand{\entailpotcf}[1]{\underset{\m{cf}}{\entailpot{#1}}}
\newcommand{\entailspan}{\vdash}
\newcommand{\entailtype}{\vdash}
\newcommand{\fpot}{\; @ \;}
\newcommand{\pay}[1]{#1^{1}}
\newcommand{\sync}[1]{#1^{2}}
\newcommand{\spanpot}[1]{\langle \pay{#1}, \sync{#1} \rangle}
\newcommand{\ichoicepot}[2]{\overset{#1}{\ichoiceop} \{ #2 \}}
\newcommand{\echoicepot}[2]{\overset{#1}{\echoiceop} \{ #2 \}}
\newcommand{\tlist}[1]{\m{list}_{#1}}
\newcommand{\plist}[2]{\m{list}_{#1}^{#2}}
\newcommand{\tdia}[1]{\Diamond #1}
\newcommand{\tbox}[1]{\Box #1}
\newcommand{\tforall}[1]{\forall . #1}
\newcommand{\texists}[1]{\exists . #1}
\newcommand{\Dia}{\Diamond}
\newcommand{\Next}{\raisebox{0.3ex}{$\scriptstyle\bigcirc$}}
\newcommand{\tdelay}[2]{
    \IfEqCase{#2}{%
        {1}{\next{#1}}%
        % you can add more cases here as desired
    }[{\Next^{#2} (#1)}]%
}%
\newcommand{\sch}[1]{\tau(#1)}
\newcommand{\lforce}[2]{[#1]_L^{#2}}
\newcommand{\rforce}[2]{[#1]_R^{#2}}
\newcommand{\force}[2]{#1 \circ (#2)}

\setlength{\inferLineSkip}{4pt}
\newcommand{\blue}[1]{{\color{blue}#1}}
\newcommand{\red}[1]{{\color{red}#1}}
\newcommand{\green}[1]{{\color{green}#1}}
\newcommand{\tick}{\blue{\m{tick}}}
\newcommand{\delay}{\red{\m{delay}}}
\newcommand{\when}[1]{\red{\m{when?}\;#1}}
\newcommand{\now}[1]{\red{\m{now!}\;#1}}
\newcommand{\noww}{\red{\m{now!}}}
\newcommand{\whenn}{\red{\m{when?}}}
% \newcommand{\vdashi}{\vdash^{\!\!{}^i}}
\newcommand{\vdashi}{\vdash^{\!\!\scriptscriptstyle i}}
\newcommand{\tock}{`}


%% Indices
\newcommand{\indv}[1]{\overline{\{#1\}}}
\newcommand{\ind}[1]{\{#1\}}


%% Syntactic Sugar
\newcommand{\config}{\mathcal{C}}
\newcommand{\cost}[2]{\mathrm{cost}(\proc{#1}{#2})}
\newcommand{\tcost}[2]{\mathrm{cost}(#1 \mapsto #2)}
\newcommand{\ccost}[1]{\mathrm{cost}(#1)}
\newcommand{\dc}{\mathcal{D}}
\newcommand{\ec}{\mathcal{E}}
\newcommand{\ac}{\mathcal{A}}
\newcommand{\st}[1]{\m{store}_{#1}}
\newcommand{\stack}[1]{\m{stack}_{#1}}
\newcommand{\queue}[1]{\m{queue}_{#1}}
\newcommand{\mapper}[1]{\m{mapper}_{#1}}
\newcommand{\fdr}[1]{\m{folder}_{#1}}
\newcommand{\lt}[1]{\m{list}_{#1}}
%\newcommand{\bits}{\m{bits}}
\newcommand{\ctr}{\m{ctr}}
\newcommand{\trans}[2]{#1 \Longrightarrow #2}
\newcommand{\typetrans}[1]{\left\lvert{#1}\right\rvert}
\newcommand{\tree}{\m{tree}}
\newcommand{\bool}{\m{bool}}
\newcommand{\delayedbox}[1]{#1 \; \m{delayed}^{\Box}}
\newcommand{\delayeddia}[1]{#1 \; \m{delayed}^{\Diamond}}
\newcommand{\dom}[1]{\m{dom}(#1)}
\newcommand{\valid}[1]{#1 \; \m{valid}}
\newcommand{\invalid}[1]{#1 \; \m{invalid}}

%% Smart Contracts
\newcommand{\addr}{\m{addr}}
\newcommand{\ether}{\m{ether}}
\newcommand{\players}{\m{players}}
\newcommand{\lottery}{\m{lottery}}
\newcommand{\tint}{\m{int}}
\newcommand{\ballot}{\m{ballot}}
\newcommand{\tbool}{\m{bool}}
\newcommand{\lc}{\tlist{\m{coin}}}
\newcommand{\auction}{\m{auction}}
\newcommand{\object}{\m{object}}

%% Typing Judgments for Servers and Clients
\newcommand{\sentailpot}[1]{\prescript{}{S}{\xvdash{#1}} \hspace{2pt}}
\newcommand{\centailpot}[1]{\prescript{}{C}{\xvdash{#1}} \hspace{2pt}}


%% Sharing
\newcommand{\down}{\downarrow^{\m{S}}_{\m{L}}}
\newcommand{\up}{\uparrow^{\m{S}}_{\m{L}}}
\newcommand{\eacquire}[2]{#1 \leftarrow \m{acquire} \; #2}
\newcommand{\eaccept}[2]{#1 \leftarrow \m{accept} \; #2}
\newcommand{\erelease}[2]{#1 \leftarrow \m{release} \; #2}
\newcommand{\edetach}[2]{#1 \leftarrow \m{detach} \; #2}


%% Subtyping
\newcommand{\subt}[2]{#1 \leq #2}
\newcommand{\wsubt}{ <: }
\newcommand{\qsubt}[1]{\overset{#1}{\leq}}


%% Latex
%\newtheorem{theorem}{Theorem}
%\newtheorem{definition}{Definition}
%\newtheorem{lemma}{Lemma}
%\newtheorem{cor}{Corollary}


%%Global Semantics
\newcommand{\sinfer}[3]
{\inferrule
{#3}
{#2}
#1}
\newcommand{\enq}[2]{\m{enq}(#1, #2)}
\newcommand{\deq}[1]{\m{deq}(#1)}
\newcommand{\nil}{[]}
\newcommand{\elem}[1]{[#1]}


%% Channel typing
\newcommand{\eqdef}{\cong}


%% Types to Processes
\newcommand{\typeProc}[2]{#1 \Longrightarrow #2}

%% AARA
\newcommand{\abs}[1]{\left\lvert #1 \right\rvert}
\newcommand{\bin}[1]{(#1)_2}
% \newcommand{\ceil}[1]{\left\lceil #1 \right\rceil}
\newcommand{\bigO}[1]{\mathcal{O}(#1)}
% \newcommand{\ignore}[1]{\textcolor{red}{#1}}
% new \oset macro
\makeatletter
\newcommand{\oset}[3][-0.7ex]{%
  \mathrel{\mathop{#3}\limits^{
    \vbox to#1{\kern-2\ex@
    \hbox{$\scriptstyle#2$}\vss}}}}
\makeatother
\newcommand{\monus}{\oset{.}{-}}

%% Indexed Types
\newcommand{\cons}{\mathcal{C}}
\newcommand{\vars}{\mathfrak{v}}
\newcommand{\Vars}{\mathcal{V}}
\newcommand{\Cons}{\mathcal{C}}
\newcommand{\Tokens}{\Gamma}
\newcommand{\K}{\gamma}
\newcommand{\Tokentypes}{\mathcal{K}}
\newcommand{\VTokens}{\mathcal{V}}
\newcommand{\TokSig}{\mathcal{S}}
\newcommand{\exchange}[3]{#1 \overset{#2}{\longrightarrow} #3}
\newcommand{\GlobalF}{\ensuremath{\mathfrak{f}}\xspace}
\newcommand{\GlobalP}{\mathfrak{p}}
\newcommand{\depth}{\mathfrak{d}}

%% Two Counter Machines
\newcommand{\ins}{\iota}
\newcommand{\tcm}{\mathcal{M}}
\newcommand{\inc}[1]{\m{inc}(#1)}
\newcommand{\dec}[1]{\m{dec}(#1)}
\newcommand{\goto}{\m{goto}}
\newcommand{\zeroc}[1]{\m{zero}(#1) ?}
\newcommand{\halt}{\m{halt}}

%% UC stuff
\newcommand{\fcomm}{\mathcal{F}_{\msf{comm}}}
%\newcommand{\B}[1]{\colorbox{gray}{#1}}
%\newcommand{\hlc}[2][yellow]{{%
%    \colorlet{foo}{#1}%
%        \sethlcolor{foo}\hl{#2}}%
%        }
%\newcommand{\hlcyan}[1]{{\sethlcolor{cyan}\hl{#1}}}
%\newcommand{\B}[1]{\hlc[pink]{#1}}
\definecolor{airforceblue}{rgb}{0.36, 0.54, 0.66}
\newcommand{\B}[1]{{\color{airforceblue}{#1}}}
\newcommand{\wt}{\circled{w}}

%% TODO
\newcommand{\ankush}[1]{\textcolor{red}{\textbf{Ankush: #1}}}


%%% Local Variables:
%%% mode: plain-tex
%%% TeX-master: "pldi19"
%%% End:


\begin{abstract}
  Universal Composability (UC) is a leading framework for modeling secure protocols, in cryptography but increasingly.
  Less well understood than traditional game-based definitions, feature communication patterns.
  In this work we continue a recent line of work on bringing a formal semantics to the UC framework.
  Our approach is to investigate adding session types to UC, as a way of annotating ideal functionality definitions with additional structure.
  We build a new language, Nomos-UC, by combining an existing session-typed language (Nomos) with a process-based core calculus for UC (ILC). This integration required solving a range of technical challenges, especially since session types impose a linear discipline while we wanted to preserve the UC lets the adversary dynamically steer the communication patterns at runtime.
  Polynomial runtime is an essential component of the security definition, yet encoding it in other formal frameworks has required placing significant restrictions on the execution modeled considered in UC.
Since we show how to faithfully encode the dynamic ``import tokens'' mechanism from UC into the resource-aware types in Nomos, we retain the best of both worlds.
To validate our design, we work through the standard theory of UC composition operators, and complete a modular application case study: a coin tossing protocol in the random oracle model using commitments as the intermediate primitive.
\end{abstract}

\maketitle


%\begin{IEEEkeywords}
%component, formatting, style, styling, insert
%\end{IEEEkeywords}

\section{Introduction}
In this work we present a programming language design based on the Universal (UC) Composability framework from cryptography.

We build on existing work, ILC, which also aims to be a programming language for UC, but start from a more powerful language, called Nomos, which incorpoates session types and work aware resource types and has been previously used for  smart contracts and distributed applications.
Both of the features of Nomos turn out to have 

Second, Nomos features a notion of Work-aware types. This is useful for capturing the notion of “locally polynomial runtime.” This allows us to model UC more faithfully than any prior work to date

As a starting point, we build a language that merges types rules from ILC into Nomos. The main design idea of ILC is that it is uses static typing rules to encode the requirements of the Interacting Turing Machines (ITMs) model, a model that is uniquely associated with UC. The ILC rules roughly ensure that simulations of the language can be carried out by probabilistic Turing machines, which is necessary for reduction to computationally hard problems, required for cryptographic security proofs. The rules from ILC are compatible with session types, so it turns out to be straightforward to merge these into nomos. The result provides benefits associated with session types, namely that it avoids potential errors from internally-inconsistent programs.

   Beyond just session types, the Work-aware component of Nomos allows us to tackle a fundamental challenge in defining a programming language for UC that ILC (and all other related work) left unfulfilled, which is to express the notion of polynomially runtime.
   
   The challenge of “polynomial runtime” in UC is that individual processes must be judged as polynomial, but when eveluated in context with other concurrently running process it is difficult to assign blame.
       The current best way to define polynomial runtime, found in the 2019 and later version of UC, is based a concept of ``import tokens.''
   We identify how to relate the “Potential” concept from Nomos, to the import tokens from UC. 
	The result is a deep connection between session type semantics and the formal foundation of UC.
	The Preservation theorem we prove associated with our type system and operational semantics proves the following: 
well-typed terms in Nomos UC are “locally polynomial time”, in the sense required of UC, meaning they do not take more steps that some polynomial function T(N) of the net number of import tokens it has received.

In addition, our language has other benefits.
The Progress theorem is useful because it gives some evidence that ideal functionalities and protocols encoded in Nomos UC cannot get stuck. Together helps confirm that the process halts in polynomial time.
TODO: Give an example of a bad machine ruled out by progress guarantee.

\ignore{
Carries forward the same metatheory guarantees as ILC. Namely: if a process terminates, then it depends only on the random coins (unlike Pi calculus, including Session-type pi calculus). Thus simulating the execution of a Nomos UC experiment can be carried out by a probabilistic polynomial time Turing machine (PPT). This is essential in UC for reduction to computationally hard problems.
}

\ignore{
The Universal Composability Framework~\cite{uc} is the popular and widely-used framework for modelling the security of cryptographic and distributed protocols.
Its novel contribution compared to other frameworks is that it provides a very strong notion of security: a UC-secure protocol is proved to be secure even when composed with arbitrary other protocols running concurrently.
This constrasts with other, property-based notions of security~\todo{need to get some citations here}.

Analyzing large and complex protocols is a difficult task made easier by UC's ideal functionality abstraction. 
However, despite this additional modularity, UC proofs and models still tend to be very complex, unwieldy, and difficult to understand.
These issues are exacerbated when new communication models are added on top of UC~\cite{katz, etc}.
Therefore, we propose a two-fold solution: a new construction for modelling different communication models that removes all model-specific code from protocols and functionalities, and an implementation of the UC framework in the Nomos language. 
}



\section{Background} \label{sec:background}
\subsection{Universal Composability}
The universal composability framework~\cite{uc} proposes a new framework for proving the security of cryptographic and distributed protocol.
Compared to previous works, the UC framework provides a stronger notion of security where protocols that are UC-secure are secure even when composed with arbitrary other protocols running concurrently. 

Such a strong notion of security is achieved through the real-ideal world paradigm.
The ideal world encompasses an ideal implementation of a protocol, called the \textit{ideal functionality} $\mathcal{F}$, which acts as a trusted third party that caputures all the desired security properties.
The ideal functionality is usually a simple definition making it trivial to prove its security properties.
The real world, on the other hand, consists of parties running an actual protocol, $\pi$, against a real adversary.

Security proofs in UC involve creating a simulator $\mathcal{S}$ in the ideal world that can simulate every potential attack on a real protocol in the real world.
If $\mathcal{S}$ can make the two worlds indistinguishable for any real world adversary $\mathcal{A}$ for all distinguishing environments $\mathcal{Z}$, then we say the protocol $\pi$ UC-emulates the ideal functionality $\mathcal{F}$.
Indistinguishability of the two worlds to any $\mathcal{Z}$ implies that the protocol $\pi$ must exhibit the same security properties as the ideal functionality $\mathcal{F}$ otherwise there should be sobe distinguishing environment. 
More formally, indistinguishability is stated:

$$ \text{EXEC}_{\mathcal{F},\mathcal{S},\Environment} \approx \text{EXEC}_{\pi,\mathcal{A},\Environment} $$

\paragraph{GUC-Framework}


\subsection{The Import Mechanism}
A notion of resource-bound computation is necessary for the UC framework to reason about computationally efficient algorithms as well as the capabilities of ITIs under a particular resource constraint.
Often we would like to reason about adversarial capabilities under such constraints and perform efficient transformations (transforming an adversary into a simulator).

Previous definitions of polynomial-time computation have taken the form of bounding the computation of an ITI by some polynomial $T$:
given an input of length $n$ the machine $\mu$ halts within $T(n)$ steps.
However, using the length of the inputs to the machine as $n$, in this case leads to an infinite runs problems identified by Canetti~\cite{uc}.
Machines that are locally $T(n)$-bounded are able to spawn other machines to the point that an infinite chain of such machines can be spawed where each is locally $T$-bounded, but the whole system of machines can not be bounded by any polynomial $T$.

Therefore, a new notion of $n$ was needed. The UC paper defines an import mechanism where the first ITI, the environment, is spawned with a polynomially amount of import which can be thought of as tokens or coins.
The environment can then activate other ITIs with some import tokens allowing them to run for $T(n')$ computationsl steps for some $T$ and some amount of import $n'$.
In this new definition, an ITI that is $T$-bounded takes at most $T(n')$ steps where $n'$ is the difference between the import it has received from incoming messages and outgoing import it's given to other machines.
This definition therefore suffices to ensure that every machine is locally bounded by some polynomial but also guarantees that the system of ITMs is bounded by a polynomial number of import tokens. 


\section{A Commitment Protocol in NomosUC} \label{sec:example}
% Comments:
% Why use session types? The advantages
% Maybe have an untyped commitment protocol
% Benefit of session types: extra type annotations, concise specification
% Does not provide a complete term, only a specification
% Might make sense to talk about extending session types with dependencies for commitment

%Just like distributed protocols, cryptographic protocols follow a predefined communication pattern.
The central focus of our work is to explore the role that \emph{session types} can play in defining and analyzing ideal functionalities.
In this section, we illustrate how session types can be used to describe protocol interfaces, what information is leaked to the adversary, and, as we'll see in a later section what its 
what its runtime requirements are.
%We do so and introduce our running example throughout the paper: a simple, linear database functionality, \Fdb
We do so and introduce our running example throughout the paper: a commitment in the random oracle model, $\Fcom$.
%Our view is that session types are especially useful as ways of annotating or analyzing the ideal functionality.
%To illustrate, we use cryptographic commitment as our main running example.
%The commitment functionality \Fcom encapsulates the security properties of a two-phase, two-party commitment,
%which, given its simplicity is an ideal learning example.


The database functionality, presented in Figure~\ref{fig:fdbideal}, implements a flat key-value store.
Parties can submit key-value pairs and \Fdb stores them in an append-only list, $\ell$.
On \m{Store}, \Fdb sends an acknowledgement back to $P_i$, and on \m{Get} it returns the key-value pair if the key exists in $\ell$ (or a negative acknowledgement if it doesn't).

\begin{figure}
\centering
\begin{minipage}{0.38\textwidth}
\begin{bbox}[title={Functionality $\F_{\m{db}}$}]

Intialize $\ell = []$

\OnInput \inmsg{\msf{Store}}{$k',v'$} from $P_i/\A$:

\qquad append $(k',v')$ to $\ell$ and \Send $\m{Ok}$ to $P_i/\A$:

\OnInput \inmsg{Get}{$k'$} from $P_i/\A$:

\qquad If $v' <- \ell(k')$ in $\ell$ then

\qquad \qquad \Send $(k', v')$ to $P_i/\A$

\qquad else

\qquad \qquad \Send $\m{No}$ to $P_i/\A$
\end{bbox}
\end{minipage}
\hspace{3em}
\begin{minipage}{0.5\textwidth}
\begin{lstlisting}[basicstyle=\scriptsize\BeraMonottFamily, frame=single, mathescape, numbers=left, xleftmargin=2em, xrightmargin=2em]
$\nproc$ Fdb[k][v]: ($\$$p2f: db[k][v]), ($\$$f2p: 1), 
  ($\$$a2f: adv[k][v]), ($\$$f2a: 1), (l: [(k,v)]) |- ($\$$c: 1) =
{
  $\ncase$ $\$$p2f (
    store => pid,(k',v') = $\nrecv$ $\$$p2f
      $\$$tb' <- pappend[(k,v)] <- $\$$tb k' v' ;
      $\$$p2f.Ok; $\nsend$ $\$$p2f pid ;
	  $\$$c $\leftarrow$ Fdb[k][v] <- $\tg{(* args *)}$ $\$$tb'
    retrieve => pid,k' = $\nrecv$ $\$$p2f ;
      b $\leftarrow$ exist $\leftarrow$ $\$$tb k' ;
      $\nif$ b $\nthen$
        v' $\leftarrow$ get $\$$tb k' ;
        $\$$p2f.yes; $\nsend$ $\$$p2f pid; $\nsend$ $\$$p2f v';
      $\nelse$
        $\$$p2f.no; $\nsend$ $\$$p2f pid ;
      $\$$c $\leftarrow$ Fdb[k][v $\leftarrow$ $\tg{(* args *)}$ 
}
\end{lstlisting}
\end{minipage}
\hspace{3em}
\begin{minipage}{0.5\textwidth}
\begin{lstlisting}[basicstyle=\scriptsize\BeraMonottFamily, frame=single, mathescape, numbers=left, xleftmargin=2em, xrightmargin=2em]
$\nproc$ somparty[k][v]: (pid: PID), ($\$$p2f: db[k][v]), 
  ($\$$f2p: 1)  |- ($\$$c: 1) =
{
  $\$$p2f.store ; $\nsend$ $\$$p2f pid ; 
  $\nsend$ $\$$p2f someK ; $\nsend$ $\$$p2f someV ;
  $\ncase$ $\$$p2f ( Ok => 1 )
}
\end{lstlisting}
\end{minipage}
caption{(a) The ideal functionality \Fdb parameterized by types for the keys and values, 
(b) corresponding code in NomosUC, and (c) a simple party that stored a key-value pair.}
\label{fig:fdbideal}
\vspace{-4mm}
\end{figure}


The communication pattern outlined above can be enforced by a \emph{binary session type}.
The key insight here is that we assign a session type to the communication channel connecting two processes. 
Communication between an arbitrary number of parties and \Fdb is captured by the following session type:
\begin{tabbing}
	$\mi{type} \; \m{db[k][v]} = \ichoice{$\=$ \; \mb{store}:$\=$\m{PID} \arrow \m{k} \arrow$ \\
	\>\>$\echoice{ \mb{OK}: \m{PID} \arrow \m{db[k][v]}},$ \\
	\>\=$ \; \mb{get}: $\=$\m{PID} \arrow \m{k} \arrow$ \\
	\>\>\>$\echoice{$\=$\mb{yes}: \m{v} \arrow \m{db[k][v]},$ \\
	\>\>\>\>$\mb{no}: \m{db[k][v]}}}$
\end{tabbing}
The session type uses two operators to distinguish which endpoint is sending or receiving the messages.
Every linear channel has a ``provider'' (the process that spawns the channel) whose writes on the channel are denoted by the type constructor $\ichoice$. 
Similarly, the other endpoint of the channel, the ``client'', writes to it with the $\echoice$ constructor. 
The session types define, at any given moment, which endpoint's turn it is to write to the channel by which type constructor is used.
In \m{db[k][v]} at first only the party can send a message but can choose between labels \mb{store} and \mb{get}. 
Next only \Fdb can send a message in response depending on the label received.
In general, the provider-client relationship in the channel ends up being unimportant in our construction, and we arbitrarily choose one ITM as the provider for
the purpose of enforcing the type. 
In the case of \m{db[k][v]}, it is implied that the party is the provider, because it uses $\ichoice$ to \mb{store} and \mb{get}, and \Fdb is the other endpoint.
In this way the session types becomes a succinct description of how a party can use a functionality, and, at a high-level, what the functionality should do.

In Figure~\ref{fig:fdbideal}(b) and (c) we show the NomosUC process for \Fdb and a party that sends it messages over a channel of type $\m{db[k][v]}$. 
At initialization, only the protocol party can send a message with \ichoice. 
The party stores on Line 4 and sends the label \mb{store} over \ic{p2f} using \isend to send the message contents in the type: a PID and a key-value pair to store.
On the \Fdb side, it waits to receive a message with a \icase match on its incoming channel (line 4). 
On \mb{store}, it \irecv s the pid and key-value pair and appends them to $\ell$. 
The session type progresses to the external choice, and \Fdb sends the \mb{Ok} acknowledgement back (line 7) with \inline{pid}.
The channel recurses back to type \m{db[k][v]} and the process calls itself again with a new list \inline{$\$$tb'} and waits for new messages.
Not all session types recurse back like \inline{db[k][v]}. Some types, for example for one-shot functionalities, eventually transition to a terminating type \m{1}.

Canonically in NomosUC, functionalities and parties are given two channels, here the two are \ic{p2f} and \ic{f2p}, to allow unidirectional communication over each if desired or necessary. 
In Figure~\ref{fig:fdbideal}, the communication pattern is easily captured by just one channel and one type, therefore \ic{f2p} is unused and typed with the terminating \m{1}. 
%In general, a functionality can be written to accept any number of channels, even one channel per party but, as we explain in Section~\ref{sec:execuc}, this requires some additional multiplexing code which can be generated at compile-time.\todo{explain in section}

To complete the example, the party can query with \inline{retrieve} and get a \mb{yes}, with the value, or a \mb{no} from \Fdb.
%In fact, the session type can not enforce that \Fdb sends \mb{yes} on a hit and \mb{no} on a miss. This is only what a correct functionality \emph{should} do. 
%An functionaliy that doesn't always behave this way satisfies some other set of properties than those intutively desired from a database and would likely fail to be emulated by a correct protocol that implements a database.

The type of \Fdb in Figure~\ref{fig:fdbideal} also indicates a channel with \A called \ic{a2f}. 
Given just the type we can infer that \A has access to the same interface as protocol parties, and, importantly, that \Fdb doesn't leak any information to it~\footnote{We exclude it to save space.}.
\emph{Leaks define a crucial part of the adversarial model for functionalities, and the session type succinctly describe it}.
%\begin{tabbing}
%	$\mi{type} \; \m{db[k][v]} = \ichoice{$\=$\textcolor{red}{\paypot{1}}$\=$ \; \mb{store}:\m{PID} \arrow \m{k} \arrow$ \\
%	\>\>$\echoice{ \mb{OK}: \m{PID} \arrow \m{db[k][v]}},$ \\
%	\>$\textcolor{red}{\paypot{1}}$\=$ \; \mb{get}: \m{PID} \arrow \m{k} \arrow$ \\
%	\>\>$\echoice{$\=$\mb{yes}: \m{v} \arrow \m{db[k][v]},$ \\
%	\>\>\>$\mb{no}: \m{db[k][v]}}}$
%\end{tabbing}


%A cryptographic commitment is a protocol consisting of a sender that knows some witness $x$ and sends a commitment message $C = f(x)$ that is
%some function of the witness such that $f^{-1}(C) \neq x$ with negligible probability. 
%The ideal functionality \Fcom in Figure~\ref{fig:fcomideal}(a) describes the properties of the commitment
%It consists of a  \emph{sender} ITM $S$ and a \emph{receiver} ITM $R$ connected to 
%the ideal functionality ITM $\Fcom$. It enforces that the committer can not equivocate on the witness $x$ once it creates a commitment.
%This property is called \emph{hiding}.
%Similarly, the receiver can not learn the witness from just the commitment, which we call the \emph{binding} property.
%%\Fcom encapsulates the security properties of a two-phase, two-party commitment: (\emph{binding}) committer can't change what they committed to, and (hiding) the receiver can't open the commitment itself. 

%The communication pattern outlined in Figure~\ref{fig:fcomideal} between the sender $S$ and $\Fcom$ (and also the receiver $R$
%and $\Fcom$) is enforced via \emph{binary session types}.
%% Session types are a type system for statically expressing bi-directional communication protocols
%% in message-passing process systems.
%The key insight here is that we assign a session type to the communication channel connecting
%two processes.
%As notation, every channel has a unique \emph{provider} process that offers the channel and a
%\emph{client} process that uses it, and the session type governs the type and direction of messages exchanged between them. 
%%the processes, with the provider and client processes performing dual send/receive actions.
%As an example, we start with the session type of the channel offered by $S$ that is used by
%$\Fcom$.
%\begin{mathpar}
%  \mi{type} \; \m{sender} = \ichoice{\mb{Commit} : \m{bit} \product \ichoice{\mb{Open} : \one}}
%\end{mathpar}
%The type constructor $\ichoiceop$ denotes an \emph{internal choice}
%\footnote{Although $\ichoiceop$ with only one choice is redundant, we still use
%it here for the purpose of exposition.}
%dictating that the provider $S$ first sends a
%$\mb{Commit}$ message to $\Fcom$.
%Next, the type constructor $\product$ denotes that $S$
%sends a value of type $\m{bit}$ ($\m{bit} \product \ldots$).
%Finally, the $\ichoiceop$ constructor
%enforces that $S$ sends $\mb{Open}$ to $\Fcom$ followed by type $\one$
%that indicates $S$ terminates and closes its channel.
%In UC, one-shot functionalities terminate after a single instance, and reactive
%functionalities persist and run many times. 
%It is important to point out that session types capture communication over one channel.
%For example, the session type of the sender does not capture what, if any, information is leaked to the adversary when \Fcom is activated.
%This helps with modularity: since one session type only captures the local communication
%between two processes, we can modify the adversary's implementation or communication interface
%without impacting the session type between $S$ and $\Fcom$.
%Local session types also do not directly capture \Fcom's security property that the same bit that was committed is the one that is opened.
%\footnote{With refinement session types~\cite{Das20CONCUR,Das20FSCD}, such advanced properties can be captured but they would significantly
%complicate the type system.}

%In this work \emph{import session types} extend the above session type $\m{sender}$ with annotations that express import tokens, which act as runtime budgets, passed over a channel between processes. 
%Though a trivia example because \Fcom does a constant amount of work, we still give some import below so that a protocol that does polynomial work can realize \Fcom as well.
%Such restrictions are important to consider when creating types.
%%Though a trivial example of import, given that $\Fcom$ is a one-shot functionality which does only a constant amount of work, the import session type for $\m{sender}$ is given below.
%\begin{mathpar}
%  \mi{type} \; \m{sender} = \textcolor{red}{\paypot^{2}} \ichoice{\mb{Commit}: \m{bit} \product \textcolor{red}{\paypot^{0}} \ichoice{\mb{Open}: \one}}
%\end{mathpar}
%The annotation \textcolor{red}{$\paypot^2$} asserts that the sender gives 1 import with \m{Commit} and 0 with \m{Open}. 
%%Despite doing constant work, requiring 0 tokens would
%%contrain protocols that can realize it (which will necessarily have the same type as \Fcom) to those that do constant work. Such restrictions are important to 
%%consider when defining import session types in NomosUC.
%The dual of $\paypot$ is given by $\getpot$ where an external choice operation can be specified with a required amount of import to be received. 
%The typing rules for both $\paypot$ and $\getpot$, and a more comprehensive discussion about the handling of import tokens, is given in Section~\ref{sec:import}.
%
%Similar to the sender, we define a channel provided by the receiver $R$ 
%used by $\Fcom$ with the following session type
%\begin{mathpar}
%	\mi{type} \; \m{receiver} = \textcolor{red}{\getpot^0} \echoice{\mb{Committed}: \textcolor{red}{\getpot^0} \echoice{\mb{Opened} : \m{bit} \arrow \one}}
%\end{mathpar}
%Dual to internal choice, the $\echoiceop$ type constructor represents \emph{external choice}
%prescribing that the provider $R$ must receive a $\mb{Committed}$ message from $\Fcom$
%followed by an $\mb{Opened}$ message (using another $\echoiceop$ constructor) from $\Fcom$.
%Then, $R$ must receive a bit from $\Fcom$ as depicted by the $\arrow$ constructor (dual to $\product$)
%followed by termination (indicated by $\one$).
%Dual to $\mb{sender}$, the receiver here expects to receive no import from $\Fcom$.

%Protocols expressed via session typed channels are realized by process implementations.
%A session-typed process \emph{uses} a set of channels in its context (similar to a function
%having arguments) and provides a unique channel (similar to a function returning a single value).
%NomosUC also allows processes to store functional data (like integers, booleans, lists, etc.)
%and either transfer them to other processes or perform local computation on them.
%The type checker guarantees that every process adheres to the protocol on every channel as defined by
%the corresponding session type.

%As an illustration, consider the $\Fcom$ process implemented in Figure~\ref{fig:fcomideal}(b)
%that \emph{uses} channels $S$ and $R$ and \emph{provides} channel \inline{fc}.
%The used channels with their types are written to the left of the turnstile
%($\vdash$) while the offered channel and type are written on the right.
%The process first case analyzes on channel $S$ branching on the
%message received.
%Since there is only one choice $\mb{Commit}$, we only have one
%branch in the definition.
%$\Fcom$ then receives the bit $b$ (line 3) on $S$, followed by sending the
%\m{Committed} message on channel $R$ to the receiver (line 4).
%Then, $\Fcom$ receives the $\mb{Open}$ message on $S$ followed by sending the
%$\mb{Opened}$ message on $R$ (line 6), followed by the bit $b$ (line 7).
%$\Fcom$ then waits for the channels $R$ and $S$ to terminate and then finally
%terminates the \inline{fc} channel (code not shown for brevity).

%The types associated with \Fcom here don't make use of the import tokens encoding we 
%introduce later in this work. An important reason for it is that most functionalities
%in NomosUC are designed to be parametric in the amount of import they, and their
%session types, require. A static amount of import for some \F constrains the 
%amount of import, or computation, that a protocol realizing \F can use. 
%Such a constraint is unnecessarily restrictive requiring multiple versions
%of the same functionality for different protocols. 

%A protocol may consist parties with different roles with different sets of inputs and messages in the protocol. 
%A session type defines the protocol for only one role in an ideal functionality and others may have their own types.
%The sender and receiver are different roles in the same protocol, and, therefore, must have their own channels to \Fcom rather than communicating over a common channel.

%The protocol initiates with $S$ sending a $\mb{commit}$ message to $\Fcom$
%indicating its intent to \emph{commit} to a bit.
%Next, $S$ sends this committed bit to $\Fcom$.
%After receiving the committed bit, $\Fcom$ sends a $\mb{commit}$ message
%to $R$ indicating that a bit has been committed to, but does not reveal
%this bit to $R$.
%At a later time, $S$ sends an $\mb{open}$ message to $\Fcom$ expressing
%that $S$ wishes to reveal the secret bit to $R$.
%Receiving this message, $\Fcom$ in turn sends an $\mb{open}$ message
%to $R$ followed by this bit.
%The protocol concludes with each party (process) terminating.
%Finally, the type $\one$ denotes termination, indicating that
%$S$ will send $\m{close}$ message to $\Fcom$.

%\begin{figure*}[!ht]
%\begin{lstlisting}[basicstyle=\small\ttfamily,frame=single]
stype sender = +{ commit : bit ^ +{ open : 1 } } 

stype receiver = &{ commit : &{ open : bit -> 1 } }

decl F_comm : (S : sender), (R : receiver) |- (F : 1)

def F <- F_comm S R =
  case s (
    commit => b = recv S ;
              R.commit ;
              case S (
                open => R.open ;
                        send R b ;
                        wait S ; wait R ; close F ) )
\end{lstlisting}

%\caption{The $\mathcal{F}_{\msf{comm}}$ commitment ideal functionality in Nomos. The types for the sender and receiver channel define what inputs they can give to the functionality and what messsages are sent from the functionality back to the receiver.}
%\label{fig:nomos:commitment}
%\end{figure*}



\section{Base System of Session Types in NomosUC} \label{sec:basic}

The core calculus of NomosUC is based on \emph{binary session types}~\cite{caires2010session}:
a type discipline for communication-centric programming derived from a Curry-Howard interpretation
of intuitionistic linear logic~\cite{girard1987linear}.
Under this correspondence, a process term $P$ is assigned to
a logical judgment of the form $A_1, \ldots A_n \vdash C$ and each antecedent as well as the succedent is
labeled with a \emph{channel} to obtain
\[
x_1 : A_1, \ldots, x_n : A_n \vdash P :: (z : C)
\]
The resulting judgment states that process $P$ \emph{provides} a service
of session type $C$ along channel $z$, \emph{using} the services of session
types $A_1, \ldots, A_n$ provided along channels $x_1, \ldots, x_n$ respectively.
We mandate all channel names to be distinct for the judgment
to be \emph{well-formed}.
The linear antecedents are often abbreviated to $\D$.

Formally, the typing judgment for processes in NomosUC is written as
$\Sg \semi k \semi \Tokens \semi \Psi \semi \D \entailpot{q}{q'} P :: (x : A)$.
$\Sg$ denotes the signature containing type and process definitions and $k$
denotes the security parameter.
Both these quantities are globally known and fixed, therefore we omit them from
most typing rules for brevity.
$\Tokens$ describes the total and current ($=$ received - sent) import tokens
of each type stored in the process (explained more in Section~\ref{sec:import}).
$\Psi$ represents the functional data structures and $\D$ collects the
session-typed channels along with an optional \emph{write token} $\wt$
(to resolve non-determinism in the semantics) used by the process.
Intuitively, the process sending a message \emph{must possess} the write
token which is then transferred to the receiver along with the write token.
Globally, the process owning the write token is activated to take the
next execution step.
Finally, $P$ is the process expression that is currently being executed and
the process offers channe $x$ of type $A$.
Similar to import tokens, the natural number annotations $q$ and $q'$ on the turnstile
denote the total and current potential stored in the process.
We will gradually explain each component of the language, initiating
with the basic system of session types.
For simplicity of exposition, we will display the yet unexplained
parts of the system in blue.

The operational semantics for session-typed programs are formalized as a
system of \emph{multiset rewriting rules}~\cite{cervesato2009relating}.
These rules consist of semantic objects $\proc{c}{P}$ and $\msg{c}{M}$ describing
process $P$ (or message $M$) providing service along channel $c$.
Remarkably, in this formulation, a message is just a particular form of process,
thereby not requiring any special rules for typing; it can be typed just as processes.
Since we track computational cost as well in NomosUC, we extend the semantic objects
to $\proc{c}{w, P}$ and $\msg{c}{w, P}$ where work counter $w$ stores the work performed
(number of computational steps executed) by process $P$ (resp. message $M$).

\subsection{Session Type Constructors}
\label{subsec:constructors}

The Curry-Howard correspondence gives each linear logic connective an
interpretation as a session type.
We follow a detailed description of each of these session type constructors,
but restricted to a subset that are sufficient for the applications of NomosUC.

\paragraph*{\textbf{Choice Operators}}
The internal choice $\ichoice{\ell : A_\ell}_{\ell \in L}$ constructor
is an $n$-ary labeled generalization of the additive disjunction $A \oplus B$.
A process that provides $x : \ichoice{\ell : A_\ell}_{\ell \in L}$ can send
any label $k \in L$ along $x$ and then continue by providing $x : A_k$. The
corresponding process is written as $(\esendl{x}{k} \semi P)$, where
$P$ is the continuation that provides $A_k$.
On the other end of the channel, the client branches on the label received along $x$.
The provider and client are typed according to the following $\oplus R$ and $\oplus L$
rules respectively.
\begin{mathpar}
  \infer[{\oplus}R]
  {\B{\Tokens \semi \Psi} \semi \wt, \D \entailpot{\B{q}}{\B{q'}} (\esendl{x}{k} \semi P) ::
    (x : \ichoice{\ell : A_\ell}_{\ell \in L})}
  {(k \in L) \qquad \B{\Tokens \semi \Psi} \semi \D \entailpot{\B{q}}{\B{q'}} P :: (x : A_k)}
\and
  \infer[{\oplus}L]
  {\B{\Tokens \semi \Psi} \semi \D, (x : \ichoice{\ell : A_\ell}_{\ell \in L})
    \entailpot{\B{q}}{\B{q'}} \ecase{x}{\ell}{Q_\ell}_{\ell \in L} :: (z : C)}
  {(\forall \ell \in L) \qquad \B{\Tokens \semi \Psi} \semi \wt, \D, (x : A_\ell)
    \entailpot{\B{q}}{\B{q'}} Q_\ell :: (z : C)}
\end{mathpar}
Additionally, the provider should possess the write token to be able to send the
label $k$. Dually, the client receives the write token with the label to continue
execution.

Operationally, since communication is asynchronous, the process
$(\esendl{c}{k} \semi P)$ sends a message $k$
along $c$ and continues as $P$ without waiting for it to be received.
As a technical device to ensure that consecutive messages on a
channel arrive in order, the sender also creates a fresh continuation
channel $c'$ so that the message $k$ is actually represented as
$(\esendl{c}{k} \semi \fwd{c}{c'})$ (read: send $k$ along $c$ and
continue as $c'$). The provider substitutes $c'$ for $c$ enforcing
that the next message is sent on $c'$.
The work counter of the process remains unaltered, and the new message
is created with work $0$.
\begin{tabbing}
$(\oplus S) : \proc{c}{w, \esendl{c}{k} \semi P} \step \proc{c'}{w, [c'/c]P},
\msg{c}{0, \esendl{c}{k} \semi \fwd{c}{c'}}$
\end{tabbing}
When the message $k$ is received along $c$, the client selects branch
$k$ and also substitutes the continuation channel $c'$ for $c$, thereby
ensuring that it receives the next message on $c'$. This implicit
substitution of the continuation channel ensures the ordering of the
messages.
The client process also collects the work performed by the message, if
there is any.
\begin{tabbing}
$(\oplus C) :$ \= $\msg{c}{w, \esendl{c}{k} \semi \fwd{c}{c'}},
\proc{d}{w', \ecase{c}{\ell}{Q_\ell}}
\step \proc{d}{w+w',[c'/c]Q_k}$
\end{tabbing}

The dual of internal choice is \emph{external choice} $\echoice{\ell :
A_\ell}_{\ell \in L}$, the $n$-ary labeled generalization of the
additive conjunction $A \with B$. This dual operator simply reverses
the role of the provider and client. The provider process of
$x : \echoice{\ell : A_\ell}_{\ell \in L}$ branches on receiving a label
using the expression $\ecase{x}{\ell}{Q_\ell}_{\ell \in L}$,
while the client sends one such label in $L$ using the expression $(\esendl{x}{k} \semi P)$.
% $k \in L$ (described in $\with R$), while the client sends this label
% (described in $\with L$).
% \begin{mathpar}
%   \footnotesize
%   \infer[\with R]
%   {\B{k \semi \Tokens \semi \Psi} \semi \D \entailpot{\B{q}}{\B{q'}} \ecase{x}{\ell}{P_\ell}_{\ell \in L} ::
%     (x : \echoice{\ell : A_\ell}_{\ell \in L})}
%   {(\forall \ell \in L) \qquad \B{k \semi \Tokens \semi \Psi} \semi \wt, \D
%     \entailpot{\B{q}}{\B{q'}} P_\ell :: (x : A_\ell)}
% \end{mathpar}
% \begin{mathpar}
%   \footnotesize
%   \infer[\with L]
%   {\B{k \semi \Tokens \semi \Psi} \semi \wt, \D, (x : \echoice{\ell : A_\ell}_{\ell \in L})
%     \entailpot{\B{q}}{\B{q'}} \esendl{x}{k} \semi Q :: (z : C)}
%   {\B{k \semi \Tokens \semi \Psi} \semi \D, (x : A_k) \entailpot{\B{q}}{\B{q'}} Q :: (z : C)}
% \end{mathpar}
Dual to internal choice, the client contains the write token which is
sent to the provider along with the label.
The operational semantics rules are also just the inverse of internal choice,
and therefore skipped for brevity.

\paragraph*{\textbf{Termination}}
The type $\one$, the multiplicative unit of linear logic, represents
termination of a process, which (due to linearity) is not allowed to use
any channels. A terminating process offering on $x : \one$ simply
closes channel $x$ while the client waits for this close message to arrive.
\begin{mathpar}
  \infer[{\one}R]
  {\B{k \semi \Tokens \semi \Psi} \semi \wt \entailpot{\B{q}}{\B{q'}} \eclose{x} :: (x : \one)}
  {\B{q = 0}}
  \and
  \infer[{\one}L]
  {\B{k \semi \Tokens \semi \Psi} \semi \D, (x : \one) \entailpot{\B{q}}{\B{q'}} (\ewait{x} \semi Q) :: (z : C)}
  {\B{k \semi \Tokens \semi \Psi} \semi \wt, \D \entailpot{\B{q}}{\B{q'}} Q :: (z : C)}
\end{mathpar}
Similar to internal choice, the closing process transfers the write
token to its waiting client along with the close message.
Additionally, the terminating process does not store
any potential since it cannot take any further execution steps
(explained more in Section~\ref{sec:import}).
% Operationally, the provider converts into a closing message
% with no continuation since the offered channel terminates.
% \begin{tabbing}
% $(\one S) : \proc{c}{\eclose{c}} \step \msg{c}{\eclose{c}}$ \\
% $(\one C) : \msg{c}{\eclose{c}}, \proc{d}{\ewait{c} \semi Q} \step
% \proc{d}{Q}$
% \end{tabbing}

% The provider receives the branching label $k$ sent by the provider. Both
% processes perform appropriate substitutions to ensure the order of messages
% sent and received is preserved.
% \[
% \begin{array}{lll}
% (\with S) & \proc{d}{\esendl{c}{k} \semi Q} \step \msg{c'}{\esendl{c}{k}
% \semi \fwd{c'}{c}}, \proc{d}{[c'/c]Q} & \fresh{c'} \\
% (\with C) & \proc{c}{\ecase{c}{\ell}{Q_\ell}_{\ell \in L}},
% \msg{c'}{\esendl{c}{k} \semi \fwd{c'}{c}} \step \proc{c'}{[c'/c]Q_k}
% \end{array}
% \]

\paragraph*{\textbf{Exchanging Functional Data}}
So far, we have discussed the channels in $\D$ in the typing judgment for NomosUC.
Now, we turn out attention to the functional layer $\Psi$ that contains the
traditional data structures and values.
Communicating a \emph{value} of the functional fragment along a channel
is expressed at the type level by adding the following two session types.
\begin{center}
\begin{minipage}{0cm}
\begin{tabbing}
$A ::= \ldots \mid \tau \arrow A \mid \tau \product A$
\end{tabbing}
\end{minipage}
\end{center}
Here, $\tau$ describes a functional type, e.g. $\m{int}, \m{bool}, \tau \; \m{list}$, etc
(we assume the language contains standard functional types).
The type $\tau \arrow A$ prescribes receiving a value of type $\tau$
with continuation type $A$, while its dual $\tau \product A$ prescribes
sending a value of type $\tau$ with continuation $A$. The corresponding
typing rules for arrow ($\arrow R, \arrow L$) are given below.
\begin{mathpar}
  \infer[\arrow R]
  {\B{\Tokens} \semi \Psi \semi \D \entailpot{\B{q}}{\B{q'}}
  \erecvch{x}{v} \semi P :: (x : \tau \arrow A)}
  {\B{\Tokens} \semi \Psi, (v : \tau) \semi \wt, \D \entailpot{\B{q}}{\B{q'}}
  P :: (x : A)}
  %
  \and
  %
  \inferrule*[right = $\arrow L$]
  {\B{r' = p+q'} \qquad
  \B{\Psi \share (\Psi_1, \Psi_2)} \qquad
  \Psi_1 \exppot{\B{p}} M : \tau \\
  \B{\Tokens} \semi \Psi_2 \semi \D, (x : A) \entailpot{\B{q}}{\B{q'}}
  Q :: (z_k : C)}
  {\B{\Tokens} \semi \Psi \semi \wt, \D, (x : \tau \arrow A)
  \entailpot{\B{q}}{\B{r'}} \esendch{x}{M} \semi Q :: (z : C)}
\end{mathpar}
As indicated in the $\arrow R$ rule, receiving a value $y : \tau$ on a channel
$x : \tau \arrow A$ adds it to the functional context $\Psi$. On the
other hand, sending (value of) expression $M$ on channel $x : \tau \arrow A$
requires that $M$ has type $\tau$ (third premise).
The premises indicated in blue describe how potential is divided across
the functional and session-typed layers and will be described further in Section~\ref{sec:import}.
Intuitively, the potential in functional context $\Psi$ is \emph{shared}
between $\Psi_1$ and $\Psi_2$ (second premise); $\Psi_1$ is used to type
$M$ while $\Psi_2$ is passed on to the continuation $Q$.
The $\product$ operator is dual to $\arrow$ reversing the roles of provider and client,
and we omit those rules for brevity.

\subsection{Expressing ITMs With Session Types}
\label{subsec:communicators}
Despite the seemingly structured nature of execution in UC, the framework is meant to
capture arbitrary connections, communication patterns, or configurations of ITMs.
However, linear session types can prove to be quite restrictive in terms of realizing
arbitrary communication patterns in practical cryptographic protocols.
In this section, we highlight a common communication pattern in UC that requires introducing
the notion of shared session types~\cite{balzer2017manifest} to be expressible in NomosUC.

We motivate the use of shared session types through an example of an ITM communication pattern
pattern common in UC, but difficult to capture with a single session type.
Imagine two ITMS $P$ and $Q$ which communicate in the following way. If $P$ is activated first,
it sends a message to $Q$ and terminates, but if $Q$ is activate first it writes a message to $P$, 
$P$ writes a message back, and communication terminates.
Session types require it to be statically known which party will write next to a channel so such
a communication pattern can not be realized by juts a single session type.

If we try to resolve this by splitting communcation among two channels, we may type them as
%\begin{center}
\vspace{2mm}

{\centering
 $\m{PtoQ} = \ichoice{\mb{``one''}: int \tensor 1}$ \\
 $\m{QtoP} = \echoice{\mb{``one''}: int \tensor \ichoice{\mb{``end''}: int \tensor 1}}$
%\end{center}
\par}
\vspace{2mm}
However, session types naturally impose a \emph{parent-child relationship} among processes by requiring
every channel to have a provider and client endpoint.
Moreover, these endpoints are fixed, i.e., a process cannot transition from being a client of
a channel to its provider, or vice-versa.
If $P$ and $Q$ are connected with channels they offer each other, the provider-client relationship
is undefined as one process must spawn the other.
In general, this restriction results in an acyclic tree-like topology among processes where parents progressively
spawn child processes becoming the client to the channel provided by the child.
%On the other hand, UC places no such restriction on communication and cyclic communication
%is quite common in UC protocols.
Thus, a direct translation from ITM terms to NomosUC expressions can result in cyclic
dependency among channels, thus causing the program to be ill-typed.

% The pattern emerges from the following code: a machine $P$ either writes to another machine $Q$ or is written to by $Q$. 
% At first glance, it is a trivial scenario, but we encounter a problem trying to encode this with a single session type.
% A example are machines $P$ and $Q$ which execute as follows:
% \begin{itemize}
% 	\item An external machines flips a bit and activates either $P$ or $Q$.
% 	\item If $P$ is activated it writes to $Q$ and the execution terminates. 
% 	\item Otherwise, if $Q$ is activated, it writes a message to $P$, $P$ writes something back, and the execution terminates.
% \end{itemize}
% A single type between $P$ and $Q$ would have to allow either of the two parties to write on the channel, but session types require it to be statically known.
%
% If we try to separate communication between $P$ and $Q$ into two uni-directional channel, we can express the session type for each of the channels:
% \begin{center}
% \parbox{0cm}{
% \begin{tabbing}
% $\m{PtoQ} = \ichoice{\mb{``one''}: int \tensor 1}$ \\
% $\m{QtoP} = \echoice{\mb{``one''}: int \tensor \ichoice{\mb{``end''}: int \tensor 1}}$
% \end{tabbing}}
% \end{center}
%
% This approach, however, poses another problem. 
% Channels are only created by a process offering them, and a process can only offer a single channel.
% Without adding any additional processes, $P$ must offer a channel to $Q$ and $Q$ offer one to $P$. 
% Each of them becomes both a provider and a client to the other, and provider/client ambiguity is not allowed in Nomos. 
% Logically, one process must have spawned the other, and such a cycle would be impossible to realize.

To enable arbitrary communication between two processes without forcing a parent-child relationship, we introduce 
the concept of \emph{providerless channels} using \emph{communicator processes} that rely on recently introduced shared session types~\cite{balzer2017manifest}.
Communicators act as a buffer between two processes allowing them to exchange messages via the communicator.
Communicators also break the parent-child relationship by offering a \emph{shared channel that can have multiple clients}
(unlike linear channels that can have only one client) that is then used by both the sender and receiver processes.

The communicator has the following polymorphic type:
%\begin{center}

{\centering
\vspace{2mm}
\parbox{0cm}{
\begin{tabbing}
$\m{comm[\tau]} = \up \echoice{$\=$\mb{SEND}: \m{\tau} \arrow \m \down \m{comm[\tau]},$\\
\>$\mb{RECV}: \ichoice{$\=$\mb{yes}: \m{\tau} \;\product \down \m{comm[\tau]},
\mb{no}: \; \down \m{comm[\tau]}}}$
\end{tabbing}}
%\end{center}
\vspace{2mm}
\par}

The communicator type relies on shared type constructors $\up$ and $\down$.
The type initiates with an $\up$ indicating that the communicator channel must be \emph{acquired} to interact with it.
Because the channel has two clients (the sender and the receiver), either of them can send this acquire request.
Once acquired, the sender uses the $\mb{SEND}$ branch of the type while the receiver interacts with the $\mb{RECV}$ branch.
In the former case, the communicator receives the $\mb{SEND}$ message from the sender followed by the actual
message of type $\tau$ as indicated by the $\arrow$ constructor.
Then the sender releases the channel so that the receiver can interact with the communicator and check, with $\mb{RECV}$ whether 
there is a message waiting or not.
%Then, the type transitions to $\down$ meaning that the sender sends a \emph{release} request to the communicator effectively
%detaching from it so that the receiver can interact with the communicator.
%In the latter case, the communicator gets the $\mb{RECV}$ request from the receiver and checks whether the there is a message
%waiting for the receiver or not.
%If a message is found, the communicator replies with a $\mb{yes}$ message followed by the message of type $\tau$, and if not,
%the $\mb{no}$ message is sent.
%Finally, in either case, the communicator detaches from the receiver as indicated by the $\down$ constructor.

Shared session types impose an \emph{acquire-release} discipline on processes; 
a client must acquire the channel offered by a shared process to interact with it
and must release this channel after the interaction.
The corresponding typing rules are
\begin{mathpar}
  \infer[\up L]
  {\Tokens \semi \Psi \semi \wt, \D, (x : \up A_L)
  \entailpot{q}{q'} \eacquire{y}{x} \semi Q :: (z : C)}
  {\Tokens \semi \Psi \semi \D, (y : A_L)
  \entailpot{q}{q'} Q :: (z : C)}
  %
  \and
  %
  \infer[\up R]
  {\Tokens \semi \Psi \semi \D \entailpot{q}{q'}
  \eaccept{y}{x} \semi P :: (x : \up A_L)}
  {\Tokens \semi \Psi \semi \wt, \D \entailpot{q}{q'} P :: (y : A_L)}
\end{mathpar}
The $\up L$ rule describes a client acquiring a shared channel $x$
and obtaining a private linear channel $y$ along which it can communicate
with the corresponding acquired process.
Correspondingly, the $\up R$ rule describes the shared process
accepting the acquire request and creating the fresh linear channel $y$.
The release-detach rules corresponding to the $\down$ type constructor
are exact dual of acquire-accept.

An important caveat here is that shared channels can introduce non-determinism
in the semantics since multiple clients can send the acquire request simultaneously.
To address this problem, we require that the acquiring client \emph{possess
the write token}.
Since write tokens are treated as a linear quantity, only one of the client can
possess it enabling only that process to acquire the shared channel.
Remarkably, this write token can resolve both read and write non-determinism
due to linearity of the channels.

The communicator type is parametric in the message type sent over it, and it is restricted to sending 
functionally typed messages only. 
Therefore, in order to continue to meaningfully use session types we give a construction, illustrated in Figure~\ref{ref:newpandq}, that realized our desired \emph{providerless channels} using communicators and some useful shell code.
We wrap each of $P$ and $Q$ in shell code, call them $S_P$ and $S_Q$, which spawn two dummy processes each.
The processes, call them $a$ and $b$, perform two tasks. First $a$ and $b$ offer, say to $P$, the desired session types of \m{PtoQ} and \m{QtoP}.
Second, $a$ and $b$ each communicate with one communicator and convert functionally typed messages received from them to ones that are send along their session typed channels to $P$ (and vice versa).
Although the process code for terms $a$ and $b$ differ based on the session type being used, they can be systematically generated as a case switch between functional and session typed messages.
\begin{figure}
	\begin{subfigure}{0.3\textwidth}
	\centering
	\includegraphics[scale=0.4]{figures/p_and_q.png}
	\caption{$P$ and $Q$ connected by two logical channels which are actually implemented by the figure on the right.}
	\label{fig:pandq}
	\end{subfigure}
	~ \ \ \ \ 
	\begin{subfigure}{0.6\textwidth}
	\centering
	\includegraphics[scale=0.4]{figures/new_p_and_q.png}
	\caption{We can realize the left be intermediating communication with communicators. The direction of messages from $P$ to $Q$ still suggest a cycle but the communicator is provider (dot) to \emph{both} shell codes.}
	\label{fig:newpandq}
	\end{subfigure}
	\caption{Two ITM configurations. One possible with ITMs (left) and one realized in NomosUC (right). Arrows indicate direction of messages and the dot indicates the provider of the channel.}
\end{figure}

In general, we represent communication between any two ITMs as two providerless channels.
We further don't restrict the types of the two channels, and allow them to each be bidirectional as well.
For example, communication between \Fcom and a protocol party need only make use of one of the channels (typed by \m{sender} if the party is the committer) and type the other one with $1$.
In such cases, we omit the second channel from the diagram entirely (see Section~\ref{sec:execuc}.
However, for simplicity, and to be as generalized as possible, when both channels are used we refer to them as being uni-directional.


\section{Import and Potential in NomosUC} \label{sec:import}
\paragraph*{\textbf{Import Tokens}}
A defining aspect of NomosUC is the representation of import tokens in the type system.
This enables a static reasoning of the import mechanism in NomosUC.
To this end, we introduce a novel token context $\Tokens$
in the process typing judgment to denote the real and virtual tokens.
This context contains the information on the total and current tokens
of each type and the \emph{security parameter}.
Before we explain the token context we first motivate the need for virtual tokens
It is a common technique in UC, especially in simulators, to internally run, or simulate, 
the code of other ITIs. In NomosUC, we wish to enable the same sandbox running of processes,
but its channels may have import requirements. It doesn't make sense to
send real import to such a process, because, intuitively the internal process should use the
import and, therefore, potential available to the ``host'' process~\footnote{As far as resource-contraints go, simulating a process should be no different from natively executing its code.}. Therefore, in 
order to satisfy the types of internally simulated processes we introduce a virtual 
tokens construction. 
By default, every process contains a unique real token type $K_0$
and corresponding number of total and current tokens $n$ and $n'$ resp.\
denoted by $K_0 \hookrightarrow (n, n')$.
There is no mechanism to create a real token; they can only be passed on to
a process during its creation, or be exchanged between processes during communication.
Virtual tokens, on the other hand, can be created (under certain conditions,
see below) by a process.
However, all tokens follow a \emph{token hierarchy}: $K_0 \to K_1 \to K_2 \to \ldots K_m$
such that we can only use tokens of type $K_i$ to withdraw tokens of type
$K_{i+1}$~\footnote{ITIs in UC can have arbitrary simulation depth, i.e. a process simulates another process which simulates another process. Despite this, we can statically define the token heirarchy because it is statically known the maximum simulation depth of any process in the UC execution.}.
In addition, we use a global function $\GlobalF$ as the connection
rate between two successive token types.

To maintain well-typedness of a process, an implicit side condition is
that the token context must always be \emph{valid}.
This involves ensuring that if the context contains $m'$ current tokens of type
$K_i$, it can only contain at most $\GlobalF(m,k)$ total tokens of type
$K_{i+1}$. The inductive rules for validity of a token context are below.
\begin{mathpar}
  \infer
  {K_0 \hookrightarrow (t_0, t_0') \;\; \m{valid}}
  {}
  \and
  \infer
  {k \and \Tokens, K_{i+1} \hookrightarrow (t_{i+1}, t_{i+1}')\;\; \m{valid}}
  {\Tokens\;\; \m{valid} \and
  K_{i} \hookrightarrow (t_i, t_i') \in \Tokens \and
  t_{i+1} \leq \GlobalF(t_i',k)}
\end{mathpar}
Since validity of a token context is a side condition, we mandate
that it is implicitly satisfied by all the process typing rules
presented in our paper.
From an implementation point-of-view, this validity check only
needs to be performed when the token context changes (in the
rules that follow).

As a first step in introducing program notation for import tokens, we need 
syntax for creating new tokens of a given token type.
We call this construct $\m{withdrawToken} \; K_i \; n \; K_{i+1}$.
\begin{mathpar}
  \inferrule*[right=$\m{tok}$]
  {k \semi \Tokens, K_{i+1} \hookrightarrow (t_{i+1} + n, t_{i+1}' + n) \semi
  \Psi \semi\wt, \D \entailpot{\B{q}}{\B{q'}} P :: (x : A)}
  {k \semi \Tokens, K_{i+1} \hookrightarrow (t_{i+1}, t_{i+1}') \semi \Psi \semi \wt, \D \entailpot{\B{q}}{\B{q'}} \hspace{4em} \\
    \hspace{5em}\m{withdrawToken} \; K_i \; n\; K_{i+1}  \semi P :: (x : A)}
\end{mathpar}
The above construct generates $n$ new tokens of type $K_{i+1}$ and adds
them to both the total and current count for $K_{i+1}$ in the token
context $\Tokens$.
The implicit side condition of the validity of the token context ensures
that $t_{i+1} + n \leq \GlobalF(t_i',k)$ where $K_i \hookrightarrow (t_i, t_i') \in \Tokens$.
If this side condition fails, the above construct would fail to typecheck.

In addition, we also introduce two dual constructs for exchanging tokens
between processes.
To this end, we first introduce two new type constructors.
\begin{center}
\begin{minipage}{0cm}
\begin{tabbing}
$A ::= \ldots \mid \tpaypot{A}{r : K} \mid \tgetpot{A}{r : K}$
\end{tabbing}
\end{minipage}
\end{center}
The provider of $x : \tgetpot{A}{r : K}$ is required to receive
$r$ import tokens of type $K$ from the client using the construct
$\eget{x}{r : K}$. Dually, the client needs to pay this import
using the construct $\epay{x}{r : K}$.
The corresponding typing rules are
\begin{mathpar}
  \footnotesize
  \infer[\getpot R]
  {k \semi \Tokens, K_i \hookrightarrow (t_i, t_i') \semi \Psi \semi \D \entailpot{\B{q}}{\B{q'}} \eget{x}{r : K_i} \semi P ::
  (x : \tgetpot{A}{r : K_i})}
  {k \semi \Tokens, K_i \hookrightarrow (t_i, t_i'+r) \semi \Psi \semi \wt, \D \entailpot{\B{q}}{\B{q'}} P :: (x : A)}
  %
  \and
  %
  \infer[\getpot L]
  {k \semi \Tokens, K_i \hookrightarrow (t_i, t_i'+r) \semi \Psi \semi \wt, \D, (x : \tgetpot{A}{r : K_i}) \entailpot{\B{q}}{\B{q'}}
  \epay{x}{r : K_i} \semi P :: (z : C)}
  {k \semi \Tokens, K_i \hookrightarrow (t_i, t_i') \semi \Psi \semi \D, (x : A) \entailpot{\B{q}}{\B{q'}} P :: (z : C)}
\end{mathpar}
In the rule $\getpot R$, process $P$ storing $(t_i, t_i')$ import tokens of type $K_i$
receives $r$ additional $K_i$ tokens adding it to the current token counter, thus
the continuation executes with $(t_i, t_i'+r)$ tokens of type $K_i$.
Note that validity of token context is trivially satisfied in this case since the
process is gaining import tokens.
%
In the dual rule $\getpot L$, a process containing $(t_i, t_i'+r)$ tokens of type $K_i$
pays $r$ units along channel $x$ leaving $(t_i, t_i')$ import tokens of type $K_i$ with
the continuation.
In this case, the validity of the token context establishes that $t_{i+1} \leq \GlobalF(t_i',k)$,
a condition that is necessary for successful typechecking.
The typing rules for the dual constructor $\tpaypot{A}{r : K}$
are the exact inverse.
Similar to prior rules, the sender transfers the write token $\wt$
along with the potential to the receiver.

The need for virtual tokens in UC arises because machines often simulate
other machines as part of their construction. The program notation for \msf{withdrawToken}
does not require an inverse to exchange tokens \textit{back} from type $K'$ to $K$.
The reason is that virtual tokens only exist to allow re-use of existing processes 
and satisfy their types. Type $K$ tokens are not deducted when new ones of type $K'$ 
are created is because, in reality, siulating a process by calling it or simply running
its code natively should be equivalent in cost. Therefore, there is also no need to 
include an inverse of \msf{withdrawToken} which exchanges from $K'$ to $K$.

\paragraph*{\textbf{Potential}}
The main purpose of import tokens is to bound the number of execution steps of ITMs.
We achieve that purpose in NomosUC by introducing the notion of \emph{potential}.
Potential is an abstract quantity represented by a natural number stored
within each process.
To take an execution step, a process consumes \emph{one} unit of potential.
Therefore, the total potential stored in a process upper bounds the total
number of execution steps that will ever be taken by the process.
Furthermore, potential is represented syntactically, thus providing a
static upper bound on the execution cost.
Since execution cost needs to eventually connect to the import tokens, all
we need is a mechanism to generate potential using import tokens.
To this end, we introduce a novel construct $\m{genPot} \; r$.
\begin{mathpar}
  \inferrule*[right=$\m{pot}$]
  {q+r \leq \GlobalF(t_{\depth}',k) \and K_{\depth} \hookrightarrow (t_{\depth}, t_{\depth}') \in \Tokens \\\\
  k \semi \Tokens \semi \Psi \semi \wt, \D \entailpot{q+r}{q'+r} P :: (x : A)}
  {k \semi \Tokens \semi \Psi \semi \wt, \D \entailpot{q}{q'} \m{genPot} \; r \semi P :: (x : A)}
\end{mathpar}
A process initially storing $(q, q')$ potential units generates $r$ potential so that
the continuation contains $(q+r, q'+r)$ potential units.
Note, however, that the maximum potential allowed is bounded by the number of import tokens
a process contains.
To this end, we introduce a \emph{token depth}: $\depth$ that signifies the number of token
types that exist in the token hierarchy.
Thus, when generating potential, the typechecker verifies that the total new potential (i.e., $q+r$)
is bounded by $\GlobalF(t_{\depth}',k)$ where $(t_{\depth}, t_{\depth}')$ is the number of tokens
of type $K_{\depth}$, the highest token in the hierarchy.

The purpose of introducing potential into NomosUC is to bound the
number of execution steps.
Therefore, we introduce the $\etick{r}$ construct that consumes $r$
potential from the stored process potential $q$, and the continuation remains with
$p = q-r$ units, as described in the rule below.
\begin{mathpar}
  \footnotesize
  \infer[\m{tick}]
  {k \semi \Tokens \semi \Psi \semi \wt, \D \entailpot{q}{q'+r} \etick{r} \semi P :: (x : A)}
  {k \semi \Tokens \semi \Psi \semi \wt, \D \entailpot{q}{q'} P :: (x : A)}
\end{mathpar}
NomosUC is equipped with a cost instrumentation engine that automatically
inserts a $\etick{1}$ construct before each primitive operation.
This enables us to simulate the cost model that counts the total number of
execution steps.
However, since ticks are not tied directly to the type system, the programmer
can modify the cost model to only count the resource they are interested in
(e.g., message exchange, process spawns, etc.).

% \begin{mathpar}
%   \D_1 \equiv_Z \D_2 \\
%   \D \overset{(import, potential, cost)}{\vDash} P :: \D' \\
%   A \equiv B \\
%   \infer[]
%   {\vars \vdash \D_1, (x : A) \equiv \D_2, (x : B)}
%   {\vars \vdash \D_1 \equiv \D_2 \and \vars \vdash A \equiv B}
% \end{mathpar}

\paragraph*{\textbf{Shared Channels}}
Until now, we have only described the linear fragment of session types
in Nomos.
Unfortunately, this fragment imposes a strong restriction on programs.
The only provision to spawn new processes is when a parent process creates a new
child process, and uses an exclusive linear channel to communicate with the child.
Thus, any two processes connected by a channel inherently maintain this parent-child
relationship.
Intuitively, this leads to a linear tree-like hierarchy among the processes,
thus preventing a cycle in the process graph.

Unfortunately, this restriction precludes practical programming scenarios
where process topologies indeed have a cyclic dependency (e.g. ring networks,
dining philosophers, etc.).
Recognizing this limitation, Balzer et al.~\cite{balzer2017manifest} proposed
a \emph{shared} extension of session types that allows arbitrary process topologies.
The types are extended as follows:
\begin{center}
\begin{minipage}{0cm}
\begin{tabbing}
$A_L ::= \down A_S \mid \ldots \text{(all linear types $A$ so far)}\ldots$\\
$A_S ::= \up A_L$
\end{tabbing}
\end{minipage}
\end{center}
We have found this extension exceedingly helpful in the design and implementation
of cryptographic protocols.

Shared session types impose an \emph{acquire-release} discipline on processes; 
a client must acquire the channel offered by a shared process to interact with it
and must release this channel after the interaction.
The corresponding typing rules are
\begin{mathpar}
  \footnotesize
  \infer[\up L]
  {k \semi \Tokens \semi \Psi \semi \wt, \D, (x : \up A_L)
  \entailpot{q}{q'} \eacquire{y}{x} \semi Q :: (z : C)}
  {k \semi \Tokens \semi \Psi \semi \D, (y : A_L)
  \entailpot{q}{q'} Q :: (z : C)}
  %
  \and
  %
  \infer[\up R]
  {k \semi \Tokens \semi \Psi \semi \D \entailpot{q}{q'}
  \eaccept{y}{x} \semi P :: (x : \up A_L)}
  {k \semi \Tokens \semi \Psi \semi \wt, \D \entailpot{q}{q'} P :: (y : A_L)}
\end{mathpar}
The $\up L$ rule describes a client acquiring a shared channel $x$
and obtaining a private linear channel $y$ along which it can communicate
with the corresponding acquired process.
Correspondingly, the $\up R$ rule describes the shared process
accepting the acquire request and creating the fresh linear channel $y$.
The release-detach rules corresponding to the $\down$ type constructor
are exact dual of acquire-accept.

An important caveat here is that shared channels can introduce non-determinism
in the semantics.
The only source of non-determinism is that a shared process can latch on to
any of the acquiring clients.
To address this problem, we require that the acquiring client \emph{possess
the write token}.
Since write tokens are treated as a linear quantity, only one of the client can
possess it enabling only that process to acquire the shared channel.
Remarkably, this write token can resolve both read and write non-determinism
due to linearity of the channels.


\paragraph*{\textbf{Process Definitions and Sandboxing}}
Process definitions have the form
$\Psi \semi \D \entailpot{q}{q'} f\{\Tokens\} :: (x : A) = P$ where $f$
is the name of the process and $P$ its definition.
We parameterize the process $f$ with the number and type of
real tokens it would need.
All definitions are collected in a fixed global process signature $\Sg$.
Also, since process definitions are mutually recursive, it is required that
for every process in the signature is well-typed w.r.t. $\Sg$.
A new instance of a defined process $f$ can be spawned with
the expression $\procdef{f\{\Tokens\}}{\overline{y}}{x} \semi Q$
where $\overline{y}$ is a sequence of variables matching the
antecedents $\Psi$ and $\D$.
Sometimes a process invocation is a \emph{tail call}, written without
a continuation as $\procdef{f\{\Tokens\}}{\overline{y}}{x}$.
This is a short-hand for
$\procdef{f\{\Tokens\}}{\overline{y}}{x'} \semi \fwd{x}{x'}$ for a
fresh variable $x'$, that is, a fresh channel is created and
immediately identified with $x$.

An important note here is that NomosUC allows executing processes in
a \emph{sandbox}.
Therefore, a process invocation can either be \emph{regular} or in a
\emph{sandbox}.
Syntactically, we use the same term for both but the two invocations
are distinguished via the token type passed into the call.
For a regular call, the parent process passes in a real token type,
while for a sandboxed call, a virtual token type is passed in.
We have a similar distinction for $\m{pay}$ and $\m{get}$ expressions:
if a real token is passed into these terms, it's a regular token
exchange; if a virtual token is passed in, it's a sandboxed $\m{pay}$
and $\m{get}$.

\subsection{Preservation and Progress}
The main type safety theorems that exhibit the deep connection between our type
system and the operational semantics are the usual \emph{type
preservation} and \emph{progress}, sometimes called \emph{session
fidelity} and \emph{deadlock freedom}, respectively.

To exhibit these theorems, we first need to introduce semantic objects
$\proc{c}{w, P}$ and $\msg{c}{w, M}$.
The former (resp. latter) denotes a process (resp. message) executing
expression $P$ (resp. $M$) offering channel $c$ and having performed
work $w$ so far.
The work counter keeps track of execution steps taken by a process,
giving rise to the following semantics rule:
\begin{tabbing}
  $(\m{tick}) : \proc{c}{w, \etick{r} \semi P} \step \proc{c}{w+r, P}$
\end{tabbing}
A multiset of such semantic objects communicating with each other
is known as a \emph{configuration}.
A configuration is typed w.r.t. a signature providing the type declaration
of each process.
A signature $\Sg$ is \emph{well formed} if
(a) every type definition $V = A_V$ is \emph{contractive},
and (b) every process definition
$\Psi \semi \D \vdash f \{\Tokens\} = P :: (x : A)$ in $\Sg$
is well typed according to the process typing judgment, i.e.
$\Tokens \semi \Psi \semi \D \vdash P :: (x : A)$.
 
A key question then is how to type these configurations.
Since they consist of both processes and messages, they
both \emph{use} and \emph{provide} a collection of channels.
Another goal with the type safety theorems is to establish a connection
between the statically determined import tokens of a process,
its total potential, and the dynamically evolving work counters
that account for the total number of execution steps.
We use the following judgment to type a configuration.
\[
\D_1 \overset{(T, Q)}{\underset{W}{\vDash}} \config :: \D_2
\]
It states that the configuration $\config$
uses the channels in the context $\D_1$ and provides the channels in
the context $\D_2$.
In addition, $T$ and $Q$denote the total number of real tokens
potential contained in a configuration.
Similarly, $W$ denotes the total work performed by a configuration.
All these quantities are computed by adding the individual tokens,
potential, and work of each semantic object.
\begin{figure}[t]
\begin{mathpar}
\infer[\m{empty}]
{\D \overset{(0, 0)}{\underset{0}{\vDash}} (\cdot) :: \D}
{}
\and
\infer[\m{compose}]
{\D_0 \overset{(T_1+T_2, Q_1+Q_2)}{\underset{W_1+W_2}{\vDash}} (\config_1 \; \config_2) :: \D_2}
{\D_0 \overset{(T_1, Q_1)}{\underset{W_1}{\vDash}} \config_1 :: \D_1 \qquad
\D_1 \overset{(T_2, Q_2)}{\underset{W_2}{\vDash}} \config_2 :: \D_2}
\and
\infer[\m{proc}]
{\D, \D_1 \overset{(t, q)}{\underset{w}{\vDash}} \proc{c}{w, P} :: (\D, (c : A) )}
{\Tokens, K_0 \hookrightarrow t \semi \cdot \semi \D_1 \entailpot{q} P :: (c : A)}
\and
\infer[\m{msg}]
{\D, \D_1 \overset{(t, q)}{\underset{w}{\vDash}} \msg{c}{w, M} :: (\D, (c : A) )}
{\Tokens, K_0 \hookrightarrow t \semi \cdot \semi \D_1 \entailpot{q} M :: (c : A)}
\end{mathpar}
\caption{Typing rules for a configuration}
\label{fig:config_typing}
\end{figure}

The configuration typing judgment is defined using
the rules presented in Figure~\ref{fig:config_typing}.
%
The rule $\m{empty}$ defines that an empty configuration
is well-typed with $(T, Q, W) = (0, 0, 0)$ and uses and
provides the same set of channels.
The $\m{compose}$ rule combines two configurations by canceling out
the common channels and adding the individual tokens, potential, and work.
The $\m{proc}$ rule creates a configuration out of a single process
and uses its tokens, potential, and work as the annotations for the
configuration.
Similarly, the $\m{msg}$ rule creates a configuration out of a single message.

\begin{theorem}[Type Preservation]
\label{thm:preservation}
Suppose we have a well-typed configuration
$\D \overset{(T_1, Q_1)}{\underset{W_1}{\vDash}} \config_1 :: \D'$ such
that there exists a polynomial $\mathfrak{p}$ such that $\mathfrak{p}(T_1) \geq Q_1+W_1$.
If $\config_1 \step \config_2$, then there exist $T_2$, $Q_2$, and $W_2$ such
that $\D \overset{(T_2, Q_2)}{\underset{W_2}{\vDash}} \config_2 :: \D'$,
and $\mathfrak{p}(T_2) \geq Q_2+W_2$.
\end{theorem}
\begin{proof}
  By case analysis on the transition rule, applying inversion to the
  given typing derivation, and then assembling a new derivation of
  $\dc$.
\end{proof}

A process or message is said to be \emph{poised} if it is trying to
communicate along the channel that it provides.  A poised process is
comparable to a value in a sequential language. A configuration is
poised if every process or message in the configuration is poised.
Conceptually, this implies that the configuration is trying to communicate
externally, i.e. along one of the channel it provides.
The progress theorem then shows that either a configuration can take a
step or it is poised.  To prove this I show first that the typing
derivation can be rearranged to go strictly from right to left and
then proceed by induction over this particular derivation.

\begin{theorem}[Global Progress]
\label{thm:progress}
\mbox{}
If $\cdot \overset{(T, Q)}{\underset{W}{\vDash}} \config :: \D$ then either
\begin{enumerate}
\item[(i)] $\config \mapsto \config'$ for some $\config'$, or
\item[(ii)] $\config$ is poised.
\end{enumerate}
\end{theorem}
\begin{proof}
By induction on the right-to-left typing of $\config$ so that either
$\config$ is empty (and therefore poised) or
$\config = (\dc\; \proc{c}{w, P})$ or
$\config = (\dc\; \msg{c}{w, M})$. By induction hypothesis, $\dc$ can
either take a step (and then so can $\config$), or $\dc$ is poised.  In
the latter case, I
analyze the cases for $P$ and $M$, applying multiple steps of
inversion to show that in each
case either $\config$ can take a step or is poised.
\end{proof}


%\subsection{UC Communicators} \label{sec:communicators}
%% all the processes conncted together leads to a cycle of linear channels ==> Z <--> P so here we use a communicator 
%The UC execution connects the protocol to the environment in both directions of communication.
%This poses a technical challenge where, if linear channels are used, the resulting topology contains a cycle of linear channels: the environmtne offers a channel to the wrapper and the wrapper to the environment.
%Such cycles violate type preservation because a client is acquiring its client~\ref{dasnomos}.
%Therefore, we use a message buffers called communicators which offered shared channels that both ends of the communication can use.
%Communicators are used in the main UC execution in Section~\ref{sec:execuc} to connect the main processes together, as well as within the \partywrapper. 
%
%A communicator has a \emph{sender} and a \emph{receiver}. 
%The shared channel offered by the communicator has the following polymorphic session type:
%\begin{tabbing}
%  $\mi{stype} \; \m{comm[K][msg]\{n\}} =$\\
%  \quad $\up \tgetpot{}{n+1: K} \echoice{$\=$\mb{push} : \m{msg} \arrow
%  \down \m{comm[msg]},$\\
%  \>$\mb{pop} : \ichoice{$\=$\mb{yesmsg} : \m{msg} \product \down \tpaypot{}{n: K} \m{comm[msg]},$\\
%  \>\>$\mb{nomsg} : \down \m{comm[msg]} }}$
%\end{tabbing}
%
%One illustration of the use of shared session types is a \emph{communicator}.
%We use communicators as message buffers between two arbitrary processes: a
%\emph{sender} and a \emph{receiver}.
%The communicator is connected to both the sender and the receiver using a shared
%channel.
%
%Intuitively, the communicator receives \emph{push} requests from the sender followed
%by receiving a message and stores them internally.
%Analogously, the communicator receives \emph{pop} requests from the receiver,
%and responds appropriately with the message if one is stored inside the communicator.
%Formally, a communicator has the following polymorphic session type
%\begin{tabbing}
%  $\mi{stype} \; \m{comm[K][msg]\{n\}} =$\\
%  \quad $\up \tgetpot{}{n+1: K} \echoice{$\=$\mb{SEND} : \m{msg} \arrow
%  \down \m{comm[msg]},$\\
%  \>$\mb{RECV} : \ichoice{$\=$\mb{yes} : \m{msg} \product \down \tpaypot{}{n: K} \m{comm[msg]},$\\
%  \>\>$\mb{no} : \down \m{comm[msg]} }}$
%\end{tabbing}
%The $\up$ indicates that it is a shared channel that must be \emph{acquired} by a process in order to send something over it.
%
%The sender can $\mb{SEND}$ a message into the communicator, and the receiver can periodically try to $\mb{RECV}$ a message from it.
%If there is a message, it responds with $\mb{yes}$, the message of the parameterized type $\m{msg}$, and the import sent with it.
%Note that the communicator retains one unit of import from every message. 
%It needs at least one because it may be activated a polynomial number of times, and, therefore a constant amount of potential is insufficient. 
%At the end of activation, the channel is released with $\down$, and another process can acquire it.

\subsection{Discussion on Realizing Import}
Our adaptation of the import mechanism to resource-aware session types concretizes some parts of the import mechanism like the run-time budget and adds new mechanisms
to facilitate common UC design pattersn such as simulation and virtualization of machines. 
It's important to validate our realization of the import mechanism by ensuring that it faithfully provides the same guarantees.
Furthermore, in this section we discuss existing pitfalls and drawbacks of prior mechanisms that import was created to over come, and assert that our implementation steers clear of them as well.

An important part of our discussion must focus around our sandboxing technique and ensuring that it does not provide a pathway for a process to run infinitely.
The infinite runs problems is a persistent issue in existing length-of-input based approaches to polynoimal time.
It naturally appears in our sandboxing mechanism by the continual creation of new virtual token types. Following from the intent of sandboxing, the NomosUC rule for a valid token context ensures that all virtual token types  \todo{ finish this by deciding which rule to update: \inline{genPot} or \inline{withdrawTokens}}

%\begin{itemize}
%\item Identify design decisions like concretizing potential, sandboxing and virtualizing with withdrawTokens, the valid token context rule, the type system in general
%\item Identify polytime concerns that need to be discussed in the context of our polytime design
%	\begin{itemize}
%	\item is PPT efficiently recognizable?
%	\item address the infinite runs problem and make sure it isn't allowed here with particular attention paid to withdrawTokens and infinite virtualizations
%	\item the type system guarantees we don't have a case where, given some polynomial, a machine just halts mid execution so we avoid any additional information that an environment can use to distinguish based on execution timing in both word
%	\end{itemize}
%\item end with the virtualization point and tie that into proposition 7 and the universal turing machine that can simulate the UC execution. This goes a long way in assuring PPT notion in NomosUC, even thought we aren't dealing exactly with ITMs here.
%\end{itemize}

%The type $\m{comm}$ is parameterized by the type $\m{msg}$, i.e., the type of
%messages in the buffer, and import type parameter, i.e. the amount of import tokens sent with
%the message. 
%The type initiates with an $\up$ denoting that $\m{comm}$ is a shared session type.
%The type prescribes that the communicator needs to be acquired by the sender (or receiver)
%for further interaction.
%Such an acquire-release discipline is automatically enforced by the shared session type.
%Once acquired, the communicator can either receive $\mb{push}$ (from sender) or
%$\mb{pop}$ requests (from receiver).
%In the former case, the communicator receives a message of type $\m{msg}$ and $n+1:K$ import tokens, and
%then detaches from the client using the dual $\down$ operator.
%In the latter case, the communicator checks if it internally contains a message
%for the receiver.
%If yes, the communicator replies with the $\mb{yesmsg}$ label followed by sending
%the message (the $\product$ constructor) and $n:K$ import tokens.
%Otherwise, the communicator replies with the $\mb{nomsg}$ label.
%In either case, the communicator then detaches from the client matching the $\down$
%operator.
%Internally, the communicator stores these messages in a first-in-first-out order.

%It is important to note that our communicators need at least 1 token of import 
%to use themselves to handle a potentially polynomial number of activations. 
%Therefore, it requires $n+1$ units of import from the sender and sends the intended
%$n$ tokens to the receiver when requested.

%The communicator is also the perfect opportunity to implement an unreliable
%message buffer that can drop or reorder messages.
%All we would need to do is change the internal implementation of the communicator
%\emph{without} changing the offered session type.


\section{Type Safety of NomosUC} \label{sec:safety}
\todo{IMPORTANT: We plan to remove Figure 2 and rely on text explanations of configurations to then present progress, preservation, local/global PPT}
The deep connection between our type system and the operational semantics are
formalized by the standard \emph{type preservation} and \emph{progress} theorems.
The preservation theorem also guarantees that the execution time is polynomial in the import tokens.
The full set of base typing rules for Nomos, which we borrow and augment with import, along with the full set of typing rules for import in processes, is in Appendices~\ref{app:basic} and Append

To express these theorems, we introduce the semantic objects
$\proc{c}{w, P}$ and $\msg{c}{w, M}$ describing process $P$ (or message $M$) providing service along channel $c$ and with work counter $w$ storing the work done. %of a NomosUC program.
A multiset of such semantic objects communicating with each other
is known as a \emph{configuration}, denoted as $\config$.
\begin{mathpar}
  \config ::= \msg{c}{w, M} \mid \proc{c}{w, P} \mid \config \; \config
\end{mathpar}
A configuration is typed w.r.t. a signature $\Sg$ containing type and process definitions.
$\Sg$ is \emph{well formed} if
(a) every type definition $V = A_V \in \Sg$ is \emph{contractive}, i.e.,
$A_V$ is not itself a type name,
and (b) every process definition
$\Psi \semi \D \vdash f \{t : \K\} :: (x : A) = P$ in $\Sg$
is well typed according to the process typing judgment, i.e.
$\Sg \semi k \semi \K \hookrightarrow (t,t) \semi \Psi \semi \D \entailpot{0}{0} P :: (x : A)$.

%\begin{figure}[t]
%\begin{mathpar}
%  \vspace{-0.7em}
%  \infer[\m{empty}]
%  {\D \overset{0}{\underset{0}{\vDash}} (\cdot) :: \D}
%  {}
%  \and\vspace{-0.6em}
%  \infer[\m{compose}]
%  {\D_0 \overset{T_1+T_2}{\underset{W_1+W_2}{\vDash}} (\config_1 \; \config_2) :: \D_2}
%  {\D_0 \overset{T_1}{\underset{W_1}{\vDash}} \config_1 :: \D_1 \qquad
%  \D_1 \overset{T_2}{\underset{W_2}{\vDash}} \config_2 :: \D_2}
%  \and\vspace{-0.6em}
%  \inferrule*[right=$\m{proc}$]
%  {\Tokens, \K_0 \hookrightarrow (t, t') \semi \cdot \semi \D_1 \entailpot{q}{q'} P :: (c : A)}
%  {\D, \D_1 \overset{t}{\underset{w}{\vDash}} \proc{c}{w, P} :: (\D, (c : A) )}
%  \and\vspace{-0.6em}
%  \inferrule*[right=$\m{msg}$]
%  {\Tokens, \K_0 \hookrightarrow (t, t') \semi \cdot \semi \D_1 \entailpot{q}{q'} M :: (c : A)}
%  {\D, \D_1 \overset{t}{\underset{w}{\vDash}} \msg{c}{w, M} :: (\D, (c : A) )}
%\end{mathpar}
%\vspace{-1.2em}
%\caption{Typing rules for a configuration}
%\vspace{-1.1em}
%\label{fig:config_typing}
%%\Description{Configuration Typing Rules}
%\end{figure}

\rmd{A key question then is how to type these configurations.
Since they consist a set of processes and messages, they
both \emph{use} and \emph{provide} a collection of channels.
Another goal with the type safety theorems is to establish a connection
between the statically determined import tokens of a process,
its total potential, and the dynamically evolving work counters
that account for the total number of execution steps.}
We use the following judgment to type a configuration w.r.t. $\Sg$
(which we omit from the rules unless necessary).
\vspace{-0.2em}
\[
\Sg \semi \D_1 \overset{T}{\underset{W}{\vDash}} \config :: \D_2
\]
\vspace{-0.2em}
It states that the configuration $\config$
uses the channels in the context $\D_1$ and provides the channels in
the context $\D_2$.
In addition, $T$, and $W$ denotes the total sum of the number of real tokens,
and work counter of each semantic object in a configuration respectively.

\rmd{The configuration typing judgment is defined using
the rules presented in Figure~\ref{fig:config_typing}.}
\ins{We relegate the $\m{proc}$ and $\m{msg}$ rules to Appendix~\ref{app:someapp} to save space.
%
The rule $\m{empty}$ defines that an empty configuration
An \m{empty} configuration is well-typed with $(T, W) = (0, 0)$ and uses and
provides the same set of channels.}
The $\m{compose}$ rule combines two configurations by canceling out
the common channels in $\D_1$ and adding the individual tokens, potential, and work.
\rmd{To build up the configuration from processes (resp. messages), we use the $\m{proc}$ (resp. $\m{msg}$) rule.}
\begin{mathpar}
  \infer[\m{compose}]
  {\D_0 \overset{T_1+T_2}{\underset{W_1+W_2}{\vDash}} (\config_1 \; \config_2) :: \D_2}
  {\D_0 \overset{T_1}{\underset{W_1}{\vDash}} \config_1 :: \D_1 \qquad
  \D_1 \overset{T_2}{\underset{W_2}{\vDash}} \config_2 :: \D_2}
\end{mathpar}
\rmd{For a process, the configuration simply checks how many real tokens $t$ it possesses and
uses it as the token count.
Note that the process could be operating in a sandbox in which case it would not possess
any globally real tokens, but its parent would cover for the cost of this sandboxed execution.
The $\m{msg}$ rule is the same as $\m{proc}$ (we do not need separate typing rules for messages).
Also, note that the functional context $\Psi$ is empty for both the $\m{proc}$ and $\m{msg}$
rules since we are typing runtime objects where all functional variables are substituted
by their values.}

\begin{lemma}[Local PPT]\label{lem:local_ppt}
  Consider a semantic object $\proc{c}{w, P}$ originating from a well-typed configuration and
  typed as $\K_0 \hookrightarrow (t, t'), \Tokens \semi \cdot \semi \D \entailpot{q}{q'} P :: (c : A)$
  where $\K_0$ is the real token type for $P$.
  Then there exists a polynomial $\poly$ such that $w \leq \poly(t, k)$.
\end{lemma}

\begin{proof}
  Since $q'$ is always non-negative, we trivially obtain $w \leq q'+w$.
  Also, note that the work counter $w$ only increments while executing $\etick{r}$
  which decrements $q'$ by $r$.
  And since $q$ is never decremented during executions, we get $q'+w \leq q$.
  Then, from the $\m{pot}$ rule, we obtain $q \leq \GlobalF(t_m', k) \leq \GlobalF(t_m, k)$.
  And, we keep following the token hierarchy to obtain
  $w \leq q \leq \GlobalF^{m}(t, k)$.
  And, thus $\poly = \GlobalF^{m}$ where $m$ is the simulation depth of the process.
  The same lemma would hold for messages as well.
\end{proof}

Next, we turn the local polytime invariant into a global polytime invariant exploiting
the super-additivity of $\GlobalF$.

\begin{theorem}[Global PPT] \label{thm:global_ppt}
  If a well-typed configuration is written as $\D \overset{T}{\underset{W}{\vDash}} \config :: \D'$,
  then $W \leq \poly(T, k)$.
\end{theorem}

\begin{proof}
  We use Lemma~\ref{lem:local_ppt} to obtain that for every semantic object $\proc{c}{w, P}$ (or $\msg{c}{w, M}$),
  $w \leq \poly(t, k)$ where $t$ is the real token quantity for that object.
  When these objects are composed, we get $W = \Sg w \leq \Sg \poly(t, k) \leq
  \poly(\Sg t, k) = \poly(T, k)$.
  Note that due to super-additivity of $\GlobalF$, we get that $\poly = \GlobalF^m$ is super-additive
  and therefore $\Sg \poly(t, k) \leq \poly(\Sg t, k)$.
\end{proof}

Finally, we establish the standard type safety theorems.

\begin{theorem}[Type Preservation]
\label{thm:preservation}
Suppose we have a well-typed configuration typed as
$\D \overset{T_1}{\underset{W_1}{\vDash}} \config_1 :: \D'$.
If $\config_1 \step \config_2$, then there exist $T_2$ and $W_2$ such
that $\D \overset{T_2}{\underset{W_2}{\vDash}} \config_2 :: \D'$.
\end{theorem}
\begin{proofsketch}
  By case analysis on the transition rule, applying inversion to the
  given typing derivation, and then assembling a new derivation of
  $\config_2$.
\end{proofsketch}

A process or message is said to be \emph{poised} if it is trying to
receive along the channel that it provides.  A poised process is
comparable to a computation expecting a value (e.g. a lambda expression).
Similarly, a poised message is trying to send along its provided channel and is equivalent to a value.
A configuration is poised if every process or message in the configuration is poised.
Conceptually, this implies that the configuration is trying to communicate
externally, i.e. along one of the channel it provides.
The progress theorem then shows that either a configuration can take a
step or it is poised.

\begin{theorem}[Global Progress]
\label{thm:progress}
\mbox{}
If $\cdot \overset{T}{\underset{W}{\vDash}} \config :: \D$ then either
\begin{enumerate}
\item[(i)] $\config \mapsto \config'$ for some $\config'$, or
\item[(ii)] $\config$ is poised.
\end{enumerate}
\end{theorem}
\begin{proof}
The proof proceeds by induction on the right-to-left typing of $\config$ so that either
$\config$ is empty (and therefore poised) or
$\config = (\dc\; \proc{c}{w, P})$ or
$\config = (\dc\; \msg{c}{w, M})$. By induction hypothesis, $\dc$ can
either take a step (and then so can $\config$), or $\dc$ is poised.  In
the latter case, we analyze the cases for $P$ and $M$, applying multiple steps of
inversion to show that in each
case either $\config$ can take a step or is poised.
\end{proof}

%\todo{Not sure where to include this discussion on crytographic reductions/hardness}
%\paragraph{Hardness Assumptions in Cryptography}
%One of the main tools for reasoning about security in cryptography is reductions. 
%Proving security usually relies on reducing an adversary that breaks the security of the protocol to one that breaks the security of a primitive that is assumed to be secure. 
%In many cases, the security is reduced to a problem with no known polynomial time solution such as discrete log or computational Diffe-Hellman (DDH).
%Although an assumption, the NomosUC model 
%
%For Computationsl Diffe-Hellman (CDH), for example, defining such a reduction in NomosUC means first implementing a process that attempts to tries to compute $g^{ab}$ from $g^a$ and $g^b$ for $g \in \mathsf{G}$ where $\mathsf{G}$ is cylic group, and then implementing the poynomial-time reduction itself.
%The NomosUC type system \todo{finish}


%\subsection{UC Communicators} \label{sec:communicators}
%% all the processes conncted together leads to a cycle of linear channels ==> Z <--> P so here we use a communicator 
%The UC execution connects the protocol to the environment in both directions of communication.
%This poses a technical challenge where, if linear channels are used, the resulting topology contains a cycle of linear channels: the environmtne offers a channel to the wrapper and the wrapper to the environment.
%Such cycles violate type preservation because a client is acquiring its client~\ref{dasnomos}.
%Therefore, we use a message buffers called communicators which offered shared channels that both ends of the communication can use.
%Communicators are used in the main UC execution in Section~\ref{sec:execuc} to connect the main processes together, as well as within the \partywrapper. 
%
%A communicator has a \emph{sender} and a \emph{receiver}. 
%The shared channel offered by the communicator has the following polymorphic session type:
%\begin{tabbing}
%  $\mi{stype} \; \m{comm[K][msg]\{n\}} =$\\
%  \quad $\up \tgetpot{}{n+1: K} \echoice{$\=$\mb{push} : \m{msg} \arrow
%  \down \m{comm[msg]},$\\
%  \>$\mb{pop} : \ichoice{$\=$\mb{yesmsg} : \m{msg} \product \down \tpaypot{}{n: K} \m{comm[msg]},$\\
%  \>\>$\mb{nomsg} : \down \m{comm[msg]} }}$
%\end{tabbing}
%
%One illustration of the use of shared session types is a \emph{communicator}.
%We use communicators as message buffers between two arbitrary processes: a
%\emph{sender} and a \emph{receiver}.
%The communicator is connected to both the sender and the receiver using a shared
%channel.
%
%Intuitively, the communicator receives \emph{push} requests from the sender followed
%by receiving a message and stores them internally.
%Analogously, the communicator receives \emph{pop} requests from the receiver,
%and responds appropriately with the message if one is stored inside the communicator.
%Formally, a communicator has the following polymorphic session type
%\begin{tabbing}
%  $\mi{stype} \; \m{comm[K][msg]\{n\}} =$\\
%  \quad $\up \tgetpot{}{n+1: K} \echoice{$\=$\mb{SEND} : \m{msg} \arrow
%  \down \m{comm[msg]},$\\
%  \>$\mb{RECV} : \ichoice{$\=$\mb{yes} : \m{msg} \product \down \tpaypot{}{n: K} \m{comm[msg]},$\\
%  \>\>$\mb{no} : \down \m{comm[msg]} }}$
%\end{tabbing}
%The $\up$ indicates that it is a shared channel that must be \emph{acquired} by a process in order to send something over it.
%
%The sender can $\mb{SEND}$ a message into the communicator, and the receiver can periodically try to $\mb{RECV}$ a message from it.
%If there is a message, it responds with $\mb{yes}$, the message of the parameterized type $\m{msg}$, and the import sent with it.
%Note that the communicator retains one unit of import from every message. 
%It needs at least one because it may be activated a polynomial number of times, and, therefore a constant amount of potential is insufficient. 
%At the end of activation, the channel is released with $\down$, and another process can acquire it.

%%\subsection{Discussion on Realizing Import}
%%In this section we present a generic way of using communicators and shared channel to realize arbitrary communication between two parties, avoid cycles, and, still, meaningfully use session types.
%%A consequence of the approach is that communicators carry only functionally typed messages and, therefore, shell code needs to convert between them and session-typed messages.
%%Now that we have introduced the import and potential mechanisms in NomosUC, we introduce a final consequence of our design.
%%
%%The communicator type, and the functional messages, restrict all messages in one direction between two parties to send a constant amount of import.
%%This means that if a protocol requires sending different import with different messages, NomosUC realizes it be sending the maximal import with every message.
%%As the intent of import is not to impose very right bounds on resource usage, we argue that this constraint only results in users defining types that give more import than absolutely necessary.


%\begin{itemize}
%\item Identify design decisions like concretizing potential, sandboxing and virtualizing with withdrawTokens, the valid token context rule, the type system in general
%\item Identify polytime concerns that need to be discussed in the context of our polytime design
%	\begin{itemize}
%	\item is PPT efficiently recognizable?
%	\item address the infinite runs problem and make sure it isn't allowed here with particular attention paid to withdrawTokens and infinite virtualizations
%	\item the type system guarantees we don't have a case where, given some polynomial, a machine just halts mid execution so we avoid any additional information that an environment can use to distinguish based on execution timing in both word
%	\end{itemize}
%\item end with the virtualization point and tie that into proposition 7 and the universal turing machine that can simulate the UC execution. This goes a long way in assuring PPT notion in NomosUC, even thought we aren't dealing exactly with ITMs here.
%\end{itemize}

%The type $\m{comm}$ is parameterized by the type $\m{msg}$, i.e., the type of
%messages in the buffer, and import type parameter, i.e. the amount of import tokens sent with
%the message. 
%The type initiates with an $\up$ denoting that $\m{comm}$ is a shared session type.
%The type prescribes that the communicator needs to be acquired by the sender (or receiver)
%for further interaction.
%Such an acquire-release discipline is automatically enforced by the shared session type.
%Once acquired, the communicator can either receive $\mb{push}$ (from sender) or
%$\mb{pop}$ requests (from receiver).
%In the former case, the communicator receives a message of type $\m{msg}$ and $n+1:K$ import tokens, and
%then detaches from the client using the dual $\down$ operator.
%In the latter case, the communicator checks if it internally contains a message
%for the receiver.
%If yes, the communicator replies with the $\mb{yesmsg}$ label followed by sending
%the message (the $\product$ constructor) and $n:K$ import tokens.
%Otherwise, the communicator replies with the $\mb{nomsg}$ label.
%In either case, the communicator then detaches from the client matching the $\down$
%operator.
%Internally, the communicator stores these messages in a first-in-first-out order.

%It is important to note that our communicators need at least 1 token of import 
%to use themselves to handle a potentially polynomial number of activations. 
%Therefore, it requires $n+1$ units of import from the sender and sends the intended
%$n$ tokens to the receiver when requested.

%The communicator is also the perfect opportunity to implement an unreliable
%message buffer that can drop or reorder messages.
%All we would need to do is change the internal implementation of the communicator
%\emph{without} changing the offered session type.


\section{The UC Experiment} \label{sec:execuc}
In this section we introduce the UC experiment in Nomos and the resulting emulation definition.
We continue on to state the dummy lemma theorem as well as a composition theorem for Nomos UC.
An important part of our definition is using code-generation techniques~\cite{somecodegeneration} to constructs some processes in the UC experiment as well ass useful operators to achieve full composition in the sense of Canetti et al.~\cite{uc}.


We first introduce some convenient notation.
For the remainder of this section, when we refer to a protocol, we actually refer to a pair of ITMs as in Definition~\ref{def:protocol}.
\begin{definition}\label{def:protocol}
A \textit{protocol} is a pair of terms ($\pi$, $\mathcal{F}$) where $\pi$ is the protocol run by honest parties and \F is an ideal functionality that parties an access.
\end{definition}
In the ideal world, $\pi$ is replaced by an an ideal protocol, \idealP, which is a dummy protocol: it frorwards message between \Environment to \F for honest parties  and between \Adversary and \F for corrupt parties.
In the real world, \F is called the \textit{hybrid functionality} and stands in for a real protocol that emulates it.
For protocols that don't make calls to any hybrid functionality, \F is just the dummy protocol which does nothing on activation by any other ITM.

\subsection{The UC Experiment}
The UC experiment is an execution of a main protocol, called the \textit{challenge protocol}, consisting of protocol parties and an ideal functionality, reacting to input by an adversary \Adversary or the \Environment.
The experiment is created by an \msf{execUC} function which spawns the environment, a special construction called the \textit{protocol wrapper}, the adversary, and any functionalities (wrapped by the \textit{functionality wrapper}: a simplified version of the \textit{protocolwrapper}).

The design of \msf{execUC} is constrained by our use of virtual tokens.
In our definition all token types must be statically initialized in order to be used.
Therefore, \msf{execUC} and, in fact, the protocol wrapper rely on code generation to create unique process definitions for a specific protocol we wish to express.
For example a protocl that simulates another protocol, which in turn might simulate other protocols, needs more virtual token types than a simple functionality like \Fcom.
In Figure~\ref{fig:execuc}, we illustrate what \msf{execUC} looks like for the commitment protocol example we've used throughout the paper.

%% EXEC UC FIGURE 
\begin{figure*}
In this section we introduce the UC experiment in Nomos and the resulting emulation definition.
We continue on to state the dummy lemma theorem as well as a composition theorem for Nomos UC.
An important part of our definition is using code-generation techniques~\cite{somecodegeneration} to constructs some processes in the UC experiment as well ass useful operators to achieve full composition in the sense of Canetti et al.~\cite{uc}.


We first introduce some convenient notation.
For the remainder of this section, when we refer to a protocol, we actually refer to a pair of ITMs as in Definition~\ref{def:protocol}.
\begin{definition}\label{def:protocol}
A \textit{protocol} is a pair of terms ($\pi$, $\mathcal{F}$) where $\pi$ is the protocol run by honest parties and \F is an ideal functionality that parties an access.
\end{definition}
In the ideal world, $\pi$ is replaced by an an ideal protocol, \idealP, which is a dummy protocol: it frorwards message between \Environment to \F for honest parties  and between \Adversary and \F for corrupt parties.
In the real world, \F is called the \textit{hybrid functionality} and stands in for a real protocol that emulates it.
For protocols that don't make calls to any hybrid functionality, \F is just the dummy protocol which does nothing on activation by any other ITM.

\subsection{The UC Experiment}
The UC experiment is an execution of a main protocol, called the \textit{challenge protocol}, consisting of protocol parties and an ideal functionality, reacting to input by an adversary \Adversary or the \Environment.
The experiment is created by an \msf{execUC} function which spawns the environment, a special construction called the \textit{protocol wrapper}, the adversary, and any functionalities (wrapped by the \textit{functionality wrapper}: a simplified version of the \textit{protocolwrapper}).

The design of \msf{execUC} is constrained by our use of virtual tokens.
In our definition all token types must be statically initialized in order to be used.
Therefore, \msf{execUC} and, in fact, the protocol wrapper rely on code generation to create unique process definitions for a specific protocol we wish to express.
For example a protocl that simulates another protocol, which in turn might simulate other protocols, needs more virtual token types than a simple functionality like \Fcom.
In Figure~\ref{fig:execuc}, we illustrate what \msf{execUC} looks like for the commitment protocol example we've used throughout the paper.

%% EXEC UC FIGURE 
\begin{figure*}
In this section we introduce the UC experiment in Nomos and the resulting emulation definition.
We continue on to state the dummy lemma theorem as well as a composition theorem for Nomos UC.
An important part of our definition is using code-generation techniques~\cite{somecodegeneration} to constructs some processes in the UC experiment as well ass useful operators to achieve full composition in the sense of Canetti et al.~\cite{uc}.


We first introduce some convenient notation.
For the remainder of this section, when we refer to a protocol, we actually refer to a pair of ITMs as in Definition~\ref{def:protocol}.
\begin{definition}\label{def:protocol}
A \textit{protocol} is a pair of terms ($\pi$, $\mathcal{F}$) where $\pi$ is the protocol run by honest parties and \F is an ideal functionality that parties an access.
\end{definition}
In the ideal world, $\pi$ is replaced by an an ideal protocol, \idealP, which is a dummy protocol: it frorwards message between \Environment to \F for honest parties  and between \Adversary and \F for corrupt parties.
In the real world, \F is called the \textit{hybrid functionality} and stands in for a real protocol that emulates it.
For protocols that don't make calls to any hybrid functionality, \F is just the dummy protocol which does nothing on activation by any other ITM.

\subsection{The UC Experiment}
The UC experiment is an execution of a main protocol, called the \textit{challenge protocol}, consisting of protocol parties and an ideal functionality, reacting to input by an adversary \Adversary or the \Environment.
The experiment is created by an \msf{execUC} function which spawns the environment, a special construction called the \textit{protocol wrapper}, the adversary, and any functionalities (wrapped by the \textit{functionality wrapper}: a simplified version of the \textit{protocolwrapper}).

The design of \msf{execUC} is constrained by our use of virtual tokens.
In our definition all token types must be statically initialized in order to be used.
Therefore, \msf{execUC} and, in fact, the protocol wrapper rely on code generation to create unique process definitions for a specific protocol we wish to express.
For example a protocl that simulates another protocol, which in turn might simulate other protocols, needs more virtual token types than a simple functionality like \Fcom.
In Figure~\ref{fig:execuc}, we illustrate what \msf{execUC} looks like for the commitment protocol example we've used throughout the paper.

%% EXEC UC FIGURE 
\begin{figure*}
\input{listings/execuc}
\caption{The \msf{execUC} function used for the two-party commitment example used througout this paper. Recall, the \msf{execUC} is customized insofar as it takes in some number of virtual token types (here, $K_1$) to enable machines that simulate other machines. In the commitment example, there is no such simulation happening at the protocol or functionality level, therefore only the real token type $K_1$ is used here. The funtion spawns all the necessary ITMs in the UC execution: the environment, the protocol wrapper, the functionalty (wrapped), and the adversary. Each is parameterized with the security parameter $k$ and a random bit sequence $\msf{rng} \in \{0,1\}^{poly(k)}$.
At the end, the environment is started and it returns a bit $b$ which is its guess for which world it is in. The full code can be found in the Appendix.}
\label{lst:execuc}
\end{figure*}

An obvious omission from the \msf{execUC} process definition is the protocol, functionality, adversary, and environments as function parameters.
The reason for the omission is that passing process definitions as parameter is not supported yet in the Nomos impementation.
Therefore, we rely on importing modules which define the relevant processes and define them in scope for \msf{execUC}.

A module representing the protocol, for example,  must define a process called \msf{PS.prot} and \msf{PS.func} for the protocol and the functionality, respectively.
The protocol wrapper and functionality wrapper manage spawning the instance(s) of the functionality and protocol parties.
Similarly, an environment \msf{PS.env} and adversary \msf{PS.adv} must be defined as well.
The message types exchanged between the processes are provided directly to \msf{execUC} as type parameters of the form \msf{p2f}, \msf{f2p}, and so on. 

The environment is spawned first and selects the session id, or \msf{sid}, for the execution and determines the corrupted parties, \msf{clist}.
The rest of the ITMs are then spawned with this \msf{sid} and are given the list of corrupt parties.
Recall that in the UC framework, corrupt parties accept input and give output to the adversary instead of the environment, and the protocol wrapper runs dummy parties in their place that forward messages between \Adversary and \F.

Finally, the environment executes its own code when activated by \msf{\$z.start} and returns a bit that indicates its guess as to which world it is operatin in: real or ideal.
Over all possible environments, security parameters $k$, and random bit sequences $r$, the output of \msf{execUC} represents an ensemble of distributions for the output bit. 

\subsection{The Protocol Wrapper}
The \msf{execUC} definition introduces a new construct called the \textit{protocol wrapper}. 
In the UC experiment, the environment can create protocol parties on the fly and none exist until the first message is written to them.
Thefeore the wrapper is intended to create new parties on demand.

The necessity of a protocol wrapper leads to an interesting problem in how channels and session typed can be used.
All communciation between protocol parties and \Environment, \F, and \Adversary is managed by the protocol wrapper, and the need to multiplex and de-multiplex communication between parties and other machines makes session types between them impossible.
It isn't possible to use multiple session-typed channels through a single communicator either, because parties can have different roles within a protocol so not even the same code is being executed (hence different session types governing the protocol) for each party.
Therefore, we define a new approach to creating protocol-specific party wrappers, but with a generic construction that can be used for any protocol.
We use the two-party commitment protocol as an example to demonstrate how the construction works and show that it is generic enough to allow code generation of a wrapper for any protocol.

Recall the session type governing the committer and receiver in the commitment protocol:
\begin{gather}
	\mi{stype} \; \m{sender} = \ichoice{\mb{commit} : \m{bit} \product \m{scommitted}} \\
	\mi{stype} \; \m{scommitted} = \ichoice{\mb{open} : 1} \\
	\mi{stype} \; \m{receiver} = \echoice{\mb{commit} : \m{rcommitted}} \\
	\mi{stype} \; \m{rcommitted} = \echoice{\mb{open} : \m{bit} \arrow \one}
\end{gather}

The commitment protocol has multiple roles, the committer and receiver, with different session types. 
The wrapper instantiates its internal channels to the parties with the appropriate session types and maintains lists that hold the channels for each possible session types.
For commitment the lists for the \msf{z2p} channels are the following:
\begin{gather}
	\m{R1L1}[\m{sender}] \\
	\m{R1L2}[\m{scommitted}] \\
	\m{R2L1}[\m{receiver}] \\
	\m{R2L2}[\m{rcommitted}] 
\end{gather}
At the start of the commitment UC execution, the \msf{z2p} channels for the committer and receiver will be in $\msf{R1L1}$ and $\msf{R2L1}$, respectively.
Recall this is an auto-generated wrapper, hence the lists are named generically: $\msf{R1L1}$ stands for list 1 or role 1 (role 1: committer. role 2: receiver). 

The channels between the protocol wrapper and the rest of the machines are still limited to functional types through a communicator. 
For example, as shown in the \msf{execUC} definition, the protocol wrapper's channel from \Environment is typed as \inline{comm[K1][Z2Pmsg[z2p]]} where \inline{z2p} is a type parameter to \msf{execUC} which is
\begin{gather}
\mi{type} \; \m{comz2p} = \m{Commit} \; \mi{of} \; \m{bit} \; | \; \m{Open}
\end{gather}
for commitment.

When the protocol wrapper receives a message for some \msf{pid}, if the party doesn't exist the protocol wrapper creates all the party's channel parameterized by the correct session types (the party's role and session types are determined by functional type of the incoming message).
The channels are stored in the appropriate lists corresponding to their type.
The session type of the message is determined by the functional type, and the session-typed message is sent along that channel.
This we can still make use of session types and environments, adversaries, or protocols that send messages out of order (i.e. are incorrect) will still \textit{fail to type check}.
After delivering the message, the channel is moved to the next list corresponding to its new type.

For outgoing messages, the party wrapper does a similar conversion where it reads the sesion typed message output by the party and converts it to the appropriate functional message type.
For the commitment example, at the beginning the type of the \inline{p2f} channel the committer has is typed as (1). 
When the party \inline{PID} sends a bit with
\begin{lstlisting}[basicstyle=\small\BeraMonottFamily, frame=single, mathescape]
$\$$f2p.commit 
$\tb{send}$ $\$$f2p b
$\tb{pay}$ {p2fn} $\$$f2p 
\end{lstlisting}
the party wrapper does
\begin{lstlisting}[basicstyle=\small\BeraMonottFamily, frame=single, mathescape]
$\$$z2p.SEND 
$\tb{send}$ $\$$z2p pid 
$\tb{send}$ $\$$z2p Commit(b)
$\tb{pay}$ {p2fn} $\$$f2p
\end{lstlisting}
sending a functional message type to the functionality.
For outgoing messages, a new process per party channel waits to read and does the inverse conversion: from session type to funtional type and attaches the party's \msf{pid} to it.

\paragraph{Functionality Wrapper}
Similar to protocol parties, we want to write functionalities using session types.
However, as described for the protocol wrapper, a static definition does not suffice to capture the UC features: dynamic number of parties and different roles (hence, different session types) per party.
For the same reasons, we also create a functionality wrapper around the ideal functionaltiy (in both the real and ideal worlds).
The key difference between the functionality wrapper and the protocol wrapper is that there is only one instance of the functionality running.
We also want to retain the the design of \Fcom presented earlier, where a functionality can be written to interact specifically with each party on separate channels. 
Therefore, the functionality wrapper also generates lists for each session type for each party role (the roles are committer and receiver in \Fcom), and de-multiplexes incoming messages.
Like the protocol wrapper code, the functionality wrapper searches for the channel of the sending \inline{PID} and attempts to send the session typed messages accross it.
The full code of the functionality wrapper for the commitment ideal world is given, in full, in the appendix. 

\subsection{Polynomial Bound}
The UC import mechanism provides a way to define polynomial time computation and resource-bounds by ensuring that a single ITM's execution is upper-bounded by some value $T(n)$ where $T$ is a polynomial and $n$ the total units of import the ITM ever receives.
In UC, it is important to reason about polynomial bounds in the security parameters $k$. Hence, the UC execution relies on an initial amount of import that is give as a polynonial in $k$. 
In NomosUC, we take advantage of the import built into the type system to ensure ITMs are PPT in the security parameter. 

\begin{definition}[PPT Term]\label{def:pptterm}
A \textit{PPT term} is a \textit{well-typed} term $e(k, r)$ that is \textit{closed} except for security parameter $k$, random bit sequence $r$.
\end{definition}

We first-define terms that are well-typed in the traditional session-types-sense in Definition~\ref{def:pptterm}, i.e. without any resource constraints~\cite{sessiontypes}.
Such terms are closed except for the security parameter $k$ and some uniformly random bit sequence $r$.

However, we also want to reason about terms that are well-typed when connected to another Nomos terms.
We introduce the term \textit{well-matched} to mean a PPT term $e$ is well-typed when connected to another term $e'$.
Simply put, the channels that $e$ and $e'$ share channels of the same type. 
Specifically, we want to exclude processes that logically share a channel, say the channel from \inline{p} to \inline{f}, but belong to different protocols (their types message types don't match).
This new definition becomes important when we discuss UC emulation below as we want to reason about environments that are \textit{well-matched} for a protocol $\pi$ or a specific adversary \Adversary.

\begin{definition}[Well-Matched]\label{def:wellmatched}
\begin{mathpar}
\footnotesize
\inferrule*[right=Well-matched]
{\Tokens_1, K \semi \Delta_1 \vdash C_1 :: \Delta_1' \semi 
\Tokens_2, K' \semi \Delta_2 \vdash C_2 :: \Delta_2' \\ \\
 S \equiv \Delta_1 \bigcap \Delta_2 \neq \emptyset}
{\Delta_1 \equiv_{S} \Delta_2 \semi K \equiv K'} 
\end{mathpar}
\end{definition}

Notice that in Definition~\ref{def:wellmatched} we are concerned with two terms that are \textit{open} even when connected. 
We only care about being well-matched, when connected to another term, on the channels over which they are connected.

Next we introduce our definition of a polynomial-bound in the security parameter $k$.
Terms that are PPT in $k$ are dubbed \textit{well-resource-typed}.
\begin{theorem}[PPT in $k$]\label{thm:ppt}
A \textit{PPT Term} $e(k, r)$ is well-resource-typed if, given initial import $n(k) \in poly(k)$, there exists a polynomial $T$ s.t. $\forall k, r, e(k, r) \{n(k)\}$ terminates in at most $T(n)$ steps. 
\end{theorem}

\begin{proof}
The Nomos type system only type checks programs for which a satisfying assignment of polynomial $T$ is possible.
Given an $n \in poly(k)$, all programs that type check must be \textit{well-resource-typed.}
%The Nomos type system guarantees that a satisfying assignment of $n$ and $T$ will correctly type-check.
%Therefore, given an initial amount of import $n(k) \in poly(k)$, the existence of some $T$ ensures that any process, regardless of its randomized execution according to the bit sequence $r$, $e$ is guarantees to be upper-bounded by $poly(k)$ satisfying the definition of probabilistic polynomial time in $k$.
\end{proof}

\subsection{Emulation}
A proof of security in the UC framework relies upon emulation of different executions.

In general, we say that a protocol $\pi$ posesses the same security properties as another protocol $\phi$ if no environment given them inputs can distinguish between them for any adversary.
In most cases we compare a real protocol $\pi$ with an idealized protocol $(\idealP, \F)$ which is actually just an ideal functionality with dummy parties.
The ideal functionality is known to achieve the desired security processes because it acts like a simple, trusted third party.
They are much simpler than protocols because they don't require any special code to handle mutually distrustful other processes, and they perform the given computation on behald of the ideal world parties.

Given the random choices ITMs in UC can make, it is clear that the outputs of \inline{execUC} in Figure~\ref{lst:execuc} produces and ensemble of distributions over all possible random bitstrings and security parameters.
Emulation, then, is about the ensembles created by two UC environments being computationally indistinguishable from each other.
We define indistinguishabiliy between ensembles in a standard way using \textit{statistical distance} in Definition~\ref{def:distance}.

\begin{definition}[Indisinguishability]\label{def:distance}
Two ensembles $\mathcal{D}_{1,k}, \mathcal{D}_{2,k}$ are indistinguishable, $\mathcal{D}_{1,k} \sim \mathcal{D}_{2,k}$, if their statistical distance is at most $negl(k), \forall k$.
\end{definition}

Before we introduce the emulation definition, we first define valid protocols, valid functionalities, and what it means for protocols, functionalities, adversaries, and environments to be well-matched with each other.
We shorten the communicator type \msf{comm} to \msf{c} in the following definitions.

\todo{Ankush: The context of a valid functionality must contain channels typed with the type parameters given by \msf{execUC}. An the machine, parameterized with security parameter $k$ and random bit sequence $r$ are bounded by some polynomial $T_\F$. $\leftarrow$ the last part is meant to capture the well-resource-typed (from the well-matched definition), but maybe we can just say $\F$ is well-resource-typed given $k$,$r$}
\begin{definition}[Valid Functionality]\label{def:validfunc}
\begin{mathpar}to one of footnotesize
\inferrule*[right=valid-F]
{\exists c_1:c[\msf{p2f}], c_2:c[\msf{f2p}], c_3: c[\msf{f2a}], c_4:c[\msf{a2f}] \in \Delta_1 \\
\Delta_1 \models (\F(k, r) : T_\F) :: \Delta_1'}
{\msf{validF}\ \F \rightarrow \Delta_1'}
\end{mathpar}
\end{definition}

\todo{The intent is the same as above here execpt for protocol having channels with the right types. Again here I could just say $\pi$ is well-resource-typed instead of the $\pi(k,r)$ that is there now.}
\begin{definition}[Valid Protocol]\label{def:validprot}
\begin{mathpar}
\footnotesize
\inferrule*[right=valid-P]
{\exists c_1: \msf{p2f}, c_2: \msf{f2p}, c_3: \msf{p2a}, c_4: \msf{a2p}, c_5: \msf{z2p}, c_6: \msf{p2z} \in \Delta_1 \\
\Delta_1 \models (\pi(k, r) : T_\pi) :: \Delta_1' }
{\msf{validP}\ \pi \rightarrow \Delta_1'}
\end{mathpar}
\end{definition}

\todo{Ankush: this defines what it means for a protocol and functionality to be well-matched. Namely, they shared channels typed according to parameters given by execUc (p2f, f2p, ...) and have the same type and import parameters on their communicators}
\begin{definition}[Well-Matched]
\begin{mathpar}
\footnotesize
\inferrule*[right=p2f match] 
{\msf{validP}\ \pi \rightarrow \D_1 \semi \msf{validF}\ \F \rightarrow \Delta_2 \\
\Delta_1:, (\msf{c}[K][\msf{f2p}]), (\msf{c}[K][\msf{p2f}]) \equiv \\
\Delta_2, (\msf{c}[K][\msf{pid \textasciicircum f2p}]), (\msf{c}[K][\msf{pid \textasciicircum p2f}])}
{\langle \pi \leftrightarrow \F \rangle}
\end{mathpar}
\end{definition}

\todo{Ankush: same for this one and the rest, as above}
\begin{definition}
\begin{mathpar}
\footnotesize
\inferrule*[right=p2a match] 
{\msf{validP}\ \pi \rightarrow \Delta_1 \semi \Adversary \rightarrow \Delta_2 \\
\Delta_1:, (\msf{c}[K][\msf{a2p}]), (\msf{c}[K][\msf{p2a}]) \equiv \\ 
\Delta_2, (\msf{c}[K][\msf{pid \textasciicircum a2p}]), (\msf{c}[K][\msf{pid \textasciicircum p2a}])}
{\langle \pi \leftrightarrow \Adversary \rangle}
\end{mathpar}
\end{definition}

\begin{definition}
\begin{mathpar}
\footnotesize
\inferrule*[right=f2a match] 
{\msf{validF}\ \F \rightarrow \Delta_1 \semi \Adversary \rightarrow \Delta_2 \\
\Delta_1:, (\msf{c}[K][\msf{a2f}]\{a2fn\}), ( \msf{c}[[K]\msf{f2a}]\{0\}) \equiv \\
 \Delta_2, (\msf{c}[K][\msf{a2f}]\{a2fn\}), ( \msf{c}[K][\msf{f2a}]\{0\})}
{\langle \F \leftrightarrow \Adversary \rangle}
\end{mathpar}
\end{definition}

\begin{definition}
\begin{mathpar}
\footnotesize
\inferrule*[right=p2z match] 
{\msf{validP}\ \pi \rightarrow \Delta_1 \semi \Environment \rightarrow \Delta_2}
{\Delta_1:, (\msf{c}[K][\msf{z2p}]), (\msf{c}[K][\msf{p2z}]) \equiv \\
 \Delta_2, (\msf{c}[K][\msf{pid \textasciicircum z2p]}), (\msf{c}[K][\msf{pid \textasciicircum p2z}])}
\end{mathpar}
\end{definition}

Indisintiguishability between two protocols is defined as follows (we shorten the communicator type \msf{comm} to \msf{c}):

\begin{definition}[Emulation]\label{def:emulation}
Given two protocols $(\pi, \F_1), (\phi, \F_2)$ that are well-resource-typed then if $\forall \Adversary$ well-matched with $(\pi, \F_1)$, $\exists \Simulator$ s.t. $\forall \Environment$ well-matched with \Adversary and $(\pi, \F_1)$: \Simulator is well-matched with $(\phi, \F_2)$, \Environment is well-matched with $(\phi, \Simulator)$, and $\msf{execUC}(\pi, \F_1, \Environment, \Adversary) \approx \msf{execUC}(\phi, \F_2, \Environment, \Simulator)$:

\begin{mathpar}
\footnotesize
	\inferrule*[right=emulate]
	{
		. \models \msf{execUC}[\Tokentypes][\alpha] :: \Delta[\Tokentypes][\alpha] \\ \\
		% Protocols that are well-matched with their functionalities
		\msf{validP}\ \pi \rightarrow \Delta_1' \semi
		\msf{validP} \phi \rightarrow \Delta_2' \semi
		\langle \pi \leftrightarrow \F_2 \rangle, \langle \phi \leftrightarrow \F_1 \rangle \\
		% Type of execUC[DELTA_pi] and execUC[DELTA_phi]
		\Delta_1'[\Tokentypes][\mathrm{T}_{\pi}] \equiv_{\Environment} \Delta_1\ 
		\semi \Delta_2'[\Tokentypes][\mathrm{T}_{\phi}] \equiv_\Environment \Delta_2 \\
		% For all A if exists well-typed A that is well-matched with real world
		\forall \Adversary, (\exists (\Delta_4, \Delta_4') | \Delta_4 \vdash \Adversary :: \Delta_4',\ \langle \Adversary \leftrightarrow \pi \rangle, \langle \Adversary \leftrightarrow \F_1 \rangle \\
		% implies simulator that is well-matched for ideal world
		\Rightarrow \exists (\Delta_3,\Delta_3') | \Delta_3 \vdash \Simulator_\Adversary :: \Delta_3', \langle \Simulator_\Adversary \leftrightarrow \phi \rangle, \langle \Simulator_\Adversary \leftrightarrow \F_2 \rangle \\
		% for all Z they that's well-matched for the real world => Z is well-matched with S and ideal world
		\forall \Environment (\langle \Environment \leftrightarrow \Adversary \rangle, \langle \Environment \leftrightarrow \pi \rangle \Rightarrow \langle \Environment \leftrightarrow \Simulator_\Adversary \rangle, \langle \Environment \leftrightarrow \phi \rangle \\
		% and emulation has to hold
		\msf{execUC} \ \pi\ \Environment\ \F_1\ \Adversary \approx\ \msf{execUC} \ \phi\ \Environment\ \F_2\ \Simulator_\Adversary))
	}
	{
		% EMULATION DEFINITION
		\lambda \Adversary . \Simulator_\Adversary \vdash (\pi, \F_1) \sim (\phi, \F_2)
	}
\end{mathpar}
\end{definition}
The definition ensures that for emulation to hold, the constructed simulator must be well-matched everywhere \Adversary is well-matched: for all environments \Adversary is well-matched with the \Simulator must also be well-matched with.

\paragraph{UC Realize}
When we talk about emulation, we particularly care about emulation with respect to an ideal protocol $\phi$ which is really just $(\idealP, \F)$ where \idealP is the protocol which forwards all messages to/from \Environment and \F.
We say the protocol $\pi$ (potentially with a hybrid functionality $\F_1$) UC-realizes an ideal functionality $\F_2$ if Definition~\ref{def:emulation} holds for $(\pi, \F_1)$ and  $\phi = (\idealP, \F_2)$

\begin{definition}[UC-Realize]
A protocol $\pi$ UC-realized an ideal functionality $\F_1$ if $(\pi, \F_2) \sim (\idealP, \F_1)$ for some $\F_2$.
\end{definition}

\subsection{Dummy Lemma}
The Dummy Lemma is an important lemma in the UC framework that requires only one simulator to work with a dummy adversary in order to prove emulation with respect to any adversary.
The proof of the lemma makes use of the \msf{withdrawTokens} program definition from Section~\ref{sec:nomosuc}.
The instruction allows for re-use of existing machines and make simulator construction to use the real-world adversary, or other sub-simulators, in a black-box manner.

The Lemma states that if dummy simulator satisfies emulation with respect to the dummy adversary, then for any \Adversary a simulator can be constructed with the dummy simulator. 
The constructed simulator simply runs \Adversary and \Dummysim internally, and it sends messages from \Environment to \Adversary and outputs of \Adversary to \Dummysim.
At a high leve, the proof relies on the emulation definition where dummy emulation covers environments that run \Adversary internally. Here, we are only moving \Adversary into the execution

\begin{theorem}[Dummy Lemma]\label{thm:dummy}
If $\exists \Dummysim$ s.t. $ \DummyAdv, \Dummysim \vdash (\pi, \F_2) \sim (\phi, \F_1)$ then $\forall \Adversary \ \exists \Simulator_\Adversary$ s.t. $\Simulator_{\Adversary} \vdash  (\pi, \F_2) \sim (\phi, \F_1)$ 
\end{theorem}

\begin{proof}
The constructed simulator $\Simulator_\Adversary$ internally simulates \Dummysim and \Adversary through a virtual token type $K'$. 
We describe the simulation pattern below to simulate messages to \Dummysim and \Adversary.
Recall that the virtual tokens consturction is a tool to make witing complex protocols easier, and has no impact on the import token requirements of the simulating machine.
Simply put, simulating as a block-box should be equivalent, with respect to import, as \Simulator running the code natively. 
The only different in running a simulation internally is additional potential usage in using \inline{$\tm{withdrawToken}$} and routing messages.

On input from \Environment on channel \msf{z2p}, \Simulator:
\begin{lstlisting}[basicstyle=\small\BeraMonottFamily, frame=single,  mathescape, label={lst:sim}]
msg = $\nrecv$ $\$$z2a ;
$\nget$ $\$$z2a {z2an : K} ;
$\tm{withdrawTokens}$ f K K1 z2an ;
$\nsend$ $\$$a_z2a msg ;
$\npay$ {z2an : K1} $\$$a_z2a ; 
\end{lstlisting}

Similarly, on output from \Adversary to a protocol party on channel \msf{a2p}
\begin{lstlisting}[basicstyle=\small\BeraMonottFamily, frame=single,  mathescape]
pid = $\tb{recv}$ $\$$a_a2p ;
msg = $\tb{recv}$ $\$$a_a2p ;
$\tb{get}$ K1 $\$$aa2p {a2pn} ;
$\tb{send}$ $\$$sd_z2a A2P(pid, msg) ;
$\npay$ $\$$sd_z2a {z2an : K1} ;
\end{lstlisting}

$\Simulator_\Adversary$ forwards input from \Environment and forwards it to the internal \Adversary. 
\Adversary output to either the protocol parties or the ideal functionality. 
\Simulator forwards this output to \Dummysim acting as input from the environment (here we fallback to the notion that \Adversary can be run internally by \Environment) and forward any outputs it creates to the intended machines.
The proof oblication here is to ensure that the constructed simulator $\Simulator_\Adversary$ is well-resource-typed for all well-resource-typed and well-matched, with \Environment, \Adversary.
The $\Simulator_\Adversary$ performs constant overhead on the simulattion of \Adversary and \Dummysim. Therefore, a sufficient bounding polynomial on the runtime of $\Simulator_\Adversary$ can be given as:
\[
T(n) = T_{\Adversary,\Dummysim}(n) + T_{\Adversary,\Dummysim}(n) + O(n)
\]
where $T_{\Adversary,\Dummysim}(n)$ is the greater of the two bounding polynomials for \Dummysim and \Adversary evaluated at $n$, and $n$ is the import that \Environment sends to \Adversary. 
The same \textit{well-resource typed} reasoning extends to the token context where amount of virtual tokens created are polyomial in number and generate potential that is bounded by the above bounding polynomial for $\Simulator_\Adversary$.
\end{proof}

\subsection{Single Composition}
In this section we present a simplified composition theorem and another theorem, which we call the \textit{squash theorem}.
These two theorems combine to prove the full generalized composition theorem as it appears in the UC framework~\cite{uc}.

The composition operator defines a way for some protocol $\rho$ that uses a functionality $\F$ to swap $\F$ for a procol $(\pi, \F')$, which realizes $\F$, such that $(\rho, \F) \sim (\phi, \F'') \Rightarrow (\rho \circ \pi, \F') \sim (\phi, \F'')$.
The $\circ$ composition operator is defined in Nomos in Figure~\ref{lst:compose}.

Recall that the Nomos language currently does not support passing processes as arguments to other processes even though the theory allows it. 
In the $\circ$ code the protocols $\pi$ and $\phi$ exist globally.

\begin{figure*}
\begin{lstlisting}[basicstyle=\small\BeraMonottFamily, frame=single,  mathescape]
$\tb{proc}$ compose[K][z2r][r2z][f2r][r2f][p2f][f2p] : 
    (pid: Int), ($\$$z_to_p: c[K][z2p]), ($\$$p_to_z: c[K][r2z]), 
    ($\$$f_to_p: c[K][f2r]), ($\$$p_to_f: c[K][r2f])  |- ($\$$D : 1) =
{
	$\$$rho_to_pi <- $\tm{createchan}$[K][p2f];
	$\$$pi_to_rho <- $\tm{createchan}$[K][f2p];

	 <- pi  <-                 $\$$rho_to_pi $\$$pi_to_rho $\$$p_to_f $\$$f_to_p ;
	 <- phi <- $\$$z_to_p $\$$p_to_z $\$$rho_to_pi $\$$pi_to_rho ; 
}
\end{lstlisting}
\caption{Composition operator in Nomos that connects a protocol $\rho$ to a protocol $\pi$ that uses some functionality $\F$. The operators creates new channels to connect the realizing $\pi$ and it's hybrid \F. Output from $\rho$ intended for the replace functionality are actually send to parties of $\rho$, and channels outgoing from the parties to the functionality are given to $\pi$.}
\label{lst:compose} 
\end{figure*}

\todo{Include a graphical illustration of wtf is going on, and going on inside the party wrapper as}

\begin{theorem}[Composition]\label{thm:singlecomp}
\begin{mathpar}
\inferrule*[right=single-compose]
{
	\F_1 \xrightarrow{\pi} \F_2 \semi \F_2 \xrightarrow{\rho} \F_3 \\
}
{
	\F_1 \xrightarrow{\rho \circ \pi} \F_3
}
\end{mathpar}

If \textit{well-typed} $(\pi, \F_1$) realizes $\F_2$ and ($\rho$, $\F_2$) realizes some $\F_3$, then $(\rho \circ \pi, \F_2)$ is \textit{well-typed} and realizes $\F_3$ when $\circ$ is defined as in Figure~\ref{lst:compose}.
\end{theorem}

\begin{proof}
The pre-condition ensures the existence of a \textit{well-resource-typed} simulator $\Simulator_\pi$ for $(\pi, \F_1) \sim (\idealP, \F_2)$. 
We construct a simulator $S$ which relies only on $\Simulator_\pi$ for:
\[
	\msf{execUC}\ (\rho \circ \pi)\ \F_1\ \Environment\ \Adversary \approx \msf{execUC}\ \idealP\ \F_3\ \Environment\ \Simulator
\]	
We don't need to perform simulation on any inputs by \Environment to the main parties of $\rho$ (it's the same protocol in both worlds).
The constructed simulator \Simulator simulates \Sim{\pi} internally and passes messages intended for the parties of $\pi$, or for $\F_2$, to \Sim{\pi} and simulates its computation.
Similariy, \Simulator sends any message from $\F_3$ to \Sim{\pi} for simulation.  
Input to any party of the main protocol $\rho$ from \Environment, or outout from them to \Simulator, are forwarded without any modification or simulation.
The constructed simulator performs constant overhead in routing messages to the simulated \Sim{\pi} and forwrading messages to/from parties of $\rho$/\Environment. 
Given that \Sim{\pi} is \textit{well-resource-typed}, with bounding polynomial $T_{\Sim{\pi}}$, it suffices to show that an additional linear term is sufficient to create a bounding polynomial for \Simulator.

\end{proof}

We give a simpler, high-level idea of the proof here which can be understood visually:
\begin{align}
& \msf{execUC} \: \Environment \, (\rho \circ \pi) \, \F_1 \, \DummyAdv \\
\equiv \; & \msf{execUC} \: (\Environment \circ \rho) \, \pi \, \F_1 \, \DummyAdv \\
\approx \; & \msf{execUC} \: (\Environment \circ \rho) \, \idealP \, \F_2 \, \Sim{\pi} \\
\equiv \; & \msf{execUC} \: \Environment \, \rho \, \F_2 \, \Sim{\pi} 
%\approx \; & \msf{execUC} \: (\Environment \circ \Sim{\pi}) \, \idealP \, \F_3 \, \Sim{\rho} \\
%\equiv \; & \msf{execUC} \: \Environment \, \idealP \, \F_3 \, (\Sim{\pi} \circ \Sim{\rho}) 
\end{align}
The $\equiv$ operator is a result of moving around ITMs (some from within other ITMs into the main UC execution) and $\sim$ refers to indistinguishability.
In line (13) above, $\rho$ is moved into the execution environment with an unchanged simulator as no additional simulation is required: the simulator allows unfettered communication between parties of $\rho$ and \Environment.

\subsection{Multisession}
The multi-session extension of a protocol or functionality, specified by the $!$ operator (such as $!\rho$ or $!\F$), allows multiple instances to be run within a sinlge ITM.
The ITM simulates multiple instances of the protocol/functionality intnerally and multiplexes input/output to/from them in same way as the party wrapper for protocol parties.
The channel from the protocol wrapper to the multisession operator can be typed as:
\begin{gather}
\mi{stype} \; \m{{P2MS}[a]\{n\}} = \echoice{\mb{push}: pid \textasciicircum ssid \textasciicircum a \arrow |\{n\}> \m{P2MS[a]\{n\}}}
\end{gather}
The operator accepts messages of the form $(\msf{ssid}, msg)$ from a particular \msf{pid}, where \msf{ssid} is a sub-session identifier.
If an instance of the functionality with $\msf{sid} := \msf{ssid}$ then $!\F$ creates one and forwards the message to it.
Additionally, $!\F$ listens for outgoing messages from each of the instances and forwards them to the outside execution.
The operator differs from the party wrapper in one crucial way: it only works with functional messages types and does not wrap around any session types like any other standalone functionality in Nomos UC.

The multisession behaves like the protocol wrapper in that we rely on code generation to create the operator for a particular functionality. 
The reason behind this is that the operator simulates many instances of a functionality and must use virtual tokens to communicate with them. 
For the commitment example we've used throughout this paper, the multisession needs only one virtual token type alongside the real token type.
The commitment functionality doesn't internally simulate any other machines and therefore does not need any virtual token type itself. 
The process definition for $!\F_\msf{com}$ is shown in Figure \ref{lst:bangf} accepting two token types: the real token type $K$ and the virtual token type $K_1$ for instances of $\F_\msf{com}$.

The communicators between \bangf and the other ITMs all use the real token type.
Only the internal channels that it creates use virtual token types.
The communication pattern between the operator and the simulated functionalities works in the same was as Listing \ref{lst:sim}.

\begin{figure*}
\begin{lstlisting}[basicstyle=\small\BeraMonottFamily, frame=single, mathescape]
type sid[a] = SID of String ^ a ;

proc bangF_1[K, K1][$p2f$][$f2p$][$a2f$][$f2a$]{$p2fn$}{$f2pn$}{$a2fn$} : 
    ($\$$pw_to_f: P2MS[K][p2f]), ($\$$f_to_pw: MS2P[K][f2p]), ($\$$f_to_a: MS2A[K][f2a]), ($\$$a_to_f: A2MS[K][a2f]),
	($\$\l1: list[sender] ), ($\$$l2: list[scommitted]), ($\$$l3: list[receiver]), ($\$$l4: list[rcommitted]) |- ($\$$ms: 1)
\end{lstlisting}
\caption{The type definition for the multisession operator for functionalities and the correspond message type and import parameters.}
\label{lst:bangf}
\end{figure*}

\begin{theorem}[PPT !]\label{thm:bangppt}
If a functionality $\F$ is well-resource-typed, then it's multisession extension $!\F$ is well-resource-typed.
\end{theorem}

\begin{proof}
A \textit{well-resource-typed} \F guarantees a polynomial $T_{\F}$ bounding its execution.
In the worse-case, the multisession operator must spawn a new instance of $\F$ an every activation. 
Let $N_{\F}$ denote the total number of instances (and, hence, number of activations) of $\F$ created by the operator.
Note that $N_{\F}$ is polynomial in the security parameter $k$ for all well-typed environments, protocols, and adversary.
Therefore, there always exists a bounding polynomial to bound a polynomial number of simulated instances of \F.
The polynomial can be given as:
$$ P_{!\F}(n) = N_{\F} P_{\F}(n) + \mathcal{O}(N_{\F}) $$
where the $\mathcal{O}(N_{\F})$ is due to the overhead of maintaining and accessing the set of all instances.

Similarly, \F being \textit{well-resource-typed} ensures a valid token context for all processes it may simulate. 
Therefore, it is clear that there exists a global connecting poltnomial $f$ that ensures a valid token context for $!\F$.
\end{proof}

\begin{theorem}[Squash Theorem]\label{thm:squash}
%If a functionality \F is well-resource-typed, then $!\F$ and $!!\F$ are well-resource-typed (by Theorem~\ref{thm:bangppt}) and $(\idealP, !!\F) \sim (\msf{squash}, !\F)$.
%\textit{Well-resource-typed} \F $\Rightarrow$ $!\F \xrightarrow{\msf{squash}} !!\F$%  $(\idealP, !!\F) \sim (\msf{squash}, !\F)$
\begin{mathpar}
\inferrule*[right=squash]
{
\textit{well-resource-typed} \; \F
}
{
!\F \xrightarrow{\msf{squash}} !!\F
}
\end{mathpar}
\end{theorem}

\begin{proof}
First we describe the \msf{squash} protocol in figure \ref{fig:squash}.
Note that $!!\F$ is nested $!$ operators. The top level process maintains multiple sessions of $!\F$ each with their own \msf{ssid}.
Functionalities in each $!\F[\msf{ssid}]$ have their own \msf{sid}. 

In $(\idealP, !!\F)$, \idealP~expects to receive messages of the form $(\msf{ssid}_1, (\msf{ssid}_2, m))$ where $\msf{ssid_2}$ is a sub-session of $\F$ (i.e. instance) inside some $!\F$ with sub-session id $\msf{ssid}_1$ inside of $!!\F$ (the message accesses functionality $!!\F[\msf{ssid}_1][\msf{ssid}_2]$).
The \msf{squash} protocol flattens the indexing of instances of \F and combines session ids $\msf{ssid}_1$ and $\msf{ssid}_2$ into a single \msf{ssid}: $\msf{ssid}_3 := \msf{ssid}_1 \cdot \msf{ssid}_2$.
If follows intuitively that the view for the environment remains the same. 

We construct a simulator such that:
\[
\msf{execUC} \, \Environment \, \idealP \, !!\F \, \Sim{\msf{squash}} \approx \msf{execUC} \, \Environment \, \msf{squash} \, !\F \DummyAdv 
\]
The simulator is very simple. 
Inputs to/from parties/\Environment for a corrupt party is forwarded unmodified.
Input intended for $!\F$ of the form $(\msf{ssid}_1 \cdot \msf{ssid}_2, msg)$ sends $(\msf{ssid}_1, (\msf{ssid}_2, msg))$ to $!!\F$. 
Output from $!!\F$ is modified inversely and sent to \Environment.

The simulator is clearly \textit{well-typed} 

\end{proof}

\subsection{UC Composition}
Composition in the UC setting is not limited to replacement of a single instance of a protocol.
Instead, it permits replacement of any number of instances of a protocol $\phi$, each with their own session id, with instances of a realizing protocol $\pi$.
This generalized form of composition follows directly from Theorems \ref{thm:singlecomp} and \ref{thm:squash}.

\begin{theorem}[Composition]\label{thm:composition}
\begin{mathpar}
\inferrule*[right=compose]
{
	%(\pi, !\F_1) \sim (\idealP, F_2) \semi (\rho, !\F_2) \sim (\idealP, \F_3) \\ 
	!\F_1 \xrightarrow{\pi} \F_2 \semi !\F_2 \xrightarrow{\rho} \F_3 \\
	%\Rightarrow \exists \Simulator(\Adversary) \vdash (\rho^{!\F_2 \rightarrow (!\pi \, \circ \, \msf{squash})}, !\F_1) \sim (\idealP, \F_3)
}
{
	!\F_1 \xrightarrow{\rho \, \circ !\pi \circ \, \msf{squash}} \F_3
	%(\rho \, \circ \, !\pi \circ \msf{squash}, !\F_1) \sim (\idealP, \F_3)
}
\end{mathpar}
\end{theorem}

\begin{proof}
The proof of full composition follows directly from the single composition Theorem~\ref{thm:singlecomp} and the Squash Theorem~\ref{thm:squash}.
By Theorem~\ref{thm:singlecomp} we can infer $!!\F_1 \xrightarrow{\rho \, \circ \, !\pi} \F_3$.
Theorem~\ref{thm:squash} allows us to ``squash'' $!!\F_1$ and construct a simulator for $!\F_1 \xrightarrow{\rho \, \circ \, !\pi \, \circ \, \msf{squash}} \F_3$
\end{proof}

\caption{The \msf{execUC} function used for the two-party commitment example used througout this paper. Recall, the \msf{execUC} is customized insofar as it takes in some number of virtual token types (here, $K_1$) to enable machines that simulate other machines. In the commitment example, there is no such simulation happening at the protocol or functionality level, therefore only the real token type $K_1$ is used here. The funtion spawns all the necessary ITMs in the UC execution: the environment, the protocol wrapper, the functionalty (wrapped), and the adversary. Each is parameterized with the security parameter $k$ and a random bit sequence $\msf{rng} \in \{0,1\}^{poly(k)}$.
At the end, the environment is started and it returns a bit $b$ which is its guess for which world it is in. The full code can be found in the Appendix.}
\label{lst:execuc}
\end{figure*}

An obvious omission from the \msf{execUC} process definition is the protocol, functionality, adversary, and environments as function parameters.
The reason for the omission is that passing process definitions as parameter is not supported yet in the Nomos impementation.
Therefore, we rely on importing modules which define the relevant processes and define them in scope for \msf{execUC}.

A module representing the protocol, for example,  must define a process called \msf{PS.prot} and \msf{PS.func} for the protocol and the functionality, respectively.
The protocol wrapper and functionality wrapper manage spawning the instance(s) of the functionality and protocol parties.
Similarly, an environment \msf{PS.env} and adversary \msf{PS.adv} must be defined as well.
The message types exchanged between the processes are provided directly to \msf{execUC} as type parameters of the form \msf{p2f}, \msf{f2p}, and so on. 

The environment is spawned first and selects the session id, or \msf{sid}, for the execution and determines the corrupted parties, \msf{clist}.
The rest of the ITMs are then spawned with this \msf{sid} and are given the list of corrupt parties.
Recall that in the UC framework, corrupt parties accept input and give output to the adversary instead of the environment, and the protocol wrapper runs dummy parties in their place that forward messages between \Adversary and \F.

Finally, the environment executes its own code when activated by \msf{\$z.start} and returns a bit that indicates its guess as to which world it is operatin in: real or ideal.
Over all possible environments, security parameters $k$, and random bit sequences $r$, the output of \msf{execUC} represents an ensemble of distributions for the output bit. 

\subsection{The Protocol Wrapper}
The \msf{execUC} definition introduces a new construct called the \textit{protocol wrapper}. 
In the UC experiment, the environment can create protocol parties on the fly and none exist until the first message is written to them.
Thefeore the wrapper is intended to create new parties on demand.

The necessity of a protocol wrapper leads to an interesting problem in how channels and session typed can be used.
All communciation between protocol parties and \Environment, \F, and \Adversary is managed by the protocol wrapper, and the need to multiplex and de-multiplex communication between parties and other machines makes session types between them impossible.
It isn't possible to use multiple session-typed channels through a single communicator either, because parties can have different roles within a protocol so not even the same code is being executed (hence different session types governing the protocol) for each party.
Therefore, we define a new approach to creating protocol-specific party wrappers, but with a generic construction that can be used for any protocol.
We use the two-party commitment protocol as an example to demonstrate how the construction works and show that it is generic enough to allow code generation of a wrapper for any protocol.

Recall the session type governing the committer and receiver in the commitment protocol:
\begin{gather}
	\mi{stype} \; \m{sender} = \ichoice{\mb{commit} : \m{bit} \product \m{scommitted}} \\
	\mi{stype} \; \m{scommitted} = \ichoice{\mb{open} : 1} \\
	\mi{stype} \; \m{receiver} = \echoice{\mb{commit} : \m{rcommitted}} \\
	\mi{stype} \; \m{rcommitted} = \echoice{\mb{open} : \m{bit} \arrow \one}
\end{gather}

The commitment protocol has multiple roles, the committer and receiver, with different session types. 
The wrapper instantiates its internal channels to the parties with the appropriate session types and maintains lists that hold the channels for each possible session types.
For commitment the lists for the \msf{z2p} channels are the following:
\begin{gather}
	\m{R1L1}[\m{sender}] \\
	\m{R1L2}[\m{scommitted}] \\
	\m{R2L1}[\m{receiver}] \\
	\m{R2L2}[\m{rcommitted}] 
\end{gather}
At the start of the commitment UC execution, the \msf{z2p} channels for the committer and receiver will be in $\msf{R1L1}$ and $\msf{R2L1}$, respectively.
Recall this is an auto-generated wrapper, hence the lists are named generically: $\msf{R1L1}$ stands for list 1 or role 1 (role 1: committer. role 2: receiver). 

The channels between the protocol wrapper and the rest of the machines are still limited to functional types through a communicator. 
For example, as shown in the \msf{execUC} definition, the protocol wrapper's channel from \Environment is typed as \inline{comm[K1][Z2Pmsg[z2p]]} where \inline{z2p} is a type parameter to \msf{execUC} which is
\begin{gather}
\mi{type} \; \m{comz2p} = \m{Commit} \; \mi{of} \; \m{bit} \; | \; \m{Open}
\end{gather}
for commitment.

When the protocol wrapper receives a message for some \msf{pid}, if the party doesn't exist the protocol wrapper creates all the party's channel parameterized by the correct session types (the party's role and session types are determined by functional type of the incoming message).
The channels are stored in the appropriate lists corresponding to their type.
The session type of the message is determined by the functional type, and the session-typed message is sent along that channel.
This we can still make use of session types and environments, adversaries, or protocols that send messages out of order (i.e. are incorrect) will still \textit{fail to type check}.
After delivering the message, the channel is moved to the next list corresponding to its new type.

For outgoing messages, the party wrapper does a similar conversion where it reads the sesion typed message output by the party and converts it to the appropriate functional message type.
For the commitment example, at the beginning the type of the \inline{p2f} channel the committer has is typed as (1). 
When the party \inline{PID} sends a bit with
\begin{lstlisting}[basicstyle=\small\BeraMonottFamily, frame=single, mathescape]
$\$$f2p.commit 
$\tb{send}$ $\$$f2p b
$\tb{pay}$ {p2fn} $\$$f2p 
\end{lstlisting}
the party wrapper does
\begin{lstlisting}[basicstyle=\small\BeraMonottFamily, frame=single, mathescape]
$\$$z2p.SEND 
$\tb{send}$ $\$$z2p pid 
$\tb{send}$ $\$$z2p Commit(b)
$\tb{pay}$ {p2fn} $\$$f2p
\end{lstlisting}
sending a functional message type to the functionality.
For outgoing messages, a new process per party channel waits to read and does the inverse conversion: from session type to funtional type and attaches the party's \msf{pid} to it.

\paragraph{Functionality Wrapper}
Similar to protocol parties, we want to write functionalities using session types.
However, as described for the protocol wrapper, a static definition does not suffice to capture the UC features: dynamic number of parties and different roles (hence, different session types) per party.
For the same reasons, we also create a functionality wrapper around the ideal functionaltiy (in both the real and ideal worlds).
The key difference between the functionality wrapper and the protocol wrapper is that there is only one instance of the functionality running.
We also want to retain the the design of \Fcom presented earlier, where a functionality can be written to interact specifically with each party on separate channels. 
Therefore, the functionality wrapper also generates lists for each session type for each party role (the roles are committer and receiver in \Fcom), and de-multiplexes incoming messages.
Like the protocol wrapper code, the functionality wrapper searches for the channel of the sending \inline{PID} and attempts to send the session typed messages accross it.
The full code of the functionality wrapper for the commitment ideal world is given, in full, in the appendix. 

\subsection{Polynomial Bound}
The UC import mechanism provides a way to define polynomial time computation and resource-bounds by ensuring that a single ITM's execution is upper-bounded by some value $T(n)$ where $T$ is a polynomial and $n$ the total units of import the ITM ever receives.
In UC, it is important to reason about polynomial bounds in the security parameters $k$. Hence, the UC execution relies on an initial amount of import that is give as a polynonial in $k$. 
In NomosUC, we take advantage of the import built into the type system to ensure ITMs are PPT in the security parameter. 

\begin{definition}[PPT Term]\label{def:pptterm}
A \textit{PPT term} is a \textit{well-typed} term $e(k, r)$ that is \textit{closed} except for security parameter $k$, random bit sequence $r$.
\end{definition}

We first-define terms that are well-typed in the traditional session-types-sense in Definition~\ref{def:pptterm}, i.e. without any resource constraints~\cite{sessiontypes}.
Such terms are closed except for the security parameter $k$ and some uniformly random bit sequence $r$.

However, we also want to reason about terms that are well-typed when connected to another Nomos terms.
We introduce the term \textit{well-matched} to mean a PPT term $e$ is well-typed when connected to another term $e'$.
Simply put, the channels that $e$ and $e'$ share channels of the same type. 
Specifically, we want to exclude processes that logically share a channel, say the channel from \inline{p} to \inline{f}, but belong to different protocols (their types message types don't match).
This new definition becomes important when we discuss UC emulation below as we want to reason about environments that are \textit{well-matched} for a protocol $\pi$ or a specific adversary \Adversary.

\begin{definition}[Well-Matched]\label{def:wellmatched}
\begin{mathpar}
\footnotesize
\inferrule*[right=Well-matched]
{\Tokens_1, K \semi \Delta_1 \vdash C_1 :: \Delta_1' \semi 
\Tokens_2, K' \semi \Delta_2 \vdash C_2 :: \Delta_2' \\ \\
 S \equiv \Delta_1 \bigcap \Delta_2 \neq \emptyset}
{\Delta_1 \equiv_{S} \Delta_2 \semi K \equiv K'} 
\end{mathpar}
\end{definition}

Notice that in Definition~\ref{def:wellmatched} we are concerned with two terms that are \textit{open} even when connected. 
We only care about being well-matched, when connected to another term, on the channels over which they are connected.

Next we introduce our definition of a polynomial-bound in the security parameter $k$.
Terms that are PPT in $k$ are dubbed \textit{well-resource-typed}.
\begin{theorem}[PPT in $k$]\label{thm:ppt}
A \textit{PPT Term} $e(k, r)$ is well-resource-typed if, given initial import $n(k) \in poly(k)$, there exists a polynomial $T$ s.t. $\forall k, r, e(k, r) \{n(k)\}$ terminates in at most $T(n)$ steps. 
\end{theorem}

\begin{proof}
The Nomos type system only type checks programs for which a satisfying assignment of polynomial $T$ is possible.
Given an $n \in poly(k)$, all programs that type check must be \textit{well-resource-typed.}
%The Nomos type system guarantees that a satisfying assignment of $n$ and $T$ will correctly type-check.
%Therefore, given an initial amount of import $n(k) \in poly(k)$, the existence of some $T$ ensures that any process, regardless of its randomized execution according to the bit sequence $r$, $e$ is guarantees to be upper-bounded by $poly(k)$ satisfying the definition of probabilistic polynomial time in $k$.
\end{proof}

\subsection{Emulation}
A proof of security in the UC framework relies upon emulation of different executions.

In general, we say that a protocol $\pi$ posesses the same security properties as another protocol $\phi$ if no environment given them inputs can distinguish between them for any adversary.
In most cases we compare a real protocol $\pi$ with an idealized protocol $(\idealP, \F)$ which is actually just an ideal functionality with dummy parties.
The ideal functionality is known to achieve the desired security processes because it acts like a simple, trusted third party.
They are much simpler than protocols because they don't require any special code to handle mutually distrustful other processes, and they perform the given computation on behald of the ideal world parties.

Given the random choices ITMs in UC can make, it is clear that the outputs of \inline{execUC} in Figure~\ref{lst:execuc} produces and ensemble of distributions over all possible random bitstrings and security parameters.
Emulation, then, is about the ensembles created by two UC environments being computationally indistinguishable from each other.
We define indistinguishabiliy between ensembles in a standard way using \textit{statistical distance} in Definition~\ref{def:distance}.

\begin{definition}[Indisinguishability]\label{def:distance}
Two ensembles $\mathcal{D}_{1,k}, \mathcal{D}_{2,k}$ are indistinguishable, $\mathcal{D}_{1,k} \sim \mathcal{D}_{2,k}$, if their statistical distance is at most $negl(k), \forall k$.
\end{definition}

Before we introduce the emulation definition, we first define valid protocols, valid functionalities, and what it means for protocols, functionalities, adversaries, and environments to be well-matched with each other.
We shorten the communicator type \msf{comm} to \msf{c} in the following definitions.

\todo{Ankush: The context of a valid functionality must contain channels typed with the type parameters given by \msf{execUC}. An the machine, parameterized with security parameter $k$ and random bit sequence $r$ are bounded by some polynomial $T_\F$. $\leftarrow$ the last part is meant to capture the well-resource-typed (from the well-matched definition), but maybe we can just say $\F$ is well-resource-typed given $k$,$r$}
\begin{definition}[Valid Functionality]\label{def:validfunc}
\begin{mathpar}to one of footnotesize
\inferrule*[right=valid-F]
{\exists c_1:c[\msf{p2f}], c_2:c[\msf{f2p}], c_3: c[\msf{f2a}], c_4:c[\msf{a2f}] \in \Delta_1 \\
\Delta_1 \models (\F(k, r) : T_\F) :: \Delta_1'}
{\msf{validF}\ \F \rightarrow \Delta_1'}
\end{mathpar}
\end{definition}

\todo{The intent is the same as above here execpt for protocol having channels with the right types. Again here I could just say $\pi$ is well-resource-typed instead of the $\pi(k,r)$ that is there now.}
\begin{definition}[Valid Protocol]\label{def:validprot}
\begin{mathpar}
\footnotesize
\inferrule*[right=valid-P]
{\exists c_1: \msf{p2f}, c_2: \msf{f2p}, c_3: \msf{p2a}, c_4: \msf{a2p}, c_5: \msf{z2p}, c_6: \msf{p2z} \in \Delta_1 \\
\Delta_1 \models (\pi(k, r) : T_\pi) :: \Delta_1' }
{\msf{validP}\ \pi \rightarrow \Delta_1'}
\end{mathpar}
\end{definition}

\todo{Ankush: this defines what it means for a protocol and functionality to be well-matched. Namely, they shared channels typed according to parameters given by execUc (p2f, f2p, ...) and have the same type and import parameters on their communicators}
\begin{definition}[Well-Matched]
\begin{mathpar}
\footnotesize
\inferrule*[right=p2f match] 
{\msf{validP}\ \pi \rightarrow \D_1 \semi \msf{validF}\ \F \rightarrow \Delta_2 \\
\Delta_1:, (\msf{c}[K][\msf{f2p}]), (\msf{c}[K][\msf{p2f}]) \equiv \\
\Delta_2, (\msf{c}[K][\msf{pid \textasciicircum f2p}]), (\msf{c}[K][\msf{pid \textasciicircum p2f}])}
{\langle \pi \leftrightarrow \F \rangle}
\end{mathpar}
\end{definition}

\todo{Ankush: same for this one and the rest, as above}
\begin{definition}
\begin{mathpar}
\footnotesize
\inferrule*[right=p2a match] 
{\msf{validP}\ \pi \rightarrow \Delta_1 \semi \Adversary \rightarrow \Delta_2 \\
\Delta_1:, (\msf{c}[K][\msf{a2p}]), (\msf{c}[K][\msf{p2a}]) \equiv \\ 
\Delta_2, (\msf{c}[K][\msf{pid \textasciicircum a2p}]), (\msf{c}[K][\msf{pid \textasciicircum p2a}])}
{\langle \pi \leftrightarrow \Adversary \rangle}
\end{mathpar}
\end{definition}

\begin{definition}
\begin{mathpar}
\footnotesize
\inferrule*[right=f2a match] 
{\msf{validF}\ \F \rightarrow \Delta_1 \semi \Adversary \rightarrow \Delta_2 \\
\Delta_1:, (\msf{c}[K][\msf{a2f}]\{a2fn\}), ( \msf{c}[[K]\msf{f2a}]\{0\}) \equiv \\
 \Delta_2, (\msf{c}[K][\msf{a2f}]\{a2fn\}), ( \msf{c}[K][\msf{f2a}]\{0\})}
{\langle \F \leftrightarrow \Adversary \rangle}
\end{mathpar}
\end{definition}

\begin{definition}
\begin{mathpar}
\footnotesize
\inferrule*[right=p2z match] 
{\msf{validP}\ \pi \rightarrow \Delta_1 \semi \Environment \rightarrow \Delta_2}
{\Delta_1:, (\msf{c}[K][\msf{z2p}]), (\msf{c}[K][\msf{p2z}]) \equiv \\
 \Delta_2, (\msf{c}[K][\msf{pid \textasciicircum z2p]}), (\msf{c}[K][\msf{pid \textasciicircum p2z}])}
\end{mathpar}
\end{definition}

Indisintiguishability between two protocols is defined as follows (we shorten the communicator type \msf{comm} to \msf{c}):

\begin{definition}[Emulation]\label{def:emulation}
Given two protocols $(\pi, \F_1), (\phi, \F_2)$ that are well-resource-typed then if $\forall \Adversary$ well-matched with $(\pi, \F_1)$, $\exists \Simulator$ s.t. $\forall \Environment$ well-matched with \Adversary and $(\pi, \F_1)$: \Simulator is well-matched with $(\phi, \F_2)$, \Environment is well-matched with $(\phi, \Simulator)$, and $\msf{execUC}(\pi, \F_1, \Environment, \Adversary) \approx \msf{execUC}(\phi, \F_2, \Environment, \Simulator)$:

\begin{mathpar}
\footnotesize
	\inferrule*[right=emulate]
	{
		. \models \msf{execUC}[\Tokentypes][\alpha] :: \Delta[\Tokentypes][\alpha] \\ \\
		% Protocols that are well-matched with their functionalities
		\msf{validP}\ \pi \rightarrow \Delta_1' \semi
		\msf{validP} \phi \rightarrow \Delta_2' \semi
		\langle \pi \leftrightarrow \F_2 \rangle, \langle \phi \leftrightarrow \F_1 \rangle \\
		% Type of execUC[DELTA_pi] and execUC[DELTA_phi]
		\Delta_1'[\Tokentypes][\mathrm{T}_{\pi}] \equiv_{\Environment} \Delta_1\ 
		\semi \Delta_2'[\Tokentypes][\mathrm{T}_{\phi}] \equiv_\Environment \Delta_2 \\
		% For all A if exists well-typed A that is well-matched with real world
		\forall \Adversary, (\exists (\Delta_4, \Delta_4') | \Delta_4 \vdash \Adversary :: \Delta_4',\ \langle \Adversary \leftrightarrow \pi \rangle, \langle \Adversary \leftrightarrow \F_1 \rangle \\
		% implies simulator that is well-matched for ideal world
		\Rightarrow \exists (\Delta_3,\Delta_3') | \Delta_3 \vdash \Simulator_\Adversary :: \Delta_3', \langle \Simulator_\Adversary \leftrightarrow \phi \rangle, \langle \Simulator_\Adversary \leftrightarrow \F_2 \rangle \\
		% for all Z they that's well-matched for the real world => Z is well-matched with S and ideal world
		\forall \Environment (\langle \Environment \leftrightarrow \Adversary \rangle, \langle \Environment \leftrightarrow \pi \rangle \Rightarrow \langle \Environment \leftrightarrow \Simulator_\Adversary \rangle, \langle \Environment \leftrightarrow \phi \rangle \\
		% and emulation has to hold
		\msf{execUC} \ \pi\ \Environment\ \F_1\ \Adversary \approx\ \msf{execUC} \ \phi\ \Environment\ \F_2\ \Simulator_\Adversary))
	}
	{
		% EMULATION DEFINITION
		\lambda \Adversary . \Simulator_\Adversary \vdash (\pi, \F_1) \sim (\phi, \F_2)
	}
\end{mathpar}
\end{definition}
The definition ensures that for emulation to hold, the constructed simulator must be well-matched everywhere \Adversary is well-matched: for all environments \Adversary is well-matched with the \Simulator must also be well-matched with.

\paragraph{UC Realize}
When we talk about emulation, we particularly care about emulation with respect to an ideal protocol $\phi$ which is really just $(\idealP, \F)$ where \idealP is the protocol which forwards all messages to/from \Environment and \F.
We say the protocol $\pi$ (potentially with a hybrid functionality $\F_1$) UC-realizes an ideal functionality $\F_2$ if Definition~\ref{def:emulation} holds for $(\pi, \F_1)$ and  $\phi = (\idealP, \F_2)$

\begin{definition}[UC-Realize]
A protocol $\pi$ UC-realized an ideal functionality $\F_1$ if $(\pi, \F_2) \sim (\idealP, \F_1)$ for some $\F_2$.
\end{definition}

\subsection{Dummy Lemma}
The Dummy Lemma is an important lemma in the UC framework that requires only one simulator to work with a dummy adversary in order to prove emulation with respect to any adversary.
The proof of the lemma makes use of the \msf{withdrawTokens} program definition from Section~\ref{sec:nomosuc}.
The instruction allows for re-use of existing machines and make simulator construction to use the real-world adversary, or other sub-simulators, in a black-box manner.

The Lemma states that if dummy simulator satisfies emulation with respect to the dummy adversary, then for any \Adversary a simulator can be constructed with the dummy simulator. 
The constructed simulator simply runs \Adversary and \Dummysim internally, and it sends messages from \Environment to \Adversary and outputs of \Adversary to \Dummysim.
At a high leve, the proof relies on the emulation definition where dummy emulation covers environments that run \Adversary internally. Here, we are only moving \Adversary into the execution

\begin{theorem}[Dummy Lemma]\label{thm:dummy}
If $\exists \Dummysim$ s.t. $ \DummyAdv, \Dummysim \vdash (\pi, \F_2) \sim (\phi, \F_1)$ then $\forall \Adversary \ \exists \Simulator_\Adversary$ s.t. $\Simulator_{\Adversary} \vdash  (\pi, \F_2) \sim (\phi, \F_1)$ 
\end{theorem}

\begin{proof}
The constructed simulator $\Simulator_\Adversary$ internally simulates \Dummysim and \Adversary through a virtual token type $K'$. 
We describe the simulation pattern below to simulate messages to \Dummysim and \Adversary.
Recall that the virtual tokens consturction is a tool to make witing complex protocols easier, and has no impact on the import token requirements of the simulating machine.
Simply put, simulating as a block-box should be equivalent, with respect to import, as \Simulator running the code natively. 
The only different in running a simulation internally is additional potential usage in using \inline{$\tm{withdrawToken}$} and routing messages.

On input from \Environment on channel \msf{z2p}, \Simulator:
\begin{lstlisting}[basicstyle=\small\BeraMonottFamily, frame=single,  mathescape, label={lst:sim}]
msg = $\nrecv$ $\$$z2a ;
$\nget$ $\$$z2a {z2an : K} ;
$\tm{withdrawTokens}$ f K K1 z2an ;
$\nsend$ $\$$a_z2a msg ;
$\npay$ {z2an : K1} $\$$a_z2a ; 
\end{lstlisting}

Similarly, on output from \Adversary to a protocol party on channel \msf{a2p}
\begin{lstlisting}[basicstyle=\small\BeraMonottFamily, frame=single,  mathescape]
pid = $\tb{recv}$ $\$$a_a2p ;
msg = $\tb{recv}$ $\$$a_a2p ;
$\tb{get}$ K1 $\$$aa2p {a2pn} ;
$\tb{send}$ $\$$sd_z2a A2P(pid, msg) ;
$\npay$ $\$$sd_z2a {z2an : K1} ;
\end{lstlisting}

$\Simulator_\Adversary$ forwards input from \Environment and forwards it to the internal \Adversary. 
\Adversary output to either the protocol parties or the ideal functionality. 
\Simulator forwards this output to \Dummysim acting as input from the environment (here we fallback to the notion that \Adversary can be run internally by \Environment) and forward any outputs it creates to the intended machines.
The proof oblication here is to ensure that the constructed simulator $\Simulator_\Adversary$ is well-resource-typed for all well-resource-typed and well-matched, with \Environment, \Adversary.
The $\Simulator_\Adversary$ performs constant overhead on the simulattion of \Adversary and \Dummysim. Therefore, a sufficient bounding polynomial on the runtime of $\Simulator_\Adversary$ can be given as:
\[
T(n) = T_{\Adversary,\Dummysim}(n) + T_{\Adversary,\Dummysim}(n) + O(n)
\]
where $T_{\Adversary,\Dummysim}(n)$ is the greater of the two bounding polynomials for \Dummysim and \Adversary evaluated at $n$, and $n$ is the import that \Environment sends to \Adversary. 
The same \textit{well-resource typed} reasoning extends to the token context where amount of virtual tokens created are polyomial in number and generate potential that is bounded by the above bounding polynomial for $\Simulator_\Adversary$.
\end{proof}

\subsection{Single Composition}
In this section we present a simplified composition theorem and another theorem, which we call the \textit{squash theorem}.
These two theorems combine to prove the full generalized composition theorem as it appears in the UC framework~\cite{uc}.

The composition operator defines a way for some protocol $\rho$ that uses a functionality $\F$ to swap $\F$ for a procol $(\pi, \F')$, which realizes $\F$, such that $(\rho, \F) \sim (\phi, \F'') \Rightarrow (\rho \circ \pi, \F') \sim (\phi, \F'')$.
The $\circ$ composition operator is defined in Nomos in Figure~\ref{lst:compose}.

Recall that the Nomos language currently does not support passing processes as arguments to other processes even though the theory allows it. 
In the $\circ$ code the protocols $\pi$ and $\phi$ exist globally.

\begin{figure*}
\begin{lstlisting}[basicstyle=\small\BeraMonottFamily, frame=single,  mathescape]
$\tb{proc}$ compose[K][z2r][r2z][f2r][r2f][p2f][f2p] : 
    (pid: Int), ($\$$z_to_p: c[K][z2p]), ($\$$p_to_z: c[K][r2z]), 
    ($\$$f_to_p: c[K][f2r]), ($\$$p_to_f: c[K][r2f])  |- ($\$$D : 1) =
{
	$\$$rho_to_pi <- $\tm{createchan}$[K][p2f];
	$\$$pi_to_rho <- $\tm{createchan}$[K][f2p];

	 <- pi  <-                 $\$$rho_to_pi $\$$pi_to_rho $\$$p_to_f $\$$f_to_p ;
	 <- phi <- $\$$z_to_p $\$$p_to_z $\$$rho_to_pi $\$$pi_to_rho ; 
}
\end{lstlisting}
\caption{Composition operator in Nomos that connects a protocol $\rho$ to a protocol $\pi$ that uses some functionality $\F$. The operators creates new channels to connect the realizing $\pi$ and it's hybrid \F. Output from $\rho$ intended for the replace functionality are actually send to parties of $\rho$, and channels outgoing from the parties to the functionality are given to $\pi$.}
\label{lst:compose} 
\end{figure*}

\todo{Include a graphical illustration of wtf is going on, and going on inside the party wrapper as}

\begin{theorem}[Composition]\label{thm:singlecomp}
\begin{mathpar}
\inferrule*[right=single-compose]
{
	\F_1 \xrightarrow{\pi} \F_2 \semi \F_2 \xrightarrow{\rho} \F_3 \\
}
{
	\F_1 \xrightarrow{\rho \circ \pi} \F_3
}
\end{mathpar}

If \textit{well-typed} $(\pi, \F_1$) realizes $\F_2$ and ($\rho$, $\F_2$) realizes some $\F_3$, then $(\rho \circ \pi, \F_2)$ is \textit{well-typed} and realizes $\F_3$ when $\circ$ is defined as in Figure~\ref{lst:compose}.
\end{theorem}

\begin{proof}
The pre-condition ensures the existence of a \textit{well-resource-typed} simulator $\Simulator_\pi$ for $(\pi, \F_1) \sim (\idealP, \F_2)$. 
We construct a simulator $S$ which relies only on $\Simulator_\pi$ for:
\[
	\msf{execUC}\ (\rho \circ \pi)\ \F_1\ \Environment\ \Adversary \approx \msf{execUC}\ \idealP\ \F_3\ \Environment\ \Simulator
\]	
We don't need to perform simulation on any inputs by \Environment to the main parties of $\rho$ (it's the same protocol in both worlds).
The constructed simulator \Simulator simulates \Sim{\pi} internally and passes messages intended for the parties of $\pi$, or for $\F_2$, to \Sim{\pi} and simulates its computation.
Similariy, \Simulator sends any message from $\F_3$ to \Sim{\pi} for simulation.  
Input to any party of the main protocol $\rho$ from \Environment, or outout from them to \Simulator, are forwarded without any modification or simulation.
The constructed simulator performs constant overhead in routing messages to the simulated \Sim{\pi} and forwrading messages to/from parties of $\rho$/\Environment. 
Given that \Sim{\pi} is \textit{well-resource-typed}, with bounding polynomial $T_{\Sim{\pi}}$, it suffices to show that an additional linear term is sufficient to create a bounding polynomial for \Simulator.

\end{proof}

We give a simpler, high-level idea of the proof here which can be understood visually:
\begin{align}
& \msf{execUC} \: \Environment \, (\rho \circ \pi) \, \F_1 \, \DummyAdv \\
\equiv \; & \msf{execUC} \: (\Environment \circ \rho) \, \pi \, \F_1 \, \DummyAdv \\
\approx \; & \msf{execUC} \: (\Environment \circ \rho) \, \idealP \, \F_2 \, \Sim{\pi} \\
\equiv \; & \msf{execUC} \: \Environment \, \rho \, \F_2 \, \Sim{\pi} 
%\approx \; & \msf{execUC} \: (\Environment \circ \Sim{\pi}) \, \idealP \, \F_3 \, \Sim{\rho} \\
%\equiv \; & \msf{execUC} \: \Environment \, \idealP \, \F_3 \, (\Sim{\pi} \circ \Sim{\rho}) 
\end{align}
The $\equiv$ operator is a result of moving around ITMs (some from within other ITMs into the main UC execution) and $\sim$ refers to indistinguishability.
In line (13) above, $\rho$ is moved into the execution environment with an unchanged simulator as no additional simulation is required: the simulator allows unfettered communication between parties of $\rho$ and \Environment.

\subsection{Multisession}
The multi-session extension of a protocol or functionality, specified by the $!$ operator (such as $!\rho$ or $!\F$), allows multiple instances to be run within a sinlge ITM.
The ITM simulates multiple instances of the protocol/functionality intnerally and multiplexes input/output to/from them in same way as the party wrapper for protocol parties.
The channel from the protocol wrapper to the multisession operator can be typed as:
\begin{gather}
\mi{stype} \; \m{{P2MS}[a]\{n\}} = \echoice{\mb{push}: pid \textasciicircum ssid \textasciicircum a \arrow |\{n\}> \m{P2MS[a]\{n\}}}
\end{gather}
The operator accepts messages of the form $(\msf{ssid}, msg)$ from a particular \msf{pid}, where \msf{ssid} is a sub-session identifier.
If an instance of the functionality with $\msf{sid} := \msf{ssid}$ then $!\F$ creates one and forwards the message to it.
Additionally, $!\F$ listens for outgoing messages from each of the instances and forwards them to the outside execution.
The operator differs from the party wrapper in one crucial way: it only works with functional messages types and does not wrap around any session types like any other standalone functionality in Nomos UC.

The multisession behaves like the protocol wrapper in that we rely on code generation to create the operator for a particular functionality. 
The reason behind this is that the operator simulates many instances of a functionality and must use virtual tokens to communicate with them. 
For the commitment example we've used throughout this paper, the multisession needs only one virtual token type alongside the real token type.
The commitment functionality doesn't internally simulate any other machines and therefore does not need any virtual token type itself. 
The process definition for $!\F_\msf{com}$ is shown in Figure \ref{lst:bangf} accepting two token types: the real token type $K$ and the virtual token type $K_1$ for instances of $\F_\msf{com}$.

The communicators between \bangf and the other ITMs all use the real token type.
Only the internal channels that it creates use virtual token types.
The communication pattern between the operator and the simulated functionalities works in the same was as Listing \ref{lst:sim}.

\begin{figure*}
\begin{lstlisting}[basicstyle=\small\BeraMonottFamily, frame=single, mathescape]
type sid[a] = SID of String ^ a ;

proc bangF_1[K, K1][$p2f$][$f2p$][$a2f$][$f2a$]{$p2fn$}{$f2pn$}{$a2fn$} : 
    ($\$$pw_to_f: P2MS[K][p2f]), ($\$$f_to_pw: MS2P[K][f2p]), ($\$$f_to_a: MS2A[K][f2a]), ($\$$a_to_f: A2MS[K][a2f]),
	($\$\l1: list[sender] ), ($\$$l2: list[scommitted]), ($\$$l3: list[receiver]), ($\$$l4: list[rcommitted]) |- ($\$$ms: 1)
\end{lstlisting}
\caption{The type definition for the multisession operator for functionalities and the correspond message type and import parameters.}
\label{lst:bangf}
\end{figure*}

\begin{theorem}[PPT !]\label{thm:bangppt}
If a functionality $\F$ is well-resource-typed, then it's multisession extension $!\F$ is well-resource-typed.
\end{theorem}

\begin{proof}
A \textit{well-resource-typed} \F guarantees a polynomial $T_{\F}$ bounding its execution.
In the worse-case, the multisession operator must spawn a new instance of $\F$ an every activation. 
Let $N_{\F}$ denote the total number of instances (and, hence, number of activations) of $\F$ created by the operator.
Note that $N_{\F}$ is polynomial in the security parameter $k$ for all well-typed environments, protocols, and adversary.
Therefore, there always exists a bounding polynomial to bound a polynomial number of simulated instances of \F.
The polynomial can be given as:
$$ P_{!\F}(n) = N_{\F} P_{\F}(n) + \mathcal{O}(N_{\F}) $$
where the $\mathcal{O}(N_{\F})$ is due to the overhead of maintaining and accessing the set of all instances.

Similarly, \F being \textit{well-resource-typed} ensures a valid token context for all processes it may simulate. 
Therefore, it is clear that there exists a global connecting poltnomial $f$ that ensures a valid token context for $!\F$.
\end{proof}

\begin{theorem}[Squash Theorem]\label{thm:squash}
%If a functionality \F is well-resource-typed, then $!\F$ and $!!\F$ are well-resource-typed (by Theorem~\ref{thm:bangppt}) and $(\idealP, !!\F) \sim (\msf{squash}, !\F)$.
%\textit{Well-resource-typed} \F $\Rightarrow$ $!\F \xrightarrow{\msf{squash}} !!\F$%  $(\idealP, !!\F) \sim (\msf{squash}, !\F)$
\begin{mathpar}
\inferrule*[right=squash]
{
\textit{well-resource-typed} \; \F
}
{
!\F \xrightarrow{\msf{squash}} !!\F
}
\end{mathpar}
\end{theorem}

\begin{proof}
First we describe the \msf{squash} protocol in figure \ref{fig:squash}.
Note that $!!\F$ is nested $!$ operators. The top level process maintains multiple sessions of $!\F$ each with their own \msf{ssid}.
Functionalities in each $!\F[\msf{ssid}]$ have their own \msf{sid}. 

In $(\idealP, !!\F)$, \idealP~expects to receive messages of the form $(\msf{ssid}_1, (\msf{ssid}_2, m))$ where $\msf{ssid_2}$ is a sub-session of $\F$ (i.e. instance) inside some $!\F$ with sub-session id $\msf{ssid}_1$ inside of $!!\F$ (the message accesses functionality $!!\F[\msf{ssid}_1][\msf{ssid}_2]$).
The \msf{squash} protocol flattens the indexing of instances of \F and combines session ids $\msf{ssid}_1$ and $\msf{ssid}_2$ into a single \msf{ssid}: $\msf{ssid}_3 := \msf{ssid}_1 \cdot \msf{ssid}_2$.
If follows intuitively that the view for the environment remains the same. 

We construct a simulator such that:
\[
\msf{execUC} \, \Environment \, \idealP \, !!\F \, \Sim{\msf{squash}} \approx \msf{execUC} \, \Environment \, \msf{squash} \, !\F \DummyAdv 
\]
The simulator is very simple. 
Inputs to/from parties/\Environment for a corrupt party is forwarded unmodified.
Input intended for $!\F$ of the form $(\msf{ssid}_1 \cdot \msf{ssid}_2, msg)$ sends $(\msf{ssid}_1, (\msf{ssid}_2, msg))$ to $!!\F$. 
Output from $!!\F$ is modified inversely and sent to \Environment.

The simulator is clearly \textit{well-typed} 

\end{proof}

\subsection{UC Composition}
Composition in the UC setting is not limited to replacement of a single instance of a protocol.
Instead, it permits replacement of any number of instances of a protocol $\phi$, each with their own session id, with instances of a realizing protocol $\pi$.
This generalized form of composition follows directly from Theorems \ref{thm:singlecomp} and \ref{thm:squash}.

\begin{theorem}[Composition]\label{thm:composition}
\begin{mathpar}
\inferrule*[right=compose]
{
	%(\pi, !\F_1) \sim (\idealP, F_2) \semi (\rho, !\F_2) \sim (\idealP, \F_3) \\ 
	!\F_1 \xrightarrow{\pi} \F_2 \semi !\F_2 \xrightarrow{\rho} \F_3 \\
	%\Rightarrow \exists \Simulator(\Adversary) \vdash (\rho^{!\F_2 \rightarrow (!\pi \, \circ \, \msf{squash})}, !\F_1) \sim (\idealP, \F_3)
}
{
	!\F_1 \xrightarrow{\rho \, \circ !\pi \circ \, \msf{squash}} \F_3
	%(\rho \, \circ \, !\pi \circ \msf{squash}, !\F_1) \sim (\idealP, \F_3)
}
\end{mathpar}
\end{theorem}

\begin{proof}
The proof of full composition follows directly from the single composition Theorem~\ref{thm:singlecomp} and the Squash Theorem~\ref{thm:squash}.
By Theorem~\ref{thm:singlecomp} we can infer $!!\F_1 \xrightarrow{\rho \, \circ \, !\pi} \F_3$.
Theorem~\ref{thm:squash} allows us to ``squash'' $!!\F_1$ and construct a simulator for $!\F_1 \xrightarrow{\rho \, \circ \, !\pi \, \circ \, \msf{squash}} \F_3$
\end{proof}

\caption{The \msf{execUC} function used for the two-party commitment example used througout this paper. Recall, the \msf{execUC} is customized insofar as it takes in some number of virtual token types (here, $K_1$) to enable machines that simulate other machines. In the commitment example, there is no such simulation happening at the protocol or functionality level, therefore only the real token type $K_1$ is used here. The funtion spawns all the necessary ITMs in the UC execution: the environment, the protocol wrapper, the functionalty (wrapped), and the adversary. Each is parameterized with the security parameter $k$ and a random bit sequence $\msf{rng} \in \{0,1\}^{poly(k)}$.
At the end, the environment is started and it returns a bit $b$ which is its guess for which world it is in. The full code can be found in the Appendix.}
\label{lst:execuc}
\end{figure*}

An obvious omission from the \msf{execUC} process definition is the protocol, functionality, adversary, and environments as function parameters.
The reason for the omission is that passing process definitions as parameter is not supported yet in the Nomos impementation.
Therefore, we rely on importing modules which define the relevant processes and define them in scope for \msf{execUC}.

A module representing the protocol, for example,  must define a process called \msf{PS.prot} and \msf{PS.func} for the protocol and the functionality, respectively.
The protocol wrapper and functionality wrapper manage spawning the instance(s) of the functionality and protocol parties.
Similarly, an environment \msf{PS.env} and adversary \msf{PS.adv} must be defined as well.
The message types exchanged between the processes are provided directly to \msf{execUC} as type parameters of the form \msf{p2f}, \msf{f2p}, and so on. 

The environment is spawned first and selects the session id, or \msf{sid}, for the execution and determines the corrupted parties, \msf{clist}.
The rest of the ITMs are then spawned with this \msf{sid} and are given the list of corrupt parties.
Recall that in the UC framework, corrupt parties accept input and give output to the adversary instead of the environment, and the protocol wrapper runs dummy parties in their place that forward messages between \Adversary and \F.

Finally, the environment executes its own code when activated by \msf{\$z.start} and returns a bit that indicates its guess as to which world it is operatin in: real or ideal.
Over all possible environments, security parameters $k$, and random bit sequences $r$, the output of \msf{execUC} represents an ensemble of distributions for the output bit. 

\subsection{The Protocol Wrapper}
The \msf{execUC} definition introduces a new construct called the \textit{protocol wrapper}. 
In the UC experiment, the environment can create protocol parties on the fly and none exist until the first message is written to them.
Thefeore the wrapper is intended to create new parties on demand.

The necessity of a protocol wrapper leads to an interesting problem in how channels and session typed can be used.
All communciation between protocol parties and \Environment, \F, and \Adversary is managed by the protocol wrapper, and the need to multiplex and de-multiplex communication between parties and other machines makes session types between them impossible.
It isn't possible to use multiple session-typed channels through a single communicator either, because parties can have different roles within a protocol so not even the same code is being executed (hence different session types governing the protocol) for each party.
Therefore, we define a new approach to creating protocol-specific party wrappers, but with a generic construction that can be used for any protocol.
We use the two-party commitment protocol as an example to demonstrate how the construction works and show that it is generic enough to allow code generation of a wrapper for any protocol.

Recall the session type governing the committer and receiver in the commitment protocol:
\begin{gather}
	\mi{stype} \; \m{sender} = \ichoice{\mb{commit} : \m{bit} \product \m{scommitted}} \\
	\mi{stype} \; \m{scommitted} = \ichoice{\mb{open} : 1} \\
	\mi{stype} \; \m{receiver} = \echoice{\mb{commit} : \m{rcommitted}} \\
	\mi{stype} \; \m{rcommitted} = \echoice{\mb{open} : \m{bit} \arrow \one}
\end{gather}

The commitment protocol has multiple roles, the committer and receiver, with different session types. 
The wrapper instantiates its internal channels to the parties with the appropriate session types and maintains lists that hold the channels for each possible session types.
For commitment the lists for the \msf{z2p} channels are the following:
\begin{gather}
	\m{R1L1}[\m{sender}] \\
	\m{R1L2}[\m{scommitted}] \\
	\m{R2L1}[\m{receiver}] \\
	\m{R2L2}[\m{rcommitted}] 
\end{gather}
At the start of the commitment UC execution, the \msf{z2p} channels for the committer and receiver will be in $\msf{R1L1}$ and $\msf{R2L1}$, respectively.
Recall this is an auto-generated wrapper, hence the lists are named generically: $\msf{R1L1}$ stands for list 1 or role 1 (role 1: committer. role 2: receiver). 

The channels between the protocol wrapper and the rest of the machines are still limited to functional types through a communicator. 
For example, as shown in the \msf{execUC} definition, the protocol wrapper's channel from \Environment is typed as \inline{comm[K1][Z2Pmsg[z2p]]} where \inline{z2p} is a type parameter to \msf{execUC} which is
\begin{gather}
\mi{type} \; \m{comz2p} = \m{Commit} \; \mi{of} \; \m{bit} \; | \; \m{Open}
\end{gather}
for commitment.

When the protocol wrapper receives a message for some \msf{pid}, if the party doesn't exist the protocol wrapper creates all the party's channel parameterized by the correct session types (the party's role and session types are determined by functional type of the incoming message).
The channels are stored in the appropriate lists corresponding to their type.
The session type of the message is determined by the functional type, and the session-typed message is sent along that channel.
This we can still make use of session types and environments, adversaries, or protocols that send messages out of order (i.e. are incorrect) will still \textit{fail to type check}.
After delivering the message, the channel is moved to the next list corresponding to its new type.

For outgoing messages, the party wrapper does a similar conversion where it reads the sesion typed message output by the party and converts it to the appropriate functional message type.
For the commitment example, at the beginning the type of the \inline{p2f} channel the committer has is typed as (1). 
When the party \inline{PID} sends a bit with
\begin{lstlisting}[basicstyle=\small\BeraMonottFamily, frame=single, mathescape]
$\$$f2p.commit 
$\tb{send}$ $\$$f2p b
$\tb{pay}$ {p2fn} $\$$f2p 
\end{lstlisting}
the party wrapper does
\begin{lstlisting}[basicstyle=\small\BeraMonottFamily, frame=single, mathescape]
$\$$z2p.SEND 
$\tb{send}$ $\$$z2p pid 
$\tb{send}$ $\$$z2p Commit(b)
$\tb{pay}$ {p2fn} $\$$f2p
\end{lstlisting}
sending a functional message type to the functionality.
For outgoing messages, a new process per party channel waits to read and does the inverse conversion: from session type to funtional type and attaches the party's \msf{pid} to it.

\paragraph{Functionality Wrapper}
Similar to protocol parties, we want to write functionalities using session types.
However, as described for the protocol wrapper, a static definition does not suffice to capture the UC features: dynamic number of parties and different roles (hence, different session types) per party.
For the same reasons, we also create a functionality wrapper around the ideal functionaltiy (in both the real and ideal worlds).
The key difference between the functionality wrapper and the protocol wrapper is that there is only one instance of the functionality running.
We also want to retain the the design of \Fcom presented earlier, where a functionality can be written to interact specifically with each party on separate channels. 
Therefore, the functionality wrapper also generates lists for each session type for each party role (the roles are committer and receiver in \Fcom), and de-multiplexes incoming messages.
Like the protocol wrapper code, the functionality wrapper searches for the channel of the sending \inline{PID} and attempts to send the session typed messages accross it.
The full code of the functionality wrapper for the commitment ideal world is given, in full, in the appendix. 

\subsection{Polynomial Bound}
The UC import mechanism provides a way to define polynomial time computation and resource-bounds by ensuring that a single ITM's execution is upper-bounded by some value $T(n)$ where $T$ is a polynomial and $n$ the total units of import the ITM ever receives.
In UC, it is important to reason about polynomial bounds in the security parameters $k$. Hence, the UC execution relies on an initial amount of import that is give as a polynonial in $k$. 
In NomosUC, we take advantage of the import built into the type system to ensure ITMs are PPT in the security parameter. 

\begin{definition}[PPT Term]\label{def:pptterm}
A \textit{PPT term} is a \textit{well-typed} term $e(k, r)$ that is \textit{closed} except for security parameter $k$, random bit sequence $r$.
\end{definition}

We first-define terms that are well-typed in the traditional session-types-sense in Definition~\ref{def:pptterm}, i.e. without any resource constraints~\cite{sessiontypes}.
Such terms are closed except for the security parameter $k$ and some uniformly random bit sequence $r$.

However, we also want to reason about terms that are well-typed when connected to another Nomos terms.
We introduce the term \textit{well-matched} to mean a PPT term $e$ is well-typed when connected to another term $e'$.
Simply put, the channels that $e$ and $e'$ share channels of the same type. 
Specifically, we want to exclude processes that logically share a channel, say the channel from \inline{p} to \inline{f}, but belong to different protocols (their types message types don't match).
This new definition becomes important when we discuss UC emulation below as we want to reason about environments that are \textit{well-matched} for a protocol $\pi$ or a specific adversary \Adversary.

\begin{definition}[Well-Matched]\label{def:wellmatched}
\begin{mathpar}
\footnotesize
\inferrule*[right=Well-matched]
{\Tokens_1, K \semi \Delta_1 \vdash C_1 :: \Delta_1' \semi 
\Tokens_2, K' \semi \Delta_2 \vdash C_2 :: \Delta_2' \\ \\
 S \equiv \Delta_1 \bigcap \Delta_2 \neq \emptyset}
{\Delta_1 \equiv_{S} \Delta_2 \semi K \equiv K'} 
\end{mathpar}
\end{definition}

Notice that in Definition~\ref{def:wellmatched} we are concerned with two terms that are \textit{open} even when connected. 
We only care about being well-matched, when connected to another term, on the channels over which they are connected.

Next we introduce our definition of a polynomial-bound in the security parameter $k$.
Terms that are PPT in $k$ are dubbed \textit{well-resource-typed}.
\begin{theorem}[PPT in $k$]\label{thm:ppt}
A \textit{PPT Term} $e(k, r)$ is well-resource-typed if, given initial import $n(k) \in poly(k)$, there exists a polynomial $T$ s.t. $\forall k, r, e(k, r) \{n(k)\}$ terminates in at most $T(n)$ steps. 
\end{theorem}

\begin{proof}
The Nomos type system only type checks programs for which a satisfying assignment of polynomial $T$ is possible.
Given an $n \in poly(k)$, all programs that type check must be \textit{well-resource-typed.}
%The Nomos type system guarantees that a satisfying assignment of $n$ and $T$ will correctly type-check.
%Therefore, given an initial amount of import $n(k) \in poly(k)$, the existence of some $T$ ensures that any process, regardless of its randomized execution according to the bit sequence $r$, $e$ is guarantees to be upper-bounded by $poly(k)$ satisfying the definition of probabilistic polynomial time in $k$.
\end{proof}

\subsection{Emulation}
A proof of security in the UC framework relies upon emulation of different executions.

In general, we say that a protocol $\pi$ posesses the same security properties as another protocol $\phi$ if no environment given them inputs can distinguish between them for any adversary.
In most cases we compare a real protocol $\pi$ with an idealized protocol $(\idealP, \F)$ which is actually just an ideal functionality with dummy parties.
The ideal functionality is known to achieve the desired security processes because it acts like a simple, trusted third party.
They are much simpler than protocols because they don't require any special code to handle mutually distrustful other processes, and they perform the given computation on behald of the ideal world parties.

Given the random choices ITMs in UC can make, it is clear that the outputs of \inline{execUC} in Figure~\ref{lst:execuc} produces and ensemble of distributions over all possible random bitstrings and security parameters.
Emulation, then, is about the ensembles created by two UC environments being computationally indistinguishable from each other.
We define indistinguishabiliy between ensembles in a standard way using \textit{statistical distance} in Definition~\ref{def:distance}.

\begin{definition}[Indisinguishability]\label{def:distance}
Two ensembles $\mathcal{D}_{1,k}, \mathcal{D}_{2,k}$ are indistinguishable, $\mathcal{D}_{1,k} \sim \mathcal{D}_{2,k}$, if their statistical distance is at most $negl(k), \forall k$.
\end{definition}

Before we introduce the emulation definition, we first define valid protocols, valid functionalities, and what it means for protocols, functionalities, adversaries, and environments to be well-matched with each other.
We shorten the communicator type \msf{comm} to \msf{c} in the following definitions.

\todo{Ankush: The context of a valid functionality must contain channels typed with the type parameters given by \msf{execUC}. An the machine, parameterized with security parameter $k$ and random bit sequence $r$ are bounded by some polynomial $T_\F$. $\leftarrow$ the last part is meant to capture the well-resource-typed (from the well-matched definition), but maybe we can just say $\F$ is well-resource-typed given $k$,$r$}
\begin{definition}[Valid Functionality]\label{def:validfunc}
\begin{mathpar}to one of footnotesize
\inferrule*[right=valid-F]
{\exists c_1:c[\msf{p2f}], c_2:c[\msf{f2p}], c_3: c[\msf{f2a}], c_4:c[\msf{a2f}] \in \Delta_1 \\
\Delta_1 \models (\F(k, r) : T_\F) :: \Delta_1'}
{\msf{validF}\ \F \rightarrow \Delta_1'}
\end{mathpar}
\end{definition}

\todo{The intent is the same as above here execpt for protocol having channels with the right types. Again here I could just say $\pi$ is well-resource-typed instead of the $\pi(k,r)$ that is there now.}
\begin{definition}[Valid Protocol]\label{def:validprot}
\begin{mathpar}
\footnotesize
\inferrule*[right=valid-P]
{\exists c_1: \msf{p2f}, c_2: \msf{f2p}, c_3: \msf{p2a}, c_4: \msf{a2p}, c_5: \msf{z2p}, c_6: \msf{p2z} \in \Delta_1 \\
\Delta_1 \models (\pi(k, r) : T_\pi) :: \Delta_1' }
{\msf{validP}\ \pi \rightarrow \Delta_1'}
\end{mathpar}
\end{definition}

\todo{Ankush: this defines what it means for a protocol and functionality to be well-matched. Namely, they shared channels typed according to parameters given by execUc (p2f, f2p, ...) and have the same type and import parameters on their communicators}
\begin{definition}[Well-Matched]
\begin{mathpar}
\footnotesize
\inferrule*[right=p2f match] 
{\msf{validP}\ \pi \rightarrow \D_1 \semi \msf{validF}\ \F \rightarrow \Delta_2 \\
\Delta_1:, (\msf{c}[K][\msf{f2p}]), (\msf{c}[K][\msf{p2f}]) \equiv \\
\Delta_2, (\msf{c}[K][\msf{pid \textasciicircum f2p}]), (\msf{c}[K][\msf{pid \textasciicircum p2f}])}
{\langle \pi \leftrightarrow \F \rangle}
\end{mathpar}
\end{definition}

\todo{Ankush: same for this one and the rest, as above}
\begin{definition}
\begin{mathpar}
\footnotesize
\inferrule*[right=p2a match] 
{\msf{validP}\ \pi \rightarrow \Delta_1 \semi \Adversary \rightarrow \Delta_2 \\
\Delta_1:, (\msf{c}[K][\msf{a2p}]), (\msf{c}[K][\msf{p2a}]) \equiv \\ 
\Delta_2, (\msf{c}[K][\msf{pid \textasciicircum a2p}]), (\msf{c}[K][\msf{pid \textasciicircum p2a}])}
{\langle \pi \leftrightarrow \Adversary \rangle}
\end{mathpar}
\end{definition}

\begin{definition}
\begin{mathpar}
\footnotesize
\inferrule*[right=f2a match] 
{\msf{validF}\ \F \rightarrow \Delta_1 \semi \Adversary \rightarrow \Delta_2 \\
\Delta_1:, (\msf{c}[K][\msf{a2f}]\{a2fn\}), ( \msf{c}[[K]\msf{f2a}]\{0\}) \equiv \\
 \Delta_2, (\msf{c}[K][\msf{a2f}]\{a2fn\}), ( \msf{c}[K][\msf{f2a}]\{0\})}
{\langle \F \leftrightarrow \Adversary \rangle}
\end{mathpar}
\end{definition}

\begin{definition}
\begin{mathpar}
\footnotesize
\inferrule*[right=p2z match] 
{\msf{validP}\ \pi \rightarrow \Delta_1 \semi \Environment \rightarrow \Delta_2}
{\Delta_1:, (\msf{c}[K][\msf{z2p}]), (\msf{c}[K][\msf{p2z}]) \equiv \\
 \Delta_2, (\msf{c}[K][\msf{pid \textasciicircum z2p]}), (\msf{c}[K][\msf{pid \textasciicircum p2z}])}
\end{mathpar}
\end{definition}

Indisintiguishability between two protocols is defined as follows (we shorten the communicator type \msf{comm} to \msf{c}):

\begin{definition}[Emulation]\label{def:emulation}
Given two protocols $(\pi, \F_1), (\phi, \F_2)$ that are well-resource-typed then if $\forall \Adversary$ well-matched with $(\pi, \F_1)$, $\exists \Simulator$ s.t. $\forall \Environment$ well-matched with \Adversary and $(\pi, \F_1)$: \Simulator is well-matched with $(\phi, \F_2)$, \Environment is well-matched with $(\phi, \Simulator)$, and $\msf{execUC}(\pi, \F_1, \Environment, \Adversary) \approx \msf{execUC}(\phi, \F_2, \Environment, \Simulator)$:

\begin{mathpar}
\footnotesize
	\inferrule*[right=emulate]
	{
		. \models \msf{execUC}[\Tokentypes][\alpha] :: \Delta[\Tokentypes][\alpha] \\ \\
		% Protocols that are well-matched with their functionalities
		\msf{validP}\ \pi \rightarrow \Delta_1' \semi
		\msf{validP} \phi \rightarrow \Delta_2' \semi
		\langle \pi \leftrightarrow \F_2 \rangle, \langle \phi \leftrightarrow \F_1 \rangle \\
		% Type of execUC[DELTA_pi] and execUC[DELTA_phi]
		\Delta_1'[\Tokentypes][\mathrm{T}_{\pi}] \equiv_{\Environment} \Delta_1\ 
		\semi \Delta_2'[\Tokentypes][\mathrm{T}_{\phi}] \equiv_\Environment \Delta_2 \\
		% For all A if exists well-typed A that is well-matched with real world
		\forall \Adversary, (\exists (\Delta_4, \Delta_4') | \Delta_4 \vdash \Adversary :: \Delta_4',\ \langle \Adversary \leftrightarrow \pi \rangle, \langle \Adversary \leftrightarrow \F_1 \rangle \\
		% implies simulator that is well-matched for ideal world
		\Rightarrow \exists (\Delta_3,\Delta_3') | \Delta_3 \vdash \Simulator_\Adversary :: \Delta_3', \langle \Simulator_\Adversary \leftrightarrow \phi \rangle, \langle \Simulator_\Adversary \leftrightarrow \F_2 \rangle \\
		% for all Z they that's well-matched for the real world => Z is well-matched with S and ideal world
		\forall \Environment (\langle \Environment \leftrightarrow \Adversary \rangle, \langle \Environment \leftrightarrow \pi \rangle \Rightarrow \langle \Environment \leftrightarrow \Simulator_\Adversary \rangle, \langle \Environment \leftrightarrow \phi \rangle \\
		% and emulation has to hold
		\msf{execUC} \ \pi\ \Environment\ \F_1\ \Adversary \approx\ \msf{execUC} \ \phi\ \Environment\ \F_2\ \Simulator_\Adversary))
	}
	{
		% EMULATION DEFINITION
		\lambda \Adversary . \Simulator_\Adversary \vdash (\pi, \F_1) \sim (\phi, \F_2)
	}
\end{mathpar}
\end{definition}
The definition ensures that for emulation to hold, the constructed simulator must be well-matched everywhere \Adversary is well-matched: for all environments \Adversary is well-matched with the \Simulator must also be well-matched with.

\paragraph{UC Realize}
When we talk about emulation, we particularly care about emulation with respect to an ideal protocol $\phi$ which is really just $(\idealP, \F)$ where \idealP is the protocol which forwards all messages to/from \Environment and \F.
We say the protocol $\pi$ (potentially with a hybrid functionality $\F_1$) UC-realizes an ideal functionality $\F_2$ if Definition~\ref{def:emulation} holds for $(\pi, \F_1)$ and  $\phi = (\idealP, \F_2)$

\begin{definition}[UC-Realize]
A protocol $\pi$ UC-realized an ideal functionality $\F_1$ if $(\pi, \F_2) \sim (\idealP, \F_1)$ for some $\F_2$.
\end{definition}

\subsection{Dummy Lemma}
The Dummy Lemma is an important lemma in the UC framework that requires only one simulator to work with a dummy adversary in order to prove emulation with respect to any adversary.
The proof of the lemma makes use of the \msf{withdrawTokens} program definition from Section~\ref{sec:nomosuc}.
The instruction allows for re-use of existing machines and make simulator construction to use the real-world adversary, or other sub-simulators, in a black-box manner.

The Lemma states that if dummy simulator satisfies emulation with respect to the dummy adversary, then for any \Adversary a simulator can be constructed with the dummy simulator. 
The constructed simulator simply runs \Adversary and \Dummysim internally, and it sends messages from \Environment to \Adversary and outputs of \Adversary to \Dummysim.
At a high leve, the proof relies on the emulation definition where dummy emulation covers environments that run \Adversary internally. Here, we are only moving \Adversary into the execution

\begin{theorem}[Dummy Lemma]\label{thm:dummy}
If $\exists \Dummysim$ s.t. $ \DummyAdv, \Dummysim \vdash (\pi, \F_2) \sim (\phi, \F_1)$ then $\forall \Adversary \ \exists \Simulator_\Adversary$ s.t. $\Simulator_{\Adversary} \vdash  (\pi, \F_2) \sim (\phi, \F_1)$ 
\end{theorem}

\begin{proof}
The constructed simulator $\Simulator_\Adversary$ internally simulates \Dummysim and \Adversary through a virtual token type $K'$. 
We describe the simulation pattern below to simulate messages to \Dummysim and \Adversary.
Recall that the virtual tokens consturction is a tool to make witing complex protocols easier, and has no impact on the import token requirements of the simulating machine.
Simply put, simulating as a block-box should be equivalent, with respect to import, as \Simulator running the code natively. 
The only different in running a simulation internally is additional potential usage in using \inline{$\tm{withdrawToken}$} and routing messages.

On input from \Environment on channel \msf{z2p}, \Simulator:
\begin{lstlisting}[basicstyle=\small\BeraMonottFamily, frame=single,  mathescape, label={lst:sim}]
msg = $\nrecv$ $\$$z2a ;
$\nget$ $\$$z2a {z2an : K} ;
$\tm{withdrawTokens}$ f K K1 z2an ;
$\nsend$ $\$$a_z2a msg ;
$\npay$ {z2an : K1} $\$$a_z2a ; 
\end{lstlisting}

Similarly, on output from \Adversary to a protocol party on channel \msf{a2p}
\begin{lstlisting}[basicstyle=\small\BeraMonottFamily, frame=single,  mathescape]
pid = $\tb{recv}$ $\$$a_a2p ;
msg = $\tb{recv}$ $\$$a_a2p ;
$\tb{get}$ K1 $\$$aa2p {a2pn} ;
$\tb{send}$ $\$$sd_z2a A2P(pid, msg) ;
$\npay$ $\$$sd_z2a {z2an : K1} ;
\end{lstlisting}

$\Simulator_\Adversary$ forwards input from \Environment and forwards it to the internal \Adversary. 
\Adversary output to either the protocol parties or the ideal functionality. 
\Simulator forwards this output to \Dummysim acting as input from the environment (here we fallback to the notion that \Adversary can be run internally by \Environment) and forward any outputs it creates to the intended machines.
The proof oblication here is to ensure that the constructed simulator $\Simulator_\Adversary$ is well-resource-typed for all well-resource-typed and well-matched, with \Environment, \Adversary.
The $\Simulator_\Adversary$ performs constant overhead on the simulattion of \Adversary and \Dummysim. Therefore, a sufficient bounding polynomial on the runtime of $\Simulator_\Adversary$ can be given as:
\[
T(n) = T_{\Adversary,\Dummysim}(n) + T_{\Adversary,\Dummysim}(n) + O(n)
\]
where $T_{\Adversary,\Dummysim}(n)$ is the greater of the two bounding polynomials for \Dummysim and \Adversary evaluated at $n$, and $n$ is the import that \Environment sends to \Adversary. 
The same \textit{well-resource typed} reasoning extends to the token context where amount of virtual tokens created are polyomial in number and generate potential that is bounded by the above bounding polynomial for $\Simulator_\Adversary$.
\end{proof}

\subsection{Single Composition}
In this section we present a simplified composition theorem and another theorem, which we call the \textit{squash theorem}.
These two theorems combine to prove the full generalized composition theorem as it appears in the UC framework~\cite{uc}.

The composition operator defines a way for some protocol $\rho$ that uses a functionality $\F$ to swap $\F$ for a procol $(\pi, \F')$, which realizes $\F$, such that $(\rho, \F) \sim (\phi, \F'') \Rightarrow (\rho \circ \pi, \F') \sim (\phi, \F'')$.
The $\circ$ composition operator is defined in Nomos in Figure~\ref{lst:compose}.

Recall that the Nomos language currently does not support passing processes as arguments to other processes even though the theory allows it. 
In the $\circ$ code the protocols $\pi$ and $\phi$ exist globally.

\begin{figure*}
\begin{lstlisting}[basicstyle=\small\BeraMonottFamily, frame=single,  mathescape]
$\tb{proc}$ compose[K][z2r][r2z][f2r][r2f][p2f][f2p] : 
    (pid: Int), ($\$$z_to_p: c[K][z2p]), ($\$$p_to_z: c[K][r2z]), 
    ($\$$f_to_p: c[K][f2r]), ($\$$p_to_f: c[K][r2f])  |- ($\$$D : 1) =
{
	$\$$rho_to_pi <- $\tm{createchan}$[K][p2f];
	$\$$pi_to_rho <- $\tm{createchan}$[K][f2p];

	 <- pi  <-                 $\$$rho_to_pi $\$$pi_to_rho $\$$p_to_f $\$$f_to_p ;
	 <- phi <- $\$$z_to_p $\$$p_to_z $\$$rho_to_pi $\$$pi_to_rho ; 
}
\end{lstlisting}
\caption{Composition operator in Nomos that connects a protocol $\rho$ to a protocol $\pi$ that uses some functionality $\F$. The operators creates new channels to connect the realizing $\pi$ and it's hybrid \F. Output from $\rho$ intended for the replace functionality are actually send to parties of $\rho$, and channels outgoing from the parties to the functionality are given to $\pi$.}
\label{lst:compose} 
\end{figure*}

\todo{Include a graphical illustration of wtf is going on, and going on inside the party wrapper as}

\begin{theorem}[Composition]\label{thm:singlecomp}
\begin{mathpar}
\inferrule*[right=single-compose]
{
	\F_1 \xrightarrow{\pi} \F_2 \semi \F_2 \xrightarrow{\rho} \F_3 \\
}
{
	\F_1 \xrightarrow{\rho \circ \pi} \F_3
}
\end{mathpar}

If \textit{well-typed} $(\pi, \F_1$) realizes $\F_2$ and ($\rho$, $\F_2$) realizes some $\F_3$, then $(\rho \circ \pi, \F_2)$ is \textit{well-typed} and realizes $\F_3$ when $\circ$ is defined as in Figure~\ref{lst:compose}.
\end{theorem}

\begin{proof}
The pre-condition ensures the existence of a \textit{well-resource-typed} simulator $\Simulator_\pi$ for $(\pi, \F_1) \sim (\idealP, \F_2)$. 
We construct a simulator $S$ which relies only on $\Simulator_\pi$ for:
\[
	\msf{execUC}\ (\rho \circ \pi)\ \F_1\ \Environment\ \Adversary \approx \msf{execUC}\ \idealP\ \F_3\ \Environment\ \Simulator
\]	
We don't need to perform simulation on any inputs by \Environment to the main parties of $\rho$ (it's the same protocol in both worlds).
The constructed simulator \Simulator simulates \Sim{\pi} internally and passes messages intended for the parties of $\pi$, or for $\F_2$, to \Sim{\pi} and simulates its computation.
Similariy, \Simulator sends any message from $\F_3$ to \Sim{\pi} for simulation.  
Input to any party of the main protocol $\rho$ from \Environment, or outout from them to \Simulator, are forwarded without any modification or simulation.
The constructed simulator performs constant overhead in routing messages to the simulated \Sim{\pi} and forwrading messages to/from parties of $\rho$/\Environment. 
Given that \Sim{\pi} is \textit{well-resource-typed}, with bounding polynomial $T_{\Sim{\pi}}$, it suffices to show that an additional linear term is sufficient to create a bounding polynomial for \Simulator.

\end{proof}

We give a simpler, high-level idea of the proof here which can be understood visually:
\begin{align}
& \msf{execUC} \: \Environment \, (\rho \circ \pi) \, \F_1 \, \DummyAdv \\
\equiv \; & \msf{execUC} \: (\Environment \circ \rho) \, \pi \, \F_1 \, \DummyAdv \\
\approx \; & \msf{execUC} \: (\Environment \circ \rho) \, \idealP \, \F_2 \, \Sim{\pi} \\
\equiv \; & \msf{execUC} \: \Environment \, \rho \, \F_2 \, \Sim{\pi} 
%\approx \; & \msf{execUC} \: (\Environment \circ \Sim{\pi}) \, \idealP \, \F_3 \, \Sim{\rho} \\
%\equiv \; & \msf{execUC} \: \Environment \, \idealP \, \F_3 \, (\Sim{\pi} \circ \Sim{\rho}) 
\end{align}
The $\equiv$ operator is a result of moving around ITMs (some from within other ITMs into the main UC execution) and $\sim$ refers to indistinguishability.
In line (13) above, $\rho$ is moved into the execution environment with an unchanged simulator as no additional simulation is required: the simulator allows unfettered communication between parties of $\rho$ and \Environment.

\subsection{Multisession}
The multi-session extension of a protocol or functionality, specified by the $!$ operator (such as $!\rho$ or $!\F$), allows multiple instances to be run within a sinlge ITM.
The ITM simulates multiple instances of the protocol/functionality intnerally and multiplexes input/output to/from them in same way as the party wrapper for protocol parties.
The channel from the protocol wrapper to the multisession operator can be typed as:
\begin{gather}
\mi{stype} \; \m{{P2MS}[a]\{n\}} = \echoice{\mb{push}: pid \textasciicircum ssid \textasciicircum a \arrow |\{n\}> \m{P2MS[a]\{n\}}}
\end{gather}
The operator accepts messages of the form $(\msf{ssid}, msg)$ from a particular \msf{pid}, where \msf{ssid} is a sub-session identifier.
If an instance of the functionality with $\msf{sid} := \msf{ssid}$ then $!\F$ creates one and forwards the message to it.
Additionally, $!\F$ listens for outgoing messages from each of the instances and forwards them to the outside execution.
The operator differs from the party wrapper in one crucial way: it only works with functional messages types and does not wrap around any session types like any other standalone functionality in Nomos UC.

The multisession behaves like the protocol wrapper in that we rely on code generation to create the operator for a particular functionality. 
The reason behind this is that the operator simulates many instances of a functionality and must use virtual tokens to communicate with them. 
For the commitment example we've used throughout this paper, the multisession needs only one virtual token type alongside the real token type.
The commitment functionality doesn't internally simulate any other machines and therefore does not need any virtual token type itself. 
The process definition for $!\F_\msf{com}$ is shown in Figure \ref{lst:bangf} accepting two token types: the real token type $K$ and the virtual token type $K_1$ for instances of $\F_\msf{com}$.

The communicators between \bangf and the other ITMs all use the real token type.
Only the internal channels that it creates use virtual token types.
The communication pattern between the operator and the simulated functionalities works in the same was as Listing \ref{lst:sim}.

\begin{figure*}
\begin{lstlisting}[basicstyle=\small\BeraMonottFamily, frame=single, mathescape]
type sid[a] = SID of String ^ a ;

proc bangF_1[K, K1][$p2f$][$f2p$][$a2f$][$f2a$]{$p2fn$}{$f2pn$}{$a2fn$} : 
    ($\$$pw_to_f: P2MS[K][p2f]), ($\$$f_to_pw: MS2P[K][f2p]), ($\$$f_to_a: MS2A[K][f2a]), ($\$$a_to_f: A2MS[K][a2f]),
	($\$\l1: list[sender] ), ($\$$l2: list[scommitted]), ($\$$l3: list[receiver]), ($\$$l4: list[rcommitted]) |- ($\$$ms: 1)
\end{lstlisting}
\caption{The type definition for the multisession operator for functionalities and the correspond message type and import parameters.}
\label{lst:bangf}
\end{figure*}

\begin{theorem}[PPT !]\label{thm:bangppt}
If a functionality $\F$ is well-resource-typed, then it's multisession extension $!\F$ is well-resource-typed.
\end{theorem}

\begin{proof}
A \textit{well-resource-typed} \F guarantees a polynomial $T_{\F}$ bounding its execution.
In the worse-case, the multisession operator must spawn a new instance of $\F$ an every activation. 
Let $N_{\F}$ denote the total number of instances (and, hence, number of activations) of $\F$ created by the operator.
Note that $N_{\F}$ is polynomial in the security parameter $k$ for all well-typed environments, protocols, and adversary.
Therefore, there always exists a bounding polynomial to bound a polynomial number of simulated instances of \F.
The polynomial can be given as:
$$ P_{!\F}(n) = N_{\F} P_{\F}(n) + \mathcal{O}(N_{\F}) $$
where the $\mathcal{O}(N_{\F})$ is due to the overhead of maintaining and accessing the set of all instances.

Similarly, \F being \textit{well-resource-typed} ensures a valid token context for all processes it may simulate. 
Therefore, it is clear that there exists a global connecting poltnomial $f$ that ensures a valid token context for $!\F$.
\end{proof}

\begin{theorem}[Squash Theorem]\label{thm:squash}
%If a functionality \F is well-resource-typed, then $!\F$ and $!!\F$ are well-resource-typed (by Theorem~\ref{thm:bangppt}) and $(\idealP, !!\F) \sim (\msf{squash}, !\F)$.
%\textit{Well-resource-typed} \F $\Rightarrow$ $!\F \xrightarrow{\msf{squash}} !!\F$%  $(\idealP, !!\F) \sim (\msf{squash}, !\F)$
\begin{mathpar}
\inferrule*[right=squash]
{
\textit{well-resource-typed} \; \F
}
{
!\F \xrightarrow{\msf{squash}} !!\F
}
\end{mathpar}
\end{theorem}

\begin{proof}
First we describe the \msf{squash} protocol in figure \ref{fig:squash}.
Note that $!!\F$ is nested $!$ operators. The top level process maintains multiple sessions of $!\F$ each with their own \msf{ssid}.
Functionalities in each $!\F[\msf{ssid}]$ have their own \msf{sid}. 

In $(\idealP, !!\F)$, \idealP~expects to receive messages of the form $(\msf{ssid}_1, (\msf{ssid}_2, m))$ where $\msf{ssid_2}$ is a sub-session of $\F$ (i.e. instance) inside some $!\F$ with sub-session id $\msf{ssid}_1$ inside of $!!\F$ (the message accesses functionality $!!\F[\msf{ssid}_1][\msf{ssid}_2]$).
The \msf{squash} protocol flattens the indexing of instances of \F and combines session ids $\msf{ssid}_1$ and $\msf{ssid}_2$ into a single \msf{ssid}: $\msf{ssid}_3 := \msf{ssid}_1 \cdot \msf{ssid}_2$.
If follows intuitively that the view for the environment remains the same. 

We construct a simulator such that:
\[
\msf{execUC} \, \Environment \, \idealP \, !!\F \, \Sim{\msf{squash}} \approx \msf{execUC} \, \Environment \, \msf{squash} \, !\F \DummyAdv 
\]
The simulator is very simple. 
Inputs to/from parties/\Environment for a corrupt party is forwarded unmodified.
Input intended for $!\F$ of the form $(\msf{ssid}_1 \cdot \msf{ssid}_2, msg)$ sends $(\msf{ssid}_1, (\msf{ssid}_2, msg))$ to $!!\F$. 
Output from $!!\F$ is modified inversely and sent to \Environment.

The simulator is clearly \textit{well-typed} 

\end{proof}

\subsection{UC Composition}
Composition in the UC setting is not limited to replacement of a single instance of a protocol.
Instead, it permits replacement of any number of instances of a protocol $\phi$, each with their own session id, with instances of a realizing protocol $\pi$.
This generalized form of composition follows directly from Theorems \ref{thm:singlecomp} and \ref{thm:squash}.

\begin{theorem}[Composition]\label{thm:composition}
\begin{mathpar}
\inferrule*[right=compose]
{
	%(\pi, !\F_1) \sim (\idealP, F_2) \semi (\rho, !\F_2) \sim (\idealP, \F_3) \\ 
	!\F_1 \xrightarrow{\pi} \F_2 \semi !\F_2 \xrightarrow{\rho} \F_3 \\
	%\Rightarrow \exists \Simulator(\Adversary) \vdash (\rho^{!\F_2 \rightarrow (!\pi \, \circ \, \msf{squash})}, !\F_1) \sim (\idealP, \F_3)
}
{
	!\F_1 \xrightarrow{\rho \, \circ !\pi \circ \, \msf{squash}} \F_3
	%(\rho \, \circ \, !\pi \circ \msf{squash}, !\F_1) \sim (\idealP, \F_3)
}
\end{mathpar}
\end{theorem}

\begin{proof}
The proof of full composition follows directly from the single composition Theorem~\ref{thm:singlecomp} and the Squash Theorem~\ref{thm:squash}.
By Theorem~\ref{thm:singlecomp} we can infer $!!\F_1 \xrightarrow{\rho \, \circ \, !\pi} \F_3$.
Theorem~\ref{thm:squash} allows us to ``squash'' $!!\F_1$ and construct a simulator for $!\F_1 \xrightarrow{\rho \, \circ \, !\pi \, \circ \, \msf{squash}} \F_3$
\end{proof}


\section{Coin Flipping and Commitment} \label{sec:commitment}
In this section we highlight the use of our token type abstract and the use of virtual tokens in NomosUC.
Specifically, we expand on the commitment ideal functionality \Fcom, used throughout this work, to show a simple example of a simulator sandboxing real protocols, other adversaries, and other simulators. 

\subsection{Commitment Protocol}
The commitment protocol to realize \Fcom exists in the \Fro-hybrid world. This means that protocol parties also have access to an idealized hash function \Fro in the real world. 
A random oracle accepts queries of size $k$ and generates a ``hash'' for them by sampleing $k$-bit randomness ($\{0,1\}^k$. 
Its hiding and binding properties ensure that a generated hash can be used to commit to knowledge of a pre-image and the real-world protocol need only send messages between the committer and receiver in the correct order. 
Let \O{x} be the reply of \Fro on query $x$. 
The committer committing to bit $b$ samples a bliding nonce $n \samplek$ and sends $c = \O{n \oplus b}$ to the receiver. When opening the commitment, the committer sends $b, n$ to the receiver who can check that $c \equiv \O{b \oplus n}$.



\section{Related Works} \label{sec:related}
There are many works that attempt to formalize the UC framework with an implementation for protocol analysis and proof generation.

One of the most relevant works to our own is EasyUC~\cite{easyuc}. 
EasyUC uses the existing EasyCrypy~\cite{easycrypt} toolset to model UC protocols and mechanize proof generation. 
It departs from EasyCrypt's limtations to game-based security definitions (lacking simulation-based composition).
However, it still lacks a notion of polynomial time. The authors, themselves, mentions that it can't detect deviant behavior like the adversary and functionality passing messages between each other indefinitely. 
Our use of the import mechainsm and session types let us reason about polynomial time in the sytem of ITMs encompassed by \msf{execUC} but also locally for \textit{open} terms. 
Furthermore, import in NomosUC lets us have guarantees of termination as well by the polyomial import constraints added to UC by Canetti et al.

Liao et al. introduce executable UC through a new process calculus called ILC~\ref{ilc}.
This work adds some notion of polynomial time although it proves to be too restrictive. 
It results from the fact that poly-time can only be reasoned about for \textit{closed} terms like a full UC execution.
In order to reason about polynomial time for a particular protocol $\pi$ we must reason over all possible other terms that connect to $\pi$ and require that it is polynomial in all such cases.
A simple ping-response server can not be proven to by poly-time in this definition for a deviant other ITM that connects to $\pi$. 
In Nomos, however, as mentioned above, open terms are limited to polytime regardless of the connected other terms because of the import mechanism and the NomosUC type system that guarantees termination. 

Other works that rely in symbolic modelling of cryptography, for example, SymbolicUC~\cite{symbolicuc}, are subsumed by the above ILC work and similarly lack any polynomial time notion. 
\todo{Say something about $\pi$-calculus with probabilistic polynomial time extensions}.


To the best of our knowledge, this is the first work to deal with the new import notion of polynomial time introduced to the UC framework in 2018.
A few other works refer to the import mechanism, but it is restricted to simply defining the import a protocol is given.
	
%easyUC:
%* can not dynamially create new instances of parties/functionalities must statically determine the number of functionalities/parties spawned
%* 
%
%
%The work of Liao et al.~\ref{ilc} is the closest to our own
%It proposes a new process calculus called ILC and a concrete implementation of the UC framework.
%The type system it introduces ensures that correctly types programs can be represented as ITMs.
%However, one drawback of the ILC work is that its polynomial time representation 
%
%
%The EasyUC approach uses the existing EasyCrypt toolset to implement model UC protocols and mechanize the generate of UC-security proofs and proofs of secure composition.
%This work aim considerably higher than our work in actually attempting to generate proofs for their protocols. 
%However, this work falls short in being able to capture any notion of resource bound computation whereas we are able to make guarnatees about polynomial bounds on our system of ITMs and even guarantee termination of programs through our realization of the import mechanism.
%The EasyUC work accepts that not even infinite loops of communication can be caught and, therefore, termination of protocols can't be guarnateed either whereas the import mechanism in Nomos ensures that such infinite loops can not stall protocol progress.

%Another work similar to our own is the Symbolic UC by B\"{o}hl and Unruh.
%This works uses an applied $\pi$-calculus to symbolically model UC protocols and analyze them.
%Similar to the EasyUC work, the goals of this work are somewhat orthogonal to the our own goals.
%However, Symbolic UC does attempt to create an implementatio of UC using the $\pi$-calculus however neglects to address any issues of polynomial runtime.
%
%Perhaps the closes work to our own is that of Liao et al.~\cite{ilc} that builds an executable version of the UC framework by introducing a new process calculus called ILC.
%ILC introduces a type system that guarantees that ILC programs (i.e. functionalities, protocols, etc) can be expressed as ITMs as in the UC framework.
%However, one drawback of ILC is that it's notion of polynomial time ends up being too restrictive.
%In ILC only closed terms without any unbonded variables, i.e. and entire UC exection of a system of ITMs, can be shown to be polynomial in their definition of polynomial time.
%Proving polynomial time for open terms, such as a protocol $\pi$, requires reasoning over all possible contexts in which the protocol could exist however such a definition of polynomial time becomes too restrictive where even a simple ping-responde server protocol wouldn't be considered polynomial time.


\section{Discussion and Future Work}
There are a few immediate next steps that can be taked to improve the consruction in this work.
Namely, we introduce the notion of providerless channels to generalize communication between two process due to the consraints provider/client relationship and the linearity of channels.
We would like to do away with this requiement for channels and subsequent work would aim to greatly simplify our construction. \todo{I believe the term is `multi-party session types` that refer to non-linear channels that don't need acquire/release?}.

We also point out that we build NomosUC on top of the type system defined by Nomos and retrofit the import mechanism to the existng potential mechanism through our virtual tokens construction.
A drawback of this approach is that have not yet implemented a type checker for NomosUC and rely on manual type checking of our code. A clear next step would be the first examine the feasibility of such a type checker and explore efficient implementations of it.

A few related works to our own, also provide automated proof generation albeit for a more consrained version of the UC framework. Our work provides tooling to specify and analyze UC definitions, however, a goal of future work is to introduce further tools like fuzz testing to provide automated guarantees for the correctness of NomosUC code. 



\bibliographystyle{acm}
\bibliography{nomosuc}

%input{figures/asyncwrapper}

\appendix

\todo{Appendix todos:
\begin{itemize}
\item full dumy lemma
\item simulator composition operator
\item multisession extension greater detail and bangppt theorem
\item squas part of multisession
\item how to change import to work with Nomos sending max every time
\item code for composed simulator
\end{itemize}}

%\section{The Functionality Wrapper} \label{app:fwrapper}
%The \fwrapper adopts a similar approach to the \partywrapper, except it only creates one instance of the underlying functionality.
For the same reason as the \partywrapper, we use an \fwrapper to enable a dynamic set of parties.
One caveat in supporting such functionalities, is that, so far, only functionalities whose type follows a specific form are allowed.
In the random oracle type given below, notice that the type before and after a party interacts with it is the same: type $\m{party}[a]$.
\begin{mathpar}
\m{party}[a] = \textcolor{red}{\getpot^1} \ichoice{\mb{hash} : \m{pid} \arrow \m{int} \product \m{hashing}[a]} \\
\m{hashing}[a] = \echoice{\mb{shash} : \m{pid} \arrow \m{int} \product \textcolor{red}{\paypot^0} \m{party}[a]} 
\end{mathpar}
A party queries a $\mb{hash}$ of an integer from \Fro and receives an integer as a response. The session type includes the \inline{pid}, which enables it to handle all parties over the same channel.

An example of the internals of the \fwrapper are showing on the left-hand-side of Figure~\ref{fig:multisession}.
Like the \inline{z2p} processes in the \partywrapper, \inline{S} and \inline{R} represent the sender and receiver in the commitment protocol, and they read from their virtual communicators and communicate with \Fcom over a session-typed channel.




\section{Shared Session Type} \label{app:sharedtypes}

Shared session types impose an \emph{acquire-release} discipline on processes; 
a client must acquire the channel offered by a shared process to interact with it
and must release this channel after the interaction.
The corresponding typing rules are
\begin{mathpar}
  \infer[\up L]
  {\Tokens \semi \Psi \semi \wt, \D, (x : \up A_L)
  \entailpot{q}{q'} \eacquire{y}{x} \semi Q :: (z : C)}
  {\Tokens \semi \Psi \semi \D, (y : A_L)
  \entailpot{q}{q'} Q :: (z : C)}
  %
  \and
  %
  \infer[\up R]
  {\Tokens \semi \Psi \semi \D \entailpot{q}{q'}
  \eaccept{y}{x} \semi P :: (x : \up A_L)}
  {\Tokens \semi \Psi \semi \wt, \D \entailpot{q}{q'} P :: (y : A_L)}
\end{mathpar}
The $\up L$ rule describes a client acquiring a shared channel $x$
and obtaining a private linear channel $y$ along which it can communicate
with the corresponding acquired process.
Correspondingly, the $\up R$ rule describes the shared process
accepting the acquire request and creating the fresh linear channel $y$.
The release-detach rules corresponding to the $\down$ type constructor
are exact dual of acquire-accept.


\section{Arbitrary Parties} \label{app:arbparties}
The \partywrapper manages creating new parties in a sandbox and routing messages to/from them an the execution. 
In this section we present a snippet of the \partywrapper code to show how it reacts to new messages, creates new parties, and delivers them the message.
The example below shows a new message from \Z to a protocol party. 
For messages from \Z, the \partywrapper receives import according to the parameterized amount for the \inline{z2p} channel (line 3).
If the party with \inline{pid} doesn't exist, then it is spawned along with its providerless channels (lines 5-7) and the channel endpoints are saved in lists (line 8).  
Finally, the message is delivered to the providerless channel with the corresponding about of \emph{virtual} tokens of type \inline{K1} (line 11-13). 
Finally, after receiving a message from the outside, the \partywrapper switches to iterating through the outgoing channels of the protocol parties starting with the index 0th channel in the \inline{z2p} list (line 20).

\begin{figure}[h]
	\centering
	\begin{lstlisting}[basicstyle=\footnotesize\BeraMonottFamily, mathescape, frame=single]
$\nmatch$ $\$$z2p, $\$$f2p, $\$$a2p (
  Z2P(pid,m),*,* =>
    $\nget$ {z2pn} K $\$$z2p ;
    $\nif$ not exists pid $\nthen$
      #z2p' $\leftarrow$ channel_init[K1][z2p]; 
      #p2z' $\leftarrow$ channel_init[K1][p2z];
      #f2p' $\leftarrow$ channel_init[K1][f2p]; 
      #p2f' $\leftarrow$ channel_init[K1][p2f];
      $\$$c' $\leftarrow$ PS.prot $\leftarrow$ k rng sid 
               #z2p' #p2z' #f2p' #p2f';
      lz2p $\leftarrow$ append lz2p #z2p'; 
      lp2z $\leftarrow$ append lp2z #p2z';
      lp2f $\leftarrow$ append lp2f #p2f'; 
      lf2p $\leftarrow$ append lf2p #f2p';
    $\nelse$ ()
    #z2p' $\leftarrow$ search lz2p pid ;
    $\nwithdraw$ K K1 z2pn
    #z2p'.SEND ; $\nsend$ #z2p m; $\npay$ {z2pn} K1 #z2p' ;
  *,F2P(pid,m),* =>
    $\tg{(* identical case *)}$
  *,*,A2P(pid,m) =>
    $\tg{(* identical case *)}$
)
$\tg{(* iterate through the p2f providerless channels}$
$\tg{starting with 0th index of the lit lp2f *)}$
$\$$ch $\leftarrow$ multiplexer_p2f[K][s][z2p,p2z][p2f,f2p] 
                     $\leftarrow$ $\tg{(*identical args*)}$ 0 ;
	\end{lstlisting}
\caption{Code for the \partywrapper receiving messages from \Z.}
\label{lst:partywrapper}
\end{figure}



\section{Dummy Lemma} \label{app:dummy}
A proof of the dummy lemma relies on creating a simulator \Sim for an arbitrary adversary given \DS.
Recall from the original discussion in Section~\ref{sec:dummy} that the intuition behind the Dummy Lemma can be seen from the basic definition of emulation.

Consider UC emulation with respect to the dummy adversary. The emulation definition quantifies over all environments. 
This includes an environment that runs any arbitrary real-world adversary \A, for the protocol, internally, and \Z forwards the outputs from the adversary to the dummy adversary and dummy simulator in the execution.
Consider the modified execution where \A is run in the real execution instead of internally: it is the adversary in the real world and it is run as part of the simulator in the ideal world.
The ideal world simulator runs \A and the dummy sim internally and passes input from \Z to \A and output from \A to \DS.
Moving ITMs from inside \Z to the real world is a common pattern in UC proofs and is used in the composition proof as well.
We depict the intuition in Figure~\ref{fig:dummylemmas}.

\begin{figure}
	\begin{subfigure}[t]{0.45\textwidth}
	\centering
	\includegraphics[scale=0.5]{figures/dummylemma_pre.pdf}
	\caption{The ideal world with an environment running an arbitrary \A internally.}
	\label{fig:dummy_pre}
	\end{subfigure}
	\hspace{2mm}
	\begin{subfigure}[t]{0.45\textwidth}
	\centering
	\includegraphics[scale=0.5]{figures/dummylemma_post.pdf}
	\caption{Moving \A into the adversary in the real and ideal world equates to an equivalent execution. Inputs and outputs to \DS are unchanged as are inputs to protocol parties and the functionality. Therefore, we expect the behavior of the execution to be unchanged.}
	\label{fig:dummy_post}
	\end{subfigure}
	\caption{Graphical illustration of the intuition behind the Dummy Lemma as relying on dummy simulation to introduce an adversary.}
	\label{fig:dummylemmas}
\end{figure}


We restate the Dummy Lemma here:
\begin{theorem}[Dummy Lemma]\label{thm:dummy}
If \ $\exists \DS$ s.t. $ \DA, \DS \vdash \F_2 \xrightarrow{\pi} \F_1$ then $\forall \A \ \exists \Sim_\A$ s.t. $\Sim_{\A} \vdash  \F_2 \xrightarrow{\pi} \F_1)$ 
\end{theorem}

\begin{proof}
The constructed simulator $\Sim_\A$ runs \A and \DS internally.
$\Sim_\A$ sandboxes the two simulators using our virtual tokens construction.
The construction simulator is straightforward, and we provide sample snippets of code from the simulator.
The adversaries \A and \DS expect to receive input from \Z through communicator.

Input from \Z is passed to \A:
\begin{lstlisting}[basicstyle=\footnotesize\BeraMonottFamily, frame=single,  mathescape]
msg = $\tb{recv}$ $\$$z_to_a ;
case msg (
  Z2A2P(pid, msg) =>
    $\tb{get}$ $\$$z_to_a {z2an : K} ;
    $\tm{withdrawTokens}$ f K K1 z2an;
    $\$$a_z2a.SEND ;
    $\nsend$ $\$$a_z2a Z2A2P(pid, msg) ;
    $\npay$ {z2an : K1} $\$$a_z2a ;
    $\$$ch <- sim_a_a2p <- ... 
  Z2A2F(msg) =>
    $\tb{get}$ $\$$z_to_a {z2an : K} ;
    $\tm{withdrawTokens}$ f K K1 z2an ;
    $\$$a_z2a.SEND ;
    $\nsend$ $\$$a_z2a Z2A2F(msg) ;
    $\npay$ {z2an : K1} $\$$a_z2a ;
    $\$$ch <- sim_a_a2f <- ... ;
\end{lstlisting}
\end{proof}
It is worth noting that in NomosUC existing processes can be run directly in a sandbox just be using a virtual token.
This is different from the UC notion of the universal turing machine where other ITMs are first encoded as data.

Output from \A to the party wrapper or \F is passed as \inline{Z2A2P} or \inline{Z2A2F} input to \DS using the same token type \inline{K1}.
Output from \DS is forward along to the outside world as adversary input to the dummy parties and the ideal functionality:
\begin{lstlisting}[basicstyle=\footnotesize\BeraMonottFamily, frame=single, mathescape]
msg = $\nrecv$ $\$$ds_a2p ;
$\nget$ $\$$ds_a2p {a2pn : K1} ;
$\$$a_to_pw.SEND ;
$\nsend$ $\$$a_to_pw msg ;
$\npay$ $\$$a_to_pw {a2fpn : K} ;
\end{lstlisting}


\section{Multisession Theorems} \label{app:ms}
In this section we give more extensive code on how the multisession extension work. We also explore Theorems \ref{thm:squash} and \ref{thm:functor} in greater detail, and, specifically address the proof obligation for both.

\subsection{!\F}
The multisession operator presents the same interface to \inline{execUC} as any other functionality.
However, the shell code that we run it inside operates directly on functional messages from its communicators with \F and the \partywrapper instead of performing any conversion to session-types.
The reason for this choice is that a session type for the operator is not very meaningful as it only provides an interface of ``input'' and ``ouput'' to/from underlying instances of the functionality.

!\F communicates with the \partywrapper through the type
\begin{lstlisting}[basicstyle=\small\BeraMonottFamily, mathescape]
$\yo{type}$ p2ms[a] = P2MS of ssid ^ a ;
$\yo{type}$ ms2p[b] = MS2P of ssid ^ b ;
\end{lstlisting}
and with \A through the type
\begin{lstlisting}[basicstyle=\small\BeraMonottFamily, mathescape]
$\yo{type}$ a2ms[a] = A2MS of ssid ^ a ;
$\yo{type}$ ms2a[b] = MS2A of ssid ^ b ;
\end{lstlisting}
both of which are parameterized by the functionality message types \inline{a} and \inline{b}.


Recall that all functionalities are run inside some shell code, and their shell code communicates with communicators to other parities--we can call this a functionality wrapper as it wraps around the actual functionality code.
The multisession operator runs each instance of the functionality inside the wrapper, and provides each with ``virtual'' communicators for \A and the \partywrapper.

The !\F, on input from the \partywrapper, reads a message from the communicator and then does the following:
\begin{lstlisting}[basicstyle=\footnotesize\BeraMonottFamily, frame=single, mathescape]
$\nproc$ f_multisession_p2f_input[K][K1]{p2fn,f2pn} : 
  (k: Int), (rng: [Bit]), (sid: session[a]), (crupt: list[pid]),
  (#p2f: comm[p2fmsg[p2ms]][K]), (#f2p: comm[f2pmsg[ms2p]][K]), 
  (#a2f: comm[a2fmsg[a2ms]][K]), (#f2a: comm[f2amsg[ms2a]][K]) |- ($\$$ch: 1) =
{
  ...
  msg = $\nrecv$ #p2f ;
  $\ncase$ msg (
    P2F(pid, P2MS(ssid, msg)) =>
      $\nif$ not ssid in $\$$p2ssid
      $\nthen$
      	#new_p2ssid <- communicator_init[K1][p2f] <- ;
      	#new_ssid2p <- communicator_init[K1][f2p] <- ;
      	#new_a2ssid <- communicator_init[K1][a2f] <- ;
      	#new_ssid2a <- communicator_init[K1][f2a] <- ;
      
      	$\$$p2ssidnew <- pappend $\$$p2ssid #new_p2ssid ;
      	$\$$ssid2pnew <- pappend $\$$ssid2p #new_ssid2p ;
      	$\$$a2ssidnew <- pappend $\$$a2ssid #new_a2ssid ;
      	$\$$ssid2anew <- pappend $\$$ssid2a #new_ssid2a ;
      
      	$\$$chprime <- f_wrapper[K1][f2p,p2f][f2a,a2f] <- 
                           k rng sid clist $\#$p2ssid $\#$ssid2p $\#$a2ssid $\#$ssid2a $\#$z ;
      $\nend$
  
      #ch <- get_channel #p2ssid ssid ;
      $\nwithdraw$ K K1 {p2fn} ;
      $\$$ch.SEND ;
      $\nsend$ $\$$ch P2F(pid, msg) ;
      $\npay$ $\$$ch {p2fn : K1} ;
  	
    $\$$ch <- f_multisession_f2p_i[K][K1] <- .... $\$$p2ssidnew $\$$ssid2pnew $\$$a2ssidnew $\$$ssid2anew 0;	
}
\end{lstlisting}

First it checks whether the \m{ssid} exists. It not it spawns new communicators for the instance \F: two for the \partywrapper and two for \A.
It then grabs the \inline{#p2f} channel for $\F_{\m{ssid}}$, creates the appropriate virtual tokens and sends the message along that channel.
Finally, after handling the incoming message, it moves to the next step of checking for outgoing messages, to the \partywrapper, from each \F by calling \inline{f_multisession_f2p_i}.
The function accepts an integer at the end, in this example 0, which is the index to check in the communicator list. Eventually all communicators in the list are checked.

Recall that the Theorem~\ref{thm:squash} realizes !!\F, a functionality wrapped twice in the multisession operator an indexed by a pair of ssids $(\m{ssid}_1 \product \m{ssid}_2)$.
The real world that realizes it is comprised of a protocol that ``flattens'' the pair of ssids $(\m{ssid}_1, \m{ssid}_2)$ into a single ssid $\m{ssid}_3 = \m{ssid}_1 || \m{ssid}_2$, the concatenation (can be any other one-to-one and invertible function on the pair) of them.

\begin{theorem}[Squash Theorem]
	\begin{mathpar}
		\inferrule*[right=squash]
		{
			\textit{well-resource-typed} \; \F
		}
		{
			!\F \xrightarrow{\msf{squash}} !!\F
		}
	\end{mathpar}
\end{theorem}

\begin{proof}
The simulator for this construction is a direct simulation where $\Sim{\m{squash}}$ takes as input one of 
\begin{itemize}
	\item \inline{Z2A2P(pid, P2MS(ssid, msg))}: $\SIM{\m{squash}}$ un-concatenates \inline{ssid} into a pair (\inline{ssid1}, \inline{ssid2}) and forward the message \inline{A2P(pid, P2MS(ssid1, P2MS(ssid2, msg)))}.
	\item \inline{Z2A2F(A2MS(ssid, msg))}: Similar to above, $\SIM{\m{squash}}$ splits the given \inline{ssid} and sends \\ \inline{A2F(A2MS(ssid1, A2MS(ssid2, msg)))}. 
\end{itemize}

\SIM{\m{squash}} also keeps one import for every message it ges and forwards the remainder to either of \F or the \partywrapper.
On input from \Z, \SIM{\m{squash}} does the following; 

\begin{lstlisting}[basicstyle=\footnotesize\BeraMonottFamily, frame=single, mathescape]
proc sim_squash[K][p2f,f2p,a2f,f2a]{p2fn,f2pn,a2fn} :
  (k: Int), (rng: [Bit]), (sid: session[a]), (crupt: list[pid]),
  (#z_to_a: comm[z2amsg]), (#a_to_z: comm[a2zmsg]) ... |- ($\$$ch: 1) =
{
  ...
  $\ncase$ #z_to_a (
    Z2A2P(pid, P2MS(ssid, msg))) =>
      ssid1, ssid2 <- split ssid
      #a_to_p.SEND ;
      $\npay$ #a_to_p {a2pn-1} ;
      $\nsend$ #a_to_p A2P(pid, P2MS(ssid2, P2MS(ssid2, msg)) ;
    Z2A2F(A2MS(ssid, msg)) =>
      ssid1,ssid2 <- split ssid
      #a_to_f.SEND ; 
      $\npay$ #a_to_f {a2fn-1} ;
	  $\nsend$ #a_to_f A2P(A2MS(ssid1, A2MS(ssid2, msg))) ;
  ...
}	  
\end{lstlisting}
\end{proof}

We further require Theorem~\ref{thm:functor} in order to conclude full UC-style composition. We restate it here
\begin{theorem}[Multisession Composition]
	\begin{mathpar}
		\inferrule*[right=MultiComp]
		{
			\F_1 \xrightarrow{\pi} \F_2
		}
		{
			!\F_1 \xrightarrow{!\pi} !\F_2
		}
	\end{mathpar}
\end{theorem}

The simulator for this theorem runs internal copies of $\Sim_\pi$ for each new \m{ssid} spawned.
It is easy to see why simulating each isolated instance should be feasible, given that the protocol instances and functionality do not share any state with each other.
However, to complete the proof we need to show a reduction that translates an environment $\Z$ that can distinguish the multi-session experiment into an environment $\Z^*$ that can distinguish real from ideal in the single session experiment.

The high level approach is as follows: $\Z^*$ can only interact with only a single instance of $\F_1$ in the real world, but it must present a view of multiple sessions $!\F_1$ to its sandbox execution of $\Z$.
To do this, $\Z^*$ will create local simulations of subsessions of $\F_1$ for all but one subsession, which it forwards to its external channel.

To complete the precise statement of this proof, we have to resort to (a variable number of) hybrid worlds~\cite{canettiUC,variablehybrids}. We sketch the proof below.
We define a family of environments $\Z^*_i$, indexed by a polynomial $i$,
\begin{itemize}
\item the first $i(k)-1$ subsessions shown to $\Z$ are a local simulation of $\F_1$.
\item the $i(k)$'th subsession is forwarded to the external channel,
\item the $i(k)+1$'th and later subsessions are a simulation of $\pi$.
\end{itemize}
Notice that $\Z^*_0$ presents a simulation of exactly the multi-session real world, while $\Z^*_q$ presents a simulation of exactly the multi-session ideal world where $q$ is a polynomial bound on the runtime resulting from $\Z$.
What we can show is that by triangle rule, if $\Z$ is a distinguisher for the multi-session experiment, there must exist some polynomial $i$ such that the real world with $\Z^*_i$ is distinguishable from the real world with $\Z^*_{i+1}$. Since the real world with $\Z^*_{i+1}$ is exactly the same as the ideal world with $\Z^*_i$, this gives us that $\Z^*_i$ is a distinguisher for real and ideal in the single-session experiment.

%\subsection{Squash Theorem}
%\begin{theorem}[Squash Theorem]\label{thm:squash}
%	\begin{mathpar}
%		\inferrule*[right=squash]
%		{
%			\textit{well-resource-typed} \; \F
%		}
%		{
%			!\F \xrightarrow{\msf{squash}} !!\F
%		}
%	\end{mathpar}
%\end{theorem}
%The Squash theorem, is more straightforward to prove than Theorem~\ref{thm:functor}.
%
%\begin{proof} (Theorem~\ref{thm:squash})
%On examination the \msf{squash} theorem does constant work per activation and concantenates two \inline{ssid}s into one for messages going to the functionality and does the inverse for messages incoming from the functionality.
%In other words, it is a direct simulation and the same ideal functionality exists if both worlds, and it is clear to see that the this theorem holds with a trivial simulator.
%First we describe the \msf{squash} protocol where $!!\F$ are nested $!$ operators.
%The protocol accepts messages intended for $!!\F$ of type \inline{P2MS[p2ms[a]][ms2p[b]]}, i.e. of the form $(\msf{ssid}_1, (\msf{ssid}_2, msg))$, and ``flattens'' them into a single message of type $\inline{P2MS[a][b]}$, i.e. of the form $(\msf{ssid}_3, msg)$.
%
%In $(\idealP, !!\F)$, \idealP~expects to receive messages of the form $(\msf{ssid}_1, (\msf{ssid}_2, m))$ where $\msf{ssid_2}$ is a sub-session of $\F$ (i.e. instance) inside some $!\F$ with sub-session id $\msf{ssid}_1$ inside of $!!\F$ (the message accesses functionality $\F[\msf{ssid}_1][\msf{ssid}_2]$).
%The \msf{squash} protocol flattens the indexing of instances of \F and combines session ids $\msf{ssid}_1$ and $\msf{ssid}_2$ into a single \msf{ssid}: $\msf{ssid}_3 := \msf{ssid}_1 \cdot \msf{ssid}_2$.
%If follows intuitively that the view for the environment remains the same. 
%
%We construct a simulator such that:
%\[
%\msf{execUC} \, \Z \, \idealP \, !!\F \, \SIM{\msf{squash}} \approx \msf{execUC} \, \Z \, \msf{squash} \, !\F \DA 
%\]
%The simulator is very simple. 
%Inputs to/from parties/\Z for a corrupt party is forwarded unmodified.
%Input intended for $!\F$ of the form $(\msf{ssid}_1 \cdot \msf{ssid}_2, msg)$ is sent as $(\msf{ssid}_1, (\msf{ssid}_2, msg))$ to $!!\F$. 
%Output from $!!\F$ is modified inversely and sent to \Z.
%
%The proposed simulator is trivially analyzed to be \textit{well-resource-typed}.
%It performs constant work per activation and does ``real'' simulation other than message modification to/from $!!\F$.
%\end{proof}


\section{Import} \label{app:import}
The most recent iteration of the Universal Composability framework introduces a new notion of polynomial time: \textit{import}.
The import mechanism allocates to the environment some integer $n$ of units of import which it can consume or send to other ITIs.
When writing to another ITI, $\mathcal{Z}$ can specify an amount of import to sent alongside the message, and the receiving ITI can now use this import accordingly.
A good analogy to make for import is to consider them as tokens that are exchanged between ITIs. 
The definition of a polynomially bounded syste of ITIs now becomes: 


\paragraph{Balanced Environments}
The import mechanism above allows the environment give the protocol arbitrarily more import than the adversary.
Such a siuation is unnatural and undesirable. Consider the following argument.

Imagine a protocol $\pi$ and another protocol $\widetilde{\pi}$ which is identical to $\pi$ in every way except that the parties of $\widetilde{\pi}$ first send a message to the adversary proportional in length to their import and expects the adversary to echo the message back. 
Then $\pi$ does not UC-emulate $\widetilde{\pi}$ according to standard UC-emulation for the following reason: $\mathcal{A}$ is an adversary tha delivers all protocol messages and halts. A simulator $\mathcal{S}$ would need sufficient import to handle these messages, but an environment $\mathcal{Z}$ can always give $\widetilde{\pi}$ more import than $\mathcal{S}$ rendering the protocol un-simulatable. 

Such a restrictive definition in clearly unnatural. Therefore, the UC framework introducec an environment constraint: \textit{balanced environments}.

\begin{definition} \label{def:balancedenvironments}
An environment is \msf{balanced} if, at a certain point of execution, it provided import $n_1,...,n_k$ to $k$ ITIs overall, then the overall import of the inputs to the adversary is at least $n_1 + ... + n_k$.
\end{definition}
 
The Nomos language has a built in notion of potential (not to be confused with potential as the basic unit of computation from Section~\ref{sec:realizeimport}) which can be exchanged over channels and passed around between the different processes in the UC execution.
The potential is part of the type of a channel as follows: 


\section{Commitment Protocol} \label{app:protcom}
In this section we expand on the real world protocol that realizes \Fcom in the \Fropp-hybrid model.
The type of \Fropp in Figure~\ref{fig:fropptype} builds on the \Fro type presented in Section~\ref{sec:commitment}. 
Because the receiver must receive a message \Fropp and potentially send it a hash query, we split communication between two uni-directional channels.
The adversary type with the functionality which that lets it query hashes isn't shown but requires 1 unit of import to be sent along with a query. 

\begin{figure}
\begin{center}
	\parbox{0cm}{
	\begin{tabbing}
		$\m{sender}[a] =  \ichoice{ $\=$ \mb{query}: \textcolor{red}{\paypot^2} \m{PID} \product \tgr{\m{Int}} \product \m{sender}[a],$ \\
		\>$\mb{sendmsg}: \textcolor{red}{\paypot^1} \m{ PID} \product \m{a} \product \m{sender}[a]}$ \\
		$\m{shash} = \echoice{ \mb{hash}: \m{ PID} \arrow \tgr{\m{Int}} \arrow \m{hash}}$ \\
		$\m{rquery} =  \ichoice{ \mb{query}: \textcolor{red}{\paypot^1} \m{PID} \product \tgr{\m{Int}} \product \m{rquery} }$ \\
		$\m{receive}[a] = \echoice{ $\=$\mb{hash}: \m{PID} \arrow \tgr{\m{Int}} \arrow \m{receive}[a],$ \\
		\>$\mb{msg}: \textcolor{red}{\getpot^1} \m{PID} \arrow \m{a} \arrow \m{receive}[a]}$
	\end{tabbing}}
\end{center}
\caption{The two types for each of the committer and receiver. The first to types are the senders type to \Fropp and the next is received from \Fropp, and the same holds for the next two and the receiver. Two units of import are sent with a message from the committer. One is sent to the receiver so that it has enough import to query the oracle and check the commitment.}
\label{fig:fropptype}
\end{figure}

Due to the fact that there is a single channel connectig the \partywrapper and the functionality, the amount of import send from the \partywrapper to the wrapped \F is constant. Similarly, a constant amount of import is sent back from \F to the \partywrapper.
Recall from Section~\ref{sec:basic}, that as a result of this construction some messages may be sent with more import to the wrapped processes than required.
In general, despite the import on the session type, setting the parameters for that constant amount of import requires careful consideration in order to ensure that the \partywrapper has enough import to handle all messages parties may send out. 

In order to realize the communication attern where the receiver gets no input (and hence no import) from \Z, the type parameters to the providerless channels between, for example, \Z and \partywrapper need to be 4 units of import.  Similarly, as the session type of \Fropp indicates, 2 import is sent from \partywrapper to wrapped \Fropp and 1 import back. 

\begin{figure}
\begin{lstlisting}[basicstyle=\footnotesize\BeraMonottFamily, frame=single, mathescape]
$\tg{(* committer code after receiving a 'commit'}$
        $\tg{message from the environment *)}$
b = $\nrecv$ $\$$z2p ;
$\nget$ $\$$z2p {2} ;
bits = sample (k-1) rng ;
$\$$p2f.query ;
$\npay$ K {2} $\$$p2f ;
$\nsend$ $\$$p2f pid ;
$\nsend$ $\$$p2f (b || bits) ;
$\ncase$ $\$$f2p (
  hash => pid = $\nrecv$ $\$$p2f ; 
    h = $\nrecv$ $\$$p2f ;
    $\$$p2f.sendmsg ;
    $\nsend$ $\$$p2f pid 2 hash;
	$\npay$ K {2} $\$$p2f ;
\end{lstlisting}
\caption{The code for the committer in $\prot{com}$ when it receives a \msf{commit} message from \Z. It obtains a hash of the message from \Fropp over \msf{p2f} and sends it to the receiver (pid=2) through the same functionality.}
\label{lst:committer}
\vspace{-2mm}
\end{figure}

\begin{figure}
\begin{lstlisting}[basicstyle=\footnotesize\BeraMonottFamily, frame=single, mathescape]
$\tg{(* receiver waiting for the commitment opening}$
        $\tg{from the random oracle channel *)}$
sender = $\nrecv$ $\$$f2p ;
p = $\nrecv$ $\tm{recv}$ $\$$f2p ;
b:hs = p
$\nget$ $\$$f2p {1} ; 
$\tg{...}$
$\tg{(* query the hash of b || hs with 1 import *)}$
$\tg{...}$
h = $\nrecv$ $\$$p2f ;
$\nif$ h == hash
$\nthen$
  $\$$z2p.open
$\nend$
\end{lstlisting}
\caption{The code for the receiver checks for a new message and receives the bit and nonce from the committer. If the hash of the bit and nonce matches the commitment it received, it returns \msf{open} to \Z to confirm the commitment.}
\label{lst:receiver}
\vspace{-3mm}
\end{figure}

In Figures~\ref{lst:committer} and \ref{lst:receiver} we see the code for the committer reacting to a $\mb{commit}$ message from \Z and the receiver reacting to an open commit from the committer, respectively. 

%\subsection{Simulation}
%Finally, we present a simulator \simcom, for the dummy adversary, for which the \Fcom is realized by \prot{com} in the \Fropp-hybrid world.
%Recall that the import requirements for the ideal world, in this case for \Fcom. Therefore, the simulator is parameterized by import parameters required in the real world for the parties of $\pi_\m{com}$ and \Fro.
%The simulator is straightforward and internally maintains a table like \Fro and responds to the environments queries for hashes. 
%When the receiver is corrupt:
%\begin{itemize}
%\item \simcom responds with \inline{P2A2Z(2, no)} to all messages by \Z to get a message from the functionality
%\item On \inline{Committed} by the ideal receiver, \simcom generates a random $r$ and sends \inline{P2A2Z(2, RHash(h))} with no import.
%\item On \inline{Open(b)} from the ideal receiver, \simcom generates a random nonce $x$ and stores \inline{b+x : h} in its \Fro table, and sends \inline{Yes(1, (b,x))} to \Z when asked for messages for the corrupt receiver.
%\end{itemize}
%
%The corrupt committer is not much different from the above case. In this case
%the simulator stores the bit $b$, the none $x$, the corresponding hash $h$, and the import that \Z uses to create a commitment.
%When the simulator receives the message to send the commitment to the receiver, it tells the ideal world committer to commit to $b$ along with 2 import given by \Z, and when it's told to open the commitment it opens it in the ideal world. 
%
%It is immediately clear that this simulator satisfied $\Fro \xrightarrow{\prot{com}} \Fcom$ for the dummy adversary.



\section{Coin Flipping} \label{app:flip}
The first protocols and primitives we examine are disttibuted coin flipping. 
Both as protocols and primitives assumed in other protocols, coin tossing is is an integral part of many distributed byzantine protocols, each relying on slightly different assumptions and fault models. 
They are a simple functionality, but they express and rely on properties such as output fairness and adversarial influence, that UC is uniquely suitable for expressing and reasoning about.
In this section, we apply our methods to a lottery protocol built atom a two-party coin tossing primitive, and show that, despite knowledge of the aforementioned bug in the old MMR protocol our methods are capable of giving meaningful feedback to the programmer suggesting a possible liveness error \footnote{Recall we employ informal methods for analyzing the protocols in this work, and, especially for liveness, can suggest possible liveness failures rather than assert them directly without knowledge of a useful predicate on the execution.}.

We focus the work in this section particularly on analyzing coin tossing across layered composition, and do this for three reasons. 
First, we believe that coin tossing is a strong representative example of the kinds of analysis possible in UC which we want to show can be meaningfully analyzed informally as well.
Second, the assumed primitives across different dirstributed protocols can vary subtley, and, if not carefully analyzed, can be disastrous for the protocols that use them.
The most prominent example of this is in the original publication of the well-known byzantine agreement protocol by  Mostefaoui et al.~\cite{mmrog} (referred to as the MMR protocol from here on out).
A work similar our own ~\cite{byzbymc}, identified a critical liveness bug that arose, because the protocol relied weak common coin that allowed the adversary to see the outcome of the coin in advance of some honest parties rather than the perfect common coin that the protocol actually required in order to terminate. 
Third, software dependencies in this setting is analgous to layered composition. 
Software packages are represented by ideal functionaliteis for development, and replaced with the underlying protocol that implements it.
Testing and identifying failures arising from dependencies that either don't realize the ideal functionality or are subtley different from the assumptions expected by the application is an important capabality for the viability of our proposition in this paper. 


\todo{keep or don't keep this, we might have this in an earlier section}
Simple examples like 2-party computation, desired properties can often be deeply interconncted such that specifying them as a list of satisfiable assertion is difficult. 
In the simplest case of two parties computing a function, notions of correctness and secrecy are connected, for example, to the choice of function being computed, what an adversary can learn about the other party's inputs before choosing its own, or what distributions do the adversary's input or the protocol's output exhibit. 
In more complex protocols we desired the analysis of properties that define notions of fairness or input availability.
Specifying these properties in a laundry list of properties can be cumbersome and error-prone, and the ideal functionality model allows expressing arbitrary properties as a computational unit.
Rather than proving specific assertions hold, UC defines security in relation to an idealized version that exhibits the desired properties implicitly.
A coin flipping protocols, references throughout this work, is a core subcomponent of many asynchronous distributed protocols, and is a useful case study for examing our implementation.

\subsection{Flipping Fairness}
The first example we study is building a lottery protocol off a two-party coin tossing protocol.
Protocols like Blum's allow two mutually distrustful parties to flip a coin over the telephone protocol, and ensures an unbiased coin only when the adversary does not prematurely abort the protocol.
Lottery protocols are more modern uses of coin flipping that highlight the influence adversaries can have in determining the output.
In most agreement protocols, the value of a coin flip isn't as important as long as all parties observe the same outcome.
In distributed protocols that operate with financial incentives, the lottery being the simplest example of one, this property becomes critically important. 
The example presented here \todo{finish}
%Flipping a coin involving several parties is a common coin protocol: a more complex and still widely used primitive in modern agreement protocols.
%They are used heavily even in modern agreement protocols that operate under the strong common coin assumption: in some round all parties observe the same coin flip. 
%This relatively straightforward example protocol exhibits many properties that, traditionally, UC is adept at modelling and analyzing like fairness, input availability, and adversarial influence on outcomes.
%In this section we use this example to demonstrate that our implementation can express such a protocol, allow for analysis of properties like fairness and input availabiltiy, and detect failrue accross composition. \todo{this last sentence needs some work, maybe mention that we compose with the lottery here?}.

\paragraph{The Ideal Coin Flip}
In Figure~\ref{fig:fflip} we show the ideal functionality for a coin flip. 
The coin flip specifies that both parties but initiate the flip, and that the adversary can have no influence on the bias of the output bit.
Unlike the eventual delivery guarantees we discuss for async protocol, the functionality allows the adversary to decide which of the two parties, if any, receive the result.
This means that $\F_\m{flip}$ allows protocols that are \emph{not fair}: they do no guarantee that if one party receives output all parties eventually receive the output. 
\begin{figure}
\centering
\begin{minipage}{0.5\textwidth}
\begin{bbox}[title={Functionality $\F_\m{flip}(S,R)$}]
~
\begin{itemize}[leftmargin=*]
\item[--] on \inmsg{init} from $S$ or $R$, if first \m{init} send back \m{Ok} otherwise, generate and store the coin flip $b \xleftarrow{\$} \{0,1\}$ and send back \m{Ok}. Then,
\item[--] on the first \inmsg{deliver}{$S$} from \A send $b$ to $S$
\item[--] on the first \inmsg{deliver}{$R$} from \A send $b$ to $R$
\end{itemize}
\end{bbox}
\end{minipage}

\caption{Ideal coin flip without fairness.}
\label{fig:fflip}
\end{figure}
\todo{It is a known result that no coin flip protocol with n/2 corruptions is unbiased, what do say about that?}

\paragraph{A Lottery from Coin Tossing}

Building a lottery from an ideal functionality that gives you an unbiased two-party coin flip is a straightforward excercise and an intuitive choice for a programmer. 
A lottery between $n$ participants is an example of a distributed protocol increasingly common in blockchain systems where financial incentives are involved.
Unlike agreement protocols where only safety is required, lotteries are less tolerant of protocol that aren't fair or protocols where the adverasary can excert significant influence on the outcome. 
The financial incentives involved in the protocol require development frameworks that can express and analyze these properties.
Importantly, modular design by relying on software packages is necessary, and UC allows the abstraction to be represented by ideal functionalities.
In the same way that theoretical definitionsr rely on ideal functionalities for assumptions such as authenticated communication, implementation with ideal functionalities is an important feature.
Simply allowing design in this way isn't meaningful without the ability to analyze protocols across composition and the replacement of ideal functionalities with protocols that attempt to realize them.
So far we can apply analysis techniques to prove emulation, but doing so across composition remains to be validated.

The first and primary property that we desire from the lottery is that the adverasry can not bias the output of the lottery.
If ther eare $n$ participants then party $p_i$ is chosen with probability $\frac{1}{n}$.
Furthermore, we define the standard $\frac{n}{3}$ fault model 
\todo{iron out the right lottery protocol}

UC tells us that we can compose and arbitrary number of instances of \Fflip in order to realize a lottery.
At a high level, our protocol \prot{lotto}, flips $\log n$ pair-wise coins in order to choose one of $n$ parties as the winner. 
\todo{should the lottery property only be that the prob of win is 1/(|honest| + 1)? because we don't care which adverasry wins?}

\begin{figure}
\centering
\begin{minipage}{0.5\textwidth}
\begin{bbox}[title={Functionality $\F_\m{lottery}(\mathcal{P} = p_1,...p_n)$}]
~
Let \honest by the set of honest parties, \crupt the corrupt ones, and $\mathcal{L}$ the set of losers.

Loop with $\mathcal{P} := \mathcal{P} \setminus \mathcal{L}$ until $\mathcal{P} = \{p_i\}$:

\hspace*{0.5cm}%
\begin{minipage}{0.8\textwidth}
\begin{itemize}[leftmargin=*]
\item[--] wait to receive \inmsg{init} from all $p_i \in \honest$, send \m{ok} back in the mean time
\item[--] wait to receive $\log |\mathcal{P}|$ tuples $(p_i, p_j)$ from $P \in \mathcal{P}$ and ($P = p_i \vee P = p_j$) identifying the individual coin flips. If one party is in more than one tuple, halt.
\item[--] For each $(p_j, p_i)$ (define $(p_0, p_1) := (p_j, p_i)$):

\quad generate $b \xleftarrow{\$} \{0, 1\}$

\quad add $p_{! b}$ to $\mathcal{L}$, and leak $(\msf{winner}, p_b)$ to \A
\end{itemize}
\end{minipage}%

When $\mathcal{P} = \{p_i\}$, leak $(\msf{winner}, p_i)$ to \A and output it to $p_i$.

\end{bbox}
\end{minipage}

\caption{The lottery ideal functionality.}
\label{fig:flotto}
\end{figure}


\subsection{Common Coins}
Many agreement protocols rely on common coins to introduce shared randomness into protocol to allow protocol parties to make common decisions without adversarial influence.
Common coins definitions vary in their guarantees of output fairness, corruption threshold, and adversarial influence on the output bit.
The protocols that use common coins are prone to using them incorrectly, as we see in the case of the original MMR agreement protocol where the common coin assumption was shown in \cite{formalbyz} to be to weak and permitted a liveness fault.
As we describe in the lottery protocol above, and in mode complex distributed protocol, 

The strongest assumtpion is a perfect common coin


%\section{Simulator Composition} \label{app:simcomp}
%Here we show in Figure~\ref{lst:simcomp} the steps involved in realizing the simulator for Theorem~\ref{thm:singlecomp}.
The simulator is straightforward and must only connect, internally, the two simulators $\Sim_\pi$ and $\Sim_\rho$.
Below we show how the simulator is initialized.
Note the internal channels created, especially on lines 18 and 22 that mold output from $\Sim_\pi$ to look like input from \Z to $\Sim_\rho$ and output from $\Sim_\rho$ (to \Z) to appear as coming from either \MX or \F to $\Sim_\pi$.

\begin{figure*}[h]
\begin{lstlisting}[basicstyle=\scriptsize\BeraMonottFamily, frame=single, mathescape, numbers=left, xleftmargin=2em, xrightmargin=2em]
$\nproc$ sim_compose[K,K1][spi,srho][z2pi,z2piP,z2piF,z2rho,z2rhoP,z2rhoF,piA2F,piA2P,rhoA2P,rhoA2F,pi2zP,pi2zF,rho2zP,
    rho2zF]
  {z2pin,pi2Fn,pi2Pn,pi2zn,F2pin,P2pin,z2rhon,rho2Pn,rho2Fn,rho2zn,P2rhon,F2rhon,}: 
  (k: Int), (rng: [Bit]), (sid: SID[spi]), (crupt: [PID]), 
  ($\$$z2a: z2a[K][z2piP][z2piF]{z2pin}), ($\$$a2z: a2z[K][pi2zP][pi2zF]{pi2zn}),
  ($\$$a2p: a2p[K][rho2P]{rho2Pn}), ($\$$p2a: p2a[K][P2rho]{P2rhon}),
  ($\$$a2f: a2f[K][rho2F]{rho2Fn}), ($\$$f2a: f2a[K][F2rho]{F2rhon}) |- ($\$$ch: 1) =
{
  $\$$z2a' <- channel_init[K1][z2a[z2piP][z2piF]]{z2pin} ; 
  $\$$a2z' <- channel_init[K1][a2z[pi2zP][pi2zF]]{pi2zn} ;
  $\$$a2f' <- channel_init[K1][a2f[rhoA2F]]{rho2Fn} ;
  $\$$f2a' <- channel_init[K1][f2a[rhoF2A]]{F2rhon} ;
  $\$$a2p' <- channel_init[K1][a2p[rhoA2P]]{rho2Pn} ;
  $\$$p2a' <- channel_init[K1][p2z[rhoP2A]]{P2rhon} ;

  $\$$pi2rhoP <- channel_init[K1][a2p[z2rhoP]]{pi2Pn} ;
  $\$$pi2rhoF <- channel_init[K1][a2f[z2rhoF]]{pi2Fn} ;
  $\$$pi2rhoPF <- wrapz2a_init[K1]{z2rhon} $\leftarrow$ $\$$pi2rhoP $\$$pi2rhoF ;
  
  $\$$Prho2pi <- channel_init[K1][p2z[rhoP2A]]{rho2Pn} ;
  $\$$Frho2pi <- channel_init[K1][f2a[rhoF2A]]{rho2Fn} ;
  $\$$PFrho2pi <- unwrapa2z_init[K1]{rho2zn} <- $\$$Prho2pi $\$$Frho2pi

  $\$$chpi <- sim_pi[K1] <- k rng sid crupt $\$$z2a' $\$$a2z' $\$$pi2rhoP $\$$Prho2pi 
    $\$$pi2rhoF $\$$Frho2pi 
  $\$$chrho <- sim_rho[K1] <- k rng sid crupt $\$$pi2rhoPF $\$$PFrho2pi $\$$a2f' $\$$f2a'
    $\$$a2p' $\$$p2a'

  $\$$c <- sim_compose_from_ext[K,K1][spi,srho][z2pi,z2piP,z2piF,z2rho,z2rhoP,z2rhoF,piA2F,piA2P,rhoA2P,rhoA2F,pi2zP,
    pi2zF,rho2zP,rho2zF]
  {z2pin,pi2Fn,pi2Pn,pi2zn,F2pin,P2pin,z2rhon,rho2Pn,rho2Fn,rho2zn,P2rhon,F2rhon} $\leftarrow$ 
    k rng sid crupt $\$$z2a $\$$a2z $\$$a2p $\$$p2a $\$$a2f $\$$f2a $\$$z2a' $\$$a2z' 
	$\$$a2f' $\$$f2a' $\$$a2p' $\$$p2a' $\$$pi2rhoPF $\$$pi2rhoP $\$$pi2rhoF $\$$Prho2pi
	$\$$Frho2pi $\$$PFrho2pi 
}

$\nproc$ sim_compose_from_ext[K,K1][spi,srho][z2pi,z2piP,z2piF,z2rho,z2rhoP,z2rhoF,piA2F,piA2P,rhoA2P,rhoA2F,pi2zP,pi2zF,
    rho2zP,rho2zF]
  {z2pin,pi2Fn,pi2Pn,pi2zn,F2pin,P2pin,z2rhon,rho2Pn,rho2Fn,rho2zn,P2rhon,F2rhon,}: 
  (k: Int), (rng: [Bit]), (sid: SID[spi]), (crupt: [PID]), 
  ($\$$z_to_a: z2a[K][z2piP][z2piF]{z2pin}), ($\$$a_to_z: a2z[K][pi2zP][pi2zF]{pi2zn}),
  ($\$$a_to_p: a2p[K][rho2P]{rho2Pn}), ($\$$p_to_a: p2a[K][P2rho]{P2rhon})
  ($\$$a_to_f: a2f[K][rho2F]{rho2Fn}), ($\$$f_to_a: f2a[K][F2rho]{F2rhon}),
  ($\$$z2a': z2a[K1][z2piP][z2piF]{z2pin}), ($\$$a2z': a2z[K1][pi2zP]][pi2zF]{pi2zn}),
  $\tg{...}$
  ($\$$p2a': p2a[K1][P2rho]{P2rhon}), ($\$$f2a': f2a[K1][F2rho]{F2rhon}),
  $\tg{...}$ |- ($\$$ch: 1) =
{
  $\tg{(* the simulator gives Z input to Spi and output from that to Srho *)}$
	$\nmatch$ $\$$z_to_a, $\$$p_to_a, $\$$f_to_a (
		$\tg{(* these are forwarded to spi without change *)}$
		Z2A2P,*,* =>
			pid,m = $\nrecv$ $\$$z_to_a; $\nget$ {z2pin} $\$$z_to_a ;
			$\nwithdraw$ K K1 z2pin ;
			$\$$z2a'.Z2A2P ; $\nsend$ $\$$z2a' (pid,m) ; $\npay$ {K1} z2pin $\$$z2a';
			$\tg{(* go to sim\_compose\_from\_spi *)}$
		Z2A2F,*,* =>
			m = $\nrecv$ $\$$z_to_a ; $\nget$ {z2pin} $\$$z_to_a ;
			$\nwithdraw$ K K1 z2spi ;
			$\$$z2a'.Z2A2F ; $\nsend$ $\$$z2a' m ; $\npay$ {K1} z2pin $\$$z2a';
			$\tg{(* go to sim\_compose\_from\_spi *)}$
		*,P2A,* => $\tg{(* this goes to s\_rho *)}$
			pid,m = $\nrecv$ $\$$p_to_a ; $\nget$ {0} $\$$p_to_a ;
			$\$$p2a'.P2A ; $\nsend$ $\$$p2a' (pid,m) ;
			$\tg{(* go to sim\_compose\_from\_srho *)}$
		*,*,F2A => $\tg{(* identical case *)}$
	)
  $\tg{(* now that we activate one of the simulators, case match on their outgoing channels *)}$

}
\end{lstlisting}
\caption{This shows how the single composition is initialized. The most complicated part is just creating all of the channels that will be used. The next process shows how external messages are read and routed to $\Sim_\pi$ or $\Sim_\rho$ depending on where the input comes from.}
\label{lst:simcomp}
\end{figure*}
%  ... ($\$$z2a: comm[K][z2a]), ($\$$a2z: comm[K][a2z]), ($a2p: comm[K][a2p]),
%  ($p2a: comm[K][p2a]), ($f2a: comm[K][f2a]), ($a2f: comm[K][a2f) |- ($ch: 1) =


%\section{Proofs}
%\begin{theorem}\label{thm:bracha:sync}
Let $\Pi_{\msf{bracha}}$ and $\mathcal{F}_{\msf{bracha}}$ be $\overline{\mathcal{W}}_{\msf{sync}}$-subroutine respecting protocols. 
Then $\Pi_{\msf{bracha}}$ EUC-realizes $\mathcal{F}_{\msf{bracha}}$ in the $\mathcal{F}^{n(n-1)}_{\msf{sync}}$-hybrid world in the static corruptions ($t < \frac{n}{3}$) model for any $\overline{\mathcal{W}}_{\msf{sync}}$-externally constrained, weakly-balanced environment $\mathcal{Z}$. We have:

$$\textsc{EXEC}^{\Wsync}_{\mathcal{F}_{\msf{bracha}}, \Ssyncbracha, \Environment} \approx \textsc{EXEC}^{\Wsync}_{\pi_{\msf{bracha}}, \Dadv, \Environment}$$
\end{theorem}

\textit{Proof of Theorem \ref{thm:bracha:sync}.}

We first construct the simulator \Ssyncbracha. 
The full pseudo-code of \Ssyncbracha~ is shown in Figure~\ref{fig:sim:bracha:sync} however, for brevity, we describe it here.

\Ssyncbracha~ runs a full simulation of the real world internally with its own copy of the the wrapper, \Wsync'. It also maintains a copy the \msf{runqueue} in \Wsync and its \msf{delay}. 
It also requests leaks from \Wsync every time is it is activted with input from the wrapper (\Advance), or the enviromnent (corrupt input, \Delay, \Exec).

\heading{Simulating Honest Input to \Fbracha.}
The honest input to \Fbracha is leaked to \Ssyncbracha~ when requesting leaks from \Wsync. 
It passes the input to it's internally simulated dealer $\mathcal{D}'$ and increments its internal \msf{delay}.

\heading{Simulating Other Codeblocks.}
For all other scheduled code blocks (from other protocol sessions), it adds those corresponding codeblocks to \Wsync' and updates its copy of \msf{runqueue} and \msf{delay}.

\heading{Simulating Corrupt Input.} When the simulator receives corrupt input for some party $\pi_i$ it simply passes the input to its internal copy of it, $\pi_i'$. In the case of Bracha, only the dealer submit input to the ideal functionality, though \Ssyncbracha~ waits to give input to the corrupted dealer until the simulation commits to an input.

\heading{Simulating a \msf{Poll} Call by the Environment.}
For every \msf{Poll} to \Wsync that does not execute a codeblock (it's delay is positive), \Ssyncbracha~ is activated with the \msf{poll} message.
The simulator updates its internal $\msf{delay}$ and adds 1 unit of delay to \Wsync if it is 0 (does not want the environment to control when the next codeblock is executed).
Then it simulates \msf{poll} to \Wsync' and stores any new leaks from it.

\heading{Handling Codeblocks in the Simulation.}
When a codeblock executes in the simulation \Ssyncbracha~ executes the corresponding codeblock in the ideal world if it was schedules by a protocol session other than the challenge protocol.
When a simulated honest challenge protocol party, $\pi_i'$, outputs a value $v$, i.e. commits to a value, \Ssyncbracha:
\begin{itemize}
\item If the dealer is corrupt, \Ssyncbracha~ gives input $v$ to the corrupt ideal world dealer if this is the first such output. It then executes the outer codeblock from \Fbracha. It then executes the codeblock that outputs $v$ to $\pi_i$ in the ideal world.
\item If the dealer is honest, the codeblocks already exist and it executes the one which delivers output to $\pi_i$.
\end{itemize}

\heading{Simulating \Exec Calls.}
When activated by \Environment with an \Exec call, it forwards the call to it's internal simulated adversary. If a codeblock is executed, it responds as specified above. 

\heading{Handling Corrupt Simulated Output.} 
\Ssyncbracha outputs all corrupt party output from the simulated adversary to the environment.


%\begin{lemma}\label{lem:enoughimport}

The \textit{weakly-balanced} relaxation guarantees a simulator \mathc{S} will always have sufficient import to delay ideal world codeblocks \textit{at least} as long as the corresponding real world code blocks.

\end{lemma}

\textit{Proof of Lemma \ref{lem:enoughimport}.}

The crux of the argument lies in ensuring that, even when minimal import is provided to the adversary in the ideal world, it has sufficient import to meaningfully delay ideal world codeblocks to ensure simulatability. 
Therefore we set up an extreme scenario:
\begin{itemize}
\item A protocol $\pi$ which schedules $n$ codeblocks and outputs to the environment in the last codeblock. An ideal protocol $\phi$ schedules a single codeblock that outputs the same message to the environment.
\item The simulator must be able to ensure that the ideal world codeblock can execute at the same time as the final codeblock in $\pi$.
\end{itemize}

In the real world with dummy adversary $\mathcal{D}$, $\mathcal{Z}$ needs to \Advance \Wasync $n+1$ times in order to force execute all codeblocks.
Scheduling $n$ codeblocks requires at least $n$ unites of import on the part of the parties, and the adversary in both worlds also receives at least $n$ units of import.
Therefore, in order to ensure indistinguishability, the simulator must be able to delay the ideal world \Wasync $n$, which it can plainly do \footnote{The adverasry is also given $k$, the security parameter, amount of import at the start of execution ensuring that a polynomial time simulator is still polynomial under this new definition.}.



\pagebreak

%\begin{figure}
%\begin{bbox}[title={\textbf{Functionality} $\F_\msf{Async} (P_i, P_j)$}]

\OnInput \inmsg{\textsc{send}}{\msf{msg}} from $P_i$:
	
	\begin{renumerate}

		\item {\bf Eventually} send \msf{msg} to $P_j$

	\end{renumerate}
	
\end{bbox}

%\end{figure}

%\begin{figure}
%\begin{bbox}[title={Functionality $\mathcal{F}_\msf{RBC} (\mathcal{D}, \mathcal{P})$}]

\OnInput \inmsg{\textsc{input}}{$m$}{$n(4n+1) \token$} from $\mathcal{D}$:

	\begin{renumerate}

		\item {\bf Leak} \inmsg{\textsc{input}}{$m$}
		
		\item {\bf For} each $P_i \in \mathcal{P}$:
		\begin{renumerate}

			\item \Send \inmsg{\textsc{schedule}}{\textsc{send}}{($m$, $P_i$)} $\rightarrow \mathcal{W}_\msf{async}$		

		\end{renumerate}
	\end{renumerate}

{\bf Code} $\textsc{send}(m, P_i)$:
	\begin{renumerate}

		\item \Send $m \rightarrow P_i$
	\end{renumerate}
\end{bbox}

%\end{figure}

%\begin{figure}
%\begin{bbox}[title={Protocol $\Pi_\msf{Bracha} (\mathcal{D}, \mathcal{P})$}]

	Define $\msf{BQ} \leftarrow \frac{n+t}{2}$

	{\bf As Dealer $\mathcal{D}$}:

	\OnInput $m$ from $\mathcal{Z}$:

	\begin{renumerate}

		\item {\bf For} each $P_i \in \mathcal{P}$:
		\begin{renumerate}

			\item \Send \inmsg{\textsc{val}}{m} $\rightarrow P_i$		

		\end{renumerate}


	\end{renumerate}

	As any party $P_i$:

	\OnInput \inmsg{\textsc{val}}{$m$} from $\mathcal{D}$ or

	\parbox{\widthof{\OnInput}}~\inmsg{\textsc{echo}}{$m$} from \msf{BQ} parties or

	\parbox{\widthof{\OnInput}}~\inmsg{\textsc{ready}}{$m$} from $\msf{BQ}-\msf{t}$ parties:

	\begin{renumerate}
	
		\item {\bf For} each $P_i \in \mathcal{P}$:

			\begin{renumerate}

				\item \Send \inmsg{\textsc{echo}}{$m$} to $P_i$

			\end{renumerate}
	\end{renumerate}

	\OnInput \inmsg{\textsc{echo}}{$m$} from \msf{BQ} parties or

	\parbox{\widthof{\OnInput}}~\inmsg{\textsc{ready}}{$m$} from $\msf{BQ}-\msf{t}$ parties:

	\begin{renumerate}
	
		\item {\bf For} each $P_i \in \mathcal{P}$:

			\begin{renumerate}

				\item \Send \inmsg{\textsc{ready}}{$m$} to $P_i$

			\end{renumerate}
	\end{renumerate}

	\OnInput \inmsg{\textsc{ready}}{$m$} from \msf{BQ} parties:
		
	\begin{renumerate}

		\item Output $m$	

	\end{renumerate}
\end{bbox}

%\end{figure}
%
%\begin{figure}
%
\begin{bbox}[title={Simulator $\mathcal{S}_\msf{Bracha} (\mathcal{D}, \mathcal{P}, \Delta)$}]

Simulate real world parties $P_1',...,P_n'$ and the simulated dealer $\mathcal{D}'$.

Init $\msf{ideal\_queue}$ := $\emptyset$, $\msf{ideal\_delay}$ := $0$, $\msf{sim\_leaks}$ := $\emptyset$

$\msf{pid\_to\_idx} := \{\}$

\vspace{2mm} \hrule \vspace{2mm}

\underline{On every activation:} \vspace{2mm}

\begin{renumerate}
	\item $\msf{leaks} \leftarrow$ \{\Send (\textsc{get-leaks}) $\rightarrow \mathcal{W}_\msf{sync}$\}
	
	\item {\bf For} $\msf{leak} \in \msf{leaks}$:
	\begin{renumerate}
		\item {\bf Match} $\msf{leak}$:
			\begin{renumerate}
				
				\item {\bf Case} (\textsc{input}, $m$) from $\F_\msf{RBC}$:

				\quad Simulate (\textsc{input}, $m$, $n(4n+1) \token$) $\rightarrow \mathcal{D}'$ 

				\item {\bf Case} (\textsc{schedule}, $P_i$, \msf{rnd}, \msf{idx}) from $\F_\msf{RBC}$:

					\quad $\msf{pid\_to\_idx}[P_i] = \msf{(rnd, idx)}$

					\quad $\msf{ideal\_delay} \pluseq 1$

				\item {\bf Case} \msf{msg} from $\F$:
					
					\quad Simulate $\F'$ leaking \msf{msg}
			\end{renumerate}
		\end{renumerate}
	%\item \Send $leaks \rightarrow \mathcal{Z}$
	\end{renumerate}


\OnInput \inmsg{\textsc{get-leaks}} from $\mathcal{Z}$:
	\begin{renumerate}
	% \item \Send $simleaks \rightarrow \mathcal{Z}$
	\item $\msf{leaks} \leftarrow$ \{Simulate (\textsc{get-leaks}) $\rightarrow \mathcal{W}'_\msf{sync}$\}
	
	\item \Send \msf{leaks} $\rightarrow \mathcal{Z}$
	\end{renumerate}

\OnInput \inmsg{\textsc{poll}} from $\mathcal{W}_\msf{sync}$:
	\begin{renumerate}
	\item Execute \msf{Poll}
	\end{renumerate}

\OnInput \inmsg{\textsc{delay}}{$d \token$} from $\mathcal{Z}$:
	\begin{renumerate}
	\item Simulate $(\textsc{delay}, d \token) \rightarrow \mathcal{W}_\msf{sync}'$

%	\item \Send $(\textsc{delay}, d \token) \rightarrow \mathcal{W}_{sync}$

%	\item $idealdelay \pluseq d$

	\item \Send $\textsc{OK} \rightarrow \mathcal{Z}$
	\end{renumerate}

\OnInput \inmsg{\textsc{exec}}{\msf{rnd}}{\msf{idx}} from $\mathcal{W}_\msf{sync}$:
	\begin{renumerate}
	\item Simulate $(\textsc{exec}, \msf{rnd}, \msf{idx}) \rightarrow \mathcal{W}_\msf{sync}'$

	%\item $msg \leftarrow$ wait for output from some simulated ITM.

	\item If output $m$ from simulated party $P_i'$:

%		\quad Call $\msf{SimGeteaks}$

		\quad Call \msf{SimPartyOutput}($m$, $P_i'$)

	\item Else if output $m$ from simulated adversary $\mathcal{A}'$:

		\quad \Send $m \rightarrow \mathcal{Z}$

%	\item Execute $\msf{SimGetLeaks}$
%
%	\item Match $msg$ with:
%
%-- \OnInput (m) from $P_i$':
%  
%	\qquad call $\msf{SimPartyOutput}(m, P_i')$
%
%-- \OnInput (m) from $\mathcal{A}'$:
%
%	\qquad \Send $m \rightarrow \mathcal{Z}$
%
	\end{renumerate}

\end{bbox}

%\end{figure}
%
%\begin{figure}
%\begin{subfigure}{\columnwidth}
%
\begin{bbox}[title={Algorithm $\msf{Poll}$}]

\begin{renumerate}

  	\item $\msf{ideal\_delay} \minuseq 1$
  	
  	\item If $\msf{ideal\_delay} = 0$:
  	 
  		\quad \Send $(\textsc{delay}, 1 \token) \rightarrow \mathcal{W}_\msf{sync}$

  		\quad $\msf{ideal\_delay} = 1$

  	\item \Send $(\textsc{poll},) \rightarrow \mathcal{W}_\msf{sync}'$
 
  	\item If output $m$ from simulated party $P_i'$:

%			\quad Call $\msf{SimGetLeaks}$

			\quad Call \msf{SimPartyOutput}($m$, $P_i'$)
		
		Else if output $m$ from simulated adversary $\mathcal{A}'$:

			\quad \Send $m \rightarrow \mathcal{Z}$

\end{renumerate}

\end{bbox}

%\end{subfigure}
%%\begin{subfigure}{\columnwidth}
%%\input{figures/algosimgetleaks}
%%\end{subfigure}
%\begin{subfigure}{\columnwidth}
%
\begin{bbox}[title={Algorithm $\msf{SimPartyOutput}(m, P_i')$}]

	\begin{renumerate}
		\item If $P_i'$ simulates a dishonest $P_i$:
		\begin{renumerate}	

			\item Send $(m, P_i) \rightarrow \mathcal{Z}$
		\end{renumerate}

		\item Else, if no dealer input to $\mathcal{F}_\msf{RBC}$:
		\begin{renumerate}

			\item \Assert $\mathcal{D}$ is dishonest

			\item $\msf{OK} \leftarrow \{ \Send (\textsc{input}, m) \rightarrow \mathcal{D} \}$

			\item $\msf{leaks} \leftarrow \{ \Send (\textsc{getleaks}) \rightarrow \mathcal{W}_\msf{sync}\}$
		
			\item {\bf For} $\msf{leak} \in \msf{leaks}$, {\bf Match} $\msf{leak}$:
						\begin{renumerate}
							\item {\bf Case} (\textsc{schedule}, $P_i$, \msf{rnd}, \msf{idx}) from $\F_\msf{RBC}$:

								\quad $\msf{pid\_to\_idx}[P_i] = \msf{(rnd, idx)}$

								\quad $\msf{ideal\_delay} \pluseq 1$

							\item {\bf Case} \msf{msg} from $\F$:
					
								\quad Simulate $\F'$ leaking \msf{msg}

						\end{renumerate}
		\end{renumerate}
		\item $\msf{rnd}, \msf{idx} \leftarrow \text{pop } \msf{pid\_to\_idx}[P_i]$

		\item Update $\msf{pid\_to\_idx}$ indices like $\mathcal{W}_\msf{sync}$

		\item $\Send (\msf{exec}, \msf{rnd}, \msf{idx}) \rightarrow \mathcal{W}_\msf{sync}$
		
	\end{renumerate}

\end{bbox}

%\end{subfigure}
%\end{figure}
%
%
%\begin{bbox}[title={Subroutine {\bf ExecuteWithTimeout} $(c, T)$}]

$t \leftarrow$ current round \# in \Wsync

\Send $(\Schedule, c, T-t) \color{red} $1 \token$ \color{black} \rightarrow \Wsync$

\OnInput \inmsg{exec} from $\F_{\msf{State}}$:

\quad $t \leftarrow$ current round \# from \Wsync
\begin{renumerate}
\item \If $t < T$:
	
	\qquad \Send $(\Schedule, c, T-t) \color{red} 1 \token \color{black} \rightarrow \Wsync$

\item \Else:
	
	\qquad Execute $c$

\end{renumerate}

\end{bbox}


\begin{bbox}[title={Functionality $\F_{\msf{State}} (\Delta, U, C, \mathcal{P} = \{P_1,...,P_n\}$}]

Initialize $\msf{state} = \emptyset, \msf{buf} = [], \msf{aux\_in} = [], \msf{ptr} = 0$

$r := 0$

\vspace{2mm} \hrule \vspace{2mm}


\OnInput \inmsg{input}{v} from $P_i$:

\quad \If first input received from $P_i$ in $r$:

\begin{renumerate}
	\item \If 
\end{renumerate}

\end{bbox}



\end{document}
