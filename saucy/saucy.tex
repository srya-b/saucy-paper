\documentclass[conference]{IEEEtran}
\IEEEoverridecommandlockouts
% The preceding line is only needed to identify funding in the first footnote. If that is unneeded, please comment it out.
\usepackage[dvipsnames]{xcolor}
\usepackage{cite}
\usepackage[most]{tcolorbox}
\usepackage{amsmath,amssymb,amsfonts,amsthm}
\usepackage{algorithmic}
\usepackage{graphicx}
\usepackage{subcaption}
\usepackage{textcomp}
\usepackage{mathtools}
\usepackage[shortlabels]{enumitem}
\usepackage{xspace}
\usepackage{listings}
\usepackage{fontenc}


%% PL packages
\usepackage{stmaryrd} 
\usepackage{proof}
\usepackage{mathpartir}
\usepackage{color}
\usepackage{xstring}

\def\BibTeX{{\rm B\kern-.05em{\sc i\kern-.025em b}\kern-.08em
    T\kern-.1667em\lower.7ex\hbox{E}\kern-.125emX}}
\definecolor{darkblue}{rgb}{0,0,0.5}
\definecolor{darkgreen}{rgb}{0,0.3,0}
\definecolor{darkpink}{rgb}{0.4,0,0.3}
\definecolor{graygreen}{rgb}{0.3,0.5,0.3}
\definecolor{grayblue}{rgb}{0.2,0.2,0.6}
\definecolor{grayred}{rgb}{0.5,0.2,0.2}

\lstset{
  backgroundcolor=\color{white},     % choose the background color; you must add \usepackage{color} or \usepackage{xcolor}; should come as last argument
  % identifierstyle=\color{red},
  basicstyle=\footnotesize\ttfamily\upshape,      % the size of the fonts that are used for the code
  breakatwhitespace=false,              % sets if automatic breaks should only happen at whitespace
  breaklines=false,                     % sets automatic line breaking
  captionpos=b,                         % sets the caption-position to bottom
  abovecaptionskip=-3 mm,
  commentstyle=\itshape\color{graygreen}, % comment style
  % escapeinside={(:}{:)},             % if you want to add LaTeX within your code
  escapechar={!},
  % extendedchars=true,              % lets you use non-ASCII characters; for 8-bits encodings only, does not work with UTF-8
  % firstnumber=1000,                % start line enumeration with line 1000
  % frame=tb,                        % adds a frame around the code
  % keepspaces=true,                 % keeps spaces in text, useful for keeping indentation of code (possibly needs columns=flexible)
  keywordstyle=\color{blue},     % keyword style
  language=Haskell,                  % the language of the code
  morekeywords={ forMseq_, generate, forAllM, fork, forever }
  %morekeywords={ Set, Tree, Leaf, Node, Applicative, fmap, liftA2, bimap, foldMap
  %             , traverse, mappend, pure, Foldable, Traversable, zero, one
  %             , Semiring, Semigroup, NonEmpty, sconcat, TSet,
  %             , SimplicialSet, TSimplicialSet, Graph, TGraph, LGraph
  %             , Map, IsString, fromString },
  deletekeywords={instance, data, where, class, filter, type, insert, delete, union, map},      % if you want to delete keywords from the given language
  emph={data, class, instance, where, type},
  emphstyle=\color{darkpink},
  numbers=none,                      % where to put the line-numbers; possible values are (none, left, right)
  % numbersep=5pt,                   % how far the line-numbers are from the code
  % numberstyle=\tiny\color{mygray}, % the style that is used for the line-numbers
  % rulecolor=\color{black},         % if not set, the frame-color may be changed on line-breaks within not-black text (e.g. comments (green here))
  % showspaces=false,                % show spaces everywhere adding particular underscores; it overrides 'showstringspaces'
  % showstringspaces=false,          % underline spaces within strings only
  % showtabs=false,                  % show tabs within strings adding particular underscores
  % stepnumber=2,                    % the step between two line-numbers. If it's 1, each line will be numbered
  stringstyle=\color{grayred},     % string literal style
  % tabsize=2,                       % sets default tabsize to 2 spaces
  % title=\lstname                   % show the filename of files included with \lstinputlisting; also try caption instead of title
  xleftmargin=10pt,
  aboveskip=8pt,
  belowskip=4pt
}

\begin{document}

\usetikzlibrary{matrix, arrows.meta, calc, positioning}
\tikzset{myarrow/.style={-Latex, rounded corners},}

\definecolor{vert}{RGB}{0,181,0}
\definecolor{oran}{RGB}{223,74,0}
\definecolor{viol}{RGB}{134,0,175}
\definecolor{roug}{RGB}{215,15,0}
\definecolor{bb}{RGB}{0,0,0}
\definecolor{gg}{RGB}{220,220,220}

\newtcolorbox[auto counter]{bbox}[2][]{%
    colback=white,
    colframe=bb,
    %colbacktitle=white!90!roug,
	colbacktitle=white!40!gg,
    coltitle=black,
    fonttitle=\footnotesize\bfseries, 
	fontupper=\footnotesize,
	fontlower=\footnotesize,
    enhanced,
    attach boxed title to top left={yshift=-2mm, xshift=0.5cm},%
    #1,% For possible options
}
\title{SAUCy: Super Awesome Universal ComposabilitY }

\newcommand{\mc}[1]{\ensuremath{\mathcal{#1}}}
\newcommand{\msf}[1]{\ensuremath{{\mathsf {#1}}}}
\newcommand{\mathc}[1]{\ensuremath{\mathcal{#1}}}
\newcommand{\tsc}[1]{\textsc{#1}}
\newcommand{\f}[1]{\ensuremath{\mathcal{#1}}\xspace}
\newcommand{\F}{\f{F}}
\newcommand{\PI}{\ensuremath{\pi}\xspace}
\newcommand{\RHO}{\ensuremath{\rho}\xspace}
\newcommand{\achan}{\ensuremath{\F_{\msf{achan}}^{p_r,p_s}}}
%\newcommand{\C}{\mathcal{C}}
\newcommand{\con}[1]{\msf{Contract_{#1}}}
%\newcommand{\Fsync}[2]{\ensuremath{\F_{\msf{sync},#1,#2}}}
\newcommand{\Fsync}[2]{\ensuremath{\F_{\msf{BD-SEC}}(#1,#2)}}
\newcommand{\Fchan}[2]{\ensuremath{\F_{\msf{chan}}(#1,#2)}}
\newcommand{\Fbdsec}{\ensuremath{\F_{\msf{BD-SEC}}^{\delta,\ell}}}
\newcommand{\Fbc}{\ensuremath{\F_{\msf{broadcast}}}}
\newcommand{\Fsfe}{\ensuremath{\F_{\msf{SFE}}}}
\newcommand{\Fstate}{\ensuremath{\F_{\msf{state}}}}
\newcommand{\Fclock}{\ensuremath{\F_{\msf{clock}}}}
\newcommand{\Frbc}{\ensuremath{\F_{\msf{rbc}}}}
\newcommand{\Fpay}{\ensuremath{\F_{\msf{pay}}}}
\newcommand{\Fcom}{\ensuremath{\F_{\msf{com}}}\xspace}
\newcommand{\Fauth}{\ensuremath{\F_{\msf{auth}}}\xspace}
\newcommand{\Fflip}{\ensuremath{\F_{\msf{coinflip}}}\xspace}
\newcommand{\Fro}{\ensuremath{\F_{\msf{RO}}}\xspace}
\renewcommand{\O}[1]{\ensuremath{\mathcal{O}(#1)}\xspace}
\newcommand{\kbits}{\ensuremath{\{0,1\}^k}\xspace}
\newcommand{\samplek}{\ensuremath{\xleftarrow{\$} \kbits}\xspace}
\newcommand{\Fsmc}{\ensuremath{\F_{\msf{SMC}}}\xspace}
\newcommand{\Fropp}{\ensuremath{\F_{\msf{P2P\hyp RO}}}\xspace}
\newcommand{\Gledger}{\ensuremath{\f{G}_{\msf{ledger}}}}
\newcommand{\Wsync}{\ensuremath{\mathcal{W}_{\msf{sync}}}}
\newcommand{\Wasync}{\ensuremath{\mathcal{W}_{\msf{async}}}}
\newcommand{\Ssyncbracha}{\ensuremath{\mathc{S}_{\msf{sbracha}}}}
\newcommand{\Fbracha}{\ensuremath{\mathcal{F}_{\msf{bracha}}}}
\newcommand{\Schedule}{\tsc{Schedule}}
\newcommand{\Delay}{\tsc{Delay}}
\newcommand{\Advance}{\tsc{Advance}}
\newcommand{\Exec}{\tsc{Exec}}
%\newcommand{\Adversary}{\ensuremath{\mathcal{A}}\xspace}
\newcommand{\A}{\ensuremath{\mathcal{A}}\xspace}
\newcommand{\DummyAdv}{\ensuremath{\mathcal{A}_\mathcal{D}}\xspace}
\newcommand{\DA}{\ensuremath{\A_\mathcal{D}}\xspace}
\newcommand{\Sim}{\ensuremath{\mathcal{S}}\xspace}
\newcommand{\SIM}[1]{\ensuremath{\mathcal{S}_{#1}}\xspace}
\newcommand{\simcom}{\SIM{\msf{com}}}
\newcommand{\cf}{\ensuremath{\mathcal{C}}\xspace}
\newcommand{\ID}[1]{\ensuremath{\mathcal{I}(#1)}\xspace}
%\newcommand{\Sim}[1][]{\ifthenelse{\equal{#1}{}}{\ensuremath{\Simulator}}{\ensuremath{\Simulator_{#1}}}}
\newcommand{\DS}{\SIM{D}\xspace}
%\newcommand{\Environment}{\ensuremath{\mathcal{Z}}\xspace}
\newcommand{\Z}{\ensuremath{\mathcal{Z}}\xspace}
\newcommand{\Partyi}{\ensuremath{P_i}}
\newcommand{\Partyj}{\ensuremath{P_j}}
\newcommand{\partywrapper}{multiplexer\xspace}
\newcommand{\pw}{\PI}
\newcommand{\fwrapper}{\todo{fwrappername}\xspace}

\newcommand{\dealer}{\ensuremath{\mathcal{D}}}
\newcommand{\globalf}[1]{\ensuremath{{\overline{\mathcal{#1}}}}}
\newcommand{\todo}[1]{\textcolor{Red}{todo: #1}}
\newcommand{\edict}{\{\}}
\newcommand{\lar}{\leftarrow}
\newcommand{\rar}{\rightarrow}
\newcommand{\Init}{{\bf \color{NavyBlue} Init}~}
\newcommand{\OnInput}{{\bf \textcolor{Black} On input}~}
\newcommand{\Allinputs}{{\bf \color{Cerulean} All other input~}}
\newcommand{\OnAdvInput}{{\bf \color{BrickRed} On input}~}
\newcommand{\heading}[1]{\textbf{#1}}
\newcommand{\Type}{\ensuremath{\yo{type}}}
\newcommand{\Stype}{\ensuremath{\yo{stype}}}
\newcommand{\bangf}{\ensuremath{!\F}}
\newcommand{\execuc}{\ensuremath{\msf{execUC}}}
\newcommand{\iexecuc}{\inline{execUC}}
\newcommand{\UC}[4]{\ensuremath{\execuc #1  #2  #3  #4}}
\newcommand{\idealP}{\ensuremath{\mathbbm{1}_d}\xspace}
%\newcommand{\prot}[1][]{\ifthenelse{\equal{\ensuremath{#1}}{}}{\ensuremath{\Pi}}{\ensuremath{\Pi_{X #1}}}}
\newcommand{\prot}[1]{\ensuremath{\pi_{\msf{#1}}}}
\newcommand{\lla}{\leftarrow}
\newcommand{\lvd}{\vdash}
\newcommand{\tb}[1]{\text{\color{royalblue}{#1}}}
\newcommand{\tgr}[1]{\text{\color{forestgreen}{#1}}}
\newcommand{\tm}[1]{\text{\color{magenta}{#1}}}
\newcommand{\tg}[1]{\text{\color{gray}{#1}}}
\newcommand{\tp}[1]{\text{\color{purp}{#1}}}
\newcommand{\nparam}[1]{\tp{#1}}
\newcommand{\tr}[1]{\text{\color{Red}{#1}}}
\newcommand{\yo}[1]{\text{\color{YellowOrange}{#1}}}
\newcommand{\inline}[1]{\lstinline[basicstyle=\footnotesize\BeraMonottFamily, mathescape]!#1!}
\newcommand{\nrecv}{\tb{recv}}
\newcommand{\nsend}{\tb{send}}
\newcommand{\nget}{\tb{get}}
\newcommand{\npay}{\tb{pay}}
\newcommand{\nsimget}{\tm{simget}}
\newcommand{\nsimpay}{\tm{simpay}}
\newcommand{\ncase}{\tm{case}}
\newcommand{\nproc}{\tb{proc}}
\newcommand{\nwithdraw}{\tm{withdrawTokens}}
\newcommand{\nif}{\yo{if}}
\newcommand{\nthen}{\yo{then}}
\newcommand{\nend}{\yo{end}}
\newcommand{\nwhile}{\yo{while}}


%\newcommand{\pluseq}{\mathrel{+}=}
%\newcommand{\minuseq}{\mathrel{-}=}
\newcommand{\Assert}{{\bf \color{BrickRed} Assert }}
\newcommand{\Require}{{\bf \color{BrickRed} Require }}

%\theoremstyle{acmdefinition}
%\newtheorem{definition}{Definition}[section]
\newtheorem{ddef}{Definition}
%\newtheorem{theorem}{Theorem}
\newtheorem{claim}{Claim}
%\newtheorem{lemma}{Lemma}

%\newlist{renumerate}{enumerate}{1}
%\setlist[renumerate]{before=\setlength{\baselineskip}{20pt}, itemsep=-2ex, topsep=-2ex}
%\newenvironment{renumerate}{\begin{enumerate}[before=\setlength{\baselineskip}{20pt},itemsep=-2ex,topsep=0pt]}{\end{enumerate}}
\newenvironment{renumerate}{\begin{enumerate}[nosep]}{\end{enumerate}}
%\newenvironment{ritemize}{\begin{itemize}[before=\setlength{\baselineskip}{20pt},itemsep=-2ex,topsep=0pt]}{\end{itemize}}
\newenvironment{ritemize}{\begin{itemize}[nosep] \renewcommand\labelitemi{--}}{\end{itemize}}

\newenvironment{mylst}{\begin{lstlisting}[basicstyle=\small\BeraMonottFamily, frame=single, mathescape]}{\end{lstlisting}}

\makeatletter
\newcommand{\inmsg}[1]{%
(#1\checknextarg}
\newcommand{\checknextarg}{\@ifnextchar\bgroup{\gobblenextarg}{)~}}
\newcommand{\gobblenextarg}[1]{, #1\@ifnextchar\bgroup{\gobblenextarg}{)~}}
\makeatother


\newcommand{\transfermsg}{\inmsg{transfer}{to}{val}{data}{from}}
\newcommand{\createmsg}{\inmsg{contract \ create}{addr}{val}{data}{private}{from}}
\newcommand{\reject}{\textbf{reject}~}
\newcommand{\ignore}{\textbf{ignore}~}
%\newcommand{\For}{\textbf{For}~}
\newcommand{\Env}{\ensuremath{\mathcal{Z}}}
%\newcommand{\While}{\textbf{While}~}
\newcommand{\Buffer}{\textbf{Buffer}~}
\newcommand{\Send}{\textbf{Send}~}
\newcommand{\Output}{\emph{Output}~}
\newcommand{\Leak}{\textbf{Leak}}
\newcommand{\Eventually}{\textbf{Eventually}~}
\newcommand{\In}{\textbf{in}~}
\newcommand{\If}{\textbf{If}~}
\newcommand{\Else}{\textbf{Else}~}
%\newcommand{\Return}{\textbf{Return}~}

\newcommand{\pluseq}{\ensuremath{\mathrel{+}=}}
\newcommand{\minuseq}{\ensuremath{\mathrel{-}=}}
\newcommand{\Adv}{\ensuremath{\mathcal{A}}}
%\newcommand{\Partyi}{\ensuremath{\mathbf{P_i=(sid,pid)}}}
\newcommand{\sid}{\ensuremath{\msf{sid}}\xspace}
\newcommand{\pid}{\ensuremath{\msf{pid}}\xspace}
\newcommand{\dquad}{\quad \quad}
\newcommand{\qqquad}{\qquad \quad}
\newcommand{\qqqquad}{\qqquad \quad}
\newcommand{\qqqqquad}{\qqqquad \quad}

\newcommand*\circled[1]{\tikz[baseline=(char.base)]{
            \node[shape=circle,draw,inner sep=1pt] (char) {#1};}}

\newcommand*\token{~\circled{t}}

\DeclarePairedDelimiter{\ceil}{\lceil}{\rceil}


\newcommand{\spheading}[1]{ %
	\rotatebox{60}{\parbox{2.5cm}{\raggedright #1}}}


\author{\IEEEauthorblockN{Anon}
\IEEEauthorblockA{\textit{Nowhere}}
}

\maketitle

\begin{abstract}
The UC framework is the gold standard for proving the security of cryptographic and distributed protocols, however on-paper UC definitions can be long, complicated, and difficult to understand and reuse.
Furthermore, protocols proven secure in UC exists largely on-paper and are often realized in software frameworks that depart entirely from the UC model. 
Not only does this diminish the usefulness of the UC proof but creates a further obligation to evaluate the security of the implemenation.
A great body of prior work attempts to provide programming support and tools for UC, but many of them propose bespoke programming languages that lack the expressiveness of more mainstream software development environments.
In this work we aim to further bridge the gap between UC theory and software implementation by proposing a new development framework augmented with novel, software-inspired network modelling, a Haskell UC implementation, and fuzz testing as a means of informal security analysis.
\end{abstract}

\begin{IEEEkeywords}
component, formatting, style, styling, insert
\end{IEEEkeywords}

\section{Introduction}
In this work we present a programming language design based on the Universal (UC) Composability framework from cryptography.

We build on existing work, ILC, which also aims to be a programming language for UC, but start from a more powerful language, called Nomos, which incorpoates session types and work aware resource types and has been previously used for  smart contracts and distributed applications.
Both of the features of Nomos turn out to have 

Second, Nomos features a notion of Work-aware types. This is useful for capturing the notion of “locally polynomial runtime.” This allows us to model UC more faithfully than any prior work to date

As a starting point, we build a language that merges types rules from ILC into Nomos. The main design idea of ILC is that it is uses static typing rules to encode the requirements of the Interacting Turing Machines (ITMs) model, a model that is uniquely associated with UC. The ILC rules roughly ensure that simulations of the language can be carried out by probabilistic Turing machines, which is necessary for reduction to computationally hard problems, required for cryptographic security proofs. The rules from ILC are compatible with session types, so it turns out to be straightforward to merge these into nomos. The result provides benefits associated with session types, namely that it avoids potential errors from internally-inconsistent programs.

   Beyond just session types, the Work-aware component of Nomos allows us to tackle a fundamental challenge in defining a programming language for UC that ILC (and all other related work) left unfulfilled, which is to express the notion of polynomially runtime.
   
   The challenge of “polynomial runtime” in UC is that individual processes must be judged as polynomial, but when eveluated in context with other concurrently running process it is difficult to assign blame.
       The current best way to define polynomial runtime, found in the 2019 and later version of UC, is based a concept of ``import tokens.''
   We identify how to relate the “Potential” concept from Nomos, to the import tokens from UC. 
	The result is a deep connection between session type semantics and the formal foundation of UC.
	The Preservation theorem we prove associated with our type system and operational semantics proves the following: 
well-typed terms in Nomos UC are “locally polynomial time”, in the sense required of UC, meaning they do not take more steps that some polynomial function T(N) of the net number of import tokens it has received.

In addition, our language has other benefits.
The Progress theorem is useful because it gives some evidence that ideal functionalities and protocols encoded in Nomos UC cannot get stuck. Together helps confirm that the process halts in polynomial time.
TODO: Give an example of a bad machine ruled out by progress guarantee.

\ignore{
Carries forward the same metatheory guarantees as ILC. Namely: if a process terminates, then it depends only on the random coins (unlike Pi calculus, including Session-type pi calculus). Thus simulating the execution of a Nomos UC experiment can be carried out by a probabilistic polynomial time Turing machine (PPT). This is essential in UC for reduction to computationally hard problems.
}

\ignore{
The Universal Composability Framework~\cite{uc} is the popular and widely-used framework for modelling the security of cryptographic and distributed protocols.
Its novel contribution compared to other frameworks is that it provides a very strong notion of security: a UC-secure protocol is proved to be secure even when composed with arbitrary other protocols running concurrently.
This constrasts with other, property-based notions of security~\todo{need to get some citations here}.

Analyzing large and complex protocols is a difficult task made easier by UC's ideal functionality abstraction. 
However, despite this additional modularity, UC proofs and models still tend to be very complex, unwieldy, and difficult to understand.
These issues are exacerbated when new communication models are added on top of UC~\cite{katz, etc}.
Therefore, we propose a two-fold solution: a new construction for modelling different communication models that removes all model-specific code from protocols and functionalities, and an implementation of the UC framework in the Nomos language. 
}



\section{Related Works}
There are many works that attempt to formalize the UC framework with an implementation for protocol analysis and proof generation.

One of the most relevant works to our own is EasyUC~\cite{easyuc}. 
EasyUC uses the existing EasyCrypy~\cite{easycrypt} toolset to model UC protocols and mechanize proof generation. 
It departs from EasyCrypt's limtations to game-based security definitions (lacking simulation-based composition).
However, it still lacks a notion of polynomial time. The authors, themselves, mentions that it can't detect deviant behavior like the adversary and functionality passing messages between each other indefinitely. 
Our use of the import mechainsm and session types let us reason about polynomial time in the sytem of ITMs encompassed by \msf{execUC} but also locally for \textit{open} terms. 
Furthermore, import in NomosUC lets us have guarantees of termination as well by the polyomial import constraints added to UC by Canetti et al.

Liao et al. introduce executable UC through a new process calculus called ILC~\ref{ilc}.
This work adds some notion of polynomial time although it proves to be too restrictive. 
It results from the fact that poly-time can only be reasoned about for \textit{closed} terms like a full UC execution.
In order to reason about polynomial time for a particular protocol $\pi$ we must reason over all possible other terms that connect to $\pi$ and require that it is polynomial in all such cases.
A simple ping-response server can not be proven to by poly-time in this definition for a deviant other ITM that connects to $\pi$. 
In Nomos, however, as mentioned above, open terms are limited to polytime regardless of the connected other terms because of the import mechanism and the NomosUC type system that guarantees termination. 

Other works that rely in symbolic modelling of cryptography, for example, SymbolicUC~\cite{symbolicuc}, are subsumed by the above ILC work and similarly lack any polynomial time notion. 
\todo{Say something about $\pi$-calculus with probabilistic polynomial time extensions}.


To the best of our knowledge, this is the first work to deal with the new import notion of polynomial time introduced to the UC framework in 2018.
A few other works refer to the import mechanism, but it is restricted to simply defining the import a protocol is given.
	
%easyUC:
%* can not dynamially create new instances of parties/functionalities must statically determine the number of functionalities/parties spawned
%* 
%
%
%The work of Liao et al.~\ref{ilc} is the closest to our own
%It proposes a new process calculus called ILC and a concrete implementation of the UC framework.
%The type system it introduces ensures that correctly types programs can be represented as ITMs.
%However, one drawback of the ILC work is that its polynomial time representation 
%
%
%The EasyUC approach uses the existing EasyCrypt toolset to implement model UC protocols and mechanize the generate of UC-security proofs and proofs of secure composition.
%This work aim considerably higher than our work in actually attempting to generate proofs for their protocols. 
%However, this work falls short in being able to capture any notion of resource bound computation whereas we are able to make guarnatees about polynomial bounds on our system of ITMs and even guarantee termination of programs through our realization of the import mechanism.
%The EasyUC work accepts that not even infinite loops of communication can be caught and, therefore, termination of protocols can't be guarnateed either whereas the import mechanism in Nomos ensures that such infinite loops can not stall protocol progress.

%Another work similar to our own is the Symbolic UC by B\"{o}hl and Unruh.
%This works uses an applied $\pi$-calculus to symbolically model UC protocols and analyze them.
%Similar to the EasyUC work, the goals of this work are somewhat orthogonal to the our own goals.
%However, Symbolic UC does attempt to create an implementatio of UC using the $\pi$-calculus however neglects to address any issues of polynomial runtime.
%
%Perhaps the closes work to our own is that of Liao et al.~\cite{ilc} that builds an executable version of the UC framework by introducing a new process calculus called ILC.
%ILC introduces a type system that guarantees that ILC programs (i.e. functionalities, protocols, etc) can be expressed as ITMs as in the UC framework.
%However, one drawback of ILC is that it's notion of polynomial time ends up being too restrictive.
%In ILC only closed terms without any unbonded variables, i.e. and entire UC exection of a system of ITMs, can be shown to be polynomial in their definition of polynomial time.
%Proving polynomial time for open terms, such as a protocol $\pi$, requires reasoning over all possible contexts in which the protocol could exist however such a definition of polynomial time becomes too restrictive where even a simple ping-responde server protocol wouldn't be considered polynomial time.


\section{Background} \label{sec:background}
\subsection{Universal Composability}
The universal composability framework~\cite{uc} proposes a new framework for proving the security of cryptographic and distributed protocol.
Compared to previous works, the UC framework provides a stronger notion of security where protocols that are UC-secure are secure even when composed with arbitrary other protocols running concurrently. 

Such a strong notion of security is achieved through the real-ideal world paradigm.
The ideal world encompasses an ideal implementation of a protocol, called the \textit{ideal functionality} $\mathcal{F}$, which acts as a trusted third party that caputures all the desired security properties.
The ideal functionality is usually a simple definition making it trivial to prove its security properties.
The real world, on the other hand, consists of parties running an actual protocol, $\pi$, against a real adversary.

Security proofs in UC involve creating a simulator $\mathcal{S}$ in the ideal world that can simulate every potential attack on a real protocol in the real world.
If $\mathcal{S}$ can make the two worlds indistinguishable for any real world adversary $\mathcal{A}$ for all distinguishing environments $\mathcal{Z}$, then we say the protocol $\pi$ UC-emulates the ideal functionality $\mathcal{F}$.
Indistinguishability of the two worlds to any $\mathcal{Z}$ implies that the protocol $\pi$ must exhibit the same security properties as the ideal functionality $\mathcal{F}$ otherwise there should be sobe distinguishing environment. 
More formally, indistinguishability is stated:

$$ \text{EXEC}_{\mathcal{F},\mathcal{S},\Environment} \approx \text{EXEC}_{\pi,\mathcal{A},\Environment} $$

\paragraph{GUC-Framework}


\subsection{The Import Mechanism}
A notion of resource-bound computation is necessary for the UC framework to reason about computationally efficient algorithms as well as the capabilities of ITIs under a particular resource constraint.
Often we would like to reason about adversarial capabilities under such constraints and perform efficient transformations (transforming an adversary into a simulator).

Previous definitions of polynomial-time computation have taken the form of bounding the computation of an ITI by some polynomial $T$:
given an input of length $n$ the machine $\mu$ halts within $T(n)$ steps.
However, using the length of the inputs to the machine as $n$, in this case leads to an infinite runs problems identified by Canetti~\cite{uc}.
Machines that are locally $T(n)$-bounded are able to spawn other machines to the point that an infinite chain of such machines can be spawed where each is locally $T$-bounded, but the whole system of machines can not be bounded by any polynomial $T$.

Therefore, a new notion of $n$ was needed. The UC paper defines an import mechanism where the first ITI, the environment, is spawned with a polynomially amount of import which can be thought of as tokens or coins.
The environment can then activate other ITIs with some import tokens allowing them to run for $T(n')$ computationsl steps for some $T$ and some amount of import $n'$.
In this new definition, an ITI that is $T$-bounded takes at most $T(n')$ steps where $n'$ is the difference between the import it has received from incoming messages and outgoing import it's given to other machines.
This definition therefore suffices to ensure that every machine is locally bounded by some polynomial but also guarantees that the system of ITMs is bounded by a polynomial number of import tokens. 


\section{Computation Models} \label{sec:wrappers}
One of the main hurdles to existing UC models is that asynchronous/synchronous communication models uncecessarily complicate the design of protocols and ideal functionalities.
It makes it difficult to understand them and more painstaking to check or falsify UC security proofs. 
In this section we introduce a new alternative for sync/async communication models in the form of two shared functionalities, which allows delayed execution of entire code blocks rather than just delayed message delivery.
Furthermore, the functionalities take advantage of the new import mechanism to achieve eventual delivery in the asynchronous world.




\section{Haskell Saucy}
In this section we outline our realization of the UC framework in Haskell.
For the sake of space we limit the presentation here to the minimum required to illusrate the construction of the asynchronous wrapper and our fuzz testing tooling. 

\subsection{Communication Between ITMs}
ITMs are realized in \us as probabilistic, polynomial time, channel passing processes. 
The are similar in spirit to ITMs, however rely on channels from the Hasjel \texttt{Control.Concurrent} library as opposed to the tapes used by ITMs. 
\todo{Make a clear distinction between this and ILC?}

Processes in \us are provided several programming abstractions through the Monad typeclass \texttt{MonadITM}. 
These abstractions allow for random coin clips (\texttt{?getBit}), import tokens (\texttt{?getTokens}), and a security parameter (\texttt{?secParam}).
These are all defined as implicit parameters so tha they can be set concretely at runtime. 
The typeclass \texttt{MonadITM} is parameteric in each of these abstraction, allowing other processes to sandbox any other process and replace them with custom implementations.
For example, a simulator may want to rewing a proodf replay a particular stream of random bits.

The same design principle extends to typeclasses for functionalities, environments, adversaries, and protocol parties. 
Critically, this gives access to a set of corrupt parties for only environments, adversaries and ideal functionalities, but, it also allows processes to sandbox them with custom SIDs (which often provide execution parameters such as the IDs of other protocol parties, corruption thresholds for distributed protocols, or even a CRS).

\todo{How much should we explain the monad typeclasses themselves?}

A simple example of an ITM is given in the code listing below. We explicitly give types for clarity.
The process \texttt{exampleITM} takes in no parameters but spawn a new process that it communicates with through a read cannel \texttt{a} and a wite channel \texttt{b}.
Throughout \us we distinguish channels are read/write channels rather than allowing two-way communication over them. 

\begin{lstlisting}
{-- Ping Pong and channel params --}
doubler i o = do
  x !\la! readChan i
  writeChan o (x*2)

exampleITM :: MonadITM m !\Ra! m ()
exampleITM = do
  a !\la! (newChan :: Int)
  b !\la! (newChan :: Int)
  fork $ doubler a b
  (writeChan a 15) :: m ()
  (output :: Int) !\la! readChan b
  liftIO $ putStrLn $ "Output: " ++ show output
  return ()
\end{lstlisting}

\begin{lstlisting}
type MonadITM m = (HasFork m,
                   ?getBit :: m Bool,
                   ?secParam :: Int,
                   ?getTokens :: m Int,
                   ?tick :: m ())
\end{lstlisting}

\paragraph{The Party Wrapper}

\subsection{Token Implementation}
At a high level, imports in this work are used primarily to demonstrate the difference between eventual delivery and no delivery in the simulated world. The simulator internally emulates the real-world protocol and uses its control over network delays to align delivery of ideal-world outputs with the real world. More specifically, given some imports, we show how much progress the real-world protocol can make and how long the simulator can delay the ideal-world output so that the transcripts remain indistinguishable. Unrelated details of program execution are omitted for simplicity of implementation.

\begin{lstlisting}
data CarryTokens a = SendTokens a 
                      deriving (Show, Eq)
data TransferTokens a = DeliverTokensWithMessage a 
                         deriving (Show, Eq)
\end{lstlisting}

Our implementation makes use of two data types. The \texttt{CarryTokens} type specifies the number of imports a message carries. The \texttt{TransferTokens} type lets a protocol party specify to a Functionality the number of imports it wants to deliver to the receiver. The \texttt{TransferTokens} type itself does not carry imports, rather it only denotes the number of imports the sender intended to send. We rely on the \texttt{CarryTokens} type to show what the sender was able to send given the imports it had. When a protocol party wants to send imports to other parties, it must use the \texttt{TransferTokens} type to specify the amount and include imports in the \texttt{CarryTokens} type. We realize this import transfer with a multicast Functionality. The Functionality accepts imports included in the \texttt{CarryTokens} type, reads the amount specified by \texttt{TransferTokens} on its input tape, and sends the specified amount to the receiving party in a best-effort attempt. That is, the Functionality sends as many imports as possible in case of import shortage.

Below shows the multicast Functionality type before and after the inclusion of imports. The \texttt{TokenFunctionality} type enables import transfer between protocol parties and functionalities by appending to the p2f and f2p channels the \texttt{CarryTokens} type. \texttt{fMulticastToken} additionally includes the \texttt{TransferTokens} type in its messages so that it is notified of the number of imports protocol parties want to transfer.

\begin{lstlisting}
fMulticast :: MonadFunctionalityAsync m t => 
  Functionality t (MulticastF2P t) 
                (MulticastA2F t) 
                (MulticastF2A t) Void Void m
\end{lstlisting}

\begin{lstlisting}
type TokenFunctionality p2f f2p a2f f2a z2f f2z m 
  = MonadFunctionality m => 
    (Chan (PID, (p2f, CarryTokens Int)), 
     Chan (PID, (f2p, CarryTokens Int))) -> 
    (Chan a2f, Chan f2a) -> 
    (Chan z2f, Chan f2z) -> 
    m ()

fMulticastToken :: MonadFunctionalityAsync 
                     m ((t, TransferTokens Int), 
                        CarryTokens Int) => 
  TokenFunctionality (t, TransferTokens Int) 
                     (MulticastF2P t) 
                     (MulticastA2F t, 
                      TransferTokens Int) 
                     (MulticastF2A t, 
                      TransferTokens Int) 
                     Void Void m
\end{lstlisting}

Below is a code snippet from \texttt{fMulticastToken}. The multicast sender leaks the message, the number of imports it carries, and the number of imports the send operation forwards to each receiver to the adversary. Multicast receivers obtain the message and imports from the Functionality so that they can initiate their own multicast operations.

\begin{lstlisting}
if pid == pidS then do
  ?leak ((m, DeliverTokensWithMessage st), 
         SendTokens a)
  forMseq_ parties $ \pidR -> do
    eventually $ do
      tk <- readIORef tokens
      if tk >=1 then do
        writeIORef tokens (max 0 (tk-1-st))
        writeChan f2p (pidR, 
                       (MulticastF2P_Deliver m, 
                        SendTokens (min st 
                                        (tk-1))))
      else ?pass
  writeChan f2p (pidS, (MulticastF2P_OK, 
                        SendTokens 0))
\end{lstlisting}

For each send, the Functionality burns 1 import as delivery cost and sends as many requested imports to the receiver as possible. Specifically, given a request \texttt{TransferTokens} \texttt{tk}, the Functionality includes in the message either exactly \texttt{tk} imports or, in case of insufficient reserves, all imports it had left. The send operation can only be initiated when the ITM holds at least enough imports to pay for the delivery cost.

\subsection{Overview of Bracha's Protocol}
The reliable broadcast protocol involves a single sender and a set of $N$ servers(receivers). The sender receives an input($v$) and broadcasts to the receivers, which either all output the same value or output nothing. At most $f$ receivers can be Byzantine corrupt, with $N\geq 3f+1$. When the sender is honest, the receivers' output should match the value $v$.

\subsubsection{Properties}
\begin{figure}
    \centering
    \begin{bbox}[title={$\mathcal{F}_\msf{RBC}(\mathcal{P})$}]

    {\bf \color{Black} On first input}~ \inmsg{\tsc{input}}{msg} from $\mathcal{D}$:

    \quad \Leak (\textsc{input}, msg) to \Adv
    
    \quad For $p_i$ in $\mathcal{P}$ :
    
        \qquad \Eventually Send msg to $p_i$

\end{bbox}
    \caption{Ideal functionality for reliable broadcast.}
    \label{fig:f_code}
\end{figure}
The ideal functionality for reliable broadcast is shown in Figure \ref{fig:f_code}. The protocol must satisfy three important properties:

\textbf{Agreement}: If any two honest servers output $v$ and $v^{\prime}$, then $v = v^{\prime}$.

\textbf{Validity}: If the sender is honest and receives input($v$), then every server outputs $v$.

\textbf{Reliability}: If any honest server outputs $v$, then every server outputs some $v^{\prime}$.

The pseudo-code for reliable broadcast is shown in Figure \ref{fig:pi_code}.
\begin{figure}
    \centering
    \begin{bbox}[title={$\mathbf{\Pi}_\msf{RBC}(\mathcal{P})$}]

{\bf \color{Black} On receiving}~ \inmsg{\tsc{input}}{msg} from $\mathcal{D}$:

    \quad For $p_i$ in $\mathcal{P}$ :
    
        \qquad Send (\textsc{echo}, msg) to $p_i$

{\bf \color{Black} On receiving}~ \inmsg{\tsc{echo}}{msg} from at least $(N+f+1)/2$ processes in $\mathcal{P}$:

    \quad If \tsc{ready} has not been sent:

        \qquad For $p_i$ in $\mathcal{P}$ :
    
            \quad \qquad Send (\textsc{ready}, msg) to $p_i$

{\bf \color{Black} On receiving}~ \inmsg{\tsc{ready}}{msg} from at least $(f+1)$ processes in $\mathcal{P}$:

    \quad If \tsc{ready} has not been sent:

        \qquad For $p_i$ in $\mathcal{P}$ :
    
            \quad \qquad Send (\textsc{ready}, msg) to $p_i$

{\bf \color{Black} On receiving}~ \inmsg{\tsc{ready}}{msg} from at least $(2f+1)$ processes in $\mathcal{P}$:

    \quad Output msg
\end{bbox}
    \caption{Pseudo-code for Bracha's protocol.}
    \label{fig:pi_code}
\end{figure}

\subsubsection{Achieving Agreement and Validity}
\textsc{echo} messages guarantee Agreement and Validity for the protocol.

A corrupt sender may send different values to the receivers, who must output the same value. This results in an "Echo" phase, where receivers asynchronously attempts to agree on an output. More specifically, receivers echo their received value, and once a party has received more than a threshold number of echoes for some value $v$, it sets $v$ to be its output.

The echo threshold must be no less than $\lceil \frac{N+f+1}{2} \rceil$ so that it can be reached by at most one candidate value. This makes sure that participating parties cannot decide on different outputs. For any pair of distinct values $v_1$ and $v_2$, each of the $N-f$ honest parties will only echo one of them, while the $f$ corrupt parties can potentially echo both. This will result in a total of at most $(N-f)*1+f*2=N+f$ echoes. $\lceil \frac{N+f+1}{2} \rceil$ is more than half of the maximum number of echoes, and thus is a correct quorum and a sufficient threshold.

\subsubsection{Achieving Reliability}
A second round of communication (i.e., \textsc{ready} messages) is added to let the parties agree on whether an output is decided and thus achieve Reliability.

The attacker may try to send just enough echoes to let honest party $p_1$ decide on a value without letting another honest party $p_2$ do so. Suppose $N=4$ and $f=1$, in a case where the sender $p_4$ is corrupt. The corrupt sender sends $v$ to honest parties $p_1$ and $p_2$ and nothing to $p_3$. The parties will eventually receive 2 echoes for $v$ from $p_1$ and $p_2$, who need $\lceil \frac{N+f+1}{2} \rceil = \frac{4+1+1}{2}=3$ echoes to decide on an output. Without the \textsc{ready} round, if the corrupt party sends an echo for $v$ to $p_1$ but not to $p_2$, $p_1$ will eventually output $v$ but $p_2$ will not. This violates Reliability.

The \textsc{ready} round ensures Reliability since when $(2f+1)$ \textsc{ready}s are received by a party, at least $(f+1)$ honest parties must have sent \textsc{ready} to be eventually delivered. All $(N-f)$ honest parties will therefore eventually qualify to multicast their \textsc{ready}, and at least $(2f+1)$ \textsc{ready} will be eventually delivered to all parties. This guarantees that every honest party will eventually output a value.

Agreement and Validity from the \textsc{echo} round are not changed by the additional \textsc{ready} messages. A party can only send \textsc{ready} in two ways:

$\mathbf{}{1}$. A party sends \textsc{ready} after receiving enough \textsc{echo} messages. The quorum still applies and Agreement and Validity still hold.

$\mathbf{2}$. A party sends \textsc{ready} after receiving at least $(f+1)$ \textsc{ready}. At least $1$ of the \textsc{ready} messages must come from an honest party, and so the party's \textsc{ready} must contain the same value as some honest party that had previously sent \textsc{ready}. We can backtrack the honest \textsc{ready} messages, and the root \textsc{ready} had to be triggered by some \textsc{echo}. This implies that the quorum still applies and Agreement and Validity still hold.

\subsubsection{UC Simulator for Bracha’s Protocol}
The simulator $\mathcal{S}$ can be designed as the following:

$\mathcal{S}$ locally emulates the real-world Bracha's protocol and $\mathcal{F}_{auth}$ in a sandbox. Since the ideal $\mathcal{F}$ leaks all information (i.e., input), $\mathcal{S}$ can run $\mathbf{\Pi}$ locally in a blackbox manner. $\mathcal{S}$ forwards messages from corrupt parties into the sandbox. Delivery instructions are processed in the sandbox.

If the sender is honest, when $\mathcal{S}$ sees an input $v$, it passes $v$ to the sandbox.

If the sender is corrupt, whenever a (simulated) honest party in the sandbox outputs $v$, $\mathcal{S}$ sends input $v$ to the ideal functionality $\mathcal{F}_{RBC}$. Since everything inside the sandbox is deterministic and uses the same inputs as the real world, this ensures that the value $v$ sent to the ideal functionality will be exactly the same as the real world output.

\subsubsection{Effects of Wrong Thresholds}
Violating any of the three properties causes the protocol to be distinguished by the environment.

Agreement can be broken by an insufficient threshold in the \textsc{echo} round, as the lowered threshold is no longer a quorum. The parties can therefore reach split decisions.

Validity can be broken by an insufficient threshold in the \textsc{ready} round, as the lowered threshold can no longer guarantee that at least one message comes from an honest party. That is, messages from corrupt parties alone can make an honest party decide on its output. Therefore the eventual output can be potentially any arbitrary value from the adversary.

Reliability can be broken, for example, by an insufficient output threshold in the \textsc{ready} round. Alternatively, it can be broken by an insufficient threshold in the \textsc{echo} round together with a large enough output threshold in the \textsc{ready} round that prevents a second message from being output. Note that showing broken Reliability in UC requires distinguishing eventual output and no output. The simulator, while still holding imports, is able to repeatedly delay an eventually delivered output and maintain an indistinguishable view.

\section{Fuzz Testing}
\todo{Still need to mention that we use it to test protocol properties and simulator proofs.}
We rely on fuzz testing as our chosen method of informal protocol analysis for a few critical reasons. 
First, there is a wealth of prior work outlinin the success of fuzz testing techniques, even again program verification, for discovering unintended behavior in code.
Second, existing work in fuzz testing often focuses purely on program binaries that do not concurrently communicate with other processes aside from system calls.
A related work to our own, by Jepsen, takes a novel direction by creating a fuzz testing framework for testing the Tendermint byzantine-fault tolerant consensus protocol. 
Here, several nodes communicate with each other through tcp/ip connections and come to consensus on the ordering of messages sent by all the nodes (refer to Section~\ref{sec:relatedworks} for a more in-depth analysis of the work). 
In this work, we attempt a similar mechanism but constrain ourselves to evaluating protocols expressed in the UC framework.
Furthermore, we replicate the work done by jepsen in our framework as a baseline validation for its capabilities. 

\subsection{QuickCheck in with UC}
The QuickCheck module provides primitives for generating input according to some rules. 
The advatage of modelling protocols within the UC framework is that the interface for the adversary's input to the protocol, and any underlying assumptions or network primitives, is made explicit from the start.
This helps constrain the set of possible inputs give to the adversary and make the framework amenable to fuzzing.

\paragraph{Always Enabled Actions}
Part of defining protocols for fuzz testing in \us requires borrowing an idea from Iron Fleet~\cite{ironfleet}.
In standard UC when ITMs normally halt when something goes wrong.
In agreement protocols, a protocol might ensure distinguishability by simply halting when something incorrect happens such as, for example, receiving a broadcast from a part that isn't the sender or running out of import. 
When writing a protocol in \us it is critical that programs don't throw errors or simply stop accepting messages from others, because such situations lead to test runs that hang indefinitely. 
Instead, all prorams need to ensure that all inputs received from other ITMs result in control being passed to another ITM.
In the case of faults, or halting, this simplifies to passing control back to the environment on any input.  \todo{make this better}


\subsection{Discovering Safety Bugs in Protocols}
We examine fuzz testing by implementing some classical and a moden byzantine agreement protocol and injecting faults into them.
Most injected bugs arise from misplaces thresholds for parties to take some action.
For example, a protocol that designed to handle $\frac{n}{2}$  

Most injected bugs arise from misplaced thresholds and incorrect assumptions about corrupt party threstholds. 



\subsection{Analyzing Liveness in Distributed Protocols}
Analysis, even informal, about liveness in protocols is a hard problem.
A large body of existing works, like IronFleet, that uses temporal logic to reason 
about some positive actions happening in a distributed protocol, but this comes at 
the cost of significant user.
In this section we explore to what extend our informal analysis of consensus and agreement
protocols, and our implementation of the import mechanism, can discover and give meaningful
feedback about liveness issues to a protocol analyst.

There are some critical limitations in what an informal analysis can achieve.
With the import mechanism, the most interest kinds are evident when the execution runs out
of import, and this leads to a problem of juggling false negatives and false positives when
asking the question: is this protocol live?
Imagine a probabilistic protocol that makes random decisions
and terminates in some expected number of rounds with byzantine agreement. 
For example, say some execution among the generated test cases outputs an error
that some ITM in the execution is out of import. The error can be explained in one of 
two ways:
\begin{enumerate}
	\item The protocol, as defined, does not get enough import from the environment,  or it doesn't pass around enough import between the parties to achieve the desired functionality. It is a randomized protocol and there may be some sequence of random choices that delays termination by a large enough amount (or for many rounds) that the import provided is insufficient. 
	\item The protocol does have a fault, and there is some sequence of random decisions the parties can make which results in the protocol no terminating in a polynomial amount of time. In reality, regardless of the polynomial import provided, there will always be some sequence of decisions that prevents poly-time termination. 
\end{enumerate}
In fact, it may even be the case that in $n$ generate test cases the faulty traces of an incorrect protocol may never be triggered.  

\paragraph{False Negatives}
False negatives occur when a truly live protocol runs out of import trying to terminate. 
In such cases, the natural next analysis step is increasing the polynomial import given
to the protocol until suc 
\todo{hypothesis is that increasing the polynomial and number of test cases reduces false negatives towards zero at the limit}

\paragraph{False Positives}
individual generated executions that report failures may be false negatives for the reasons above.
A fuzz testing run that returns no failure can be false positive as the failure trace hasn't been discovered. 
\todo{hypothesis is that increasing the polynomial and number of test cases approaches a constant upper bound in the limit as the actual traces with never terminate happen infinitely often}



%\section{Broadcast Protocols} \label{sec:rbc}
%\input{sections/bcast}

\section*{Acknowledgment}

\section*{References}

\appendix


\pagebreak

\end{document}
