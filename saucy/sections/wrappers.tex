The usual structure of modelling asynchronous communication is thorugh adversarial
delay of messages between parties. This commonly takes the form of the functinality
passing control to the adversary and waiting for it to deliver the message, like the
simple \Fchan functionality in Figure~\ref{fig:fchanadv}, or a functionality that 
maintains counters for adverasrial delay and requires parties to repeatedly query for
new messages like the \Fchan varianed in \ref{fig:fchanpoll}. 

The former approach gives the adversary too much power in the execution. A naive adversary
can take control from \Fchan, never do anything else, and make it impossible to provide
a termination guarantee for a protocol in ths model. The latter approach requires protocols
to written to handle the specifics of the communication model, such as polling and waiting
for message retrieval. From a programming point of view it complicates environments
with repeated inputs to honest parties, and from a theoretical point of view it clouds 
protocol specifications with unecessary model-specific code. We see this in the example in
Figure~\ref{fig:protpoll}. 

Asynchronous communcation in UC aims to achieve \emph{eventual delivery} guarantees by 
using a polynomial time notion to conclude an adversary can only impose polynomial delay and 
thus messages are deliever \emph{eventually}. In the first model described above an adversary
that does nothing when handed control decidedly make it impossible for the protocol to
terminate. Similarly, the second model of delay counting and polling relies on an outdated
polynomial time notion that is known to have issues enabling non-polynomial behavior 
from a system of ITMs~\ref{uc}. 

In this section, we introduce a new method of capturing comuniation and \emph{computation} 
models in UC. Our method uses wrappers to encode all model-specific behavior, uses keywords
in both sofware and on-paper that compile to and abstract away the model, and we use
the import mechanism to achieve polynomial guarantees like eventual delivery. We point 
out that the abstraction presented here implements adversarially delayed computation
rather than just communication, and it can be used to abstract arbitrarily complicated
computation models like multi-party computation \todo{throwing that in there because it's cool}. 

\subsection{Asynchronous Computation}
The asynchronous wrapper acts as a shell process $\msf{SH}(.)$ around some existing ITM $M$. 
It maintains a queue of \emph{shceduled} code blocks that $M$ can add to, and it extends an 
interace to \A to query the state of the queue, delay the execution 


It accepts messages from $M$ to schedule the adversarially delayed computation, and it extends
an interface to \A that allows it to delay code blocks, query the state of adversarial delay,
and deliver computation at will. 

