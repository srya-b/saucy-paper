Realizing asynchronous distributd protocols first requires an eventual delivery model for UC.
In one of the more recent updates to the UC framework, a new notion of polynomial time was introduced, the import mechanism, which drastically changes the way computational bounds and PPT adversaries are defined.
One of the main constributions of this work, which we present here, is dissecting the import mechanism to understand how eventual delivery can be defined in this work, how existing formalisms fall short in this new setting, and how liveness is defined.

Our contribuitions toward this goal is higlighting a fundamental limitation of the constraints added alongside import in realizing eventual delivery at all.
We cast this limitation first in the context of existing work, and then isolate the required change in the UC definition in order to support eventual delivery.
Next, we introduce a new model for capturing eventual delivery networks which deviates from the constraints described by UC and uses import directly.
Finally, we discuss liveness properties of protocols in this new setting. 

We summarize, here, the description of import provided in Section~\ref{sec:background}.
Import can be thought of as tokens which ITMs exchange in order to give each other runtime. 
With any message sent from one ITM to another, an additional field can be set indicating the import sent along with the message.
At the start of a UC execution, the environment is given all the import that will ever be available to the rest of the ITMs in the system. 
As \Z sends messages to others it gives them import, enabling them to perform polynomial computation and sent import to others.

Alongside the import mechanism, UC introduces two constraints on ITM systems that fundamentally 


The focus of this work is on UC as on UC as a software development tool, and we design a new asynchronous model of computation that overcomes theoretical limitations of existing approaches and make design choices that reduce code size and complexity.
Our asynchronous model borrows from existing work, but makes the additional abstraction to ascynrhonous code execution: a familiar abstraction for programmers.
We show how our use of import acheives the necessary eventual delivery guarantees.
To the best of our knowledge, this is the first work to relate import to asynchronicity and eventual delivery.

\todo{I want to keep some text like this here so people don't think that import and our wrapper not being too useful for liveness analysis is the same as us just not designing a good wrapper in the first place.}
Our method of designing this model is to achieve the smallest and more intuitive model which uses import in a clear way.
We avoid designing a model specifically for the benefit of fuzz testing or informal analysis so its contribution remains clear and extends beyond the results of our experiments.
%The focus on this work is on realizing UC as a development framework for asynchronous distributed protocols.
%Asynchronous protocols rely on an \emph{eventual delivery} guarantees that has been studied closely in UC literature.
%In this section we introduce a new model for capturing eventual delivery in asynchronous networks that integrates with the import notion of polynomial time, and, to the best of our knowledge, are the first to concretize its relation to eventual delivery.
%We more closely connect the import mechanism to our eventual delivery notion to enable its use for program analysis later in this section--mainly aimed at enabling liveness analysis for protocols implemented in \us.
%Our model design also reflects the best practices we identify for designing UC models, that least constrains the programmer in protocol development and testing compared to existing models.
%Finally, we define our notion of eventual delivery within this model and its relationship to the import mechanism.

%\todo{We the coin flip functionality and protocol as our running example to motivate and explain our design.}
%
%\todo{I completely forgot to say we're trying to unconstrain the environment so programs and their test cases aren't cluttered with internal UC mechanics}

\subsection{Status Quo}
One of the research questions we highlight from the beginning is identifying roadblocks to UC adoption as a development framework, and proposing a set of best practices that streamline the development experience and abstract away as much of the underlying UC framework from the programmer.
We begin by first describing the most prevalent existing design for asynchronous networks in UC.
The formulation of asynchronous messaging introduces by Canetti in UC~\cite{uc} is a weaker assumption than eventual delivery, because it allows the adversary to drop messages entirely.
Although a valid adversarial model, it isn't very useful to realize the protocols we want, therefore, additional mechanisms are needed on top of the base framework to ensure eventual delivery

Two prominent works by Katz et al~\cite{katzuc} and Coretti et al.~\cite{corettimpc} propose models for asynchronous communication with eventual delivery that closely resemble one another~\footnote{The work of \cite{corettimpc} modified the design of \cite{katzuc} into a single functionality for repeated message sending between a sender and a receiver rather than a one-shot, single-message functionality.}.
Both models use a ``fetch'' model of messaging where receivers repeatedly ask the functionality for a new messages. 
On ``fetch'', the functionality decrements a delay counter $D$ that the adverasry can arbitrarily add to.
This model is summarized in Figure~\ref{fig:fchanpoll} with details about party identities removed for clarity and brevity.
We describe the notion of eventual delivery the mode's eventual delivery guarantee later in this section and discuss the consequences of its design and other common design patterns common in UC here. 

A fetch model adds additional constraints to environments to ensure protocol parties are sufficiently activated to fetch the messages requires, as well as perform all the actions they would like to do. 
Similarly, the environment test case must ensure the adversary is sufficiently activated in order to execute its strategy. 
This notion of ``sufficient activations'' would require a programmer to carefully instrument test cases so that the execution proceeds correctly according to these implicit requirements.
Rather than part of the automation of the frameowrk, requiring intervention by the programmer to simultaneously define a meaningful test case and juggle these additional requirements imposes a significant hurdle---especially looking ahead to generating test cases with fuzzing.

%The fetch model introduces a couple of constrainst on the design of environment test cases. 
%The first is that the protocol party and adversary both rely on the environment to activate at the right time and give them enough activations over the course of the execution.
%For example, the adversary needs to be periodically activated and activated enough times, according to some abstract notion of the environment knowing when and how many times this must be done.
%Similarly, the environment must ensure all protocol parties are repeatedly activated enough time for messages to be received and for the protocol to move on to the next state, phase, or round. 
%The environment has to orchestrate the mechanism, and the whole frameowork, rather than it being made automatic by design.
%From a programming point of view, this requires careful consideration when testing protocols, writing adverasaries, and especially when applying fuzzt esting like we do later in this work.
%Part of the difficulty is the state space explosion for fuzzing from the new inputs and the times the must be given.
%The other part is that executions where these constraints are not met are of no use for analysis because it isn't clear that the model is adequately captured.

\paragraph{Design Decisions for Simplicity and Clarity}
A major roadblock we identify for realizing UC for development is that many design patterns that commonly appear in UC definitions, in literature, translate to requiring significant user interaction with the underlying framework.
In the model above, there are a few subtle design choices that greatly impact the way a programmer must define code. 
We highlight them, and point out a set of best practices that our asynchronous computation model, introduces later in this section, makes use of.

The above definition requires the protocol to repeatedly ask for new messages, and the ideal functionality never gives control back to the protocol party. 
When sending messages to multiple parties, this design requires that protocol parties must be sufficiently activated by the environment in order to retrieve messages and send messages to multiple parties one-by-one. 
This process can be abstracted to some degree for simple messages passing but the protocol code must still accept null-message inputs in order to continue performing ``pending actions'': a rather unnatural method of programming in modern computing. 
For more complex functionalities, say for flipping many coins among many parties, the enviropnment must ensure that it correctly orchestrates the execution to satisfy the implicit requirements that the model imposes in order for the execution to be ``correct''. 
Bringing this to the fuzz testing domain, user defined environments and environment generators require careful design to ensure these contraints are adhered to correctly.
This extends to activating the adversary a sufficient numebr of times to allow the ideal world simulator proof to perform its function.
Orchestrating the different parts of the UC framework alongside developing and testing an implementation does little to make UC more accessible. 
Looking beyond just the asynchronous modelling we focus on here, more complex assumptions, such as realizing synchronous computation and its notion of ``input completeness'', only amplifies these drawbacks.

In general, designs that minimize programmer-interference and maximize automation are preferred in the programming environment. 
Below, we introduce our asynchonrous model which defines and implements a set of best practices that accomplish this.

%\todo{Not doing this correctly in fuzz testing means that the simualtor proofs themselves don't work unless the environment activates them enough}

\begin{figure}
\begin{bbox}[title={$\mathcal{F}_\msf{Chan}(p_s, p_r)$}]

Initialize $M=\bot$ and $D := 0$

\begin{ritemize}
\item \OnInput \inmsg{\msf{m}} from $p_s$: Set $M = m$ and send $(\msf{sent}, \ell(m))$ to \A

\item \OnInput \inmsg{\msf{fetch}} from $p_r$: set $D := D+1$ 
	
	\quad If $D = 0$: Send (\msf{sent}, $M$) to $p_r$

\item \OnInput \inmsg{\msf{delay}}{t} from \A if $T$ is in unary: 
	
		\quad \quad $D := D+T$
		
		\quad \quad Send (\msf{delay-set}) to \A	
\end{ritemize}	
\end{bbox}

\caption{A concise eventual delivery channel from \cite{katzuc}}
\label{fig:fchanpoll}
\end{figure}

% define eventual delivery

%Given that the polynomials bouding different ITMs don't have to be the same, it stands to reason that the adversary in this model will never have more computational power than the protocol parties. 
%The only other entity with possible more computational power is the environment. 
%An approach to remedying this behavior can be attempting to relax the balanced environment constraint and ensure the protocol parties get more import than \A, but it is unclear how to do that without giving the enviromment the power to give the protocol parties arbitrarily more computation than the adversary--the scenario mentioned above that motivated balanced environments in the first place.
%Unlike in our approach, which we outline in this section, there aren't network-specific inputs that \Z gives to the protocol parties that can be exempt from the balanced environment constraint.
%Generally speaking, there doesn't appear to be simple or neat solution to working the existing model into the import setting without designing an entirely new eventual delivery model based on import--like we do in this work.


% previous versions of do somethin with counters but rely on outdate notions of polynomial time
%The intended eventual delivery guaranteed in this functionality arises from the fact that the adverasry is PPT and will eventually no longer be able to add to the message counter. \todo{verify and talk to andrew} 
%Simply put, the assumption relies on the fact that the environment gives more computation power to the protocol parties (through activation) than it does the adverasry.
%If we extend this notion to import, constrained by \emph{balanced environments}, we run into an issue where the adversary always has at least as much ability to delay as the protocol parties.
%An environment may give \A the same import but cease to activate it so that it's unable to delay messages further, but this fundamentally changes what we mean by eventual delivery: from having to do with polynomial time to environments not activiting an adversary enough.
%\plan{say something about this is not realistic and doesn't track with what eventual delivery is all about}

%\paragraph{Definitional Clutter}
%Another downside of prior models, especially with respect to implementation and development, is the additiona constraints it imposes on protocol parties and on the environment.
%In the polling model, the protocol itself is required to include code that enforces the eventual delivery properties and interact with the network in a way that is unnatural in any real world setting.
%Intuitively, the protocol at hand should play no part in the network model it exists in, and all assumptions and mechanics of the underlying model or network should be abstracted away to the framework. 
%Even if the code for handling polling is abstracted away from the protocol code, the environment is still required to ensure ``enough activations'' of protocol parties and the adversary in order to ensure protocol progress or meaningful adversarial action.
%An environment that never activates the adversary creates a network with no delay and first-in-first-out messaging.
%Translating such an environment to an implementation, and to fuzz testing as we intend to do, means unnecessary instrumentation to ensure enough activations of all ITMs interleaved with adversarial actions and real protocol input.
%We prefer a construction in which most of this process is automated and allows enviromments and protocols to be as concise and application-focused as possible.
%As we will show later in this section our construction greatly simplifies this definition accross all the ITMs in the execution.

%\plan{we aim to move as much work of the network assumptions into the model itself and push it out of the environment or protocol parties and onto the adversary}

%Eventual delivery in such a functionality, arises from the fact that the adversary is PPT and therefore can not delay message delivery an infinite amount.
%Simply put, the assumption is that protocol parties, and by extension the environment that repeatedly activates them, can continue to poll the funtionality after the adversary has exhausted its polynomial resourced.
%How this connection plays out with the import mechanism is unclear without specific details on how import is used, how it is consumed, and how it is received by the ideal functionality.
%\plan{quantifying this is hard and we prefer to directly bake polynomial time into the model rather than talking about infinite time horizons}
%
%The definition of import in UC is accompanied by a constrainted on environments that send enough import to the adversary. 
%UC defines environments that give at least as much import to the adversary as they give to protocol parties as \emph{balanced environments}.
%When we go back to the definition of asynchronous networks in \ref{fig:fchanpoll}, it is unclear how protocol parties can eventually ensure delivery under this constraint.
%Clearly, this simple notion of asynchronous network isn't good enough to encode liveness.
%\todo{This statement here foreshadows our relaxation of balanced environments. We can always relax balanced environments and still do a polling based fChan, so that's no a reason but we want a closer connection to poly time anyway.}

\subsection{\fwrapper}
% from a programming point of view we'd like to say something about protocols without having to talk about infinite time horizons
Our model takes inspiration from the downsides outlined above, and we introduce the design of \fwrapper with four main ideas.
First, protocol and ideal functionality code should have minimal direct interaction with the network model. 
Second, import plays a direct role in the function of the asynchronous network and its eventual delivery guarantees. 
All parties, including the adversary and environment, directly use import to interact with the mechanism and delay or force progress in \fwrapper.
Third, we abstract asynchrony to the execution of blocks of code rather than just messages.
Asychronous code definitions, and execution, are already familiar programming constructs for developers and result in clearer and more concise definitions--especially compared to the previous models (above).
Fourth, we implement a set of best practices for how different parts of the UC framework should implemented, and use these to minimize the aforementioned constraints on environments and protocols.
These decisions firmly promote programming patterns more closely related to existing development rather than an entirely unfamiliar set of constraints to work with. 

We design our construction as a wrapper around protocols, a common design pattern in UC, and a wrapper around ideal functionalities.
The wrappers offer new functionality to the underlying ITMs, and redfines exsting behavior by intercepting and reinterpretting outgoing messages.
The wrapper around the ideal functionality redefines how leaks to the adversary are handled, offers a new functionality to asynchronously execute code, provides the adversary with an interface to interact with the mechanism, and interacts with the protocol wrapper in a unique way we describe later.

Asynchronous execution in the ideal functionality is handled by a new function call every ITM has access to: \texttt{eventually}.
In theoretical notation, this is synactic sugar that replaces a message addressed to a specific ITM identifier which the wrapper intercepts.
The design is theoretically sound within the UC framework, however, we elide those details in favor of focusiong on its impact to development.
\begin{figure}
\begin{lstlisting}
fAuth sender (p2f, f2p) (a2f, f2a) = do
  forever $ do
    (pid, m) <- readChan p2f
    if pid == sender then
      leak m
      eventually $ writeChan f2p (pidR, m)
      writeChan f2p (pid, Ok)
    else ?pass
\end{lstlisting}

\end{figure}
At first glance, the simplified code is seen in Figure~\ref{fig:fchaneventually} where we define the same channel as in Figure~\ref{fig:fchanpoll} with our framework is trivially simple because it abstracts all model specific code.

We show a pseudocode definition of \fwrapper in Figure~\ref{fig:wrapper} for illustration and work through its different mechanics below. 
Alongside just code execution and implementing an asynchronous network, \fwrapper includes additional features that we introduce in order to simplify UC ITMs.
Namely, we choose to buffer leaks and introduce a new method of protocol parties passing control back to the environment with an activation path that goes through the adversary.

%There are two main ideas that motivate the construction of \fwrapper.
%The first is mentioned above: the applications being defined and developed should be agnostic to the network at hand, ideally.
%This is critical to ensure protocol and functionality definitions aren't unecessarily tied to a specific delivery mechanism or clouded by network-specific code/functionality.
%It also corresponds to an intuitive assumption about real protocols, in practice, that assume the environment they operate in will eventually force their message to be delivered. 
%The second idea is that import should be explicitly used as the main accounting tool in \fwrapper.
%Using import to account for delay and delivery ensures a more fundamental connection between import and evental delivery, and enables programmers to interact with the polynomial time definition and reaosn about protocol liveness in terms of the import required.\todo{i want to say something here about import as some function of the protocol parameters} 

\paragraph{Asynchronous Code Execution} \plan{Wax a little about the benefits of this over just messages: stateful execution in the future, can access state in the future when the adversary decides it's time}
Our first departure from existing models is extending asynchronous message-passing to asynchronout code execution.
The wrapper around the ideal functionality defines a keyword \texttt{eventually} that schedules a block of code to be executed with adversarial delay.
In the simple channel functionality \Fchan, the change to the functionality is laregly a costmetic, and it simplifies the definition, as shown in Figure~\ref{fig:fchaneventually}.
The advantage of this design is best realized by more complex ideal functionalities. 
For example a functionality that must perform varied stateful computations, receive multiple inputs from parties over time, and maintain a causal relationship between computations. 
The obvious place this occurs is in a blockchain functionality. 
Abstracting to code execution simplifies the machinery required to perform asyncrhonous computation, and it is a programming abstraction already familiar to most developers.
Furthermore, as we show in this section, the definition removes any trace of the underlying network model from ideal functionality and protocol code.
It enables programs to be entirely model independent and allowes software packages to be reused under arbitrarily stronger network assumptions.

\todo{include a figure for processing blockchain transactions, something real simple}

The parts of our proposed construction that deal with the evntual delivery guaratntee are neatly summarized in Figure~\ref{fig:wrapper}.
As expected from prior work, it mantains and queue and a counter.
The queue contains information about the asynchronously scheduled code block, and the counter is used by the adversary to add delay.
The delay in the queue is at least as much as the total number of items ever queued, so that simulation in the ideal world is possible, and the adversary can spend import to add to the delay.
In our model, the environment takes the role of protocol parties in attempting to force progress and decrement the counter.
When the delay reaches zero, the first item is popped off the queue and executed. 
Below, we first describe the remaining design details around the model, and then address the eventual delivery guarantee. 

%When the inner ITM is activated with this information, it executes the scheduled code written on its tape, but otherwise uses the values present on the tape at the time of execution.
%An ITM schedules a code block by attempted to write a message of an ITM with a special identifier.
%The wrapper interprets this information, saves the encoded callback information, and appends it to the \texttt{runqueue}.
%Every time a codeblock is added to the queue, the delay parameter is incremented as well.
%Finally, the information about the scheduled code block is added to the leak buffer for the adversary to read.
%For example, an authenticated broadcast functionality with no secrecy guarantee schedules a message to be send to each receiver and the adversary is still leaked the content of the messagesa long with information indicated that new code blocks have been scheduled.
%It is important to note that the leaks from the functionality and leaks from \fwrapper are different, and a functionality that leaks no information will still leak when a code block is scheduled as this is the minimal information that the adversary always needs.
%\todo{which to include, a conde example of Eventually or a pseudocode of eventually}
%%% scheduling asynchronous execution through `eventually` and its analog in implementation
%
%\todo{Figure out how import is charged}
%We introduce a keyword \Eventually that takes replcaes an explicit message from the ITM to \fwrapper in pseudocode, and takes the form of an implicit function available to all asynchronous ITMs in \us. 
%When an ITM schedules code to be executed asynchronously, it passes the necessary information to \fwrapper to activate the ITM again and execute the desired code. 
%It also results in the delay counter being incremented by one and control being returned to the calling ITM.
%An environment that tries to force progress in the execution spends one unit of import to \MakeProgress and decrement the counter.
%Conversely, the adversary can spend $n$ import to add $n$ delay to the counter. 
%This is an critical departure from UC where enviroment's can't normally communicate directly with the ideal functionality, and we discuss how composition changes when the environment is additionally only able to send \MakeProgress messages to \fwrapper surrounding the ideal functionality. 
%Nevertheless, we argue that it is essential that the environment be able to force progress in the execution, because it removes shifts network and model-specific complexity out of protocol and functionality specifications, and it more closely resembles how networks function in real life. 

\paragraph{Activating the Adversary and the Environment}
Our first simplifying design choice is defining a new way of returning control back to the environment.
In UC when an ITM doesn't activate any other ITM it passes control directly back to the environment.
Rather than giving control back to the environment, our wrapper around ITMs redefines this message as always going through the wrapper around the ideal functionality and activating the adversary.
Making this change primarily eases constraints on the environment to ensure the adversary is sufficiently activated and between every activation. 
The model ensures that the adversary is automatically activated after every honest party activation by the environment allowing it to always react immediately to new information. 

By redefining adversarial activation, and making it automatic, we're able to take further decisions that enable simpler and more natural protocol code.
The most common design for leaks is to activating the adversary directly like in \Fchan. 
This serves a common purpose of allowing the adversary to react, and allows the simulator to learn what is happening in the ideal world execution.
The method of activation above allows us to safely relegate adversarial leaks to be buffered in the ideal functionality, and default to the ideal functionality always returning control to the protocol party that called it.
Ensuring adversarial action above allows all ideal functionalities to always buffer leaks, by default, and always return control back to the calling party.
Without giving control back to the calling party, even things as simple as multicasting a message to $n$ parties requires the protocol to pause, wait for enough empty message activations from the environment to send the remaining messages, and implement similar logic to handle any action that requires multiple activations of the ideal functionality.
This is an unecessary artifact of the framework that we don't want application code to deal with. 

Though these are simple design decisions, they combine to greatly impact how protocols are defined. The pitfalls of existing design decisions and their negative impact on modularity are only apparent when attempting to realize UC protocols in an implementation.

%The UC framework defines a specific message that allows any ITM to given control back to the environment if it doesnt' write to any other ITM.
%In the standard UC implementation, every ITM has access to an implicit function \texttt{?pass}, as seen in Figure~\ref{fig:fchaneventually}.
%Our asynchronous wrapper around protocol parties and ideal functuionalities intercepts and redefines this mechanism to always activate the adversary first.
%When a protocol party wants to give control back to \Z, the wrapper activates the functionality with a special messages, and the functionality's wrapper forwards this to the adversary.
%In this way, any time the environment is activates, except when a party directly outputs a message to \Z, the adversary is automatically activated. 
%In general, every time an activation of the execution, by \Z, doesn't result in honest party output, \A is always the last ITM activated. 
%This does not interfere with the standard emulation definition as \Z is always eventually activated, and it ensures the programmer test cases (environments) focus only on protocol interaction rather than ensuring ``correct'' environments. \todo{environments other wise that don't activate an adversary enough or a protocol party enough, then they aren't even correct and its results unimportant}

%The UC framework uses a special entity called the control function that ITMs activate when they send a message to another ITM.
%An ITM that doesn't send a message to any other ITM at the end of its activation activates and gives control back to the environment~\footnote{This is not a direct message sent to \Z, because UC has a special \emph{control function} that handles this activation.}.
%We propose a new way to pass control back to \Z by explicitly activating \fwrapper and tracing activation through the adverasary.
%\fwrapper encodes a new message, called \msf{Pass}, that the code wrapping an asynchronoug protocol party uses as a replacement for returning control to the environment. 
%When the party \msf{Pass}es, it activates the wrapper around \F, which passes it through and activates the asynchronout adversary with a \msf{Pass} messages.
%
%We opt for this design choice for practical reasons that impact the design of protocols and constraints on environments.
%As mentioned earlier in this section, previous formulations of eventual delivery impose additional constraints on environments and protocol parties to ensure proper funtioning of the model.
%For example, and environment that doesn't activate the adverasry ``enough'' times results in a semantically useless execution. 
%Requiring environments needlessly forces a developer to think about the model and reason about activations for testing, and, especially for fuzz testing, exacerbates the difficulty of generating valid environments.
%With this design, the adverasry is always activated after every path of activation and given the opportunity to react to some new execution in the system.
%This is especially important for the ideal world adversary's ability to internally simulate the real world in lock-step with the ideal world without relying the environment to activate it. 


%\paragraph{Best Practice: Leaking to the Adversary}
%By redefining adversarial activation, and making is automatic, we're able to take further decisions that enable simpler and more natural protocol code.
%The most common design for leaks is to activating the adversary directly like in \Fchan. 
%This serves a common purpose of allowing the adversary to react, and allows the simulator to learn what is happening in the ideal world execution.
%Like we mention above, this leads to an unnatural protocol design.
%The method of activation above allows us to safely relegate adversarial leaks to be buffered in the ideal functionality, and default to the ideal functionality always returning control to the protocol party that called it.
%Simple actions like sending messages to $n$ parties can be easily done, and the protocol, as a whole, can perform all intended actions without also juggling ``losing activation'' and waiting for activation to continue executing pending activations.
%This is an unecessary artifact of the framework that we don't want programmers to deal with. 
%Our goal is to make UC programming as relatable for developers as possible.


%\paragraph{Eventual Delivery}
%In order to achieve eventual delivery, we depart from UC in some critical and nuanced ways.
%The first, and most important, design decision is that we shift the burden of eventual delivery from the adverasry competing against protocol parties to the adversary competing against the environment. 
%In order to achieve this, we allow the environment to activate \fwrapper around the functionality in one sepcific way. 
%Rather than protocol parties activate by \Z to poll for new messages, \Z sends a special \msf{MakeProgress} message to decrement the counter and force execution of the next asynchronous code block in the queue.
%Shifting the responsibility to the environment tracks with our initial criticism that network model behavior should exista the framework level rather than in protocol.
%We relax the balanced environments constraint and do not give the adversary the import that \Z spends in activate \fwrapper to \msf{MakeProgress}.
%Recall the constraint requires \A to get at least as much input as the rest of the execution, and in the pure UC framework this constitutes import given to the protocol parties only (as \Z can not activate any other main ITM).
%The import given by \Z to the wrapper around the ideal functionality is crucially not available to the ideal functionality to perform computation with.

\begin{minipage}{0.5\textwidth}
\begin{bbox}[title={Functionality $\F_\m{Multicast}(S, \mathcal{P})$}]
~
\begin{itemize}[leftmargin=*]
\item[--] on \inmsg{send}{$m$} from $S$ leak $(S,m)$ to $\mathcal{A}$ and

\qquad \qquad For $p \in \mathcal{P}$: \emph{eventually}: Send $m$ to $p$

\item[--] Send \m{Ok} back to $S$
\end{itemize}
\end{bbox}
\end{minipage}


\subsection{Eventual Delivery}
Previous iterations of asynchronous networks rely on an outdated notion of polynomial time in UC.
The eventual delivery model of Katz~\cite{katzuc} functions with the length-of-input notion of polytime.
The crux of this model is that the adversary must encode the delay parameter $T$ in unary, so it must write a polynomial number of bits for a polynomial amount of delay.
Expected behavior is that the computational power given to the protocol parties (to ``fetch'' new messages) is more than the input given to the adversary--given to both by the environment.

Directly translating this model to the import setting poses challenges necessitating a new construction.
A basic assumption made in this model is that the environment can always give more runtime to protocol parties, to allo \msf{fetch} calls, than the adversary.
The constraints of the import mechanism described in Section~\ref{sec:background} challenge this approach because the adversary is always given the same import as the rest of the execution.
This is for a good reason so that it can always adequately simulate the execution internally.
It isn't clear that relaxing the constraint for this model is an easy solution, because it would require specifying specific messages from \Z to \A or $\pi$ to be exempt.
Which messages should be exempt is a protocol-specfic question and not easily generalizable.
Fudamentally, the problem being encountered here is that \A is competing against protocol parties in this model, and there doesn't appear to be a simple or neat solution to rework the existing model without redesigning it from the ground up.

\paragraph{Our Eventual Delivery}
In order to achieve eventual delivery, we depart from UC in some critical and nuanced ways.
The first, and most important, design decision is that we shift the burden of eventual delivery from the protocol parties competing with the adversary to the adversary competing against the environment. 
In order to achieve this, we allow the environment to activate \fwrapper around the ideal functionality in only one specific way: a \msf{MakeProgress} message which consumes an import and decrements the delay.

Rather than protocol parties activated by \Z to poll for new messages, \Z sends a special \msf{MakeProgress} message to decrement the counter and force execution of the next asynchronous code block in the queue.
Shifting the responsibility to the environment tracks with our initial criticism that network model behavior should exist oustide the application layer.
We relax the balanced environments constraint and do not give the adversary the import that \Z spends in activate \fwrapper to \msf{MakeProgress}.
We retain the original intent of the balanced enviromment constraint as far as simulatability is concerned by still ensuring the adversary gets the same import from \Z as the protocol parties, but we add an additional rule to the constraint ensuring \msf{MakeProgress} calles from \Z to \F don't also give import to \A.

\todo{concluding paragraph to tie back to RQ}
%\plan{deprecated below}
%
%The usual structure of modelling asynchronous communication is thorugh adversarial
%delay of messages between parties. This commonly takes the form of the functinality
%passing control to the adversary and waiting for it to deliver the message, like the
%simple \Fchan functionality in Figure~\ref{fig:fchanadv}, or a functionality that 
%maintains counters for adverasrial delay and requires parties to repeatedly query for
%new messages like the \Fchan varianed in \ref{fig:fchanpoll}. 
%
%
%Asynchronous communcation in UC aims to achieve \emph{eventual delivery} guarantees by 
%using a polynomial time notion to conclude an adversary can only impose polynomial delay and 
%thus messages are deliever \emph{eventually}. In the first model described above an adversary
%that does nothing when handed control decidedly make it impossible for the protocol to
%terminate. Similarly, the second model of delay counting and polling relies on an outdated
%polynomial time notion that is known to have issues enabling non-polynomial behavior 
%from a system of ITMs~\ref{uc}. 
%
%In this section, we introduce a new method of capturing comuniation and \emph{computation} 
%models in UC. Our method uses wrappers to encode all model-specific behavior, uses keywords
%in both sofware and on-paper that compile to and abstract away the model, and we use
%the import mechanism to achieve polynomial guarantees like eventual delivery. We point 
%out that the abstraction presented here implements adversarially delayed computation
%rather than just communication, and it can be used to abstract arbitrarily complicated
%computation models like multi-party computation \todo{throwing that in there because it's cool}. 
%
%\subsection{Asynchronous Computation}
%The asynchronous wrapper acts as a shell process $\msf{SH}(.)$ around some existing ITM $M$. 
%It maintains a queue of \emph{shceduled} code blocks that $M$ can add to, and it extends an 
%interace to \A to query the state of the queue, delay the execution 
%
%
%It accepts messages from $M$ to schedule the adversarially delayed computation, and it extends
%an interface to \A that allows it to delay code blocks, query the state of adversarial delay,
%and deliver computation at will. 

