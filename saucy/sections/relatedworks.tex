There is a lot of work on the academic side of UC creating programming tools
and programming languages that apply formal methods to writing and proving UC
definitions~\cite{liao2019ilc, canetti2019easyuc} or to creating new
implemented cryptographic frameworks~\cite{barbosa2021mechanized,
morrisett2021ipdl}.   These works make the job of cryptographic proving easier
academics already familiar UC or other existing proving grameworks, but it is
unlikely bespoke languages or tooling will ever find their way out of
acacdemia.  On the other side of the bridge, there are many extant development
frameworks highlighted in literature that perscribe a common set of design
principles for distributed systems and create software tooling to encourage or
enforce them, but as we describe below many of them fail to meet the needs of
modern distributed protocols.  We cover these works in the remainder of the
section.

\subsection{Existing Frameworks}
Across the literature we survey, there are a few common themes of designs or
recommendations that these works share in common when it comes to developing
distributed systems.  Our discussion of these principles centers around
outlining the existing frameworks and their commonalities, drawing a parallel
to design with UC, and highlighting key areas where they fall short of the
needs of modern distributed systems.  We show that UC, at the very least,
exhibits many of the design principles detailed in existing frameworks, and, at
best, is uniquely poised as the only framework that allows for a generalized
approach to analyzing distributed systems with advantages that we demonstrate
in the rest of this work.

\paragraph{Layered Programming and Concurrency}
Layered programming refers to clean abstractions through well-defined
interfaces between different parts of a distributed protocol such as a
consensus layer and an application layer or a network layer and a message
handling layer.  The goal, in related literature, is to minimize shared state
across layers and implement layers as atomic operations that can't be
interrupted \cite{killian2007mace, bolosky2007farsite}.  Concurrency management
comes into play here as well, as every work we study highlights highly
concurrent code as both a source of new bugs and a hurdle to identifying, or
isolating, existing bugs in protocols.  To address this concern, existing
frameworks recommend design patterns that minimize concurrency within a program
and maximize the atomicity of actions~\cite{bolosky2007farsite} while other
works enforce it directly through formalism \cite{killian2007mace,
wilcox2015verdi}.  The UC framework places strong restrictions on data sharing
between diffferent protocols through the ideal functionality abstraction, and
its communication framework simulates running protocols in a highly concurrent
environment (with potentially many interruptions) while enforcing atomicity of
action.

\paragraph{Specification Language}
The third characteristic of a development framework that we identify is its
specification language.  Many of the related works rely on either a state
machine representation \cite{killian2007mace}, higher-order abstractions like
threshold automata~\cite{tholoniat2022formal}, TLA style
specifications~\cite{hawblitze2015ironfleet, bolosky2007farsite} of safety and
liveness properties, or other bespoke models~\cite{wilcox2015verdi}.  The
specification language helps encode behavior of the protocol when its correct,
and how protocols recover from crashes or adversarial action.  It goes
hand-in-hand with the fault models that the framework can express, and this is
a place where UC stands out from existing work.  Most works model only
crash-faults and the assumption simplifies the modelling language used.
TLA-style specifications, for example, are very good at specifying safety and
liveness properties but cannot support any kind of byzantine behavior let alone
specify probabilistic properties such as the distribution of the output of a
common coin protocol among a set of mutually distrustful parties.  Crash faults
also enable other works to specify properties and program state for single
nodes rather than whole executions.  Even the works that support byzantine
reasoning are limited in the kinds of properties or types of protocols they can
model.

The work by Tholoniat and Grimolie~\cite{tholoniat2022formal} models byzantine
distributed systems using threshold automata which describe protocols as a
system of counters with guarded transitions based on the numbers of types of
messages sent by parties.  It is a powerful, and useful, abstraction but
suffers from limititations that the ideal functionality model overcomes.  The
first is that for any moderately complex system, say a repeated consensus
protocol, the state space of the automata can explode, and specifying such a
state transition function can be cumbersome to do explicitly.  The ideal
functionality model on the other hand, describes the protocols as a minimal
program and the security relation of output indistinguishability is a
simplifying specification language.  Further, despite supporting byzantine
protocols, threshold automata are not useful for protocols where particular
distributions of the protocols output is desired, such as the coin flipping
protocol we implement or the shared common coin used in byzantine agreement
protocols.  Threshold automata can validate that the protocol follows a correct
path of execution based on guards but can not make any statement about the
output of repeated runs of it.  Recent innovations introduce probabilistic
threshold automata Threshold automata are also defined via fixed number of
honest and corrupt parties, and, therefore, struggle to model permissionless
protocols where parties can come and go.  Finally, the method of verification
of the consensus protocol in \cite{tholoniate2022formal} requires an additional
constraint of the adversary such that it is round-rigid: all parties complete
round $r-1$ before any party begins round $r$. 
%\todo{confirm, but: in the Ben-Or example that we study we isolate a bug in
%the protocol exactly through a distinguishing environment which violated
%round-rigidity}.  
The ideal functionality model and real-ideal relationship imposes no such
restriction on protocols.



%\paragraph{Common Desiderata}
%We discuss the common design principles espoused in existing work to make the point that UC at least satisfied framework design decisions accepted to be practical and useful for safe programming in existing literature rather than claim them as unique advantages to UC over others.
%The first is layered programming.
%Layered programming refers to large protocols composed from smaller atomica (and modular) programs with well-defined interfaces between different parts such as a consensus and an application layer or a network layer and a message handling layer.
%Layering protocols 
%
%The first is concurrency.
%Highly concurrent implementations are seen as both a source of new bugs and a hurdle to identifying, or isolating, existing bugs in the protocols.
%Its prevalence has led to a new term has being coined to refer to bugs that don't manifest in testing but appear in deployed systems due to new execution paths created by concurrency: \emph{heisenbugs}.
%To address this concern, existing frameworks recommend design patterns that minimize concurrency and maximize atomicity of the actions a protocol takes~\cite{farsite}, while others enforce it through programming abstractions~\cite{mace,verdi}.
%UC enforces similar concurrency minimzation through its communication rules that create, effectively, single-threaded programs.
%
%The second common theme goes hand in hand with concurrency: layered programming.
%Often state isn't shared accross layers and actions within a layer can't be interrupted and are atomic with respect to other layers.
%This is an intuitive design goal for developers in any setting, and UC exhibits this through the use of ideal functionalities to abstract and prove, in isolation, the security of a protocol.
%In implementation, the same layered programming principles carry over, however security under composition is a formal guarantee that informal analysis alone can't guarantee for protocol implementations.

%\paragraph{Comparison}
%Third is a specification language for the desired properties of the system.
%Many of the related works rely on either a state machine representation \cite{mace}, higher-order abstractions like threshold automata~\cite{formalbyz}, TLA style specifications~\cite{ironfleet,farsite} of safety and liveness properties, or other bespoke models~\cite{verdi}.
%The specification language goes hand in hand with the fault-models that the works consider.
%Most works model only crash-faults and the assumption simplifies the modelling language used.
%TLA-style specifications, for example, are very good at specifying safety and liveness properties but cannot support any kind of byzantine behavior let alone specify probabilistic properties such as the distribution of the output of a common coin protocol among a set of mutually distrustful parties.
%Crash faults also enable other works to specify properties and program state for single nodes rather than whole executions. 
%Even the works that support byzantine reasoning are limited in the kinds of properties or types of protocols they can model.
%
%The work by Tholoniat and Grimolie~\cite{formalbyz} models byzantine distributed systems using threshold automata which describe protocols as a system of counters with guarded transitions based on the numbers of types of messages sent by parties.
%It is a powerful, and useful, abstraction but suffers from limititations that the ideal functionality model overcomes.
%The first is that for any moderately complex system, say a repeated consensus protocol, the state space of the automata can explode, and specifying such a state transition function can be cumbersome to do explicitly. 
%The ideal functionality model on the other hand, describes the protocols as a minimal program and the security relation of output indistinguishability is a simplifying specification language.
%Further, despite supporting byzantine protocols, threshold automata are not useful for protocols where particular distributions of the protocols output is desired, such as the coin flipping protocol we implement or the shared common coin use by many modern asynchronous byzantine agreement (ABA) protocols.
%Threshold automata can validate that the protocol follows a correct path of execution based on guards but can not make any statement about the output of repeated runs of it.
%Recent innovations introduce probabilistic threshold automata 
%Threshold automata are also defined via fixed number of honest and corrupt parties, and, therefore, struggle to model permissionless protocols where parties can come and go.
%Finally, the method of verification of the consensus protocol in \cite{formalbyz} requires an additional constraint of the adversary such that it is round-rigid: all parties complete round $r-1$ before any party begins round $r$. \todo{confirm, but: in the Ben-Or example that we study we isolate a bug in the protocol exactly through a distinguishing environment which violated round-rigidity}.
%The ideal functionality model and real-ideal relationship imposes no such restriction on protocols.

\paragraph{Formal Verification}
Some of the works, especially the works that use TLA for specifying properties,
propose formal verification of their implementations
\cite{hawblitze2015ironfleet, bolosky2007farsite} Although this provides a
strong guarantee on the implemented protocols, it does little to aid in the
composability of verified code.  Notably, TLA specifications can be difficult
to compose generically, and can require manual intervention to create
larger-scaled specs for a composed protocol.  Even two of the same protocols
running side-by-side must be respecified, and re-verified, or defined in a more
restrictive formal logic on top of TLA which provides some form of generic
composability.  This further begs the question whether informal analysis
techniques, like we use in this work, are better from the perspective of
software development where repeated watsted effort is a hurdle to modular
development.

