% RESEARCH QUESTIONS:
%     (RQ1). Is UC suitable and practical as a development framework rather than only a theoretical framework?
%             (i). Can existing UC models/techniques be improved for an engineering purpose?
%             (ii). What are the advantages of using UC for development?
%     (RQ2). Is UC as a development framework compatible with existing informal analysis techniques?
%             (i).  Fuzz testing is widely used, is a successful analysis tool, and is itself an engineering undertaking. Can we successful apply fuzzing to UC?
%             (ii).  Does the ideal functionality model and out realization of import aid in analyzng liveness in distributed protocols?

% UC is leading framework but only recently we see it being used for distributed systems
% UC is good at expressing certain kind of properties easily through an ideal functionality 
Universal composability is the leading framework for defining the security of message-passing cryptographic protocols between mutually distrustful parties.
Though largely used for cryptography, it has experienced a surge of interest in the domain of asynchronous distributed systems literature within the last 10 years.
This is due, in large part, to the rising interest in modelling decentralized byzantine protocols, namely blockchain protocols~\cite{kosba2016hawk, badertscher2018ouroboros, miller2019sprites, badertscher2024bitcoin, dziembowski2018fairswap, aumayr2021blitz, kiayias2020composable}.
%The framework's appeal is two-fold. 
The framework has widespread appeal for a few reasons. 
First, the ideal functionality model for specifying protocol behavior allows the designer to express properties not easily captured via a property-based definitions, clearly define adversarial capabilities, and reason about what the adversary learns, and it reduces security definitions down to an indistinguishability relationship to an ideal execution.
Second, the modularity encouraged by the framework, and the subsequent composition operators and theorems, are well-suited to distributed systems where protocols sit on a bed of several other layered distributed protocols, and creating composable security proofs in isolation is a big advantage.
Finally, the computational model of UC allows proving liveness of a protocol within an eventual delivery system against a computationally bounded adversary--a core innovation of the framework.

% Async systems well-suited to the UC domain
% little work has attempted using informal setting and why? 
Protocols like distributed randomness generation (two-party coin flips, $n$-party common coin protocols) exhibit properties such as fairness, that may bounds on adversarial influence on protocol output, and require a specific output distribution. 
These requirements make asynchornous systems particularly well-suited to UC.
Most modern byzantine protocols rely such protocols as subroutines, or themselves exhibit similar properties, and the UC framework provides a generalized approach to specifying and defining their security in relation to a computationally bounded adversary and environment.
Despite these advantages of the framework, little work has explored the use of UC as an analysis \emph{and implementation} framework for distributed protocols, and to the best of our knowledge no generalizable framework currently exists which exhibits all of these features.
Implementations of UC definitions are often done in frameworks that bare little resemblance to it, and the resulting code has a weak connection to the proofs in literature.
The primary reason for the disparity is that UC is a subtle and complex framework that is considered accessible only to seasoned researchers or experts on the topic.
Furthermore the question remains whether the framework remains useful or even advantageous in an informal setting where proofs are replaced by software testing. 
UC definitions, themselves, also pose a challenge because existing proofs, ideal functionalities, and protocols can be large, difficult to understand, and often contain framework-specific details alongside core functionality~\cite{badertscher2024bitcoin, badertscher2018ouroboros}
%In this work we implement UC in a mainstream language, introduce a new model for asynchronous networks, propose new UC design principles to simplify and isolate user code from framework-specific code, and examine the framework through a few case studies.

% existing UC programming tools and their push towards more formality 
Existing literature focuses on creating programming languages for UC or formal programming tools that express UC in existing languages~\cite{canetti2019easyuc, liao2019ilc}.
Despite enabling executable UC definitions, most work aims to enable formal analysis or proof generation targetted towards researchers and do very little to bridge the game between theory and implementation.

%However, these works exacerbate the biggest challenge to UC adoption oustide of academics: is requires a lot of esoteric knowledge and expertise to parse the subtle details of UC proofs, constructions, and mechanisms.
%Translating UC definitions of protocols from literature to an implementation presents its own challenges.
%Definitions are highly technical and full of framework-specific or model-specific details, and implementing them in a non-UC programming framework requires significant tranlation work.
%Existing UC definitions for distributed protocols, existing largely in the blockchain space~\cite{bitcoinledger, ouroboros, perun}, consist of ideal functionalities, sub-protocols, and proofs that are many pages long. 
%Understanding and attempting to falsify such weildy definitions can be difficult, let alone implementing them correctly. 

% highlights of drawbacks of existing development/analysis frameworks in distributed systems
% TODO: talk about the analysis drawbacks as well
%Existing frameworks and tooling for implementing and analyzing distributed protocols fall short of performing such analysis.
%Several works, for example, rely on TLA-style specifications which are able to specify safety and liveness properties but not properties about probability distributions~\cite{a, bunch, of, them}.
%Furthermore, many of them tolerate only crash faults and are unable to express any properties related to byzantine adversaries.
%For example, even simple two-party com
%putations (2PC) require reasoning about how the adversary chooses its input, what it learns, and whether inputs of other parties is kept secret.

% the central claim and research questions of this work
In this work we instantiate UC as a development framework for asynchronous distributed protocols and explore its use for implementing and analyzing them.
Notably, we take the opposite approach of existing works and only consider informal analysis of protocols within the framework.
Concretely our research questions are stated below, and we expand on, and discuss, them in the remainder of this section.
\begin{enumerate}
\item How are asynchronous eventual delivery protocols realized in UC, and what are the design consequences?
\item What are the advantages or drawbacks of developing protocols in UC compared to existing frameworks?
\item To what extent can the advantages of the real-ideal paradigm and ideal functionality model be captured by informal analysis of UC implementations?
\end{enumerate}
Our focus on \emph{informal analysis} extends from our stated goal of minimizing the barriers to accessibility.
We select fuzz testing as our informal analysis tool of choice, because test case generation and analysis fits well within UC's emulation definition (more detail in Section~\ref{sec:background}) for ensuring indistinguishability of a protocol with an ideal program for all environments (the test cases of UC) and adversaries.

% talking about the first research question
Our first challenge in implementing a UC framework for asynchronous development is contending with the new \emph{import mechanism} for polynomial time in UC and understanding the consequences of its design for eventual delivery and liveness.
Existing asynchronous models in UC work with the outdated lenght-of-input polynomial time notion, which has been shown to permit infinite executions through a straightforward protocol.
Existing definitions also impose additional design constraints on ideal functionalities, protocols, and environments that unnecessarily force framework and model-specific code to exist within application code.
The import mechanism centers around import, the fundamental unit of potential runtime that is exchanges as ``tokens'' between machines, giving them the potential to perform polynomial work based on the net import held. 
To the best of our knowledge, no work has explored how eventual delivery is defined with import, what the consequences of it are for protocol design, and how liveness properties are expressed.
Concretely, we propose a new eventual delivery model that uses import, and we outline a set of design decisions that minimize its impact on user-defined code and maximally automate parts of the framework that previously required manual instrumentation.
Further, we discover that the existing constraints of balanced environments and ``parameterized systems'' around import pose hurdles to realizing eventual delivery and require a minor departure from the framework as defined by Canetti~\cite{uc}. 
They prohibit the necessary difference in computational ability between the adversary and the rest of the execution that is required to achieve eventual delivery.
Finally, we apply fuzzing and informal analysis towards testing the liveness properties of distributed protocols.

% second research querstion
As far as we know, no work has explored using UC as an implementation framework relying only on existing informal analysis tools, commonplace in modern software engineering to test and located bugs. 
Of course, such approaches can not guarantee correct code, but we believe that programming within UC simplifies the required assertion to the single property of UC-indistinguishability of a protocol with an ideal functionality.
Relying on ideal functionalities to describe protocols, and the real-ideal paradigm to test them, is an unexplored concept, and a generalizable approach to distributed systems programming that we believe is an improvement on the status quo.
We address this claim, and the associated research question above, by specifically examining whether relying on UC emulation (real-ideal indistinguishability), suffices to catch implementation bugs in UC implementations. 
Specifically, do implementation-level bugs manigest themselves as distinguishing environments (test cases) that fuzzing can find.
Furthermore, we confirm that UC implementations, and specifically fuzz testing, allows implementations to analyze protocols for properties related to fairness and probability distribution.\todo{maybe mention existing work}.
The experimental method in this work consists of implementating candidate distributed protocols, injecting bugs into them which we believe are representative of reality, and examining the false negatives and false positives with the lens of our research questions.
Our findings suggest that for non-liveness bugs, even simple fuzz test generators are capable of discovering distinguishing environments for most bugs that lead to protocol property violation.
\todo{say something about the other bugs that don't manifest themselves are a result of un-stated properties about efficiency of a protocol or distributions of output}.
\todo{rephrase the following better as higher-level concluding thoughts}
The cases in which we fail are a consequence of our fuzzing strategies, and this is expected because fuzzing is only a method of finding bugs not ruling them out.
In the case of liveness-related properties, cycle-checking is made easy as part of the UC proof obligation for real-ideal emulation, but neither our eventual delivery model nor our attempts at being clever with import yield much success beyond that. 

Our attention now turns to ideal functionalities and protocols that express properties relating to output fairness, adversarial influence on input/output (related to adversarial knowledge), and their effect on the distribution of outputs.
The example we use is the canonical coin flipping primitve that is used throughout distributed systems literature and construct a ``fair'' lottery out of it.
The lottery is a canonical example of the kinds of byzantine protocols common in the decentralized setting, but is a protocol not readily expressable in existing distributed development frameworks.
\todo{finish}
The canonical example we use here is the coin flips and ``fair'' lotteries which have become important byzantine protocols in the blockchain space, and implementation frameworks must be able to express thes properties in a protocol's specification, 
Some non-blockchain protocols, such as the ABA protocol we study in this work, make use of distributed protocols to generate shared common coins that must specify what the adversary learns about the execution and its ability to influence the outcome. 
Existing frameworks fail to express and/or analyze protocols that require such definitions.
Not only does the ideal functionality enable greater expressiveness than existing specification frameworks in this situation, but UC's 


%Exising development and analysis frameworks fall short of desirable properties in that most lack support for byzantine protocols and for obvious reasons.
%The disparity is easily explained by the prevalence of crash-fault distributed systems in production over byzantine.
%Those that support, such as the work of Tholoniat and Gremolie~\cite{formalbyz, bymc} can not capture properties 
%and those that do, namely the work of Tholoniat and Gremolie~\cite{formalbyz, bymc} can not capture properties about adversarial influence on, or the distribution of, protocols output. 
%A simple lottery protocol, common in decentralized systens, requires fairness in availability and distribution of output, but existing framewor
%
%
%Next we implement and examine candidate byzantine protocols in this framework and analyze them for bug detection.
%There is a wealth of related frameworks that aim to support layered programming, propose a specification language for checking implementations, and some that formally verify implementations.
%The main drawback of existing works is the lack of support for byzantine protocols, and the disparity can be explained by the relative prevalence of crash-fault distributed systems compared to byzantine ones.
%%importance of crash-fault distributed systems in large scale computing compared to byzantine systems that have only recently gained widespread popularity.
%Even so, the underlying principles and techniques that these works make use of, like TLA logic to express properties, are fundamentally incompatible with the needs of the byzantine setting.
%Some works do aim at byzantine modelling for distrtibuted systems still fall short of the expressive power of UC that we aim to realize.
%For example, work by Tholoniat and Gramolie~\cite{formalbyz, bymc} uses threshold automata which can suffer from state space explosion for distributed systems with many states, can not model permissionless systems where parties may join or leave the protocol, and requires consensus algorithms be analyzed under a round-rigid adversary condition. 
%A protocol which may violate safety under an adversary that can delay parties long enough for them to fall behind can not be caught under this constraint.
%The ideal functionality model, on the other hand, expresses a desired protocols as a minimal program that expresses the properties of the protocol as a whole rather than specifying the behavior of only a single node, as in several existing works. 
%It also explicitly encodes the adversary's capabilities, and so can express desired properties about what the adversary learns about an execution of a protocol and how it can influence its outcome. 


% non-byz: fairness isn't important in non-byz and probability distributions aren't important: xor of all inputs is random and works 
% verdi: semantics for different network models, verified system transformers, extend with non-determinism for byzantine faults
% mace: single node specification because that suffices under all honest assumption
% formal byzanting verification: 

%We make the claim that a UC development framework can be used for engineering and testing distributed protocols informally, and 
%A UC-based development framework may greatly benefit programmers by offering a generalizable approach to protocol definition and testing against byzantine adversaries, and it may greatly benefit the broader UC community to create software artifacts that are easy to play with, test, and use as building blocks.
%We explore this claim by answering the following research questions:
%\begin{enumerate}
%\item Can we make better design choices to realize the UC framework and necessary network models to mimize the exposure of protocol and functionality code (user-implemented) to the underlying framework?
%\item Is the real-ideal paradigm useful for testing and analyzing protocols with informal analysis tools commonly used in software development?
%\end{enumerate}
%Security under composition is a more formal statement, and it's not clear whether there exists a straightforward validation method to substantiate it in this setting.
%
%
%The first observation we make is that US definitions tend to carry framework-specific or model-specific code alongside application-specific code.
%From a theoretical perspective, this is fine, but from a programming perspective we want minimal intereference from the framework or network model in the application itself.
%Programs and applications need to be used by other protocols, understood by programmers, and not require re-engineering for every new setting it might be used in.
%In this work, we focus on realizing and analyzing asynchronous distributed protocols, and apply these design principles to propose a new model that guarantees eventual delivery.
%Our model takes inspiration from existing works, but makes uses the import mechanism for polynomial time rather than previous, flawed, notions used in current literature.
%The proposed construction defines a wrapper around ideal functionalities, expands upon thw drawbacks of conventional UC design, and defines a new set of simple but effective design choices to automate and shift UC compleixy away from program code.
%As we discuss later in this work, the our design choices removes significant responsiblity from both protocos and environments to orchestrate the execution correctly (we defined what this means later).
%The benefits of our construction also carry over our validation strategy for the second question we answer.
%
%\todo{explain why distributed systems}
%
%In order for UC to be a useful development framework, its inherent advantages mentioned above should manifest themselves as advantages for engineering protocols.
%The obvious structural advantages give a generalized approach to writing better, more modular, code, and the ideal functionality is a more concise and expressive representation of a software package for others to interact with. 
%Despite the lack of formal guarantees on security under composition, programming within the structure provides at least some informal relationship between the ideal functuionaliy building blocks and the protocol being realized.
%The more impactful benefits are taking advantage of the real-ideal relationship as the security definition of an implementation, the environment's ability to control the adversary and adaptively choose inputs for the execution, and the 
%Finally, we care that informal analysis tools and software testing practices can be effective at analyzing protocols written with UC by relying on the indistinguishability relationship.
%
%\todo{mention that we care about bugs manifesting in the real-ideal relationship}
%We validate our claim and answer our second research question by implenting UC in a mainstream programming language, implementing asynchronous distributed protocols, and analyzing them using fuzz testing. 
%Fuzz testing is an ideal choice for UC analysis for a few reasons. 
%First, UC emulation is all about ensuring a property of the real and ideal executions holds for all environments, and relying on test case generation to create environments is a natural fit. 
%Second, fuzz testing is an effective way method of discovering bugs, even compared to formal approaches like symbolic execution, and it is widely used today in software engineering that programmers are familiar with.
%Third, the design principles that UC encourages lend themselves well to fuzz testing because protocols can be made easily separable through the ideal functionality abstraction, and the smaller state space should increase the abilities of fuzzing.  
%
%In our experiments we implement three asynchronous byzantine protocols of differing complexity. 
%The most relevant and complex protocol, an asynchronous binary agreement protocol (ABA), is an intended improvement to original version of the well-known agreement protocol by Mostefaoui et al. (dubbed the MMR protocol). 
%We implement these three protocols, and implement a fuzz testing framework for generating environments and adversarial strategies.
%In order to detect protocol implementation-level bugs through the real-ideal relationship, we inject bugs representative of those expected during development and run the faulty protocols through our fuzzing apparatus. 
%We first examine the protocols for failures unrelated to liveness or termination, and find a previously unknown vulnerability in the ABA protocol that our fuzz tester discovered.
%Furthermore, we find that for all three protocols our fuzzers only returned true positives, and failed to find a distinguishing environment in only one case (discussion in Section~\ref{sec:aba}.
%Next, we turn our attention to using our new import-based approach to polynomial time to see whether the real-ideal relationship can be useful for analyzing the liveness properties of protocols.
%Our initial approach of using the simulator proof mechanism for detecting liveness in protocols proved successful, but we observe that non-determinism in protocols makes it difficult to avoid false positives.
%Finally, we ask whether import and the environment's ability to interact with it directly aids in detecting liveness bugs.
%Our experiment shows that import ends up being minimally useful in detecting liveness bugs more successfully than the naive simulator approach, however, it is still useful for providing insight into how and when a protocol terminates with respect to the distribution of honest party inputs and adversaries. 
%
%The remainder of the paper is structured as follows.
%Section \ref{sec:background} and \ref{sec:relatedwork} describe the UC framework in more detail, along with the import mechanism, and discusses existing work in informal protocol analysis, previously proposed generalizable frameworks for distributed programming, and existing models for asynchronous networks in UC.
%Section \ref{sec:wrapper} introduces our asynchronous network model, it's unique abstraction to asyncrhonous code execution, and how it achieves eventual delivery.
%Section \ref{sec:aba} dives into our experiments with fuzzing, our setup, how our fuzzing apparatusm works, and the results of our experiments. 
%Finally, in Section \ref{sec:liveness} we examine the open problem of liveness analysis using fuzzing and our proposed eventual delivery model.

