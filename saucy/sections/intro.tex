Universal Composability (UC)~\cite{canettiUC} is the leading framework for defining security properties of cryptographic protocols.
It is considered the strongest definitional model since it guarantees the security properties hold even when the protocol is arbitrarily composed with
multiple concurrently-executing sessions of other protocol.
UC has gained popularity for analyzing cryptographic protocols due to its \emph{ideal world/real world} simulation mechanism.
In contrast to game-based cryptography where security properties are defined via attack games,
UC defines the \emph{ideal functionality}, a trusted third party that serves as the \emph{protocol specification}.
A core feature of the frameowk is its modularity where complex protocols are defined in terms of simple, ideal functionality building blocks that are secure by definition. 
A major drawback of UC is that security proofs can be quite complicated and difficult to analyze. 
Many related works \cite{ilc, easyuc, ipdl, etc} attempt to formalize UC security through a new programming language or defining UC security in an existing formal verification lanuage, and,  
although useful, such tools frequently never make their way to software engineering practice because 1. niche programming languages aren't suited to large scale development and can be difficult to use, and 2. the added proof obligations on the programmer can be extensive~\cite{ironfleet}.
Utimately, UC-secure prtocols end up being implemented in software frameworks that do not replicate UC, and, therefore, my invalidate on-paper proofs without further security proofs of the code.

We address this gap in UC-driven software develoment by exploring how inforal security analysis of UC definitons, in our implementation of UC, can identify, and aid in eliminating, security vulnerabilities.
A key controbution of our software development framework for UC is proposing a noval new abstraction for capturing network assumptions. Our abstraction makes use of the novel import mechanism 
for polynomial time and fits nicely as a software abstraction. Our implementation and network model are, to the best of our knowledge, the first concretiziations of the import mechanism in this way.
Prior attempts at modelling asynchrnous networks, for example, focus primarily on adversarial delay of messages between parites. Such notions can require
protocols to encode signficant model-specific behavior which clutters functionalitiy (and protocol) definitions, places unecessary restrictions on protocol design, and is counter-productive for modular and reusable code. 
Out abstraction, on the other hand, acts are a wrapper around ITMs and offers a notion of \emph{asynchronous computation} in a way that is UC-compatible and firs well within a software framework. 
Not only is it natural for software development, but it also reduces functionality and protocol code to be almost model-agnostic. 
We use our network model of computation to focus solely on modeling and analzing distributed protocols in this work.
UC is most notably a framework for cryptographic protocols, however, in recent years the emergence of decentralized systems has renewed focus on modelling the security of asynchronous byzantine networks in UC~\cite{many,cit,ations}. 
Decentralized, namely blockchain, systems are highly modular with many protocols sharing state in unexpected ways and relying on numerous shared distrbuted sub protocols.
Naturally, compositional security in UC is ideal for capturing such protocols. 

We opt for fuzz testing, by generating environments, as our analysis tool for three important reasons. First, numerous prior work has demonstrated the success of fuzz testing at identifyin software bugs. 
Some work suggests fuzzing is comparable in success to even formal approach such as symbolic execution. 
Second, the UC framework lends itself to compact definitions that compose through ideal functionalities making the input space for protocol parties and adversaries much smaller. 
Combined with our simple network model, UC modularity makes it easier for generated environmenst to explore more of the state space for a particular protocol or simulator proof. \todo{is this setting us up with an obligation to prove this statement with some coverage testing?}
Third, the real/ideal paradigm, and the ideal functionality, provide a built-in specification against which protocols can be tested and a method for comparing the two (the UC experiment). \todo{this last one is the least good the point is that we don't have to define state machines and added spec on top of a protocol, they should already exist from on-paper definitions it isn't an additional obligation to the programmer when using haskell saucy fuzzing.}

Among the few related works that apply fuzzing to distributed systems, the work by Jepsen goes so far as to apply their methdology to a decentralized byzantine consensus protocol called Tendermint.
Jepsen deploys compiled Tenderming binaries and tests that operations on a distributed database are linearizable under various network conditions and limited byzantine behavior. 
As they admit in their results, the byzantine behavior that they capture is limited to replicated simple scenarios signing keys for multiple nodes. Designing a ful byzantine node for Tendermint is a considerable engineering effort.
Part of the hurdle is that testing a monolothic application like the Tendermint binary requires instrumenting and implementing every sub-component in the whole protocol.
Conversely, it is not possible, within \us, to test secrity under clock skews or race conditions in multi-threaded handling of network messages like Jepsen is able todo.
Hoever, this isn't a limitation of \us, but highlights a key distinction between us and works like Jepsen. \us is a \emph{development framework} rather than only a testing framework.
An application like Tendermint, implemented in \us, ensures that sub-components like network handling and clock timing are implemented and tested in isolation.
As mentioned in the previous paragraph, this makes allows us to test race conditions, should they arise, and capture byzantine complicated byzantine behaviort that Jepsen does not. 
\todo{feels like the point is there but not stated clearly enough. I'm trying to relate to the point of reduced input space from the previous paragraph to talk about why byzantine modelling is easy in \us and that not being able to do clock-skew type testing is a consequencer of Jepsen not being a development framework and having to work with existing monolithic binaries.}

\subsection{Below this is just notes}

\us is a not only a testing framework but a \emph{development framework} where subcomponents and sub protocols are created and tested in isolation.

Combined with paper and pencil proofs, we can reply on the composition theorem to test more complex protocols like Tendermint without worrying about the full code stack.
For example, even though \us can doesn't allow testing of race conditions resulting from multi-threaded handling of network messages 

our framework ensures that such sub-components of Tenderming are created and tested
within UC and composed correctly. 

Jepsen fuzz test a compiled binary for Tendermint and check that updates to a distributed database are linearizable under various network conditions and some byzantine behavior.



fuzz testing is a proven technique even when compared to formal analysis tool. this shows promise as a simple informal tool for protocol analysis
compared to things like jepsen running and testing protocols is much easier without any of the engineering effort require because of the subtle ways in which UC is designed to mimic a realistic computing environment and network
fuzzing byzantine messages is easier with UC for a few reasons:
* the simple interface for adversarial behavior makes it easy to play with message ordering and deliver / dropping messages
* the ideal functionality abstraction removes sub-protocol detail making the space of byzantine messgaes much smaller / manageable for fuzz testing

this leads into another point about comparing with jepsen. they test existing compiled binaries and so end up testing the full stack of code as part of their testing
they can test race conditions in how messages are received, for exampe, and standalone UC fuzz testing can not replicate that
but this is precisely where UC shines: not just testing code but developing it within the UC framework all such sub-protocols are tested and designed individually and larger, composed protocols only testing teir behavior with the ideal functionality model 



We address the state of software development around UC by exploring the extent to which informal security analysis of UC definitions, in our UC implementation can identify, and aid in elminiating, security vulnerabilities such as safety, correctness, and liveness in distributed systems .\todo{this first line needs to be better}.  
A key contribution of our software development framework around UC begins with proposing a novel abstraction for capturing different network models.
Prior attempts to capture, for example, asynchronous communication focus primarily on adversarial delay of messages between parties. 
Such notions can require protocols to encode significant model-specifi behavior in their definition which lutters definitions, places unecessary restrictions on protocol design, and is counter-productive for modular and reusable code. 
Our abstraction, on the other hand, acts as a wrapper around ITMs and extends the abstraction to asynchronous \emph{computation}: adverasrially delayed code rather than messages.  
Not only is this abstraction more natural for a software setting, but it also reduces functionalities and protocols to become almost model-agnostic in their definition. 
For asynchronous networks we dub our construction the \emph{asynchronous wrapper}. 
It uses the import mechanism introduced in UC to provide \emph{eventual} deliver guarantees.
To the best of our knowledge, it is the first concretization of the import mechanism with eventual delivery. 



\todo{a statement wrapping all this up in a key takeaway and positive result of this work in the context of the state of UC}.
