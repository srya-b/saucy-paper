%In the previous section, we explore how we can use fuzz testing and our implementation to successfully test and analyze protocol properties across laybered composition.
%We specifically focused on properties around fairness and adversarial distributions and tested for design-level bugs arising from subtle differences in the different models of coin flipping.
In this section, we set out to examine the real-ideal definition of security as
a candidate for testing UC implemenations.  There are two questions we seek to
answer towards this goal.  First, is the emulation definition capable of
detecting implementation-level bugs?  Put another way, do implementation bugs
manifest themselves as disinguishing environments that can be checked relying only
on real-ideal emulation?  Second, do the inherent advantages of the ideal functionality
model of specificaiton translate to advantages in a development setting?  
As discussed earlier, specifying protocol behavior as a program, the ideal functionality, makes specification easier than property-based definitions for protocols where properties like safety, secrecy, and fairness might be highly intertwined and not easily separable.
%As mentioned earlier in this work, the frameworks security definition and 
%adversarial modeling choices allow for more sophisicated properties to
%be expressed and checked compared to other frameworks. 

In general, we are also interested in testing implementations that aim to achieve
output fairness or limit adversarial influence on some distribution.  We
observe that the properties of a simple coin tossing protocol can't be
specified or checked by existing work. To the best of our knowledge no
frameworks, especially implemenation frameworks, exist that provide a
generalizable method of testing arbitrary byzantine distributed protocols.  To
confirm or refute out hypothesis, we apply fuzz testing as our analysis
methodology for analyzing protocol failues in UC, because analysis by automated
test case generation seems a natural fit for UC's emulation definition
(indistinguishability against all possible enviroments).  We explain our
approach in more detail below, and put off discussing liveness failures until
the next section.

Our application of fuzz testing proves very successful for the safety-like
properties we are concerned with here.  For each of the example protocols we
use (described below) we observe a high true positive rate and almost no false
negatives.  The most notable result of this sesion is discovering a previously
unknown bug in the ABA paper by Crain~\cite{aba}, and it is a critical positive
result for the larger case we are making in this work.  We confirm this bug
with both the protocol pseudocode and the text describing it in detail.  The
outcomes of fuzz testing are discussed in their respective subsections below.

Ths section is laid out as follows. First we introduce our methodology for
evaluating our UC implementation. Second, we highlight the three candidate
protocols we will examine. Then we describe our approach to defining test case
generators and more detail on how we go about testing each protocol. For
brevity we only focus on the ABA protocol in this section, as it is the most
interesting example, present the high-level results from the other two
protocols, but relegate them to the appendix.  Third, we look at a coin toss
example and build a lottery protocol out of it.  Recall that these protocols
differ from traditional distributed protocols in that they require notions of
fair distribution.  They are especially important in the blockchain domain due
to the financial incentives involved, and are representative of the kinds of
protocols that form the backbone of many decentralized applications.
%In this section, we set out to examine whether fuzz testing can be a useful
%tool in UC to detect implementation-level bugs.  Specifically, we keep our
%focus on the real-ideal paradigm and ask whether implementation-level bugs
%manifest themselves as dinstiguishing environments, and, therefore, ask whether
%the real-ideal assertion is uniquely useful for finding implementation level
%bugs.  In particular, we care that bugs in the protocol which do lead to
%failures in properties specified in an ideal functionality can manifest as
%distinguishing environments using existing analysis methods.  Alongside the
%real-ideal relationship, we want to show that the environment and adversary in
%UC create a generalizable method of testing protocols against a strong
%byzantine adversary.  %To the best of our knowledge no frameworks like UC
%exists that proposes a generalized and uniform way of specifying distributed
%systems, their properties, and their security that enables straightforward
%composition of protocol implementations.  To confirm or refute our hypothesis,
%we apply fuzz testing as our analysis methodology because the approach of test
%case genration is a natural fit for UC's emulation definition of ensuring
%indistinguishability against \emph{all possible environments}.  We explain the
%experimental setup in more detail below, and put off discussing liveness
%failures until the....

%Of course, the ideal functionality is not capable of capturing all desirable properties of protocol. 
%For example, it may be useful for capture properties predicated on the protocol being in a specific state in a specific round.
%We discuss out ability to assert these kinds of properties at the end of the section.

%%%%% Experimental setup: implement protocols, implement generators, choose bugs that may or may not cause failure, apply fuzzing, and determing true/false positive/negative
\paragraph{Experimental Setup}
We answer the first question above by pitting UC and fuzz testing against
implementations of three candidate distributed protocols.  The first step is
correctly implementing the three protocols in question: Bracha's well-known
broadcast primitive~\cite{bracha}, Ben-Or's randomized agreement
algorithm~\cite{benor}, and an optimized version~\cite{aba} of the well known
asynchronoud byzantine agreement (ABA)  protocol by Motefaoui et
al.~\cite{mmr}.  next, a set of environment generations (our test cases for
fuzzing) are designed that aim to discover points of failure by exploring
different adversarial input strategies and schedules.  The implementations are
then perturbed by implementation bugs which we choose with the belief that they
are representative of common bugs encountered in the wild.  Many of the
injected bugs are generic and likely to occur in any protocol implementation,
and the others are specific to the protocol in question.  We strictly adhere to
this order of steps to ensure that the tests and generators we define aren't
influenced by the bugs we choose in way that tailors them to a positive outcome
for this work.  Finally, the generators are applied to the broken
implementations. The results of our fuzz testing, specifically examining where,
when and why we encounter true or false positives and negatives, gives insight
into using the UC framework as an implementation and analysis tool.

%%%%% what do false positives and true negatives mean and they aren't useful to answering this question, we assume we start with a correctly implemented protocol
\paragraph{Classifying Experimental Outcomes}
True positives are classified as bugs for which a distinguishing environment
exists and was produced by fuzzing, and a false negative means the former
condition is true but our fuzzer didn't find one.  A true negative means our
fuzzer corretly fails to find a distinguishing test case (none exists), and a
false positive means a distinguishing test case was found by the fuzzer where
none should exist.  False positives in our case are unexpected, and we larlgely
categorize them as a failure on our part to implement an adequate simulator for
the real-ideal experiment or a previously correct implementation of the
protocol.  False positives include cases in which bugs we inject cause only
code-level failures (e.g. index out of bounds errors or concurrent failures)
rather than protocol-level failures and are discarded because they don't
advance our understanding of our research question.  %False positives
encountered in our analysis of liveness in the next section are not discarded.
True negatives are the class of outcomes where we expect a correct protocol to
be correct. 
%These test cases are sanity checks for our implementation and on our methodolody that it is able to recognize passing tests.

%%%%% Which protocols do we choose, focus on ABA, and we find a bug in ABA
\paragraph{The Protocols}
In this work, we implement three byzantine protocols: Bracha's broadcast
protocol~\cite{brachabcast}, Ben-Or's randomized agreement
protocol~\cite{benoragreement}, and an ABA protocol~\cite{aba} that is a more
modern optimization of the well-known byzantine agreement protocol by
Mostefaoui et al.~\cite{mmr} (dubbed the MMR protocol).  We choose these
protocols for a few reasons.  We wanted both probabilistic and deterinistic
protocols, protocols that used randomness in different ways from one another,
and were different enough from each other to require different approaches to
their generators.  Bracha's broadcast is the simplest of the three protocols,
and deterministic, making is a perfec first protocol to tackle as we get our
bearings.  The Ben-Or protocol is slightly more complex and uses local
randomness generated within parties to ultimately make decisions.  The ABA
protocol is the most complex of the three (also the most modern), uses a
broadcast sub-protocol, and relies on distributed randomness generation through
the use of a common coin amont the parties.  Naturally, we expect this protocol
to have the greatest surface area for bugs and faults.  We also expect bugs to
manifest themselves more quickly in this protocol owing to its shared
randomness, and, therefore, smaller expected runtime, and it's fine-tuned
parameters (for efficiency) compared to the Ben-Or protocol.  For the remainder
of this section we limit the discussion to the ABA example and give only high
level results for the other two protocols.
%Our testing generators need to be broad and general purpose to find failures resulting from a variety of different reachable states in the protocol.
%We also expect that liveness analysis, which we discuss in the next section, should also be affected by the difference in length of the protocols.

%\todo{when to mention this result??}
%Through our fuzz testing of the protocol, we discover a previously unknown bug in the ABA protocol.
%This is a positive result for the case we are making, as implementations, and their analyses, of UC definitions is critical to achieving better protocol definitions and possibly enabling more complete analysis by being concrete.
%We confirm this bug with both the protocol pseudocode and the text describing it in detail.

\subsection{Asynchronous Byzantine Agreement (ABA)}
%\plan{Explain the ABA protocol we use and the messages involved. \m{EST} messages and \m{AUX} messages and how a value is decided by theh coin flip}
The ABA protocol has a set of $n$ parties each propose a binary input $x_i$,
and eventually all parties decide on a single value.  More formally, the
protocol aims to guarantee the following properties:
\begin{itemize}
\item \emph{Termination}: Every non-faulty process eventually decides on a
value.
\item \emph{Agreement}: No two non-fault processes decide on different values.
\item \emph{Validity}: If all non-faulty processes propose the same value, no
other value can be decided.
\end{itemize}
All three of these properties can be captured by the ideal functionality for
binary agreement in Figure~\ref{fig:faba}.  The protocol description given by
the ideal functionality lets the adversary determine the input as long as the
input must have been given by at least 1 honest party (more than $t$ number of
the input).  Otherwise the decision is is determined by which input is given by
the honest parties. The eventual send at the end of the functionality is handled
by our asynchronout model. 

\begin{figure}
\begin{minipage}{0.5\textwidth}
\begin{bbox}[title={Functionality $\F_\m{aba}(\mathcal{P})$}]
~
Let \honest be the set of honest parties and \crupt the corrupt ones

Let $\msf{nt} = \msf{nf} := 0$ be the number of ``1'' and ``0'' inputs from the parties, respectively

Let $\msf{adv} := \bot$ and $\msf{decision} := \bot$

\begin{itemize}
\item[--] on \inmsg{$v_i$} from \A, set $\msf{adv} = v_i$

\item[--] wait to receive \inmsg{input}{$v_i$} from all $p_i \in \mathcal{P}$, increament \msf{nt} if $v_i = 1$ or \msf{nf} analogously. Send \m{ok} back in the mean time.
    
\item[--] once received from all $\mathcal{P}$: 
\begin{itemize}
\item if $\msf{nt} > t$ and $\msf{nf} > t$ and $\msf{adv} \neq \bot$:

\quad set $\msf{decision} := \msf{adv}$

else if $\msf{nt} > t$ then set $\msf{decision} := 1$

else if $\msf{nf} > t$ then set $\msf{decision} := 0$

\item for $p \in \mathcal{P}$: Eventually send \m{decision} to $p$
\end{itemize}

\end{itemize}

\end{bbox}
\end{minipage}

\caption{Ideal functionality for an ABA protocol with $t < \frac{n}{3}$. The
adversary gets to choose the agreed upon value given a sufficient number of
either value.}
\label{fig:faba}
\end{figure}

%%%%%% Describe SBroadcast
\paragraph{Broadcast Primitive}
The ABA protocol relies on instances of a broadcast primitive called
\msf{SBroadcast} (shortened \msf{SBCast}).  Two \msf{SBCast} instances are
running at any time one for each possible input value 1 or 0.  The two
primitives determine whether $2t+1$ parties have broadcast a value with
$\m{EST}(\cdot)$ messages and ``deliver'' the value to the main protocol if so.
%The primitives listens for other parties proposing a particular value and
``delivers'' the value if $2t+1$ parties have similarly broadcast the value
with $\msf{EST}(\cdot)$ messages.  At the start of the protocol, a party with
input $x$ creates an instance $\msf{SBCast}(x, True)$ which starts by
broadcasting $x$ and an instance $\msf{SBCast}(\not x, False)$ only listens and
echos $\not x$ if at least one honest party broadcasts it.  When a vaue is
broadcast as a proposed value we say the protocol ``supports'' this value. The
delivered values and the output of the common coin, which determines what
values parties propose in the next round, determine what new instances of
\m{SBCast} are created.

%%%%% What the main protocol does once a value is delivered
\paragraph{Main ABA Protocol}
The main protocol tracks what values have been delivered (\emph{bin\_ptr}$[b]=
True$ if so) and how many $\m{AUX}(b)$ messages have been received.  Once one
of the values, $b$, is delivered it broadcasts $\m{AUX}(b)$ as the value it
tries to decide an waits to receive $n-t$ $\m{AUX}(\cdot)$ messages of any
kind.  It waits till there exists a set \emph{view} exists where for $b \in
view$, $bin\_ptr[b] = True$ and an $\m{AUX}(b)$ message was received for it.
The next step is waiting for the output of the common coin once it has been
called by $t+1$ honest parties.  For some party, if $c \in view$ they support
it next round by only listening for $\not c$ in case others support it. If
$view = \{c\}$ the party also decides $c$.  Otherwise the party ``supports''
$\not c$ next round with a new broadcast.  At a high level, parties only
propose the opposite of the common coin in the next round if $c \notin view$.
The values of $bin\_ptr$ are unchanged from round to round unless a new
instance of $\m{SBCast}$ overrides it to $False$.  The notable departure from
the original MMR protocol~\cite{oldmmr} is the distinct instances of
\msf{SBCast} that can persist through to future rounds.

%The protocol begins with the two instances of \msf{SBcast} above, and waits until some value is ``delivered'' (i.e. some \emph{bin\_ptr}$[b] = True$).
%It broadcasts an $\msf{AUX}(b)$ for the first value delivered and waits until it receives $n-t$ $\m{AUX}(\cdot)$ messages such that a set \emph{view} can be formed where for $b \in view$, $bin\_ptr[b] = True$ and an $\m{AUX}(b)$ messages was received.
%Once a satisfying set is found, call the common coin.
%The strong common coin only returns a value after being called by $t+1$ honest parties.
%Based on the result of the coin, $c$, and the set \emph{view}, either decide on a value if $view = {s}$ and listen with $\msf{SBcast}(\not s, False)$, have $view = {0,1}$ and support $c$ with $\msf{SBcast}(\not c, False)$, or have $view = {\not c}$ and support $\not c$ again next round with $\msf{SBCast}(\not s, True)$.
%The key idea here is that parties only propose the opposite of the common coin if the coin value was never delivered or seen in $\msf{AUX}(\cdot)$ messages.
%Otherwise, they go into the next round with $bin\_ptr[c] = True$ and listen passively for $\not c$ with a new instance of \msf{SBCast}.

 
%%%%% A note on the types of failures: difference between property failures w.r.t ideal functionalities and throwing errors based on assumptions being violated
%%%%% We don't care about the throwing errors because it has nothing to do with the real-ideal paradigm or the properties of the protocol it's just bad code on our part
%\paragraph{Types Of Failures}
%In development, we expect two types of failures to arise as a result of injecting bugs into our protocol code.
%The first is the one that we care most about: ones that lead to execution output that allows an environment to distinguish between the real and ideal worlds.
%The other is more fundamental failures such as deadlocks arising from the concurrency used within our implementation that isn't designed to handle the bugs that we inject.
%These failures are a result of our method of injecting bugs rather than a consequence of the bugs themselves.
%For example, a particular part of our correct implementation can make assumptions on a list never being empty or a map always containing a specific key, and such such assumptions may fail as a result of a bug that we inject.
%Though important, we don't consider such failures in this section, because they result from our own failure to implement a correct protocol rather than distinguishing behavior that the injected bug causes to occur. 
%It could also be considered a failure on our part to inject bugs in a way that also corrects all such assumptions the code makes about it.
%In these cases, we ignore these failures and fix them.
%\todo{whats a better way to word the reason for why we don't consider such bugs}
%
%\todo{ false negatives where real-ideal couldn't catch it, but also the ideal functionality can't define it, but it is a bug according to the paper specification like delaying decision by a few rounds. }

%\paragraph{Implementing Protocols in UC}
% The only extra care required is the write token execution ordering that had to be followed, but we still had to create threads to handle incoming messages and handlers for those messages and then a main thread that waits for those handlers and when to write output to the environment was tricky but nothing too tricky with a bunch of IORefs and channels for inter thread communication. Threads weren't all just running and reading a variable they had to be acitvated first so there had to be a path from one process to every other process including ?pass
% STEP: I don't think this is worth talking about we don't really care about it beyong it being a different way of programming and is it worth it?

%%%% Describe generators
\subsection{Environment Generators}
The generators we define for all the protocols produce environments that range
from completely random (but still non-trivial) inputs to what we call
structured environments that follow the protocol but try to force specific
deviant protocol states.  As we show in this section, the small design surface
encouraged by UC compositional design results in indivually testable protocols
with a relatively small state space and makes devising and executing
adversarial strategies, as fuzz test generators, straightforward.  For example,
despite ABA being a relatively complex distributed protocol with many moving
parts, we were able to detect most of the injected implementation bugs by
introducing small variants to a single specific generator.

%%%% fuzzing caveat the "art" of fuzzing
\paragraph{A Limtation, and Future Workk}
We do not employ any tooling to determine the coverage acheived by our fuzzers.
We design generators that we believe cover the various adversarial strategies
out there and should suffice to catch all of the implementation bugs we inject.
In places where we encounter false negatives, where our generators aren't able
to create the necessary conditions for the indistinguishability failure
associated with a bug, we address the reasons and whether this is a fundemental
limtiation of analysis or failure of our approach.  An obvious place for future
work is to add more sophisticated testing techniques, like code coverage
metrics, into our analysis framework.
% TODO: save for later
%\begin{center}
%\emph{$\diamond$ Fuzz testing for the protocol still requires insight into the protocol mechanism in order to understand what search space is interesting to explore.}
%\end{center}

%%%%% we design generators AGNOSTIC of the bugs we choose
%Due to our experimental strategy, we are careful to design generators that explore protocol states that are intuitively most likely to exhibit protocol property violations, rather than designing generators to target the specific bugs we choose to inject.
%For searching for safety violations in agreement protocols, for example, it is intuitive that violations are more likely to be found when parties are partitioned on input and selectively receive messages from within their partition, rather than randomly delivering messages between arbitrary pairs of parties.
%The latter strategy will also, eventually, lead to scenarios where a safety violation may occur, however, it spends a lot of time exploring states that aren't useful.
%The liveness bug in the oritinal MMR protocol, identified in \cite{formalbyuzantine}, follows an input trace that does exactly these high level partitioning strategy.
%Our generators, take these high level strategies and use randomness to explore different combinations of inputs.
%Later in this section, in our discussion of analyzing liveness for distributed protocols, we inject a bug that reduces the ABA protocol to the faulty MMR protocol, and our fuzzing output suggests its existence.
%\todo{make sure this aligns with what we say in the later section}

%%%% The most basic generators and they are useful for easy to catch bugs like threshold parameter tweaking and types of failures mentioned above that have nothing to do with real-ideal or the properties
%%%% these never present as distinguishing environments because the program crashes, therefore we fix and move on

\begin{figure*}
\begin{lstlisting}
-- Step 1: give parties arbitrary input
forM honest $ \h !$\rightarrow$! do
  b !$\leftarrow$! generate arbitrary
  writeChan z2p (h, ABAInput b)

forM [1..rounds] $ \r !$\rightarrow$! do
  c !$\leftarrow$! queueSize
  -- Step 2: choose some way to deliver messages
  case deliveryStrategy of
  	-- shuffle all current queue items and deliver them
    AllRandom !$\rightarrow$! (shuffleListM c) !$>>=$! deliver
	-- deliver queue items in order, the 0th index
    Sequential !$\rightarrow$! forM [1..c] $ deliver 0
	-- delivering all ESTs first (shuffled) then all the AUX messages
    ProtocolOrder -> do
      (allEsts r) >>= deliver . shuffleM
      (allAuxs r) >>= deliver . shuffleM
\end{lstlisting}


\caption{A simple ``dumb'' generator that doesn't to any protocol specific work
or targetting, but just loops and schedules message delive}
\label{lst:simplegen}
\end{figure*}

\paragraph{Simple Generators}
Thre are two kinds of environment generators we define.  The first type are
simple, or ``dumb'', generators that aren't reacting to the state of the
protocol, but randomly choose some scheduling strategy, a priori, before
observing protocol state, arbitrarily choose byzantine messages for corrupt
parties, and always deliver all messages to ensure that the protocol can make
progress.  In most cases, this means that all blocks in round $r$ are executed
before any in round $r+1$.  In other cases, the environment makes random
decisions to deliver all current scheduled blocks in the queue or throw
everything against the wall and don't make any guarantees other than taking all
actions at random.  We create these generators with the expectation \emph{that
they will help us find simple bugs, ensure that our correct implementation
doesn't return any true positives, and that our testing infrastructure doesn't
return any false positives}.  They can also be considered sanity check test
cases on our implementation of both the protocols and our fuzzing framework.
In Figure~\ref{lst:simplegen}, we show an example of a dumb generator that
performs a simple set of actions.  As we will find out in the next section,
where we study liveness properties, environments that always ensure protocol
progress are especially useful.  Simple generators mostly operate in the
all-honest setting, chooses random input, and then proceeds for some number of
rounds and delivers all the code currently waiting in the queue (hence checking
our protocol at least does what it's supposed to in the happiest case).

%\todo{Mention here that we test reactively rather than determining an input trace before hand and executing it. This is a key difference between what fuzz testing normally does.}
%We run these generators in the three important failure models: all honest, crash fault, and byzantine.
%We further divide these environments by the message deliver strategies they use.
%The generators proceed in loops where different choices are made on delivering messages.
%\emph{These environments, which we consider ``first-pass enviromments'' are useful in detecting the simplest design and implementation bugs that cause errors to be thrown, cause the protocol to enter into a locked state, or never make it past the first phase or round.}

\begin{figure*}
\begin{lstlisting}
-- partition parties on the input they receive
pidsT !$\leftarrow$! partition honest
pidsF !$\leftarrow$! honest \\ pidsT

-- start execution by only giving broadcast messages within partitions
estTtoT !$\leftarrow$! intersectM (estTrue 1) (byReceivers pidsT)
estFtoF !$\leftarrow$! intersectM (estFalse 1) (byReceivers pidsF)
deliver (estTtoT ++ estFtoF)

forM [1..rounds] $ \r -> do
  -- step 1: let some parties receive both inputs in broadcast
  partition !$\leftarrow$! subset honest
  -- deliver the intersection of messages for p and any arbitrary EST message
  forM partition $ \p -> (intersectM (byReceiver p) (arbitraryEst 1)) !$\gg =$! deliver
       
  -- step 2: give corrupt EST messages to the same group
  -- generate 5 messages, shuffle them and execute
  (cuptEstMsg partition inputs r 5) !$\gg =$! execCmd . shuffleM

  -- step 3: give crupt AUX messages and deliver all AUX messages to all
  -- generate 5 byzantine messages and shuffle them with deliver instructions for remaining AUX messages
  cruptAuxs !$\leftarrow$! cruptAuxMsg honest inputs r 5
  auxs !$\leftarrow$! allAuxs r
  shuffleM (auxs ++ cruptAuxs) !$\gg =$! execCmd

  -- step 4: give all remaining EST messages to all parties in this round
  -- deliver them in random order
  allEsts r !$\gg =$! execCmd . shuffleM
\end{lstlisting}

\caption{The code for the the most basic ``smart'' generator we use for
checking ABA. We exclude print statements and the environment setup code that
is common to all generators.}
\label{lst:genaba}
\end{figure*}

\paragraph{Smarter Generators}
The generator in Figure~\ref{lst:genaba} is a slightly smarter generator which
targets a failure mode where parties might decide on different values or never
decide any value but confirm both.  Partition the parties on input and only
deliver broadcast messages within each partiion for the respective inupts.
Choose some random set of parties, som possibly that are inter partitions, and
allow them to exchange broadcast messages so that some parties might deliver
two values or don't decide on anything this round.  Finally, give all parties
enough \msf{AUX} messages so that they progress to the common coin and
eventually start the next round.  In this round, some party may decide the
output of the common coin, and in the next round the generator selectively give
messages so that parties that didn't decide this round may decide a different
coin value if the protocol is problematic.  Variants of this environment turn
out to be sufficient for catching a lot of the bugs we inject into the protocol
that do cause indistinguishability failures.  \emph{In general, we see that
understanding the decision points of the protocol make defining useful
generators easy for safety-like faults.} Crucially, note that the generator
isn't assuming the existence of a particular bug but targetting a particulat
protocol state.  Variants of this generator are different in whether they
deliver the remaining \msf{EST} messages within the same round, or are more
selective, based on the common coin in round $r-1$, about which messages are
given to which parties in round $r$.

%In Figure~\ref{lst:genaba}, we show an example of a smart generator for the ABA protocol.
%The generator doesn't target any specific vulnerabilities, but targes an intuitive place where protocol failures are likely to happen.
%For the safety property (and even for liveness properties we discuss later) it is clear from understanding the protocol that the key to forcing failure should be cases where the \emph{view} determined by parties prevents them from deciding or allows a party to decide a different value in the different round.
%There are simpler scenarios where liveness or safety failures may occur, but the intuition is that this is where the more subtle failures can be forced.
%The generator in Figure~\ref{fig:something} does exactlty this

%First, the generator chooses a random partition of the protocol parties based on the input it will give them.
%It them delivers \msf{SBroadcast} messages between node in the same partition. When enough nodes are in each partition this ensures the different partitions \msf{SBDeliver} their own values.
%Next, it chooses a subset of the honest parties and gives them all \msf{SBroadcast} messages as well as byzantine \msf{EST} messages.
%This may cause some parties in either partition to deliver both values.
%Finally, it generates corrupt \msf{AUX} messages of arbitrary value, and delivers all \msf{AUX} messages pending in the execution.
%Finally, it delivers any remaining \msf{EST} messages in that round before looping.

%The generator in Figure~\ref{fig:genaba} is quite versatile as it gives a best attempt to ensure the protocol makes progress (all messages within the same round are eventually delivered before any from a future round) and can even explore scenariors where single parties are isolated and specifically targeted over multiple rounds.

%\emph{$\diamond$ For distributed protocols, the modularity of UC encourages small design, and a small surface for potential bugs and failures that are more easily detectable than testing monolithic code bases.}

\plan{Andrew: a paragraph commented out about the output quickcheck gives when a failure is detected, should I include something like that?}
%\paragraph{Fuzzing Output}
%We check properties by finding distinguishing environments between the two worlds. 
%The first step is that our generator makes random choices based on its structure and outputs the execution trace that it runs against the real world.
%This trace, along with the randomness used in the real world, is replayed in the ideal world. 
%The two executions output a transcript of the outputs seen, by \Z, from the honest parties and the adversary in each execution.
%In the case of ABA, we replay the randomness used in the real world, therefore it suffices to simple check equality of the transcripts.
%If the transcripts are identical, our fuzz tester considers this successful and move on to the next generated environment.
%If the equality check fails, the fuzz testing halts, and outputs the environment's input trace to the programmer.
%For example, an input trace of a failing test case in the ABA protocol consist of inputs of the form:
%\begin{lstlisting}
%Right (CmdDeliver 2,0)
%Right (CmdGetLeaks,0)
%Right (CmdDeliver 6,0)
%Left (CmdEst ("(\"sbcast\",\"Dave\",1,False)","(\"Dave\",[\"Alice\",\"Bob\",\"Charlie\",\"Dave\"],\"\")") "Alice" 1 False 64,0)
%\end{lstlisting}
%A sequence if deliver commands for specific indices in the runqueue, adversary queries for leaks, and byzanting input.
%The byzanting input (the last) command is the most verbose owing to the identify structure of the underlying \Fchan ideal functionality being parameterized by the sender (\emph{Dave}), the receiving parties (\emph{Alice, Bob, Charlie, and Dave}), the round number (1), the input (Flase or 0), and the recipient (\emph{Alice}).
%In reality, the testing framework we have implemented is smart enough to output more meaningful representations in the input trace such as (sender, recipient, round, msg) tuples, but the raw information above is what the programmer can actually use as input to replay an execution and investigate the failure.

%\paragraph{Even More specialized Generators}
%As we mention above, we define several environemnt generators for the ABA protocol. 
%The more specific generators that we define perform specific actions such as choosing one party, or a set of parties, to isolate regardless of input.
%The test generator that suggests the same liveness bug that was found in the original MMR protocol works by selecting one party arbitrarily out of four and attempting to force is to always deliver the value opposite the coin toss.
%The hypothesis behind this environment is: \emph{if we can successfuly force one party to always deliver a value in a round that is opposite to the result of the coin flip then it will never decide any value}.

\subsection{Detecting Bugs}
%The common use case of QuickCheck is definging a set of properties that assert specific things about the output of programs.
%Optionally, QuickCheck includes a set of combinators to combine properties, define predicates for ``useful'' properties, and quantify results/failures over different test case generators.
Traditionally, fuzz testing with QuickCheck, at a basic level, requires
defining the desired properties of the program in terms of generators, and
combinators to combine them, for input and an assertion of the desired
property.  In UC, we are specifically interested in the indistinguishability
property of a real-ideal execution, so our assertions in this section checks
for equality between the transcripts output by the two executions.  Our ability
to replay randomness in the simulator for the real world ensures that for full
information protocols (all of the protocols we look at) the transcripts will be
exactly equal despite their use of randomness.  The QuickCheck properties we
define for every protocol (for safety-like properties) all make the exact same
assertion: equality of transcripts.

\paragraph{An Existing Safety Bug in ABA by Crain~\cite{aba}}
The first step in our experiments was to attempt to validate our ``correct''
implementation of the protocol with our fuzzing apparatus.  Specifically, we
want to ensure that our generators allow the protocol to progress through
rounds while still executing the strategies that we encode and assert that we
no distinguishing environments are found.  The Bracha broadcast and BenOr
protocol returned no positives assertions of failure, but we did observe one in
the ABA protocol.  Specifically, we ran the generator defined in
Figure~\ref{lst:genaba} on our correct (or so we thought) implementation of the
ABA protocol and arrived at a distringuishing environment after several hundred
iterations.  The output that our fuzzer provides is the input trace of the
distinguishing environment which we replayed and discovered a design flaw in
the ABA protocol rather than a failure of our implementation.  The input trace
we present is simplified for the $n=4$ case rather than the $n=6$ case that the
trace returned.  Parties $p_1, p_2, p_3$ have inputs $1, 1, 0$ respectively.
The adversary makes $p_1, p_2$ deliver $1$ and $p_3$ deliver $0$ by controlling
the \msf{EST} messages they received.  As a result, $p_1, p_2$ broadcast
$\msf{AUX}(0)$ and $p_3$ broadcasts $\msf{AUX}(1)$.  The adversary forces $p_2$
to also deliver $1$, and then delivers all $\msf{AUX}(\cdot)$ messages to all
parties.  Parties flip the common coin $c$. If the outcome of the coin is
$c=0$:
\begin{itemize}
    \item $p_3$ decides $0$ because $v_1 = \{0\}$. $p_1$ starts next rounds by
        proposing and broadcasting $0$ because $v_2 = \{1\}$. $p_2$ switches to
        supporting $0$ (without broadcasting) because $v_3 = \{0,1\}$. 
    \item In the next round the adversary makes $p_1,p_3$ deliver $0$ and $p_1$
        decides $0$ if $c_{r+1} = 1$ %\todo{finish and confirm}
\end{itemize}
Thus two parties were able to decide two different values.  It is clear from
the execution trace that parties ignore the existence of other delivered values
when receiving an $\msf{AUX}(\cdot)$ message that contains them.  We identify
that the bug, which we confirm from both the pseudocode and text description of
the protocol, is that the set \emph{view} is ``computed from the values
included with $\msf{AUX}[r](b)$ messages received $(n-t)$ processes for which
the corresponding \emph{bin\_ptr} variables point to a true Boolean''.  If we
alter this statement to require that for every $\msf{AUX}(b)$ received the
protocol waits until $bin\_ptr[b] = True$, the bug is resolved and no safety
violation is observed.  We correct this bug, re-run all of our fuzzing models
on our correct implementation to find any other such failures before proceeding
to inject bugs into it.

\paragraph{Detecting Bugs}
The bugs that we inject into the protocol span common bugs we expect to appear
in any asynchronous distrubuted protocol and some protocol-specific bugs.
Common bugs that we intuitively think may commonly occur are ones like
incorrectly set threstholds for phase/state changes, mishandling message
validation, or erroneously modifying state accross rounds.  The more subtle
bugs that we inject are protocol-specific and do things like invert the
\msf{supportCoin} bit at the end of every round in ABA, or revert the protocol
back to the original (flawed) MMR protocol by spawning two new instances of
\msf{SBCast} every round.  We summarize the bugs we inject into the ABA
protocol and the outcome of our fuzz testing on them in the table in
Figure~\ref{table:aba}. Some of the bugs listed cause liveness failures, and we
put off discussing them until the next section.

% smaller protocols testing in isolation => smaller surface
% understanding the protocol's intent => the approach taken here vs in BenOr was different because we understand that with the common coin, the protocol should decide quickly so we can limit our environments to ones that don't need to always guarantee they execute some long strategy or continue to meaningfull explore the protocol beyond a few rounds
%                                     => discount less test cases or need to refactor test cases to not waste effort
% smaller UC like protocols => places where failures are likely to occur are easy to determine and target
% real ideal paradigm => we never had to read the transcripts to extract useful infromation and check specific properties and preconditions, just assert equivalence no need to actually deal with the protocol output in a protocol-specific way
% fuzzing lets us quantify over what input distributions produce an error and determining the source of the failure is made asier
First, we inject all bugs on their own and check for distinguishing
environments.  As we expected for both ABA and the BenOr protocol, we never
observe any false positives.  In the case of ABA, our fuzzing apparatus was
able to find every true positives and never returned a false negatives.  In the
case of BenOr, we return true positives for the simlpest bug (threshold
perturbation and improper validation of round numbers in received messages) but
return false negatives for some other relatively simple bugs.

We highlight a specific bug and test case in the BenOr protocol because it
backs up our previous intuition about this protocol, and confirms that well
designed generators can produce distinsguishing environments that perform
long-range exploits.  As we expected the BenOr protocol was resilient to most
bugs due to its tight corruption threshold, and the distinguishing environment
we did find came as a result of a targetted generator that attempt to censor
specific parties in every round--effectively leaving their messages undelivered
in the queue until later.  The bug injected removes proper message validation
for round numbers in the protocol. This is a relatively simple bug that is
easily exploited in the case of ABA \todo{(below)}, but BenOr's safety
thresholds mean finding a safety violation is non-trivial.  Exploiting the bug
requires an attack where parties must be evenly split, the party inputs are
preserved across the coin flips, and pre-decision messages are saved (and
replayed) from sufficient numbers of parties in previous rounds to force two
parties to decide two different values in the same round, Our environment test
cases, though seemingly suited to this exploit, still took several hundreds of
test cases before performing this exploit for even a few rounds, let alone the
7 rounds required for $n=6$ parties.  Nonetheless, the success of our fuzzing
apparatus in discovering even long-range attacks makes that case that fuzz
testing with UC can work through large state space explosion with well designed
generators. 

In summary, Ben-Ors low corruption threshold and use of local coin flips that
the adversary can't influence, mean it is very safe, and many of the bugs we
injected into the implementation never produced distinguishing environments but
only delayed termination/decision.

%\todo{Mention approaching testing by controlling the randomness tapes of particular parties?}

\paragraph{Fuzzing for Property-Based Definitions}
As an aside to the aim of this work, we discuss here that UC does not limit the
ability to test specific properties about a protocol that aren't expressed in
the ideal functionality.  The ABA protocol, for example, has several lemmas and
theoresm on paper reasoning about performance based on specific state
transitions of collections of parties.  An implementation resulting from
literature requires that the implementation framework can also assert desired
state transitions or reason about how performant the protocol is.  We used the
unaltered UC implementation and found that we are able to check specific state
transition by relying on the adversary and the leaks it receives from the
message-passing functionality.  This ensures that aside from just being to
check if a protocol is safe or eventuall terminates (we discuss this in the
next section), we can still do traditional test of specific desired properties
that aren't expressable in the ideal functionality.

%The distinguishing environment discovered for the BenOr protocol was the most complex test case generated so far by any of our generators.
%Briefly, the distinguishing environent that we found executed a clever strategy, dependent on the sequence of coin flips made by the parties.
%With $n=6$ parties, 1 corrupt party, and an even split in 0 and 1 inputs, the environment relies on a specific sequence of coin flips made by the parties.
%Specifically, it requires that the parties keep their original proposed value for a few rounds and one party switches at some point.
%The adversary strategy that our generators would have to achieve requires the parties to be evenly split on their input, and the coin flips to maintain the size of that split over several rounds.
%The adversary then collects and saves all proposed decision messages from one of the $n$ parties in each of $n$ rounds. 
%In the $n+1$th round, the adversary gives deciding messages of 0 to three parties and gives deciding messages for 1 for the other two honest parties. 
%The protocol requires $\frac{n+t}{2}$ messages attempting to decide a value in order for any party to decide, and the environment was able to sufficiently buffer enough such messages from each of the parties and finally deliver them to allow two parties to decide on different values in the same round.

%In comparison to ABA, where protocol termination is expected to happen in $O(1)$ time, creating generators to exploit bugs proved much easier because the probabilities invovled are greater.
%Recognizing this fact, the generators we defined for BenOr were far more specific than those for ABA.
%The generator that found this distinguishing environment, for example, was specifically set up as a one that would arbitrarily choose parties to censor communication between in specific rounds and make random decisions of when to deliver the backlog of messages.
%Without creating a generator like this, we aren't hopeful that this failure would be found without considerable time to ensure more properly crafted general-purpose generators.
%
%In comparison, the more modern and complex ABA protocol proved much easier to analyze and find distinguishing environments from.
%As shown in Table~\ref{table:aba} just about all of the bugs which produces failures were discovered using simple environments rather than the more specific targetted ones for the failures we study in this section.
%Even the bugs that exist only in the SBroadcast, similar to the primitives BVBroadcast used in \cite{formalbyz}, were easy to catch and identify from analysis of the ABA protocol. 


%It is clear that fuzzing can be used to find distinguishing environments caused by implementation level bugs relatively easily, but smarter generator techniques are required to catch bugs in randomized protocols with long expected runtimes. 
%For example, a simple construction that may turn the round number bug from a false negative to a true positive is the ability to prematurely terminate test cases when it's goal is no longer achievable. 
%In the case of BenOr terminating test cases early when the local coin flips don't maintain the protocol state that we want may help.

The success of fuzzing for finding distinguishing environments for non-liveness bugs conforms to our original hypothesis.
These are a few key takeaways.
\begin{enumerate}
\item Protocols like ABA, which aim to be efficient and performant, simple generators suffice to catch many bugs. These protocols terminate quickly making requiring less sophistication of generators. On the opposite end, protocols like Ben-Or with long expected termination time require far more sophistication. Regardless, our fuzzing approach was able to find long-range attacks against the flawed implementations.
\item Intimate protocol understanding is required to even begin to catch bugs in more expansive protocols like BenOr where local randomness balloons the state space. More dynamic and adaptive adversaries are needed to find bugs that take several rounds to discover. 
\item How the protocol uses randomness, shared or local, has a big impact on the ability of generators to target and explore interesting state space. Distributed decision making, for coin fips for example, open the protocol to adversarial influence on outcome. 
\item The real-ideal paradigm suffices to catch most of the safety-like bugs we inject and for protocols that are finely tuned like ABA. The failure to catch certain bugs in all the protocols studied indicates that existing practices of code coverage and more sophisiticated methodology for implementing generators is required. This is a ripe direction of work in the future. We anticipate bringing in more of the existing wisdom and techniques from fuzzing should open the doors to better analysis for increasingly complex protocols.
\item Traditional testing techniques are still possible with UC and fuzzing by examining the adversary's leaks and asserting specific state transitions.
\end{enumerate}

In the next section, we cover a more challenge class of properties: liveness properties.
They are key to the definitions of asynchronous distributed protocols, and we examine whether the real-ideal paradigm combined with fuzzing and our asynchronous model can assist developers in finding liveness bugs.

%In the case of BenOr, many of the bugs, aside from the obvious threshold perturbations, failed to produce a distinguishing environment at all.
%Surprisingly we encounter a false negative, a bug that we couldn't manually find a disttinguishing environment for, that our fuzzing apparatus found.
%This has to do with a simple bug where the protocol doesn't sufficiently validate the round numbers  
%
%This raises an important point that design for shorter protocols, ABA, can be significantly simple because failures are likely to arise quickly if they do exist. 
%For longer time frame protocols with a lot of unshared randomness, it is important to design protocols which can persist over many rounds and find a problem.
%There appears to be a tradeoff in longer running environments and adversarially complex environments. 
%Engineering a generator to attempt a strategy that is specific, revert to making progress if unsuccessful, and attempting again every round is difficult. 
%Even more so for bugs that require a very specific environment to appear but the protocol can take exponentially many rounds. 
%
%Surprisingly, even a simple, and intuitively problematic bug, not validating round numbers in messages is a fals negative case, because the corruption threshold, round transition thresholds, and decision threstholds made it impossible to collect sufficient messages from previous rounds to replay and force a distinguishing event like a safety violation.
%To the best of our attempts, our fuzz testing never produced a false negative, and this follows our intuition based on the protocol's design to be more secure than needed with a low corruption threshold~\footnote{The author even stats that the $\frac{n}{5}$ corrupt bound may not be tight.}.
%A conclusiton that we draw from this result is that an unexpected output from testing may be a signal that the protocol can be made more robust against corruptions withtout sacrificing the desired properties.
%We repeat the same experiments by combining bugs that were true negatives by themselves and are equally successful in detection.
%
%
%
%
%
%This matches our expectation that modern protocols are susceptible to failure even with small perturbations.
%Compared to BenOr, many of the bugs we injected did not lead to failures, and we attribute this to the ``research'' nature of the protocol. 
%The author also admits that the corruption bound is not nececssarily tight, therefore, the protocol is secure even against bugs that we would expect to cause failure in ABA.
%
%The effort we put into constructing simple yet effective generators makes that case that following modular UC protocol designs yields easier to test protocols. 
%Their input space is limited, they permit a few crucial places where failures are likely to occur, and testing is mroe easily able to force deviant behavior.
%In general, understanding the protcol well is still requires, as naive generative techniques, ``dumb generators'', aren't as useful for more subtle bugs.

% results for ABA => injecting by themselves we were able to catch every bug that caused a distinguishability failure, bugs that did not force failures that present themselves in the output that the environment observes were not caught
% for such bugs we must fall back to traditional test as these have more to do with either causing the protocol to not finish or delaying decision by a few rounds. The latter means the protocol still does waht the ideal functionality specifies
% because these properties can't be expressed in the ideal functionality. Protocols that provide additional guarantees like termination after a specific event must be specified as property and checked through the adversarial leaks observed


%Clearly, the incorrect thresholds are the most likely place to discover failures in the protocol.
%In the ABA protocol above, we expect it to cause failures violating all three properties: \emph{validity}, \emph{termination}, and \emph{safety}.
%For the other bugs, we arrive at a \emph{limitation of our validation strategy}:
%\begin{center}
%\emph{$\diamond$ Bugs for which a distinguishing environment may exist, but the generators we devise don't discover them.}
%\end{center}
%We encounter this limitatiom much more readily when we focus on liveness and termination properties.
%
%Trying to prove, analytically, the existence of distinguishing environments as a result of specific bugs is proeblematic as well, because it leans too far into defining generators for bugs.
%Despitie this limitation, we are still able to detect distinguishing environments for different combinations of injected faults and identify, using the input trace, where the bugs are.

%\paragraph{Incorrect Thresholds}
%\plan{commented out: a paragraph giving a trace of an execution that violates safety from incorrect thresholds. Since we have the above, we don't need another execution trace.}
%We give an example of an execution trace identified by a straightforward generator that is designed to explore safety violations.
%Our implementation of the ABA protocol implements SB-Broadcast a part of the protocol code, rather than treat is as an ideal functionality primitive. 
%We run our suite of generators against a faulty version of the SB-Broadcast where its thresholds are set too small to cause a violation of its \emph{validity} property.
%Our generators, capture this and present at trace with four parties, $[A, B, C, D]$, a byzantine party $D$, with inputs $[0,1,1]$.
%The adversary first delivers all EST messages within the input partitions: $A$ receives its own $\m{EST}(0)$ and $B,D$ receive their own $\m{EST}(1)$.
%The corrupt $D$ also gives a $\m{EST}(0)$ message to $A$.
%An instance of SB-Broadcast delivers for all honest parties. 
%$A$ broadcasts $\m{AUX}(0)$ and $B,D$ broadcast $\m{AUX}(1)$. 
%Corrupt $D$ gives $\m{EST}(0)$ messages to $B,D$ so that they deliver both $\{0,1\}$ and their $view = \{0,1\}$.
%The adversary scheduler delivers all $\m{AUX}(\cdot)$ messages to all parties.
%When the coin flip in this round is 0, $A$ decides 0 and parties $B,D$ continue to the next round and continue to support their original input, 1.
%The adversary targets the set of parties with input, this round, opposite the coin flip.
%It does the same above steps but this time for $B,C$ and gets them to deliver only 1, receive $n-t$ $\m{AUX}(\cdot)$ messages and decide 1 if the coin flip in this round is 1.

%\paragraph{Identifying the Bugs from a Trace}
%\plan{Commented out: paragraph on the process of going from output trace from quickcheck to replaying and identifying where the bug occurs}
%The trace above, replayed in both worlds, only outputs environment transcripts that differ.
%Discovering the source of the discrepency consits of replaying the input trace, input by input (or binary search), to find the point of disagreement.
%From there, it is clear what property was violated, in this case safety, and the states of the protocol can be observed.
%The above fault occurs in a two-round window so observing the set of messages received before the fault and the state of each party is easy.
%Obervation returns back that parties $B,C$ delivered both 0 and 1, but $A$ manages to deliver 0 without any input from $B,C$.
%It is immediatley clear that SB-Broadcast has the bug, and the bug is that $A$ delivers a value without waiting for enough inputs.
%Hance it is clear that the threshold for delivery is set too low. 

%Some of the bugs we inject into our ABA protocol are as follows
%\begin{enumerate}
%\item incorrect thresholds parameters: there three thresholds of importance (two in SB-broadcast and one in the main protocol) and we test against different combinations of these being set too high or too low
%\item round handling: the ABA protocol specifically keeps state, and instances of SB-broadcast, around from previous rounds and the programmer isn't careful about how round numbers are handled with incoming messages
%\item incorrectly resetting state at every round: resetting the broadcast primitive for the non-proposed value like the old MMR protocol
%\item wrong common coin assumption: the $t+1$ strong common coin requires $t+1$ \emph{non-faulty} parties to call rather than any $t+1$ parties to call it
%\item handling the AUX messages that are received incorrectly (which ones make it into the set \emph{view})
%\end{enumerate}
%Intuitively, incorrectly setting the threshold parameters is perhaps the simplest bug that would cause failure. 
%In the case of thresholds that are too low, the protocol can't make progress, and in the case they are too low, the parties can decide different values.
%
%Another intuition and \emph{limitation of our approach} is that any of these injected bugs, on their own, may be sufficient to cause a failure in the protocol, but we simply don't define a clever enough generator to force the distinguishing environment to appear

\begin{figure*}
\centering
\begin{minipage}{\textwidth}
\begin{center}
\begin{tabularx}{\textwidth}{ || p{0.2\textwidth} | p{0.1\textwidth} | p{0.1\textwidth} | p{0.5\textwidth} || }
\hline \hline
Bug(s) & Result & Generators & Remarks
\\ \hline
Message Validation ABA & TP & Simple  & Trivially found safety violation.
\\ \hline
Message Validation Ben Or & {\color{blue} FN} & - & The probabilities involved in producing the distinguishing environment and our generators need to be better and smarter about exploration.
\\ \hline
Reset \emph{bin\_ptr} [ABA] & TP / (Partial FP) & simple & liveness failure: partial FP because even inputs that result in decisions don't always output due to the structure of our implementation
\\ \hline
{\color{Maroon} Discovered: Any $(n-t)$ $\msf{AUX}(\cdot)$ [ABA]} & {\color{Maroon} Expected FP found FN} & {\color{Maroon} Targeted}  & {\color{Maroon} The protocol, as stated, exchibits a safety violation, and this was discovered as a bug in the ``correct'' protocol}
\\ \hline
Old MMR bug [ABA] & Partial FN & - & Our generators exploit it but not enough to force simulator to raise a failure.
\\ \hline 
Thresh. ABA Too Low / Too High & TP / (TP/TN) & Simple / simple & common coin eases ability to detect / Deadlocks in some cases + simulator without manual inspection highlighted the failure.
\\ \hline
Thresh. BenOr Too Low / Too High & TN / (TP/TN) & Targetted / Simple & Protocol over secure no exploit found without significant change in thresholds. / Deadlocks are easy to spot and live locks require manual inspection. 
\\ \hline
%{\begin{tabularx}{\textwidth}{ p{0.5\textwidth} p{0.5\textwidth} }
%\multirow{2}{1em}{Thresh.} & Too low \\ & Too high \\
%\end{tabularx}}
%& 
%{\begin{tabularx}{\textwidth}{ p{\textwidth} }
%TP \\ TP / TN
%\end{tabularx}} & 
%{\begin{tabularx}{\textwidth}{ p{\textwidth} }
%every environment \\ dumb generators
%\end{tabularx}} &
%{\begin{tabularx}{\textwidth}{ p{\textwidth} }
%safety violations \\
%some implut distributions, protocol is locked
%\end{tabularx}} \\ \hline
%every environment & Too low: safety violations \\
%& Too high: true positive / true negatives & dumb generators & Too high: some input distributions, protocol is locked
%\\ \hline
%\multirow{2}{4em}{ABA thresh. perturbation [ABA]} & Too low: true negative & every environment & Too low: safety violations \\
%& Too high: true positive / true negative & dumb generators & Too high: some distributions yield a locked protocol state
%\\ \hline
Inverted \m{supportCoin} [ABA] & TP & Simple & Simulator loop checking identifies this as a repeated cycle occuring too often to discount as randomness. 
\\ \hline
%Should present a liveness failure under specific conditions. It's rare to see exxecutions that exploit this, but some observations when partyh input is split equally. Looping behavior is a good indicator if searched over varying numbers of rounds to discound false positives. \\
\hline
\end{tabularx}
\end{center}

\end{minipage}
\caption{Summary of the bugs played with for the ABA protocol.}
\label{table:aba}
\end{figure*}

%\paragraph{Bugs: Thresholds and Message Validation}
%These simple bugs are are successfully detected even with our most simple generators.
%For example, even in the all-honest case, a threshold set too low might enable a party to deliver both values in the broadcast primitive when only one should have been.
%\todo{finish this}
%
%\paragraph{Bugs: Reinitializing Broadcast for Both Inputs}
%\todo{This is reduces the protocol back to the MMR case along with a bad coin toss}
%
%\paragraph{Bugs: Bad Common Coin Assumption}
%We combine the two above, and find that under a specific combination of input distribution and adversarial scheduling strategies, there are spurious cases of no termination.
%Although our tester can not definitively confirm that a liveness error has occurred, the appearance of liveness failure in this one specific setup suggests the execution trace is worth examining.
%Through this pre-defined generator, we \emph{discover a known bug} in the literature that plagued the original MMR protocol (the one that this combination of injected faults reduces our ABA protocol to).



