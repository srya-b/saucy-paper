\begin{figure}
\centering
\begin{minipage}{0.38\textwidth}
\begin{bbox}[title={Functionality $\F_{\m{com}}(S, R)$}]\\
\textbf{on input} \inmsg{\m{Commit}}{$b$} from $S$:\\
\hspace*{1em} store $b$ and output (\m{Committed}) to $R$.\\
 Then, \textbf{on input} \inmsg{\m{Open}} from $S$:\\
 \hspace*{1em} send $(\m{Open}, b)$ to $R$.
\end{bbox}
\end{minipage}
\hspace{3em}
\begin{minipage}{0.5\textwidth}
\begin{lstlisting}[basicstyle=\scriptsize\BeraMonottFamily, frame=single, mathescape, numbers=left, xleftmargin=2em, xrightmargin=2em]
$\nproc$ $\tm{Fcom}$: (k: $\tgr{int}$), (rng: [Bit]), (sid: SID),
(S: sender), (R: receiver)  |- (fc: 1) = {
  $\ncase$ S (
    Commit => b = $\nrecv$ S ;
              R.Committed ;
              $\ncase$ S (
                Open => R.Open ;
                        $\nsend$ R b ;
              )
  )
}
\end{lstlisting}
\end{minipage}
\caption{(a) Pseudocode for a one-shot \Fcom parameterized by a sender $S$ and receiver $R$,
and (b) the corresponding code in NomosUC}
\label{fig:fcomideal}
\vspace{-4mm}
%%\Description{Ideal Fcom}
\end{figure}
