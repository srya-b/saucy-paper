% Comments:
% Why use session types? The advantages
% Maybe have an untyped commitment protocol
% Benefit of session types: extra type annotations, concise specification
% Does not provide a complete term, only a specification
% Might make sense to talk about extending session types with dependencies for commitment

%Just like distributed protocols, cryptographic protocols follow a predefined communication pattern.
The central focus of our work is to explore the role that \emph{session types} can play in defining and analyzing ideal functionalities.
In this section we introduce how session types define communication between different processes, and how they are used to decribe protocol interaction and define adversarial leakage through our running example in this work: a commitment in the random oracle model, $\Fcom$.
%We do so and introduce our running example throughout the paper: a simple, linear database functionality, \Fdb
%Our view is that session types are especially useful as ways of annotating or analyzing the ideal functionality.
%To illustrate, we use cryptographic commitment as our main running example.
%The commitment functionality \Fcom encapsulates the security properties of a two-phase, two-party commitment,
%which, given its simplicity is an ideal learning example.

The commitment functionality is a straightforward cryptographic primitive that underlies many protocols.
The version presenting in Figure~\ref{fig:fcom} is commitment to a single bit from a sender $S$ (the committer) to the receiver $R$ (the verifier).
The simple pseudocode highlights ideal functionality modelling as it implicitly encodes the following properties.
The commitment is \emph{binding} so the sender can not equivocate on the bit that was committed and it is \emph{hiding} such that the adverasry never learned anything about the committed value beyond when the commitment takes place. 
If nothing is leaked to the adversary, only real protocols where the adverasry can not observe even network messages could realize this commitment.

\centering
\begin{bbox}[title={Functionality $\F_{\m{com}}(S, R)$}]
~
\begin{itemize}
\item[--] On input \inmsg{\m{commit}}{$b$} from $S$, store $b$ leak $\m{committed}$ to \A and send $\m{committed}$ to $R$.
\item[--] On input \inmsg{\m{open}} from $S$, send $(\m{open}, b)$ to $R$
\end{itemize}
%
%\OnInput \inmsg{\msf{Commit}}{$b$} from $S$: 
%
%\qquad store $b$ and \Send $(\m{Committed})$ to $R$
%
%then \OnInput \inmsg{Open} from $S$: 
%
%\qquad \Send $(\m{Opened}, b)$ to $R$
\end{bbox}
%\caption{(a) Pseudocode for \Fcom parameterized by sender $S$ and receiver $R$,
%and (b) corresponding code in NomosUC}
\caption{Pseudocode for a single-shot bit commitment from $S$ to $R$.}
\label{fig:fcomideal}
%\vspace{-4mm}
%%%\Description{Ideal Fcom}
%$\nproc$ Fcom: (S: sender), (R: receiver)  |- (fc: 1) =
%  $\ncase$ S (
%    Commit => b = $\nrecv$ S ;
%              $\nget$ {2} S ;
%              R.Committed ;
%              $\ncase$ S (
%                Open => R.Opened ;
%                        $\nsend$ R b ; ))


The communication pattern is summarized by the following binary session types outlining the communication between \Fcom and each of $S$, $R$, and \A. 
Ignore the red highlighted annotations for now, we will clarify their meaning later one.

\scalebox{0.9}{
\begin{mathpar}
  \mi{type} \; \m{sender} = \textcolor{red}{\getpot^2} \ichoice{\mb{commit} : \m{bit} \product \ichoice{\textcolor{red}{\getpot^0}\mb{Open} : \one}}
\end{mathpar}
}
\scalebox{0.9}{
\begin{mathpar}
  \mi{type} \; \m{receiver} = \textcolor{red}{\paypot^1} \echoice{\mb{rcommit} \arrow \echoice{\textcolor{red}{\paypot^0}\mb{ropen}: \m{bit} : \one}}
\end{mathpar}
}
\scalebox{0.9}{
\begin{mathpar}
  \mi{type} \; \m{adv} = \echoice{\mb{acommit}: \ichoice{\mb{ok}: \one}}
\end{mathpar}
}


The key insight is that the session types described communication between two connected processes, and the different choice operators, $\ichoice$ (internal choice) and $\echoice$ (external choice) determine which process is sending the message.
In the above, the sender's type uses only $\ichoice$ indicating it only ever sends messages to \Fcom. 
The receiver on the other hand is only receiving messages.
$S$ sends a the \emph{label} \m{commit} over the channel along with the bit, and \Fcom sends the label \m{rcommit} to $R$.
Notably, \Fro does not leak any information to \A, which is intuitive considering it represents a hash function rather than some possible network activity. 

The adversary's type indicates that it first receives notification of a commitment from \Fcom and responds with an ok. 
In UC terms, the Ok is \Fcom's signal to output to the receiver. We'll see why this is important later. 
\todo{mention that one is called the client and provider or save until we need to in providerless channels?}

Binary session types require that only one process can write on a channel at any given point. 
A single session type is insufficient to allow either $S$ or \Fcom to write a message.
In such a case, more than one channel is used. 
\todo{mention polymorphic null type to control the order of commit and rcommit?}
%Every linear channel has a ``provider'' (the process that spawns the channel) whose writes on the channel are denoted by the type constructor $\ichoice$. 
%Similarly, the other endpoint of the channel, the ``client'', writes to it with the $\echoice$ constructor. 

\centering
\begin{bbox}[title={Functionality $\F_{\m{RO}}$}]
~
Initialize empty table $\ell \leftarrow \emptyset$
\begin{itemize}
\item[--] On input \inmsg{\m{query}}{$x$} from $P_i$ send $h'$ to $P_i$ if $(x,h')$ exists in $\ell$, else add $(x, h \xleftarrow{\$}\{0,1\}^k)$ to $\ell$ and send it to $P_i$
\end{itemize}
\end{bbox}
\caption{Psueocode for the random oracle.}
\label{fig:fro}


\paragraph{Random Oracle}
The commitment example we realize in this paper operates in the random oracle model.\footnote{Commitment schemes can also exist in the CRS model where all processes have access to some shared information that encodes information about elliptic curves and generators on that curve. See Pedersen commitments.}
A random oracle, $\Fro$, is an idealized hash function represented as an ideal functionality in the real world.
Protocol parties can query it to hash abitrary preimages, and it ensures that identical queries always give identical responses. 
The session type(s) below adequately describe the functionality accompanied by the pseudocode in Figure~\ref{fig:fro}.

\scalebox{0.9}{
\parbox{0cm}{
\begin{tabbing}
    $\m{oracle}[a] = \textcolor{red}{\getpot^1} \ichoice{$\=$\mb{query}: \m{pid} \product \m{a} \tensor$ \\
    \>$\textcolor{red}{\paypot^1} \echoice{\mb{hash}: \m{pid} \arrow \m{int} \tensor \m{oracle[a]}}}$ 
\end{tabbing}}
}

The random oracle hashes any arbitrary data type, by the type parameter \m{a}, and always returns an integer.
\Fro receives label \mb{query} on its channel along with the identifier of the calling party and the pre-image to be hashed. 
Unlike \Fcom, where there are only two parties whose identities are known, \Fro accepts messages from an arbitrary number of parties, therefore, we include the \m{pid} in the message rather than attempt to give every party its own channel to \Fro.
After returning the resulting \m{hash} to \m{pid}, the type recurses back to the beginning (back to \m{oracle}) to accept more queries rather than terminating like \Fcom.

\paragraph{NomosUC Process Code}
In the code snippet in Figure~\ref{fig:fcom} we show the NomosUC process for the case of the sender sending a commitment in real world protocol, \protcom, that realizes \Fcom.
We ignore the \inline{pay} and \inline{get} until we introduce import in the next section. 
The type definition of the process include the standard arguments given to all processes: securyity parameter \m{k}, a stream of random bits \m{rng}, and a session it \m{sid} that can be used to encode execution parameters like the identities of the sender/receiver.
Canonically, all protocols have bidirectional channels available between both \Z and \F, but whether both are used is a design choice, but in this case, the sender needs only one channel \inline{$\$$p2f} of type \m{sender} to \Fcom. 
The sender first receives label \m{commit} from \Z (line 5) and the \m{bit} it has to commit to (line 7).
It hashes \m{bit} together with a blinding factor \m{r} by sending the label \m{query} on channel \inline{$\$$p2f} to \Fro (lines 8-10);
On receivin the hash, it activates the functionality again with label \m{sendmsg} to send \m{Commit(h)} to the receiver. 
This concludes the commit step and the sender goes on to wait for an \m{open} message from \Z before it sends both \m{bit} and \m{r} to the receiver to ``open'' the commitment.

\todo{modify RO type to have sendmsg, talk more about the session type as a descriptor of the functionality, more exposition about our goals rather than just technical detail}

%Canonically in NomosUC, functionalities and parties are given two channels, here the two are \ic{p2f} and \ic{f2p}, to allow unidirectional communication over each if desired or necessary. 
%In Figure~\ref{fig:fdbideal}, the communication pattern is easily captured by just one channel and one type, therefore \ic{f2p} is unused and typed with the terminating \m{1}. 
%In general, a functionality can be written to accept any number of channels, even one channel per party but, as we explain in Section~\ref{sec:execuc}, this requires some additional multiplexing code which can be generated at compile-time.\todo{explain in section}

%In fact, the session type can not enforce that \Fdb sends \mb{yes} on a hit and \mb{no} on a miss. This is only what a correct functionality \emph{should} do. 
%An functionaliy that doesn't always behave this way satisfies some other set of properties than those intutively desired from a database and would likely fail to be emulated by a correct protocol that implements a database.
%The type of \Fdb in Figure~\ref{fig:fdbideal} also indicates a channel with \A called \ic{a2f}. 
%Given just the type we can infer that \A has access to the same interface as protocol parties, and, importantly, that \Fdb doesn't leak any information to it~\footnote{We exclude it to save space.}.
%\emph{Leaks define a crucial part of the adversarial model for functionalities, and the session type succinctly describe it}.
%\begin{tabbing}
%   $\mi{type} \; \m{db[k][v]} = \ichoice{$\=$\textcolor{red}{\paypot{1}}$\=$ \; \mb{store}:\m{PID} \arrow \m{k} \arrow$ \\
%   \>\>$\echoice{ \mb{OK}: \m{PID} \arrow \m{db[k][v]}},$ \\
%   \>$\textcolor{red}{\paypot{1}}$\=$ \; \mb{get}: \m{PID} \arrow \m{k} \arrow$ \\
%   \>\>$\echoice{$\=$\mb{yes}: \m{v} \arrow \m{db[k][v]},$ \\
%   \>\>\>$\mb{no}: \m{db[k][v]}}}$
%\end{tabbing}


%A cryptographic commitment is a protocol consisting of a sender that knows some witness $x$ and sends a commitment message $C = f(x)$ that is
%some function of the witness such that $f^{-1}(C) \neq x$ with negligible probability. 
%The ideal functionality \Fcom in Figure~\ref{fig:fcomideal}(a) describes the properties of the commitment
%It consists of a  \emph{sender} ITM $S$ and a \emph{receiver} ITM $R$ connected to 
%the ideal functionality ITM $\Fcom$. It enforces that the committer can not equivocate on the witness $x$ once it creates a commitment.
%This property is called \emph{hiding}.
%Similarly, the receiver can not learn the witness from just the commitment, which we call the \emph{binding} property.
%%\Fcom encapsulates the security properties of a two-phase, two-party commitment: (\emph{binding}) committer can't change what they committed to, and (hiding) the receiver can't open the commitment itself. 

%The communication pattern outlined in Figure~\ref{fig:fcomideal} between the sender $S$ and $\Fcom$ (and also the receiver $R$
%and $\Fcom$) is enforced via \emph{binary session types}.
%% Session types are a type system for statically expressing bi-directional communication protocols
%% in message-passing process systems.
%The key insight here is that we assign a session type to the communication channel connecting
%two processes.
%As notation, every channel has a unique \emph{provider} process that offers the channel and a
%\emph{client} process that uses it, and the session type governs the type and direction of messages exchanged between them. 
%%the processes, with the provider and client processes performing dual send/receive actions.
%As an example, we start with the session type of the channel offered by $S$ that is used by
%$\Fcom$.
%\begin{mathpar}
%  \mi{type} \; \m{sender} = \ichoice{\mb{Commit} : \m{bit} \product \ichoice{\mb{Open} : \one}}
%\end{mathpar}
%The type constructor $\ichoiceop$ denotes an \emph{internal choice}
%\footnote{Although $\ichoiceop$ with only one choice is redundant, we still use
%it here for the purpose of exposition.}
%dictating that the provider $S$ first sends a
%$\mb{Commit}$ message to $\Fcom$.
%Next, the type constructor $\product$ denotes that $S$
%sends a value of type $\m{bit}$ ($\m{bit} \product \ldots$).
%Finally, the $\ichoiceop$ constructor
%enforces that $S$ sends $\mb{Open}$ to $\Fcom$ followed by type $\one$
%that indicates $S$ terminates and closes its channel.
%In UC, one-shot functionalities terminate after a single instance, and reactive
%functionalities persist and run many times. 
%It is important to point out that session types capture communication over one channel.
%For example, the session type of the sender does not capture what, if any, information is leaked to the adversary when \Fcom is activated.
%This helps with modularity: since one session type only captures the local communication
%between two processes, we can modify the adversary's implementation or communication interface
%without impacting the session type between $S$ and $\Fcom$.
%Local session types also do not directly capture \Fcom's security property that the same bit that was committed is the one that is opened.
%\footnote{With refinement session types~\cite{Das20CONCUR,Das20FSCD}, such advanced properties can be captured but they would significantly
%complicate the type system.}

%In this work \emph{import session types} extend the above session type $\m{sender}$ with annotations that express import tokens, which act as runtime budgets, passed over a channel between processes. 
%Though a trivia example because \Fcom does a constant amount of work, we still give some import below so that a protocol that does polynomial work can realize \Fcom as well.
%Such restrictions are important to consider when creating types.
%%Though a trivial example of import, given that $\Fcom$ is a one-shot functionality which does only a constant amount of work, the import session type for $\m{sender}$ is given below.
%\begin{mathpar}
%  \mi{type} \; \m{sender} = \textcolor{red}{\paypot^{2}} \ichoice{\mb{Commit}: \m{bit} \product \textcolor{red}{\paypot^{0}} \ichoice{\mb{Open}: \one}}
%\end{mathpar}
%The annotation \textcolor{red}{$\paypot^2$} asserts that the sender gives 1 import with \m{Commit} and 0 with \m{Open}. 
%%Despite doing constant work, requiring 0 tokens would
%%contrain protocols that can realize it (which will necessarily have the same type as \Fcom) to those that do constant work. Such restrictions are important to 
%%consider when defining import session types in NomosUC.
%The dual of $\paypot$ is given by $\getpot$ where an external choice operation can be specified with a required amount of import to be received. 
%The typing rules for both $\paypot$ and $\getpot$, and a more comprehensive discussion about the handling of import tokens, is given in Section~\ref{sec:import}.
%
%Similar to the sender, we define a channel provided by the receiver $R$ 
%used by $\Fcom$ with the following session type
%\begin{mathpar}
%   \mi{type} \; \m{receiver} = \textcolor{red}{\getpot^0} \echoice{\mb{Committed}: \textcolor{red}{\getpot^0} \echoice{\mb{Opened} : \m{bit} \arrow \one}}
%\end{mathpar}
%Dual to internal choice, the $\echoiceop$ type constructor represents \emph{external choice}
%prescribing that the provider $R$ must receive a $\mb{Committed}$ message from $\Fcom$
%followed by an $\mb{Opened}$ message (using another $\echoiceop$ constructor) from $\Fcom$.
%Then, $R$ must receive a bit from $\Fcom$ as depicted by the $\arrow$ constructor (dual to $\product$)
%followed by termination (indicated by $\one$).
%Dual to $\mb{sender}$, the receiver here expects to receive no import from $\Fcom$.

%Protocols expressed via session typed channels are realized by process implementations.
%A session-typed process \emph{uses} a set of channels in its context (similar to a function
%having arguments) and provides a unique channel (similar to a function returning a single value).
%NomosUC also allows processes to store functional data (like integers, booleans, lists, etc.)
%and either transfer them to other processes or perform local computation on them.
%The type checker guarantees that every process adheres to the protocol on every channel as defined by
%the corresponding session type.

%As an illustration, consider the $\Fcom$ process implemented in Figure~\ref{fig:fcomideal}(b)
%that \emph{uses} channels $S$ and $R$ and \emph{provides} channel \inline{fc}.
%The used channels with their types are written to the left of the turnstile
%($\vdash$) while the offered channel and type are written on the right.
%The process first case analyzes on channel $S$ branching on the
%message received.
%Since there is only one choice $\mb{Commit}$, we only have one
%branch in the definition.
%$\Fcom$ then receives the bit $b$ (line 3) on $S$, followed by sending the
%\m{Committed} message on channel $R$ to the receiver (line 4).
%Then, $\Fcom$ receives the $\mb{Open}$ message on $S$ followed by sending the
%$\mb{Opened}$ message on $R$ (line 6), followed by the bit $b$ (line 7).
%$\Fcom$ then waits for the channels $R$ and $S$ to terminate and then finally
%terminates the \inline{fc} channel (code not shown for brevity).

%The types associated with \Fcom here don't make use of the import tokens encoding we 
%introduce later in this work. An important reason for it is that most functionalities
%in NomosUC are designed to be parametric in the amount of import they, and their
%session types, require. A static amount of import for some \F constrains the 
%amount of import, or computation, that a protocol realizing \F can use. 
%Such a constraint is unnecessarily restrictive requiring multiple versions
%of the same functionality for different protocols. 

%A protocol may consist parties with different roles with different sets of inputs and messages in the protocol. 
%A session type defines the protocol for only one role in an ideal functionality and others may have their own types.
%The sender and receiver are different roles in the same protocol, and, therefore, must have their own channels to \Fcom rather than communicating over a common channel.

%The protocol initiates with $S$ sending a $\mb{commit}$ message to $\Fcom$
%indicating its intent to \emph{commit} to a bit.
%Next, $S$ sends this committed bit to $\Fcom$.
%After receiving the committed bit, $\Fcom$ sends a $\mb{commit}$ message
%to $R$ indicating that a bit has been committed to, but does not reveal
%this bit to $R$.
%At a later time, $S$ sends an $\mb{open}$ message to $\Fcom$ expressing
%that $S$ wishes to reveal the secret bit to $R$.
%Receiving this message, $\Fcom$ in turn sends an $\mb{open}$ message
%to $R$ followed by this bit.
%The protocol concludes with each party (process) terminating.
%Finally, the type $\one$ denotes termination, indicating that
%$S$ will send $\m{close}$ message to $\Fcom$.

%\begin{figure*}[!ht]
%\begin{lstlisting}[basicstyle=\small\ttfamily,frame=single]
stype sender = +{ commit : bit ^ +{ open : 1 } } 

stype receiver = &{ commit : &{ open : bit -> 1 } }

decl F_comm : (S : sender), (R : receiver) |- (F : 1)

def F <- F_comm S R =
  case s (
    commit => b = recv S ;
              R.commit ;
              case S (
                open => R.open ;
                        send R b ;
                        wait S ; wait R ; close F ) )
\end{lstlisting}

%\caption{The $\mathcal{F}_{\msf{comm}}$ commitment ideal functionality in Nomos. The types for the sender and receiver channel define what inputs they can give to the functionality and what messsages are sent from the functionality back to the receiver.}
%\label{fig:nomos:commitment}
%\end{figure*}

