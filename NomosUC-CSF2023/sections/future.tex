The immediate next steps coming from this work are cleark.
First, we would like to remove the provider/client constraints of linear session types, while maintaining the import mechanism, so that providerless channels are no longer needed in our construction.
This would greatly simplify our construction and allow for easier expression communication patterns. 

The NomosUC polynomial time notion arises from retrofitting the import mechanism to the potential mechanism already present in Nomos' resource aware session types through our token heirarchy.
A clear next step would be to, first, examine the feasibility of a type checker for the existing import tokens mechanism, and, two, create import session types from the ground up .

An ambitious, but still distant, goal is to create software tools around NomosUC to do some automated testing or proof generation like other related works to our own. For example, tools that fuzz test functionality/protocol definitions or provider correctness guarantees to NomosUC code. 

