This area of research presents several next steps and possibility for futurework building on NomosUC.

On of the immediate next steps is to overcome the existing Nomos provider-client constraints. 
Currently, we introduce a several channels and a couple of commuicators to generalize communication between two processes. 
This is due in large part to linear channels being unique to a specific provider and client.
We would like to do away withthis requirement in subsequent works and grealy simplify our construction.

We also point out that we build NomosUC on top of the type system defined by Nomos, however, we do not yet implement a type checker for NomosUC. We rely on the typing rules and theorems to validate our approach and manually typed checked the functionalities, protocols, and simulator code we implemented.
A next step in this direction is attempting to implement the type checker for NomosUC or evaluating the feasibility of doing so.

A few works related to our own, described in Section~\ref{sec:related}, attempt to automate proof generation for composable security. Our work provides tooling to specify and analyze UC definitions, however, a goal of future work is to introduce fuzz testing of protocols, functionalities, and emulation in NomosUC.

