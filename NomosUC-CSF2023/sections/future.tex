In this work we focused on adapting binary session types to the UC setting. The decision was motivated, in part, by Nomos~\cite{das2018work, dasnomos} whose resource-aware binary session types
fit nicely into the import mechanism in UC.
We want to explore adding import on top of other formulation of session types, like multiparty session types.
Resource-aware session types in Nomos also came with an implementation of their powerful type system, so, while this work is only on paper implementing NomosUC type system is a natural next step.

We also want to expand our encoding of UC in NomosUC by expressing synchronous and asyncronous models of communication and computation. 
Specifically, creating programming conventions and models around the NomosUC to express such models.
Furthermore, we want to add better tooling around our UC definitions such as automated property-based or model-based testing of functionality/protcol specifications and fuzz testing. 
Tooling like this around NomosUC makes a stronger case for it as a practical language for implementing cryptographic/distributed protocols.
