The database functionality from the previous section performs potentally polynomial amount of work on every activation: protocol parties activate \m{store} to \Fdb a polynomial number of times in the security parameter $k$.
In this section we introduce the import mechanism for polynomial time~\cite{uc}, and show how NomosUC realizes import, a new kind of \emph{import session type}, and how NomosUC enforeces polynomial time behavior with the database example.
We motivate every part of our design by working through adding import to \Fdb and a simulator that \emph{sandboxes} it.

\subsection{Giving Import to \Fdb}
The import mechanism allows an ITM to perform computation that is polynomial in the \emph{net} import is owns (and the security parameter $k$): sum of import received -- the sum of import sent out.
In order for \Fdb to iterate over a list that's polynomial in $k$ it must always have at least 1 import token in its possession. 
We specify this iport requirement by adding an import requirement to the communication between \Fdb and the protocol parties, and between \Fdb and \A.

We introduce \emph{import session types (IST)} to express and exchange import between processes. 
Unlike resource-aware session types, resource sent with import session types implies computation polynomial in the amount sent rather than linear. 
Expressing import in IST is straightforward. 
We introduce to new type constructors from resource-aware session types:  $\getpot$ and $\paypot$.
The \Fdb session type becomes:
\begin{tabbing}
    $\mi{type} \; \m{db[k][v]} = \ichoice{$\=$\textcolor{red}{\paypot{1}}$\=$ \; \mb{store}:\m{PID} \arrow \m{k} \arrow$ \\
    \>\>$\echoice{ \mb{OK}: \m{PID} \arrow \m{db[k][v]}},$ \\
    \>$\textcolor{red}{\paypot{1}}$\=$ \; \mb{get}: \m{PID} \arrow \m{k} \arrow$ \\
    \>\>$\echoice{$\=$\mb{yes}: \m{v} \arrow \m{db[k][v]},$ \\
    \>\>\>$\mb{no}: \m{db[k][v]}}}$
\end{tabbing}
As the ``provider'', the calling party sends import using the $\paypot$ operator and \Fdb sends import using the $\getpot$ operator.
In this case, the type enforces that 1 import token is sent when a party takes the $\ichoice$ and calls \Fdb.
In the other direction, \Fdb never sends any import back to the party. 
Rather than explicitly place a $\textcolor{red}{\getpot{0}}$ we omit the operator altogether.

In NomosUC process code, we introduce two new operations to let processes exchange the import that the IST requires.
The same \Fdb process code from Section~\ref{sec:example} is augmented with the commands \inline{$\nget$} and \inline{$\npay$}. Below we show the case of \inline{store} in \Fdb and the corresponding party paying the import.
\begin{lstlisting}[basicstyle=\scriptsize\BeraMonottFamily, frame=single, mathescape, numbers=left, xleftmargin=2em, xrightmargin=2em]
$\tg{...}$
$\ncase$ $\$$p2f (
  store => pid,(k',v') = $\nrecv$ $\$$p2f
    $\tr{get {1} \$p2f}$
    $\$$tb' <- pappend[(k,v)] <- $\$$tb k' v' ;
    $\$$p2f.Ok; $\nsend$ $\$$p2f pid ;
    $\$$c $\leftarrow$ Fdb[k][v] <- $\tg{(* args *)}$ $\$$tb'
$\tg{...}$
\end{lstlisting}
%$\nproc$ Fdb[k][v]: ($\$$p2f: db[k][v]), ($\$$f2p: 1), 
%  ($\$$a2f: adv[k][v]), ($\$$f2a: 1), (l: [(k,v)]) |- ($\$$c: 1) =
%{

%    retrieve => pid,k' = $\nrecv$ $\$$p2f ;
%      $\tr{get {1} \$p2f}$
%      b $\leftarrow$ exist $\leftarrow$ $\$$tb k' ;
%      $\nif$ b $\nthen$
%        v' $\leftarrow$ get $\$$tb k' ;
%        $\$$p2f.yes; $\nsend$ $\$$p2f pid; $\nsend$ $\$$p2f v';
%      $\nelse$
%        $\$$p2f.no; $\nsend$ $\$$p2f pid ;
%      $\$$c $\leftarrow$ Fdb[k][v $\leftarrow$ $\tg{(* args *)}$ 
\begin{lstlisting}[basicstyle=\scriptsize\BeraMonottFamily, frame=single, mathescape, numbers=left, xleftmargin=2em, xrightmargin=2em]
$\$$p2f.store ; $\tr{pay {1} \$p2f}$
$\nsend$ $\$$p2f pid ; 
$\nsend$ $\$$p2f someK ; $\nsend$ $\$$p2f someV ;
$\ncase$ $\$$p2f ( Ok => 1 )
\end{lstlisting}
%$\nproc$ somparty[k][v]: (pid: PID), ($\$$p2f: db[k][v]), 
%  ($\$$f2p: 1)  |- ($\$$c: 1) =
%{
%}

\subsection{Potential Mechanism}
Import in UC allows for polynomial runtime, and, in a sense, import is \emph{consumed} when a computation is performed. 
The UC security definition only cares that there is \emph{some} polynomial to bound an ITM's runtime.
In NomosUC, we are explicit about the polynomial used, and type-checking a collection of processes relies on a concrete polynomial that runtime is checked agains.
In order to actually account for runtime, another concrete value is needed: \emph{potential}.

The potential mechanism directly determines how much import a process has actually consumed by measuring how much comutation it does and is still able to do.
Specifically, we want to ensure the following situation is caught by NomosUC.
Two processes $A$ and $B$ shouldn't be able to send 1 import back and forth and \emph{both} perform polynomial from that 1 token.

%For example, when a process $A$ sends import to another process, the type system must ensure
%that the computation already performed by $A$ is still polynomially bounded by the import remaining after sending import.
%In other words, two processes $A$ and $B$ shouldn't be able to send 1 import back and forth and perform polynomial from that 1 token.
%$A$ performing a polynomial computation with the single import and sending it to $B$ should fail to type check. 
Potential is related to import by the following statement: if a process
possesses $n`$ net import then it has potential of $T(n')$, and it can not take more than $T(n')$ computational steps.
Potential is never exchanged between processes and is checked in relation to the net import a process has. If a process performs 
$X < T(n)$ steps and sends an import to another process such that $X > T(n-1)$, NomosUC fails to type-check the process (we leave all typing rules to the end of the section).

\paragraph{Why another resrouce?} 
Why is it necessary to add another resource on top of import rather than directly account of runtime through import?
In the general case of bounding runtime, a single linear resource is sufficient to control the runtime of processes (like the work by Das et al.~\cite{rast}), however, in UC we are concerned with capturing a realistic computation model and doing so leaks information that otherwise shouldn't be known. 
The motivation is stated in the original UC~\cite{uc} work and we restate it here for clarity.
At a high-level, allowing directly controlling runtime through import allows an ITM to determine the exact number of steps another ITM can take.
The gives the calling ITM about other ITMs that it should not have. 
For example, imagine two executions (say an ideal and real world for which an emuation proof exists) where the environment can control exactly the number of steps each execution is allowed to take with import. 
The different ITMs may terminate at different times, and by carefully choosing the import to give \Z can distinguish between the two worlds.
Intuitively, leaking this information to an environment doesn't capture a realistic model.
\footnote{The specific example given by Canetti~\cite{UC}: ``For instance, the initial ITI $I$ can start in a rejecting state, and then pass control to another ITI $M$. If $I$ ever gets activated again, it moves to an accepting state. Whether $I$ is activated again depends on the running time of $M$. If it exceeds the computation given to it (known by $I$) then $I$ accepts depending on information that should not be ``legitimately know'' to $I$.}.

NomosUC processes must explicitly state how much potential they use through two new keywords: \inline{$\ngenpot$} and \inline{$\ntick$}.
The type system tracks, for every process, the cumulative amount of potential generated by the process $q$ and the current remaining potential $q'$. We elaborate on the details of the type system in a later section.
A NomosUC process generate $r$ potential with \inline{$\ngenpot$}(r) resulting in $q+r$ and $q'+r$ potential for the process. 
The process can then account for performing $r$ work by doing \inline{$\ntick$(r)} leaving the cummulative potential the same but the remaining potential is reduced to $q'-r$.

Below we show just the case of \inline{store} for \Fdb using the new potential keywords. 
When appending to the end of the list \Fdb must iterate over the whole list (depending on how \inline{pappend} is implemented).
Therefore, conservatively, \Fdb generates enough potential to expend unit cost for accessing each item in the list (line 4).
In this code section, we rely on \inline{pappend} to execute \inline{$\ntick$} operations of it own, and \Fdb and deducts potential before performing a message send back to the party (line 7).
\begin{lstlisting}[basicstyle=\scriptsize\BeraMonottFamily, frame=single, mathescape, numbers=left, xleftmargin=2em, xrightmargin=2em]
$\tg{...}$
$\ncase$ $\$$p2f (
  store => pid,(k',v') = $\nrecv$ $\$$p2f
    $\tr{get {1} \$p2f}$
    $\ngenpot$(length l)
    $\$$tb' <- pappend[(k,v)] <- $\$$tb k' v' ;
    $\ntick$(1)
    $\$$p2f.Ok; $\nsend$ $\$$p2f pid ;
    $\$$c $\leftarrow$ Fdb[k][v] <- $\tg{(* args *)}$ $\$$tb'
$\tg{...}$
\end{lstlisting}
In general, instead of manually inserting a \itick, NomosUC can be instrumented to insert a \itick before every operation that it performs. \todo{maybe in the example we actually puts ticks everywhere like even a \itick(1) before calling \inline{pappend} as people might ask that function calls are not counted to you can call a function unbounded number of times or another nit pick like that.}

Additionaly, abstracting import into a sub-resource called potential makes the UC framework all the more expressive.
If import was used to count and bound computation directly, rather than through a polynomial, the \Fdb example becomes untennable. 
A party querying \Fdb has to know how long the list is so it sends enough import to iterate over it, and it doesn't know how many items other parties may have added.
This leads to a bizarre setting where parties must attempt to learn information about the list size or keep trying higher increments of import to complete a query.
Constraining runtime in this way makes the framework unnecessarily cumbersome.

%Imagine the database session type specified 1 import token is exchanged in both directions: 1 token from parties when querying and 1 token back from \Fdb with the result. 
%Without a concrete resource to track the actual com
%For example, without the potential mechanism the following processes set would type check but obviously violates the rules of import. 
%
%\begin{tabbing}
%   $\mi{type} \; \m{atob} = \ichoice{\textcolor{red}{\paypot{1}} \; \mb{give}: \echoice{\textcolor{red}{\getpot{1}} \; \mb{receive}: \m{atob}}}$
%\end{tabbing}
%
%\begin{lstlisting}[basicstyle=\footnotesize\BeraMonottFamily, mathescape, frame=single]
%$\nproc$ A : (bigNumr: Int), ($\$$a2b: atob) |- ($\$$ch: 1) = {
%  $\text{\color{Red}{genPot k}}$
%  let bigNum = bigNum * 2 ;
%  $\$$a2b.give; $\npay$ {1} $\$$a2b ;
%  $\ncase$ $\$$a2b ( receive => $\nget$ {1} $\$$a2v )
%  $\$$ch <- A bigNum $\$$a2b ;
%}
%
%$\nproc$ B : (bigNum: Int), ($\$$a2b: atob) |- ($\$$ch: 1) = {
%  $\ncase$ $\$$a2b (
%    give => $\nget$ {1} $\$$a2b ;
%      $\text{\color{Red}{genPot k}}$
%      $\nlet$ bigNum = bigNum * 2 ;
%      $\$$a2b.receive ; $\npay$ {1} $\$$a2b ;
%      $\$$ch <- B <- bigNum $\$$a2b ;
%  )
%}
%\end{lstlisting}
%
%Without additional rules for accounting for potential (lines 2 and 12), and \emph{consuming} import, the above system would violate the basic import mechanism as defined by UC.

%Furthermore, abstracting potential from import has the added benefit of a more expressive framework.
%Take the database ideal functionality below as an example. 
%When a party performs a read, it doesn't know the size of the list before-hand. 
%Using only import as the accounting for computation, the protocol party would always have to give a precise amount of import to ensure the whole list can be read. 
%With potential, a single unit of import is enough for the functionality to determine how much polynomial computation (in this case, linear) needs to be done.  
%
%\begin{tabbing}
%   $\mi{type} \; \m{db[k][v]} = \ichoice{$\=$\textcolor{red}{\paypot{1}}$\=$ \; \mb{store}:\m{PID} \arrow \m{k} \arrow$ \\
%   \>\>$\echoice{ \mb{OK}: \m{PID} \arrow \m{db[k][v]}},$ \\
%   \>$\textcolor{red}{\paypot{1}}$\=$ \; \mb{get}: \m{PID} \arrow \m{k} \arrow$ \\
%   \>\>$\echoice{$\=$\mb{yes}: \m{v} \arrow \m{db[k][v]},$ \\
%   \>\>\>$\mb{no}: \m{db[k][v]}}}$
%\end{tabbing}
%
%\todo{fdatabase is nontrivial, how much of it to include? maybe only where it iterates over the list and generates n potential to do it}
%\begin{bbox}[title={Functionality $\F_{\msf{db}}$}]
%
%Initialize list $l := []$
%
%\OnInput \inmsg{add}{$x$} form $P_i$:
%   \begin{ritemize}
%       \item Append $x$ to $\ell$
%       \item \Send $ok \rightarrow P_i$
%   \end{ritemize}
%
%\OnInput \inmsg{get}from $P_i$:
%   \begin{ritemize}
%       \item \Send $\ell \rightarrow P_i$
%   \end{ritemize}
%\end{bbox}

\subsection{Virtual Tokens}
Our realization of import so far adheres to the design specified by Canetti et al~\cite{uc}.
We encounter a subtle problem when implementing simulators that internally run other ITMs, functionalities, or protocols.
Sandboxing, as it is called, is a common design pattern for UC simulators. 
In particular, ``overriding'' the random tapes of ITMs is useful for rewinding and replaying proofs.
Similarly, running the real-world internally is a a crucial part of many simulators, especially for protocols of complete information protocols such as byzantine agreement protocols without any hiding. 

In the UC framework, a simulated ITM like \Fdb is considered a subroutine of the calling ITM (in this case a simulator \Sim).
The sum of the work performed by \Fdb and \Sim is bounded by a polynomial in the number of import \Sim has.
In NomosUC, if we want to run \Fdb as a subroutine, we're faced with a session type \m{db[k][v]} which requires one unit of import sent over the channel with every message from \Sim.
In the listing below, the simulator is running \Fdb internally and communicating with it over an internal channel \inline{$\$$fdb} (line 5). 
Almost like a dummy adversary, this \Sim passes messages from \Z, intended for \Fdb in the real world, to its internal simulation of \Fdb (line 12).
%\Sim relies on the existing process to emulate the outputs \Z would see in the real world communicating with the dummy adverasry and \Fdb.
On line 9, \Sim gets 1 unit of import from \Z, and on line 13 it's forced to give up its import to make the call to \Fdb.
Without calling \ipay, \Sim would fail to type-check due to the session type of \Fdb.
Ultimately, this is no different from running \Fdb as a separate process, and if we want to commulatively bound the work done by \Fdb and \Sim as a single process this raises a few important questions:
does \Sim lose import and \Fdb gains import? do they both share the same import because \Fdb is sandboxed? can we tell the type system to ignore import in some cases?

\begin{lstlisting}[basicstyle=\scriptsize\BeraMonottFamily, frame=single, mathescape, numbers=left, xleftmargin=2em, xrightmargin=2em]
$\nproc$ sim_db[k][v] : 
  $\tg{(* the usual params *)}$
  ($\$$z2a: Z2A[a2p,a2f]), ($\$$a2z: A2Z[p2a,f2a]),
  ($\$$p2a: a2p), ($\$$f2a: f2a),
  ($\$$fdb: db[k][v]) $\tg{(*...*)}$ |- ($\$$c: 1) =
{
 $\nmatch$ $\$$z2a,$\$$p2a,$\$$f2a (
    Z2A2F,*,* =>
      $\nget$ {1} $\$$z2a
      msg = $\nrecv$ $\$$z2a
      $\ncase$ msg (
        Query(k) => $\$$db.query
          $\npay$ {1} $\$$db
          $\nsend$ $\$$db k
          $\ncase$ $\$$db (
            yes => v = $\nrecv$ $\$$db 
            _ => ()
          x $\leftarrow$ do_poly_work ;
        Store(k,v) => $\$$db.store
          $\npay$ {1} $\$$db
          $\nsend$ $\$$db k ; $\nsend$ $\$$db v
          $\ncase$ $\$$db ( Ok => 
            x $\leftarrow$ do_poly_work ;
            $\$$c <- sim_db[k][v] $\tg{(* args *)}$ )

}
\end{lstlisting}

More generally, a functionality, in the real world, whose output depends on the amount of computation it's able to do must receive the same amount resources in the ideal world.
In the real world the dummy adversary, \DA, receiving $X$ import from \Z gives all $X$ to the functionality.
In the ideal world \Sim runs the the real-world internally, receives the same $X$ import as \DA, from \Z, and must give \Fdb all $X$ import to ensure the same output is produced. 
It is therefore forced to give up it's entire runtime budget, just as \DA does, and therefore can't do any additional polynomial work of its own. 
This results in a big constraint on the types of functionalities and emulation proofs that can be realized by NomosUC.

%% | describe the a simulated \Fdb now without virtual tokens and showcase the issue
%% | and we need to be able to reuse process definitions rather than make special virtual one
In order to support the sandboxing, we add \emph{vitual import tokens} to NomosUC.
Rather a single type of token, we allows NomosUC processes to create fake import tokens to give to sandboxed processes and tie the amount of virtual import created to the real import tokens a process holds.
Tokens types are represented by a token heirarchy 
\vspace{-0.5em}
\begin{mathpar}
  \mi{token types\;}\;\K_0 \to \K_1 \to ... \to \K_m
  \vspace{-0.5em}
\end{mathpar}
Where $\K_0$ are real import tokens and $\K_{i>0}$ are all virtual.
A depth $m$ token heirarchy conveys that the deepest ``simulation dept'' is $m$.
Every process is parameterized by a token type that it considers real tokens and can create vitual tokens of one depth higher.
For example, the simulator above is given the parameter $\K_0$ by \execuc:
\begin{lstlisting}[basicstyle=\scriptsize\BeraMonottFamily, frame=single, mathescape, numbers=left, xleftmargin=2em, xrightmargin=2em]
$\$$c $\leftarrow$ sim_db$\tr{[K0]}$[k][v] $\tg{(* the usual params *)}$
            $\tr{\^{}\^{}\^{}}$
  ($\$$z2a: Z2A[a2p,a2f]), ($\$$a2z: A2Z[p2a,f2a]),
  ($\$$p2a: a2p), ($\$$f2a: f2a) $\tg{...}$
\end{lstlisting}
The simulator, in turn, spawns an instance of \Fdb with a with  virtual token type, and \Fdb works with type $K_1$ as its real token type.
\begin{lstlisting}[basicstyle=\scriptsize\BeraMonottFamily, frame=single, mathescape, numbers=left, xleftmargin=2em, xrightmargin=2em]
$\$$fdb $\leftarrow$ Fdb$\tr{[K1]}$[k][v] $\leftarrow$ $\tg{...}$
\end{lstlisting}

The token heirarchy is a global definition and is used by all processes. This means that two different processes using real import token, $K_0$, sandbox processes using the same virtual token $K_1$.
The heirarchy has to be statically defined in order for the type system to make a judgement on the polynomial runtime of the processes involved.
\todo{the point about token depth being known statically from a process's definition and non-deterministic levels of sandboxing don't really exist. An itm can sandbox a random number of ITMs but they all exist at the same depth if they are to communicate with each other and even if they aren't you gain no expressiveness from non-deterministic virtualization.}

\paragraph{Creating Virtual Import}
A new NomosUC operation \inline{$\nwithdraw$ K0 K1 n} allows a process, using its real token type $\K_0$, to create $n$ new tokens of type $\K_1$, and we add a new parameter to $\npay$ and $\nget$ which specifies the token type being sent or received. 
A process $P$ can can spawn a process $P'$ with a virtual type $K_1$ and can pay/receive virtual tokens to/from $P'$.
When communiating with augmented \Fdb (above), \Sim (below) first creates the virtual tokens it will give \Fdb (line 5) and then sends tokens of that type over its channel with it (line 6).
\begin{lstlisting}[basicstyle=\scriptsize\BeraMonottFamily, frame=single, mathescape, numbers=left, xleftmargin=2em, xrightmargin=2em]
$\nget$ {1} $\$$z2a
msg = $\nrecv$ $\$$z2a
$\ncase$ msg (
  Query(k) => 
    $\nwithdraw$ K0 K1 1$
    $\$$db.query
    $\npay$ $\tm{K1}$ {1} $\$$db
    $\nsend$ $\$$db k
    $\ncase$ $\$$db (
      yes => v = $\nrecv$ $\$$db 
      _ => ()
    x $\leftarrow$ do_poly_work ;
\end{lstlisting}

\paragraph{Relating virtual tokens to real work.}
Without relating virtual tokens to the amount of import that the sanboxing process has, we encounter an infinite runs problem where processes can continue to create virtual tokens and processes to perform super-polynomial work.
NomosUC constrains the number of virtual tokens created through a globally known polynomial \GlobalF.
The type system requires that the virtual tokens of type $i+1$, $t_{i+1}$, created by a process are upper bounded by $\GlobalF(t_i, k)$ where $t_i$ are the real tokens of the process, of type $i$, and $k$ is the security parameter. 
This ensures that over all virtual processes within a ``real'' process the amount of tokens is never super-polynomial in the number of real import.

Processes using virtual tokens create potential in the same way, via \inline{$\ngenpot$}. 
Although the potential generated by sandboxed processes is the same as any other potential (we don't explicitly define potential types), the type system only compares the potential created/used to a polynomial in that process's token type.\todo{is there an example here that makes this last point clearer?}. 
The typing rules described next clarify on how the type system determines whether a process satisfieds import-based polynomial runtime.

Our treatment of virtual tokens bounded by a polynomial in the import a process has appears, on the surface to be identical to how we treat potential.
A natural counter-point may be to treat potential as another virtual token in the heirarchy: make the last token in the heirarchy the potential.
\todo{i don't know if i have a good answer for this other than it's easier. tokens don't know how deep the virtual tokens go. you can define one token type that everyone knows as potential but then there isn't really that much of a difference with that versus treating it differently. we want only one potential type to be used for compuation cost regadless of the token type of the process using it.}

%% | we need a different kind of virtual import to give to \Fb to satisfy its type and make the type system happy

%% | the keywords to create virtua tokens

%% | how do we make sure they don't exceed the polynomial constraints of the calling process?

\subsection{Typing Rules for Import and NomosUC}
\todo{where to include the base session types stuff from Nomos?}
We introduce some of the base type system from Nomos, but focus mainly on the the new additions for handling import, potential, and virtual tokens. 
The typing rules for NomosUC borrow heavily from Nomos, and in this section we highlight only the significant additions/changes and leave the remainder for the appendix.

% | token heirarch and token validity context 
Session types are derived from a Curry-Howard interpretation of intuitionistic linear logic. 
Under this correspondence, a process term is assigned the followeng judgement: 
\[
(x_1 : A_1), (x_2 : A_2), \ldots, (x_n : A_n) \vdash P :: (z : C)
\]
which states that process $P$ \emph{provides} a service
of session type $C$ along channel $z$, \emph{using} the services of session
types $A_1, \ldots, A_n$ provided along channels $x_1, \ldots, x_n$ respectively.
For a \emph{well-formed} judgment, all channel names need to be distinct.
The linear antecedents are often abbreviated to $\D$.

The NomosUC judgment has some additional components
\[
\Sg \semi k \semi \Tokens \semi \Psi \semi \D \entailpot{q}{q'} P :: (x : A)
\]
$\Sg$ denotes the signature containing type and process definitions and $k$
denotes the security parameter.
Both these quantities are globally known and fixed, therefore we omit them from
most typing rules for brevity.
$\Tokens$ describes the total (ever received) and current ($=$ received - sent) import tokens
of each type stored in the process (explained more in Section~\ref{sec:import}).
$\Psi$ represents the functional data structures and $\D$ collects the
session-typed channels along with an optional \emph{write token} $\wt$
(to resolve non-determinism in the semantics) used by the process.
Finally we denote the total potential created and the potential remaining for $P$ with $q$ and $q'$ above and below the turnstile.

The token context $\Tokens$ tracks the number of tokens of each type from the token heirarchy decribed in the previous section.
For each token type $\gamma_i$, $\Tokens$ stores a type $(t_i,t_i')$ which are the total tokens of type $i$ and the tokens of type $i$ currently owned by the process, respectively. 
We define a side condition for a process to be well-typed, and it is that its token context must always be valid \emph{w.r.t the security parameter $k$}.
To this end we define a globally known polynomial $\GlobalF : (\msf{nat}, \msf{nat}) \rightarrow \msf{nat}$ as bound between two successive token types. UC requires this function to be \emph{super additive}, i.e. $\GlobalF(x+y,k) \geq \GlobalF(x,k) + \GlobalF(y,k)$.
The polynomial constrains the created tokens $t_{i+1} \leq \GlobalF(t_i, k)$.
We express this condition with the following inductive rules.
\begin{mathpar}
  \infer
  {\K_0 \hookrightarrow (t_0, t_0') \;\; \m{valid}(k)}
  {}
  \and
  \infer
  {\Tokens, \K_{i+1} \hookrightarrow (t_{i+1}, t_{i+1}')\;\; \m{valid}(k)}
  {\Tokens\;\; \m{valid}(k) \and
  \K_{i} \hookrightarrow (t_i, t_i') \in \Tokens \and
  t_{i+1} \leq \GlobalF(t_i',k)}
\end{mathpar}
We mandate that this condition is satisfied by all the process typing rules presented in this paper.

% | creating virtual tokens: withdraw tokens 
The first token-specific typing rule we present is creating new tokens with \inline{$\nwithdraw$ K0 K1 n}.
\begin{mathpar} \small
  \inferrule*[right=$\m{tok}$]
  {\textcolor{blue}{\Tokens, \K_{i+1} \hookrightarrow} (t_{\textcolor{blue}{i+1}} + n, t_{\textcolor{blue}{i+1}}' + n) \semi
  \Psi \semi\wt, \D \entailpot{\B{q}}{\B{q'}} P :: (x : A)}
  {\textcolor{blue}{\Tokens, \K_{i+1} \hookrightarrow} (t_{\textcolor{blue}{i+1}}, t_{\textcolor{blue}{i+1}}') \semi \Psi \semi \wt, \D \entailpot{\B{q}}{\B{q'}} \hspace{4em} \\
    \hspace{5em}\m{withdrawToken} \; \K_i \; n\; \K_{i+1}  \semi P :: (x : A)}
% \vspace{-0.5em}
\end{mathpar}
We highlight the importance of the constraint that only tokens of type $\gamma_i$ can create tokens of type $\gamma_{i+1}$.
Here the ``real'' token type for $P$ is $\gamma_i$ and executing $\m{withdrawToken} K_i K_{i+1} n$ increases only the virtual token counts $t_{i+1}$ and $t_{i+1}'$ by $n$. 
The side condition of token validity ensures that $t_{i+1} + n \leq \GlobalF(t_i, k)$ where $K_i \hookrightarrow (t_i, t_i') \in \Gamma$. 

% | how does this manifest in the rules for choice: the new type constructors for sending import 
\paragraph{Exchanging Import Tokens}
Here we present the full typing rule for $\getpot$ for each side of the channel.
We highlight in blue the additions when taking token types into account.
\begin{mathpar} \small
  \infer[\getpot R]
  {\textcolor{blue}{\Tokens, \K_i \hookrightarrow} (t_i, t_i') \semi \Psi \semi \D \entailpot{\B{q}}{\B{q'}} \eget{x}{r \textcolor{blue}{: \K_i}} \semi P ::
  (x : \tgetpot{A}{r \textcolor{blue}{: \K_i}})}
  {\textcolor{blue}{\Tokens, \K_i \hookrightarrow} (t_i, t_i'+r) \semi \Psi \semi \wt, \D \entailpot{\B{q}}{\B{q'}} P :: (x : A)}
  %
  \and
  %
  \inferrule*[Right=$\getpot L$]
  {\textcolor{blue}{\Tokens, \K_i \hookrightarrow} (t_i, t_i') \semi \Psi \semi \D, (x : A) \entailpot{\B{q}}{\B{q'}} P :: (z : C)}
  {\textcolor{blue}{\Tokens, \K_i \hookrightarrow} (t_i, t_i'+r) \semi \Psi \semi \wt, \D, (x : \tgetpot{A}{r \textcolor{blue}{: \K_i}}) \\\ \entailpot{\B{q}}{\B{q'}} 
  \epay{x}{r \textcolor{blue}{: \K_i}} \semi P :: (z : C)}
\end{mathpar}
In the rule $\getpot R$, process $P$ storing $(t_i, t_i')$ import tokens of type $\K_i$
receives $r$ additional $\K_i$ tokens adding it to the current token counter, thus
the continuation executes with $(t_i, t_i'+r)$ tokens of type $\K_i$.
Note that validity of token context is trivially satisfied in this case since the
process is gaining import tokens.
%
In the dual rule $\getpot L$, a process containing $(t_i, t_i'+r)$ tokens of type $\K_i$
pays $r$ units along channel $x$ leaving $(t_i, t_i')$ import tokens of type $\K_i$ with
the continuation.
In this case, the validity of the token context establishes that $t_{i+1} \leq \GlobalF(t_i',k)$,
a condition that is necessary for successful typechecking.
The typing rules for the dual constructor $\tpaypot{A}{r : \K}$
are the exact inverse and omitted for brevity.
Similar to prior rules, the sender of the import tokens transfers the write token $\wt$
along with the import to the receiver.

% | the write tokens is added: UC requires one activation at any time and so the write-token is required in the context of processes when external/internal choice are used

% | potential: genpot, tick 
\paragraph{Potential}
Potential is another key design in import session types.
The main operation to interact with the potential mechanism is processes generated potential to be used with \igenpot and consuming it with \itick.
\begin{mathpar}
  \inferrule*[right=$\m{pot}$]
  {q+r \leq \GlobalF(t_{m}',k) \and K_{m} \hookrightarrow (t_{m}, t_{m}') \in \Tokens \\\
  k \semi \Tokens \semi \Psi \semi \wt, \D \entailpot{q+r}{q'+r} P :: (x : A)}
  {k \semi \Tokens \semi \Psi \semi \wt, \D \entailpot{q}{q'} \m{genPot} \; r \semi P :: (x : A)}
\end{mathpar}
A process initially storing $(q, q')$ potential units generates $r$ potential so that
the continuation contains $(q+r, q'+r)$ potential units.
Note, however, that the maximum potential that can be stored in a process is bounded by $\GlobalF(t_{m}',k)$
where $\GlobalF$ is the connection rate, $m$ is the simulation depth, and $k$ is the security parameter.
This restricts us from generating an unbounded amount of potential which could have violated the
polynomial execution cost bound.

Processes explicitly account for their own computation by executing \itick for every syntactic construct. NomosUC can be instrumented to automatically insert these before/after every operation.
\begin{mathpar}
  \infer[\m{tick}]
  {\Tokens \semi \Psi \semi \wt, \D \entailpot{q}{q'+r} \etick{r} \semi P :: (x : A)}
  {\Tokens \semi \Psi \semi \wt, \D \entailpot{q}{q'} P :: (x : A)}
\end{mathpar}
Note how the process starts with $q'+r$ potential units, and executing $\etick{r}$
consumes $r$ units leaving $q'$ potential for the continuation.
Note that this is the only rule that increments the work counter of a process.
And since this operation consumes $r$ units of potential, we can infer
that the total sum of potential and work of a closed set of processes will always be bounded.

%%%%%%%%%%%%%%%%%%%%%%%%%%%%%%%%%%%%%%%%%%%%%%%%%%%%%%%%%%%%%%%%%%%%%%%%%%%%%%%%%%%%%%%%%%%%%%%%%%%%%%%%%%%%%%%%%%%%%%%%%%%%%%%%%%%%%%%%%%%%%%%%%%%%%%%%%%%%
%%%%%%%%%%%%%%%%%%%%%%%%%%%%%%%%%%%%%%%%%%%%%%%%%%%%%%%%%%%%%%%%%%%%%%%%%%%%%%%%%%%%%%%%%%%%%%%%%%%%%%%%%%%%%%%%%%%%%%%%%%%%%%%%%%%%%%%%%%%%%%%%%%%%%%%%%%%%
%%%%%%%%%%%%%%%%%%%%%%%%%%%%%%%%%%%%%%%%%%%%%%%%%%%%%%%%%%%%%%%%%%%%%%%%%%%%%%%%%%%%%%%%%%%%%%%%%%%%%%%%%%%%%%%%%%%%%%%%%%%%%%%%%%%%%%%%%%%%%%%%%%%%%%%%%%%%

