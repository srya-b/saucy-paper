In this section we highlight the use of our token type abstract and the use of virtual tokens in NomosUC.
Specifically, we expand on the commitment ideal functionality \Fcom, used throughout this work, to show a simple example of a simulator sandboxing real protocols, other adversaries, and other simulators. 

\subsection{Commitment Protocol}
The commitment protocol to realize \Fcom exists in the \Fro-hybrid world. This means that protocol parties also have access to an idealized hash function \Fro in the real world. 
A random oracle accepts queries of size $k$ and generates a ``hash'' for them by sampleing $k$-bit randomness ($\{0,1\}^k$. 
Its hiding and binding properties ensure that a generated hash can be used to commit to knowledge of a pre-image and the real-world protocol need only send messages between the committer and receiver in the correct order. 
Let \O{x} be the reply of \Fro on query $x$. 
The committer committing to bit $b$ samples a bliding nonce $n \samplek$ and sends $c = \O{n \oplus b}$ to the receiver. When opening the commitment, the committer sends $b, n$ to the receiver who can check that $c \equiv \O{b \oplus n}$.
