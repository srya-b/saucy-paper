%In this section we work through the entire commitment example that has seen used throughout this paper, and we show composition by realizing a coin flipping ideal functionality \Fflip and a protocol realizing it in the \Fcom-hybrid world.
%Along th way, we address a new communication pattern that enables session types to enforce communication ordering over multipe channels and identify to set import requiremets for functionalities for composition.
%We start by presenting the random oracle functionality~\Fro which is used to realized \Fcom.

In this section we introduce a coin flip functionality \Fflip and discuss how composition affects its import requirements, and we introduce a design pattern where session types can enforce message ordering across multiple channels. 
\Fflip is realized by protocol $\prot{flip}$ in the \Fcom-hybrid world, and \Fcom is realized by $\prot{com}$ in the \Fropp-hybrid world.
We discuss applying composition to realize \Fflip with \prot{flip} composed with \prot{com}.
We push the realizing protocols, simulators, and process code to Appendix~\ref{app:flip} and focus here on the \Fflip and \Fropp session types.

\subsection{The Random Oracle}
The random oracle functionality captures an idealized hash function. It samples random strings of length $k$ as ``hash values`` and stores them in a table for deterministic hashes.
It allows both protocol parties and the adversary to request hashes from it, and we augment it with a single communication channel allowing parties to send messages to each other.
We refer to the augmented functionality as \Fropp from now on.

The type of \Fropp is described by two session types in Figure~\ref{fig:rotype}: one for each channel between \Fropp and a protocol party.
\begin{figure}
\begin{center}
\parbox{0cm}{
\begin{tabbing}
$\m{sandh}[a] = \textcolor{red}{\getpot^2} \ichoice{$\=$\mb{oracle}: \m{pid} \arrow \m{int} \tensor \echoice{$\=$\mb{hash}: \m{pid} \arrow \m{int} \tensor \textcolor{red}{\paypot^1} \m{hands[a]}},$ \\
\>$\mb{send}: \m{pid} \arrow \m{pid} \arrow \m{a} \product \m{sandh}[a]}$ \\
$\m{recv}[a] = \textcolor{red}{\getpot^2} \echoice{ \mb{msg}: \m{pid} \arrow \m{pid} \arrow \m{pid} \product \textcolor{red}{\paypot^1} \m{recv}[a]}$ \\
\end{tabbing}}
\end{center}
\caption{The session type for the random oracle. Two channels for each party to both query hashes and send/receive messages.}
\label{fig:rotype}
\end{figure}
%The random oracle is different from \Fcom in that it has one channel to all parties to use. This is due to the fact that its function is the same for all parties.
%Recall the session type and its structure discussed in Section~\ref{sec:execuc}. The augmented session type is given below:
%The design of the random oracle is different from \Fcom in that it has only one channel for all parties to communicate over.
%We discussed the unique structure of the session type for \Fro in Section~\ref{sec:execuc}: its type before and after interaction with a party is the same.
%This enables a dynamic set of parties to communicate with it by moving the \inline{pid} of the message sender/receiver into the type.
%Our augmented functionality's type retains this feature, as described by its session type:
%\begin{mathpar}
The type requires 2 units of import to be given to it when querying a hash or sending a message.
We justify 2 import tokens because the receiver of the message may need to perform polynomial work on activation by \Fropp, but it is never activated by \Z.
In this case, as shown by the commitment protocol in the appendix, it relies on the received message, alone, for runtime resources.

%Similarly, the functional types are given by:
%%One side effect of the session types is that we modify the standard UC channel to require receivers to ask for new messages sent to them.
%%We cannot directly deliver messages to their receivers, because the committer's and receiver's \inline{p2f} channel would end up with different types and back to \inline{party[a]}.
%%The corresponding functional message type between the protocol wrapper and functionality is also updated with inputs for the channel:
%\begin{lstlisting}[basicstyle=\scriptsize\BeraMonottFamily, mathescape]
%$\Type$ rop2f[a] = QHash of $\tgr{Int}$ | Send of pid ^ a 
%               | Recv ;
%$\Type$ rof2p[a] = RHash of $\tgr{Int}$ | Yes of pid ^ a 
%               | No ;
%\end{lstlisting}
%\paragraph*{\textbf{Commitment Protocol}}
%The real world commitment protocol is constructed in the random oracle model in the way of ~\cite{hofheinzcommitment}.
%Its incoming channel from \Z is typed identically to \Fcom to ensure that emulation and composition hold.
%
%%We show a snippet of the sender code for computing the commitment for its input bit in Figure~\ref{lst:committer}, and the relevant code the the receiver checking the commitment in Figure~\ref{lst:receiver}.
%The sender accepts a bit from \Z and generates a $k$-bit nonce to blind the bit through a \inline{sample} of randomness~\footnote{Blinding is necessary otherwise \A knows the pre-image and can query \Fro for its hash value.}. 
%It creates the commitment by sending \Fropp the blinded bit and receiving a hash value from \inline{p2f}.
%Finally it sends the hash to the receiver (which has pid=2).
%Conversely, the receiver must poll \Fropp for the commitment \inline{h} message, notify \Z of the commitment, and when it receives the bit and the nonce it checks that its hash with the commitment.
%We provide snippets of code for the most interesting parts of the committer and the receiver, creating a commitment and checking a commitment respectively, in Figures~\ref{lst:committer} and \ref{lst:receiver} in Appendix~\ref{app:protcol}.
%In total the committer is given 2 units of import: 1 for receive a hash and 1 to give to the receiver so that it can compute a hash as well.
%%$\tb{case}$ $\$$z2p (
%%  commit => 
%%The protocol works as follows:
%%\begin{enumerate}
%%\item When the committer receives a \inline{Commit(b)} message from \Z, it samples some random bits $r$ and generates a hash $h$ by sending \inline{SHash(b + r)} to \Fro.
%%\item It then sends the commitment to the receiver who notifies \Z with a \inline{committed} message.
%%\item Finally, when \Z instructs the committer to \inline{Open} the commitment, it sends bit \inline{b} and randomness \inline{r} to the receiver. The receiver checks the commitment, with \inline{b} and \inline{r}, against \Fro and outputs \inline{Open(b)} to \Z if it checks out.
%%\end{enumerate}
%
%We elide presenting an overview of the simulator to realized $\Fro \xrightarrow{\pi_\m{com}} \Fcom$ in this section and focus, instead, on demonstrating our main result: composition.


\subsection{Coin Flipping}
We present secure coin flipping here as another case study and one that makes use of our composition operator. 
Additionally, this example makes use of a neat trick we use to get more guarantees out of the NomosUC type system.

Securely flipping a coin is a basic cryptographic primitive whose ideal functionality \Fflip is captured by the session types in Figure~\ref{fig:fliptype}.
It's a 2-party protocol where one party is the initiator of the flip and the other is a receiver.
The desired security property is that the coin flip is entirely unbiased by either of the two parties. The corresponding ideal functionality \Fflip code samples a bit from from its random tape and returns it as the coin flip.
Unlike \Fropp, \Fflip does not output directly to either of the parties unless polled with a \m{getflip} message.

\begin{figure}
\begin{center}
\parbox{0cm}{
\begin{tabbing}
	$\m{flip[K]\{n\}\{m\}} = \textcolor{red}{\getpot^n} \; \ichoice{\mb{init}: K \arrow \ichoice{\mb{getflip}:  \echoice{ $\=$\textcolor{red}{\getpot^m}\mb{yes}: Bit \product 1,$ \\
	\>$\textcolor{red}{\getpot^m} \mb{no}: \m{flipped}[K]}}}$ \\
	$\m{receiver[K]\{n\}\{m\}} = \textcolor{red}{\paypot^n} \ichoice{\mb{getflip}: \echoice{$\=$ \textcolor{red}{\getpot^m}\mb{yes}: Bit \product 1,$\\
	\>$\textcolor{red}{\paypot^m} \mb{no}: \m{receiver[K]}}}$ \\
	$\m{adv[K]\{n\}} = \textcolor{red}{\getpot^n}\echoice{\mb{flip}: K \arrow \echoice{\mb{flipok}: \ichoice{$\=$\mb{yes}: 1,$\\
	\>$\mb{no}: 1}}}$
\end{tabbing}}
\end{center}
\caption{The session types describing \Fflip's channels with the flipper, receiver, and the adversary.}
\label{fig:fliptype}
\vspace{-4mm}
\end{figure}

The session type for \Fflip additionally accepts a type parameter $K$, takes in a value of type $K$ from the flipper, and sends it to the receiver when outputting the flip.
We use this $K$ as a means of ordering communication across both the flipper's and receiver's channel to \Fflip.
Recall that session types can normally only enforce message ordering on one channel, but the type system prevents \Fflip from sending a value of type $K$ to the receiver until the sender concretizes $K$ by sending it to \Fflip.
Using this design pattern, we are able to capture more of the intended behavior of a functionality in the type system.

\Fflip is also parametric in the amount of import it sends/receives unlike \Fropp. The reason is that even though \Fflip is one-shot and does constant work,
statically defining it with zero import imposes the constraint that only protocols which also do constant work (and rely on functionalities that take zero import) 
can realize it. 
This negatively impacts the reusability of NomosUC ideal functionality definitions and limits the protocols that can be proven secure.

This is best seen when composing \prot{flip} with \prot{com} in the \Fropp-hybrid to realize \Fflip. \Fropp requires 2 units of import on every input, and, therefore \prot{flip} (and \Fflip as they share the same type) also require at least 2 units of import in order for one to be indistinguishable from the other.
In the Appendix we describe, in detail, realizing \Fflip using the composition theorem and replacing \Fcom in the real world with $\prot{com}$ in the \Fropp-hybrid world.
We also provide a protocol for realizing both \Fflip and \Fcom (via simulators) and a simulator composition operator in Appendix~\ref{app:flip}.

%\begin{figure}
%\centering
%\begin{lstlisting}[basicstyle=\footnotesize\BeraMonottFamily, frame=single, mathescape]
%$\Type$ flipper[K] = +{ init: K -> flipped } ;
%$\Type$ fflipped = +{ getflip: &{ flip: Bit * 1 ,
%                                  noflip: fflipped }} ;
%$\Type$ receiver[K] = +{ getflip: &{ flip: K -> Bit -> 1 ,
%                                     noflip: recever[K] }} ;
%$\Type$ adv[K] = &{ flipped: K -> deliver } ;
%$\Type$ deliver = &{ askflip: +{ yes: deliver,
%                                 no: deliver }}
%\end{lstlisting}
%\end{figure}

%Upon a \inline{getflip} request, the adversary is activated and asked whether to deliver the outcome as shown in Figure~\ref{fig:optional}. 
%\begin{figure}
%\centering
%\begin{lstlisting}[basicstyle=\small\BeraMonottFamily, frame=single, mathescape]
%$\ncase$ $\$$F (
%  getflip =>
%    $\$$A.askflip ;
%    $\ncase$ $\$$A (
%      yes =>
%        $\$$F.flip ; send $\$$F b ;
%      no =>
%        $\tg{(* loop and wait for getflip *)}$
%    )
%)
%\end{lstlisting}
%\caption{} \label{fig:optional}
%\end{figure}


%Here we demonstrate how to compose the simulators from the two experiments to create a simulator to prove Theorem~\ref{thm:compose}.
%Two simulators being composed are: \SIM{com} for $\Fro \xrightarrow{\prot{com}} \Fcom$ and \SIM{flip} for $\Fcom \xrightarrow{\prot{flip}} \Fflip$. 
%The code for the composed is very similar to the simulator for the Dummy Lemma described in Section~\ref{sec:dummy} and expanded on in Appendix~\ref{app:dummy}.
%The exact connects are different but it follows identical virtualization and sandboxing.
%Therefore, we elide any code snippets from this section and, instead, present a high-level description of the simulator.
%
%For the sake of generality, we refer to protocols $\pi$, $\rho$, and functionalities $\F_1$, $\F_2$, and $\F_3$, where $\rho$ is \prot{flip}, $\pi$ is \prot{com}, $\F_1$ is \Fro, $\F_2$ is \Fcom, and $\F_3$ is \Fflip.
%The numbered steps in the description below correspond to the numbered arrows in Figure~\ref{fig:simcomp}.
%\begin{enumerate}
%\item Input from \Z: \inline{Z2A2P(p, msg)} for dummy parties of $\F_1$ and \inline{Z2A2F(msg)} for $\F_1$  are forwarded to \SIM{\pi}, and, Outputs from \SIM{\pi}: \inline{F2A2Z(msg)} and \inline{P2A2Z(p,msg)} are forwarded to \Z unaltered.
%\item Inputs from \SIM{\pi}: \inline{A2F(msg)} for $\F_2$ and \inline{A2P(msg)} for dummy parties of $\F_2$ are forwarded to \SIM{\rho} as \inline{Z2A2F(msg)} \inline{Z2A2P(p,msg)}, respectively.
%\item Outputs from \SIM{\rho}: \inline{P2A2Z(p,msg)} from simulated parties of $\rho$  and \inline{F2A2Z(msg)} from the simulated $\F_2$ are forwarded to \SIM{\pi} as \inline{P2A(p,m)} and \inline{F2A(m)}, respectively.
%\item Inputs from \SIM{\rho}: \inline{A2F(msg)} for $\F_3$ and \inline{A2P(p,msg)} for dummy parties of $\F_3$ are forwarded unaltered, and, Outputs from $\F_3$ and its ideal parties: \inline{F2A(msg)} from $\F_3$ and \inline{P2A(p,msg)} from its ideal parties is forwarded to \SIM{\rho} unaltered.
%\end{enumerate}

