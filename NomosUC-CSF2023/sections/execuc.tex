In this section we highlight how \m{execUC} is encoded in NomosUC.
Recall from Section~\ref{sec:basic} that linear session types impose constraints on what forms of communication can be captured, and pose challenges to supporting 
a dynamic number of protocol parties, or functionalities, at runtime. 
We detail how our encoding realizes dynamic parties through the use of virtual tokens in our \partywrapper, and validate our encoding by realizing a few critical results from the UC literature: the Dummy Lemma, a composition operator, and a multisession operator. 
These three results allow us to ultimately conclude that NomosUC realizes full UC composition.
We also run through a case of a coin-flip protocol, composed with the cryptographic commitment, in the Appendix. 

\subsection{The UC Experiment}
The UC experiment is instantiated by the \msf{execUC} function. 
Its definition is straightforward owing to our use of providerless channels. 
\m{execUC} is parameterized by at least one virtual token type to allow for sandboxing (usually by the adversary or environment) and message type parameters for the protocol in question. 
It also takes in a security parameter $k$ and a random bit string $r$ that is used as a source of randomness for all future processes.
\begin{center}
\vspace{-2mm}
\parbox{0cm}{
\begin{tabbing} 
 $\m{execout}[a] = \textcolor{red}{\getpot^n} \echoice{\mb{exec}: $\=$ \ichoice{ \mb{out}: Bit \product 1}}$ 
 \end{tabbing}}
\vspace{-2mm}
\end{center}

Its type $\m{execout}[a]$ accepts $n$ import tokens, and returns an output \inline{Bit} as the result from \Z. 
The NomosUC type system requires explicit polynomials for checking runtime bounds, and this allows us to ensure a runtime that is $poly(k)$ as long as the the number of tokens $n \in poly(k)$.

All main processes in NomosUC are wrapped according to the providerless channel specification in Section~\ref{sec:basic}. 
Therefore, \m{execUC} creates only the shared parts of the providerless channel and passes them as input to the wrapped processes. The wrapped processes 
create the intermediate processes which offer the session type to the actual process.
For example \Z and \A are connected by the following channels:
\begin{lstlisting}[basicstyle=\footnotesize\BeraMonottFamily, mathescape]
#ztoa $\leftarrow$ channel_init[$\tp{K}$][$\tp{z2a}$]{$\tp{z2an}$}
#atoz $\leftarrow$ channel_init[$\tp{K}$][$\tp{a2z}$]{$\tp{a2zn}$}
$\tg{...}$
$\$$z <- Env[K][K1] k rng #ztop #ptoz #ztoa #atoz ;
\end{lstlisting}
The channels are typed with the real token type \inline{K} and the import / message types required for the protocol being analyzed.

\paragraph{Execution Parameters}
The environment, \Z, is the first spawned process and defines the main parameters for the rest of the execution: the session ID and the set of corrupted parties.  
According to its type \m{EtoZ} (below), \Z sends the SID (a user-defined type) and the corruption list to \m{execUC}, and \m{execUC} passes them as parameters to \F, \A and the \partywrapper.  
\Z starts on input \mb{start} with $n$ tokens. 
\begin{center}
\vspace{-2mm}
\parbox{0cm}{
\begin{tabbing}
 $\m{EtoZ}[a] = \ichoice{\mb{init}: $\=$ SID[a] \arrow [PID] \arrow$ \\
\>$\textcolor{red}{\getpot^n} \echoice{\mb{start} \arrow \ichoice{\mb{out}: Bit \arrow 1}}}$
 \end{tabbing}}
\vspace{-2mm}
\end{center}

Finally, the output bit, \Z's guess of which world it is in, forms the basis of the definition of indistinguishability.

\subsection{The \partywrapper}
We create the \partywrapper to handle spawning parties, create their channels, and connect them to the rest of the experiment.
Instead of parties directly communicating with \F and \Z through providerless channels, the \partywrapper runs them virtually and presents a ``real'' view to them.
The virtual tokens construct enables us to virtualize the protocol parties and satisfy the import their session types require while sandboxing them. 
The \partywrapper, in turn, is connected by single channels each of \F, \Z, and \A and routes communication to/from the appropriate protocol party. 
Its design forces the wrapper around \F to do the same. It must de-multiplex inputs from the \partywrapper and react similarly to messages fro newly created parties. 

\subsection{Emulation}
The central security definition in UC is indistinguishability between the real and ideal world experiments.
It is defined in terms of the ensemble of distributions created by the output bits from the partial term $(\m{execUC}\ \pi\ \F)$ over all possible random inputs and security parameters. 
We say that two worlds are indistinguishable if $\forall \A\ \exists \Sim\ \forall Z$ the ensembles have  \emph{statistical difference} that is negligible in $k$ (Definition~\ref{def:emulation}).

\begin{definition}[Emulation]\label{def:emulation}
If two protocols $(\pi, \F_1)$, $(\phi, \F_2)$, which we refer to only by \PI and $\phi$, emulated each other, then $\forall \A$ of type $\Delta_3'$ well-matched with \PI, there must $\exists \Sim$ of the same type,  well-matched with $\phi$, s.t. $\forall \Z$ : $\msf{execUC}(\pi, \F_1, \Z, \A)$ $\approx$ $\msf{execUC}(\phi, \F_2, \Z, \Sim)$:

\begin{mathpar}
	\footnotesize
	\inferrule*[right=emulate]
	{
		\F_1 : \Delta_{\F_1}', \F_2 : \Delta_{\F_2}' \semi
		\Delta_{\F_1}' \vdash \pi : \Delta_1' \semi \Delta_{\F_2}' \vdash \phi : \Delta_2' \semi \\
		\forall \A \ . \ \Delta_4, \Delta_1' \vdash \A :: \Delta_3', \matched{\A}{\pi}, \matched{\A}{\F_1} \\
		\Rightarrow \exists \Sim_\A \ . \ (\Delta_3, \Delta_2' \vdash \Sim_\A :: \Delta_3'), \matched{\Sim_\A}{\phi}, \matched{\Sim_\A}{\F_2} \semi \\
		\forall \Z \ . \ \matched{\Z}{\pi}, \matched{\Z}{\phi} \Rightarrow \\
		\msf{execUC} \ \pi\ \F_1\ \Z\ \A \sim \msf{execUC} \ \phi\ \F_2\ \Z\ \Sim_\A
	}
	{
		% EMULATION DEFINITION
		\lambda \A \, . \, \Sim_\A \vdash (\pi, \F_1) \sim (\phi, \F_2)
	}
\end{mathpar}
\end{definition}
The notation $e \leftrightarrow e'$ is used to denote two \emph{well-matched} terms: a configuration connected the two is well-typed. 

\subsection{Evaluation}
An important validation of our approach is the the Dummy Lemma~\ref{thm:dummythicclemma} which shows that a simulator, \DS, for the dummy adversary, suffices to prove emulation. 
\begin{theorem}[Dummy Lemma]\label{thm:dummythicclemma}
If \ $\exists \DS$ s.t. $ \DA, \DS \vdash \F_2 \xrightarrow{\pi} \F_1$ then $\forall \A \ \exists \Sim_\A$ s.t. $\Sim_{\A} \vdash  \F_2 \xrightarrow{\pi} \F_1)$ 
\end{theorem}
The proof of the Lemma is in the form of a constructor for a simulator for any arbitrary real world \A (in Appendix~\ref{sec:dummy}).
Though a simple Lemma, and simulator construction, the simplicity of the resulting simulator proof highlights the simplicity and expressiveness that our token heirarcy and sandboxing afford. 

\subsection{Single Composition}
In this section we present a composition operator for protocols that completes Theorem~\ref{thm:singlecomp}.

The composition allows replacement of a single ideal functionality $\F_2$ with a protocol $\pi$ that realizes it in the $\F_1$-hybrid world. 
A protocol party $\rho_i$ that gives input to $\F_2$ instead gives input to party $\pi_i$ through a providerless channel in the \partywrapper. 
Like \m{execUC} (full code in the Appendix), the operator connects wrapped processes with providerless channels as shown below. 
\begin{lstlisting}[basicstyle=\footnotesize\BeraMonottFamily, mathescape, frame=single]
$\nproc$ compose[$\tg{...}$][rho2f,f2rho]{rho2fn,f2rhon}
  $\tg{(* args *)}$ |- ($\$$ch: 1) = {
  #rho2pi $\leftarrow$ channel_init[K][rho2f]{rho2fn} ;
  #pi2rho $\leftarrow$ channel_init[K][f2rho]{f2rhon} ;
  
  $\$$r $\leftarrow$ rho $\leftarrow$ k rng sid pid #z2p #p2z #rho2pi #pi2rho; 
  $\$$p $\leftarrow$ pi $\leftarrow$ k rng sid pid #rho2pi #pi2rho #p2f #f2p;
}
\end{lstlisting}

Proving UC security under composition requires creating a simulator for $\rho \circ \pi$ that realizes $\F_3$. 
This is done by connecting the two simulators, \SIM{\pi} and \SIM{\rho}, in the same way as the Dummy Lemma: sandboxed \SIM{\pi} receives input from \Z, gives output to \SIM{\rho} which gives output to the ideal world.  
Also like the Dummy Lemma, the simulator construction provided in the Appendix is agnostic to the types of the composed protocols and can be easily specified thanks to our token heirarchy. 

\subsection{UC Composition}

\begin{theorem}[Composition]\label{thm:composition}
\begin{mathpar}
\inferrule*[right=compose]
{
	%(\pi, !\F_1) \sim (\idealP, F_2) \semi (\rho, !\F_2) \sim (\idealP, \F_3) \\ 
	!\F_1 \xrightarrow{\pi} \F_2 \semi !\F_2 \xrightarrow{\rho} \F_3 \\
	%\Rightarrow \exists \Sim(\A) \vdash (\rho^{!\F_2 \rightarrow (!\pi \, \circ \, \msf{squash})}, !\F_1) \sim (\idealP, \F_3)
}
{
	!\F_1 \xrightarrow{\rho \, \circ !\pi \circ \, \msf{squash}} \F_3
	%(\rho \, \circ \, !\pi \circ \msf{squash}, !\F_1) \sim (\idealP, \F_3)
}
\end{mathpar}
\end{theorem}

Full composition, Theorem~\ref{thm:composition}, extends Theorem~\ref{thm:singlecomp} to replace multipe concurrent instances of a functionality with a realizing protocol.
We provide it by first proving a much simpler theorem arising from the multisession operator ($!$), Theorem~\ref{thm:functor}, and Theorem \ref{thm:squash}.
Theorem~\ref{thm:squash} trivially shows a protocol \m{squash} such that $!\F \xrightarrow{\m{squash}} !!\F$, and the proof of Theorem~\ref{thm:functor} is trivially proven by replicating \SIM{\pi} for every instance of $\F_2$.  

\begin{theorem}[Multisession Composition]\label{thm:functor}
	\begin{mathpar}
		\inferrule*[right=MultiComp]
		{
			\F_1 \xrightarrow{\pi} \F_2
		}
		{
			!\F_1 \xrightarrow{!\pi} !\F_2
		}
	\end{mathpar}
\end{theorem}

We can now prove Theorem~\ref{thm:composition} in NomosUC:
\begin{proof}
By Theorem~\ref{thm:singlecomp} we have that $\F_1 \xrightarrow{\pi} \F_2$. Combine it with Theorem~\ref{thm:functor} and conclude that $!!\F_1 \xrightarrow{\rho \circ !\pi} !\F_3$. 
Finally, squash two $!!$ operators into one with Theorem~\ref{thm:squash} to get $!\F_1 \xrightarrow{\rho \circ !\pi \circ \m{squash}} \F_3$.
\end{proof}

