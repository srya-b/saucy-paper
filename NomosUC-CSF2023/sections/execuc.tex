So far we have introduced a type system and process calculus for writing and checking runtime that conforms to the import mechanism for polynomial runtime.
In this section, we show that NomosUC can realize any arbitrary ITM configuration, as possible in UC, and that we can easily encode the set of fundamental proofs and lemmas of the framework and a generic composition operator.
The main challenges we overcome in realizing arbitrary ITM systems is grappling with the parent-child relationship between processes (and the provider-client reationship in channels) that prevents cyclical topologies in the normal case. 
We introduce a simple construction called \emph{providerless channels} which relies on shared session types and uses our virtual token construction. 
Second we discuss how the dummy lemma, a composition operator, and the composition theorem can be realized by the introduction of \emph{providerless channels}.

\subsection{Capturing the UC Computation Model}
The first step to capturing full UC is realizing any arbitrary ITM system in NomosUC.
The ITM computational model is extremeley flexible to capture as universal a setting as possible.
Channels in NomosUC are linear, created by some process known as the provider, and impose a strict parent-child relationship between processes. 
Unlike the ITM model, this model prohibits cyclical topologies: a configuration where a process $A$ offers a channel to $C$, which offers a channel to $B$, which offers a channel to $A$ can not be realize by NomosUC.
We overcome this constraint by introducing \emph{providerless channels} using \emph{communicator processes} that rely on shared session types~\cite{balzer2017manifest}.
We will unpack these terms one by one.

Communicators act as a buffer between two processes.
The channels that communicators offer have a shared session type meaning that multiple clients can send it messages.
This relaxes the strict parent-child relationship by allowing processes $A$, $B$, and $C$ to communicate with each other through any number of communicators using an acquire/release mechanism. 
The communicator has the following polymorphic type:

\vspace{-1mm}
{\centering
\parbox{0cm}{
\begin{tabbing}
$\m{comm[\tau]\{n\}} = \up \echoice{$\=$ \textcolor{red}{\getpot^n} \mb{push}: \m{\tau} \arrow \m \down \m{comm[\tau]},$\\
\>$\textcolor{red}{\getpot^0} \mb{poll}: \ichoice{$\=$\textcolor{red}{\paypot^{n-1}} \mb{yes}: \m{\tau} \;\product \down \m{comm[\tau]},$\\
\>\>$\textcolor{red}{\paypot^0} \mb{no}: \; \down \m{comm[\tau]}}}$
\end{tabbing}}
\par}

\todo{where do we introduce the write-tokem, we must do it}
The communicator type uses the type constructors $\up$ and $\down$ to indicate that 
access to the channel must be \emph{acquired} and eventually \emph{released}, respectively.
The acquire-release paradigm, along with the write token, is necessary for shared session types in order to prevent non-determinism.
Communicators are written to by a sender that acquires the channel and $\mb{push}$es a functionally 
typed message along with some amount of import indicated by a parameter to the type above. 
A receiver acquires the 
channel once the sender releases it and asks the communicator for new messages with $\mb{poll}$
If there is a message, a $\mb{yes}$ label is received, it is sent with some import
otherwise the channel takes the $\mb{no}$ branch.
We provide the full typing rule of shared session types in Appendix~\ref{app:typing_rules}.
\begin{figure}
	\begin{subfigure}{0.5\textwidth}
	\begin{center}
	\includegraphics[scale=0.5]{figures/p_and_q.pdf}
	\caption{Goal: $P$, $Q$ connected by channels labeled with their types.}
	\label{fig:pandq}
	\end{center}
	\vspace{0.1em}
	\end{subfigure}
	\begin{subfigure}{0.5\textwidth}
	\begin{center}
	\includegraphics[scale=0.4]{figures/newPandQ.pdf}
	\caption{Internal implementation of channels using communication\\wrapper
	with channels labeled with their session types.}
	\label{fig:newpandq}
	\end{center}
	\end{subfigure}
	\caption{Two ITM configurations. One possible with ITMs (top) and one realized in NomosUC (bottom). Arrows indicate direction of messages
	and labels indicate types.}
	\vspace{-1em}
\end{figure}
The downside of communicators is that all communication ends up using the same session type $\m{comm[m]}$ and we lose the advantages of expressing and enforcing interactive protocols. 

We abstract normal channels into \emph{providerless channels}: a black-box connecting some processes $P$ and $Q$, which internally consists of communicators and dummy processes. 
The dummy processes offer a channel of the session type expected by $P$ or $Q$ and intermediate communication between them and the communicator connecting them. 
In Figure~\ref{fig:newpandq} we show an illustration of a providerless channel between the two processes. 
Here, the processes intend to communicate through two unidiretional channels in \ref{fig:pandq} (forming a cycle) with the session types \m{PtoQ} and \m{QtoP}.
The dummy processes $a_P$ and $a_Q$ offer the channels to $P$ with the requisite session types.
On message from $P$, $a_P$ case matches on the label in the session type and sends a functionally typed message of type $\tau_\m{PtoQ}$ to the communicator. 
Similarly, when $a_Q$ receives a message from the communicator, it case matches on the message in the buffer and sends the corresponding label on the linear channel, of type \m{PtoQ}, to $Q$. 


%The dummy processes consist only of case matches and, therefore, given a simple mapping between the labels of \m{PtoQ} and $\tau_\m{PtoQ}$, $a_P$ is trivially generated at compile-time.
The advantage of this construction is that it is simple. 
Instantiating Figure~\ref{fig:newpandq} begins with creating the communicators, then a wrapper around $P$ spawns $a_P$/$b_P$, connects them to \m{C} and connects their offered channels to $P$. 
The dummy processes are trivial as they only consist of a case match between a session type label and a functionally typed message for the communicator.
\emph{Therefore, rather than force the programmer to create it themselves, NomosUC can trivially generate the dummy processes at compile time given the relevant types.}
We emphasize this point as it is crucial in understanding the simpicity of composition: the key words in code are replaced by generated processes at compile-time.

Furthermore, the type system can still enforce message ordering (as well as the $\paypot{R}$ and $\getpot{R}$ type constructors), and a system of processes can be arbitrarily connected. 
For example, if \Fdb from the previous section is connected to a party with a providerless channel, 
$\m{db}[k][v]$ remains the same and $\tau_\m{db[k][v]} =$ Store \inline{of PID k v |} OK \inline{of PID |} Get \inline{of PID k |} Yes \inline{of v |} No. 


\todo{mention somewhere here that the type only accepts a constant import, but we don't care about precise time-bounds only as far as the import in the type tells the programmer what kind of computation the functionality/protocol expects to do.}
Intuitively, this construction lets NomosUC processes realize arbitrary ITM systems thus capturing the generality of that computation model. 
In Appendix~\ref{app:itm} we sketch how NomosUC processes can realize any ITM system and vice versa (NomoUC $C' \Leftrightarrow$ ITMs $M'$).

\paragraph*{\textbf{Dynamic Parties}}
With providerless channels and sandboxxing, NomosUC is able to overcome a major limitation of prior UC formalisms: no dynamic creation of protocol parties or functionalities (and ITMs in general).
The \partywrapper internally creates and runs all protocol parties and acts as a single endpoint for communicatio with \Z, \F, and \A. 
Directly connecting a dynamic number of parties to other external processes in UC is cumbersome as all processes must be made aware of the new party and its channels, and some code must handle the creation of the new parties.
Like addressing in EasyUC (and EasyCrypt), the \partywrapper acts almost like an addressing interface where messages include the receipient's \inline{PID}.
We reuse the labelling from Figure~\ref{fig:newpandq}.
The \partywrapper communicates using non-descriptive session types where messages are multiplexed by \m{PID}.
For example, communication from \partywrapper to \Fdb is mediated by a providerless channel where the session type is
\begin{tabbing}
	$\mi{type} \; \m{P2F[\tau]\{n\}} = \ichoice{\textcolor{red}{\paypot{n}} \; \mb{p2f}: \m{PID} \arrow \m{\tau} \arrow \m{P2F[\tau]\{n\}}}$
\end{tabbing}
and where $\tau =$ \inline{Store $\tg{ of }$ k v | Get $\tg{ of }$ k} and $n = 1$, and the opposite direction is intuitive. 

New parties are spawned the first time they're sent a message like in UC. 
The \partywrapper creates the providerless channels to give to the sandboxed processes and routes the incoming message to it--all with a virtual token type.
It then listens for outgoing messages and relays them to the external recipient, this time with real import. 
An example of how the \partywrapper handles incoming messages and party creation is given in Appendix~\ref{app:arbparties}. 
As expected, sandboxing and providerless channels make the construction straightforward.
\todo{if we have space we should put that example here.}

%%%%%%%%%%%%%%%%%%
\todo{no more channels vs tapes}
%\paragraph*{\textbf{On Channels vs Tapes}}
%\todo{This is the passage from the paper: this modeling does not allow representing realistic
%situations where the number and makeup of components changes as the system evolves. It also does
%not capture commonplace situations where the sender has only partial information on the identity
%or code of the recipient. It also doesn’t account for the cost of message addressing and delivery; in
%a dynamically growing systems this complexity may be an important factor. Finally, it does not
%account for dynamic generation of new programs.}
%The UC framework specifically addresses prior models of distributed computation that model communication through names channels, as we do in NomosUC.
%The work suggests that though such a model is clean an elegant it doesn't allow scenarios where a sender may not have complete information about the identity or code of the receiver.
%Furthermore, it doesn't account for situations where the components in a system of ITMs evolves and changes, for example dynamic generation of new programs.
%%%%%%%%%%%%%%%%%%

\subsection{The UC Experiment}
The definition of \m{execUC} is straightforward owing to our mechanism of providerless channels. 
\m{execUC} is aways parameterized by at least one virtual token type to allow for sandboxing (specifically for the \partywrapper)
and message type parameters for the protocol in question. 
It also takes in a security parameter $k$ and a random bit string $r$ that is used as a source of
randomness for all future processes.
\m{execUC} offers the following type:
\begin{center}
\vspace{-2mm}
\parbox{0cm}{
\begin{tabbing} 
 $\m{execout}[\K][a]\{n\} = \echoice{ \textcolor{red}{\getpot^{\{n : \K\}}}\mb{exec}: \m{Int} $\=$ \ \ichoice{ \mb{out}: \m{Bit} \product 1}}$ 
 \end{tabbing}}
\vspace{-2mm}
\end{center}
The type is straightforward: the UC experiment is started with some initial amount of tokens $n$ (a user-defined parameter) and an \m{Int} security parameter, and it eventually returns the output \m{Bit} from \Z as its guess or which world it is in.
As long as $n$ is polynomial in the security parameter, the type system guarantees the execution is $poly(k)$.

As described in the providerless channels paragraph, \m{execUC} calls a \inline{channel_init} and connects them to wrapped processes such as \inline{wrap_adv}.
The providerless channel construction replaces these calls, with the generated portions of providerless channels.
\emph{In general, for the rest of this work, when we refer to processes we refer to their wrapped versions as defined by providerless channels.}

%\m{execUC} only spawns processes already wrapped according to the providerless channel specification in Section~\ref{sec:basic}.
%Therefore, it only has to spawn the part of the channel that connects wrapped processes.
%We make this separation because the wrapped processes code is autogenerated given a specification of the session type and functional
%type associated with the process.

%All main processes in NomosUC are wrapped according to the providerless channel specification in Section~\ref{sec:basic}. 
%Therefore, \m{execUC} creates only the part of the providerless channel (e.g. $\m{PtoQ}$ channel from Figure~\ref{fig:newpandq})
%and passes them as input to the communicator wrappers.
%The wrapper creates the intermediate processes and the shared session types providing the linear channel to \m{execUC}.
%For example \Z and \A are connected by the following channels:
%\begin{lstlisting}[basicstyle=\footnotesize\BeraMonottFamily, mathescape]
%$\$$ztoa $\leftarrow$ channel_init[$\tp{G}$][$\tp{z2a}$]{$\tp{z2an}$}
%$\$$atoz $\leftarrow$ channel_init[$\tp{G}$][$\tp{a2z}$]{$\tp{a2zn}$}
%$\tg{...}$
%$\$$z <- env[G] k rng $\$$ztop $\$$ptoz $\$$ztoa $\$$atoz ;
%\end{lstlisting}

%%%%%%%%%%%%%%%%%%%%%%%%%%%%%%%
\todo{I removed the whole part about environment and its type, seems not as important.}
%The environment, \Z, is the first process spawned and its session type states it defines an session id (\m{SID}) and list of corrupt parties (\m{[PID]}) used for this execution.
%When finished it outputs a \m{Bit} decision that is forwarded out by \m{execUC}.
%%Once channels are created, the environment \Z is the first process to be spawned.
%%As it's type \m{EtoZ} indicates below, it determines the main parameters of the experiment: the session id (\m{SID}) according to a user-defined type, and 
%%it determines the set of corrupt parties. 
%%\Z starts execution given some import $n$ by \m{execUC}.
%%These are given by \m{execUC} to the rest of the processes it spawns. 
%\begin{center}
%\vspace{-2mm}
%\parbox{0cm}{
%\begin{tabbing}
% $\m{EtoZ}[a]$ = $\m{SID}[a] \arrow [\m{PID}] \arrow \echoice{\textcolor{red}{\getpot^n} \mb{start}: \m{Bit} \arrow 1}$
% \end{tabbing}}
%\vspace{-2mm}
%\end{center}
%%%%%%%%%%%%%%%%%%%%%%%%%5

\paragraph*{\textbf{Emulation}}
The central security definition in UC is indistinguishability between the real and ideal world experiments.
It is defined in terms of the ensemble of distributions created by the output bits from the partial term
$(\m{execUC}\ \pi\ \F)$ over all possible random inputs and security parameters. 
We say that two worlds are indistinguishable if $\forall \A\ \exists \Sim\ \forall Z$
the \emph{statistical difference} in ensembles from the two worlds is negligible in $k$ (see
Definition~\ref{def:emulation} below).

\begin{definition}[Emulation]\label{def:emulation}
If two protocols $(\pi, \F_1)$ and $(\phi, \F_2)$, which we refer to only
by \PI and $\phi$, emulated each other, then $\forall \A$ of type $\Delta_3'$ well-matched with \PI, there must $\exists \Sim$ of the same type,  well-matched with $\phi$, s.t. $\forall \Z$ : $\msf{execUC}(\pi, \F_1, \Z, \A)$ $\approx$ $\msf{execUC}(\phi, \F_2, \Z, \Sim)$:

\begin{mathpar}
	\footnotesize
	\inferrule*[right=emulate]
	{
		\F_1 : \Delta_{\F_1}', \F_2 : \Delta_{\F_2}' \semi
		\Delta_{\F_1}' \vdash \pi : \Delta_1' \semi \Delta_{\F_2}' \vdash \phi : \Delta_2' \semi \\
		\forall \A \ . \ \Delta_4, \Delta_1' \vdash \A :: \Delta_3', \matched{\A}{\pi}, \matched{\A}{\F_1} \\
		\Rightarrow \exists \Sim_\A \ . \ (\Delta_3, \Delta_2' \vdash \Sim_\A :: \Delta_3'), \matched{\Sim_\A}{\phi}, \matched{\Sim_\A}{\F_2} \semi \\
		\forall \Z \ . \ \matched{\Z}{\pi}, \matched{\Z}{\phi} \Rightarrow \\
		\msf{execUC} \ \pi\ \F_1\ \Z\ \A \sim \msf{execUC} \ \phi\ \F_2\ \Z\ \Sim_\A
	}
	{
		% EMULATION DEFINITION
		\lambda \A \, . \, \Sim_\A \vdash (\pi, \F_1) \sim (\phi, \F_2)
	}
\end{mathpar}
\end{definition}
The notation $e \leftrightarrow e'$ is used to denote two \emph{well-matched} process terms meaning
that they have the \emph{same type on all the channels} used and provided.
Without a formal logic for security proofs, the emulation definition enforces that the environment sees the same pattern of output by requiring the session types between the worlds match 

% \paragraph*{\textbf{Dummy Lemma}}
\begin{theorem}[Dummy Lemma]\label{thm:dummythicclemma}
If \ $\exists \DS$ s.t. $ \DA, \DS \vdash \F_2 \xrightarrow{\pi} \F_1$ then $\forall \A \ \exists \Sim_\A$ s.t. $\Sim_{\A} \vdash  \F_2 \xrightarrow{\pi} \F_1)$ 
\end{theorem}
An important validation of our approach is the Dummy Lemma which shows that a simulator \DS for the dummy adversary, suffices to prove emulation. 
At a high-level, \DS works for all \Z even a \Z that internally runs any possible \A and gives its output to \DS.
The proof is the simulator constructor which runs any other \A and \DS within a sandbox and forwards messages between them (described in Appendix~\ref{sec:dummy}).

\paragraph*{\textbf{Single Composition}}
Recalling Theorem~\ref{thm:singlecomp}, composition allows replacement of a single ideal functionality $\F_2$ with a protocol $\pi$ that realizes it in the $\F_1$-hybrid world. 
A protocol party $\rho_i$ that gives input to $\F_2$ instead gives input to party $\pi_i$ through a providerless channel in the \partywrapper. 

\begin{lstlisting}[basicstyle=\footnotesize\BeraMonottFamily, mathescape, frame=single]
$\nproc$ compose[K,K1,s][z2rho,rho2z][pi2f,f2pi]$\tg{...}$
  (k: Int), (rng: [Bit]), (sid: SID[s]), (pid: PID),
  (#z2p: comm[K][z2rho]{z2rhon}), $\tg{...}$ 
  (#p2f: comm[K][pi2f]{pi2fn}) $\tg{...}$
    $\vdash$ ($\$$ch: 1) =
{
  #rhop2f $\leftarrow$ channel_init[K][rho2f]{rho2fn} ;
  #piz2p $\leftarrow$ wrap_z2p[K][rho2f]{rho2fn} $\leftarrow$ #rhop2f ;
  #piz2p $\leftarrow$ channel_init[K][rho2f]{rho2fn} ;
  #rhof2p $\leftarrow$ wrap_p2z[K][rho2f]{rho2fm $\leftarrow$ #piz2p ;

  $\leftarrow$ gen_wrapper_rho[K1] $\leftarrow$ 
    k rng sid pid #z2p #p2z #rho2pi #pi2rho ;
  $\leftarrow$ gen_wrapper_pi[K1] $\leftarrow$ 
    k rng sid pid #rho2pi #pi2rho #p2f #f2p ; 
}
\end{lstlisting}
Providerless channels make connecting parties simple while still enforcing their session types and import requirements. 
Above, \inline{compose} relies on generated providerless channels connecting $\rho$ and $\pi$ (wrapped) together where $\rho$ sends to $\pi$ rather than the functionality it is replaced. 
The pair form a single protocol whose instances are spawned as a normal protocol by the \partywrapper: $\rho$ communicates with \Z, $\pi$ communicates with \F, and the composed protocol is wrapped and connected to the outside with providerless channels of its own.

%The operator interacts with the wrapped versions of the constituent protocols $\rho$ and $\pi$, and its parametric type ensures that even multisession versions of a protocol $\pi$ can be composed with $\rho$.
%It creates channels between the two parties and wraps the message sent from $\rho$ to appear as input from \Z to $\pi$. 
%
%In the next section we talk briefly about building a zero-knowledge UC experient by applying the multisession extension of $!\Fcom$ used throughout this work.
%We talk about it at a high-level and relegate a more detailed specification of the protocol in the appendix.

Proving UC security under composition requires creating a simulator for $\rho \circ \pi$ that realizes $\F_3$. 
This is done by connecting the two simulators, \SIM{\pi} and \SIM{\rho}, in the same way as the Dummy Lemma: sandboxed \SIM{\pi} receives input from \Z, gives output to \SIM{\rho} which gives output to the ideal world.  
Also like the Dummy Lemma, the simulator construction provided in the Appendix is agnostic to the types of the composed protocols and can be easily specified thanks to our token hierarchy. 

% \paragraph*{\textbf{UC Composition}}
\paragraph*{\textbf{Parallel Composition}}
UC composition extends Theorem~\ref{thm:singlecomp} to replace multiple concurrent instances of a functionality with a realizing protocol. 
The multisession operator $!$ when applied to a functionality ($!\F$) or protocol ($!\pi$), allows for the creation of arbitrary many instances of the \F or $\pi$.
Messages from a protocol to $!\F$ include an additional user-defined session id that distinguishes different instances. 
The multisession composition theorem below, Theorem~\ref{thm:functor}, shows that UC emulation holds under the multisession operator. 
The simulator for this theorem is trivial as it spawns a new simulator for every new instances of $!\F$/$!\pi$.
We use Theorem~\ref{thm:functor}, whose simulator is found in Appendix~\ref{app:ms} and a simpler Theorem~\ref{thm:squash}, which allows $!!\F \xrightarrow{\msf{squash}} !\F$, 
to show that the UC composition Theorem~\ref{thm:composition} holds below.

\begin{theorem}[Multisession Composition]\label{thm:functor}
\vspace{-0.5em}
	\begin{mathpar}
		\inferrule*[right=MultiComp]
		{
			\F_1 \xrightarrow{\pi} \F_2
		}
		{
			!\F_1 \xrightarrow{!\pi} !\F_2
		}
	\end{mathpar}
\end{theorem}

%\begin{theorem}[Composition]\label{thm:composition}
%\vspace{-0.5em}
%\begin{mathpar}
%\inferrule*[right=compose]
%{
%	%(\pi, !\F_1) \sim (\idealP, F_2) \semi (\rho, !\F_2) \sim (\idealP, \F_3) \\ 
%	!\F_1 \xrightarrow{\pi} \F_2 \and !\F_2 \xrightarrow{\rho} \F_3 \\
%	%\Rightarrow \exists \Sim(\A) \vdash (\rho^{!\F_2 \rightarrow (!\pi \, \circ \, \msf{squash})}, !\F_1) \sim (\idealP, \F_3)
%}
%{
%	!\F_1 \xrightarrow{\rho \, \circ !\pi \circ \, \msf{squash}} \F_3
%	%(\rho \, \circ \, !\pi \circ \msf{squash}, !\F_1) \sim (\idealP, \F_3)
%}
%\end{mathpar}
%\end{theorem}

With these preliminary theories we can combine their simulator proofs through our generic composition operator to show that the composition operator defined in NomosUC satisfies the UC composition theorem:
\begin{proof}
By Theorem~\ref{thm:singlecomp} we have that $\F_1 \xrightarrow{\pi} \F_2$. Combine it with Theorem~\ref{thm:functor} and conclude that $!!\F_1 \xrightarrow{\rho \circ !\pi} !\F_3$. 
Finally, squash two $!!$ operators into one with Theorem~\ref{thm:squash} to get $!\F_1 \xrightarrow{\rho \circ !\pi \circ \m{squash}} \F_3$.
\end{proof}


%$!\F_1 \xrightarrow{\pi} \F_2, \, !\F_2 \xrightarrow{\rho} \F_3 \xRightarrow[\left( !!\F_2 \xrightarrow{!\pi} !\F_2 \right)]{\textsc{Multi-Comp}} \; !!\F_1 \xrightarrow{\rho \circ !\pi} \F_3$ \\ \\
%
%$\xRightarrow[\left( !\F_1 \xrightarrow{\msf{squash}} !!\F_1 \right)]{\textsc{Squash}} \; !\F_1 \xrightarrow{\rho \circ !\pi \circ \msf{squash}} \F_3$
%
