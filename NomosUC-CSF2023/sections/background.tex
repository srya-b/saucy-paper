The UC framework~\cite{canettiUC} defines security using
a real world / ideal world paradigm:
the real world features a protocol $\pi$ and an arbitrary adversary \A that controls corrupt parties;
To achieve the ideal/real security relation, we roughly need to show that ``any possible attack in the real world can also be exhibited
in the ideal world''.
This means constructing a simulator \Sim that plays the role of an adversary in the ideal world.
What makes UC the strongest definitional model is that the inputs and number of honest parties are chosen \emph{adaptively}
and \emph{dynamically}, by a distinguishing environment \Z, which can also communicate with the adversary and interleave its interactions with honest parties.

The UC framework is defined on top of a communication model called interactive turing machines (ITMs), in which multiple Turing-complete processes run concurrently in a system and communicate by passing messages over channels~\cite{canettiUC}.
%Although the Turing-complete computations can be instantiated in any reasonable core calculus, the approach to message passing in ITMs has some essential but subtle restrictions.
%In order to do cryptographic analysis, we need to make reductions to (ordinary sequential) probabilistic polynomial time computations (PPTs).
%This rules out, for example, the ordinary semantics of $\pi$-calculus, which introduces unbounded non-determinism with the possible scheduler choices.
%This adversary model leads to a flexible approach to composition, which we'll say more about shortly.
The UC security definition is more formally defined by the following relationship between two UC exections. 
Here, \m{execUC} defines a UC experiment that takes four parameters and creates the main ITMs of the execution, namely the environment \Z, an adversary \A, and an ideal functionality \F.  
Parties running the protocol $\pi$ are created at runtime when they are sent messages. 

We say that a protocol $\pi$ realizes an ideal functionality $\F_2$
if for every adversary $\A$, we can construct a simulator $\Sim$, such that the real and ideal worlds are indistinguishable to any environment \Z:
\begin{equation}
  \label{eqn:emulation}
  \forall \A \; \exists \Sim \; \forall \Z, \; \; \msf{execUC} \; \Z \, \idealP \, \F_2 \, \Sim \sim \msf{execUC} \; \Z \, \pi \, \F_1 \, \A
\end{equation}
At the end of its execution \Z outputs a bit as its guess for which world it is in.
We denote indistinguishability with the $\sim$ symbol, and it means that the ensembles of distributions resulting from the output of \Z in each execution, over all possible randomness and security parameters, have \emph{statistical difference} that is negligible in $k$. 

Generally, emulation deals with arbitrary protocols in both worlds, however, for UC realization (above) specifically the protocol in the real world is \idealP, or the \emph{dummy protocol}, that forwards messages between $\F_2$, \Z,  an \A. 
Similarly, the real world can also contain an ideal functionality $\F_1$ that may captures network assumptions or assumed primitives the protocol relies on. 
This real world is also called the $\F_1$-hybrid world.
Defining \msf{execUC} in our core calculus, and especially reconciling its dynamic nature with our static type system will be the central technical challenge we tackle later.

\paragraph*{\textbf{Composition Notation}}
The UC framework is designed to encourage a highly compositional and modular design approach, where we analyze single-instance protocols in isolation, then apply generic composition operators to build more complex systems.
Encoding the standard theory of UC composition in our core calculus is one of the main ways we validate the expressiveness of our language design.
Here we summarize the main composition theorems using category notation, where objects are functionalities and arrows are protocols.
First, if $\pi$ realizes a single instance of $\F_2$ in the $\F_1$-hybrid world according to the definition of \msf{execUC} above, we use the notation:
\[
	\F_1 \xrightarrow{\pi} \F_2
\]
%This means in the real world $\pi$ makes use of a single instance of $\F_1$, and the ideal world consists of a single instance of $\F_2$.
The first composition theorem \ref{thm:singlecomp} states that this relation is transitive, where $\rho \circ \pi$ is a generic composition operator that combines two protocols by connecting the $\rho$ to $\pi$ where $\rho$'s communication with $\F_2$ is relayed as input to $\pi$. 
%$\rho \Leftarrow \F$ channels in $\rho$ to the $\pi \Leftarrow \Z$ channel of $F$. That is, when protocol $\rho$ communicates with its ideal functionality, it is is relayed as subroutine input to $\pi$.
\begin{theorem}[Composition]\label{thm:singlecomp}
\begin{mathpar}
\inferrule*[right=single-compose]
{
	\F_1 \xrightarrow{\pi} \F_2 \semi \F_2 \xrightarrow{\rho} \F_3 \\
}
{
	\F_1 \xrightarrow{\rho \circ \pi} \F_3
}
\end{mathpar}
\end{theorem}
To prove this theorem requires constructing a straightforward simulator that combines the underlying simulators of $\pi$ and $\rho$, and the complete security reduction involves translating a distinguisher $Z$ for the combined protocol $\rho \circ \pi$ to a distinguisher $Z^*$ for either $\pi$ or $\rho$ individually.
Although the proof is straightforward, the precise statement of it in our framework serves as good validation for the expressiveness of our framework: our theorem and proof are parametric in the session type of the protocol, i.e., the theorem places no restrictions on the communication pattern used by the underlying protocol and functionalities.
We further validate our approach by expressing the multi-session extension, $!\F$, of \F that let's \Z spawn arbitrary parallel sessions of \F and its corresponding composition theorem: 
$!\F_1 \xrightarrow{\pi} \F_2$, $!\F_2 \xrightarrow{\phi} \F_3 \implies !\F_1 \xrightarrow{\phi \circ !\pi} F_3$.

%The flexibility of the UC framework extends to allow \Z to spawn an arbitrary number of sessions of some \F, denoted $!\F$, in parallel, and the theorem 
%$!\F_1 \xrightarrow{\pi} \F_2$, $!\F_2 \xrightarrow{\phi} \F_3 \implies !\F_1 \xrightarrow{\phi \circ !\pi} F_3$ ensures that composition holds even in this setting.
%This theorem allows analysis of a single protocol in isolation to guarantee security under parallel composition, and it is crucial in proving full UC composition that ensures security under composition 
%with parellel sessions of arbitrary protocols.

%The next generic operation is the multi-session extension of $\F$, denoted $!\F$, which provides $\pi$ with an arbitrary number of instances of $\F$, each tagged with a separate \textsf{sid} (for \emph{session identifier}).
%Here is a central aspect of UC's flexibility, that the environment gets to determine at runtime the exact number and values of \textsf{sid}'s, with no static bound required.
%The Universal Composition theorem says that composition even holds in this setting, which we state as
%$!\F_1 \xrightarrow{\pi} \F_2$, $!\F_2 \xrightarrow{\phi} \F_3 \implies !\F_1 \xrightarrow{\phi \circ !\pi} F_3$.
%This theorem is essential to the appeal of UC as a framework because it encourages simple analysis of a single protocol in isolation (the proof that $\pi$ realizes one instance of $\F_2$), which is then safe to use in protocols like $\phi$ that rely on multiple concurrently-running subsessions of it.

\paragraph{Universal Composability and ITMs}

%Our approach, following ILC~\cite{ilc}, is to encode the ITMs framework as faithfully as possible.
%%This is because the final step in a UC proof is to show that a distinguishing environment Z can be leveraged to construct a polynomial time solution to a hard problem like Discrete Log.
%The basic rule that ITMs follow is that only one process is active at a given time. 
%Specifically whenever a process writes to one of its outgoing channels, the unique process that holds the read end of that channel is immediately activated next.
%In this way the message scheduling is essentially deterministic so it can be easily simulated by a sequential computation.
%This discipline means that modeling inherently non-deterministic phenomena, like network schedulers, requires us to explicitly offer choices to an adversary process defined in our model.
The ITM model has the advantage that only one process is active at a given time. It results in message scheduling that is essentially deterministic and easily simulated by sequential computation. 
Despite having an established message pattern, it is still not straightforward to define a notion of polynomial runtime for ITM systems.
We need a way to make local judgements about each of \A, \Z, $\pi$, and \F and conclude the entire $\msf{execUC}$ overall is polynomial time. 
%Looking at the order of quantifiers in the UC emulation definition from Relation~\ref{eqn:emulation}, we need a way to make local judgments about each $\A$, $\Z$, $\pi$, and $\F$ individual, and conclude that the entire $\msf{execUC} \; \Z \; \pi \; \A \; \F$ experiment overall is polynomial time as a result.
%We follow Canetti's approach~\cite{canettiUC}, which is to keep track at runtime of quantity called ``import tokens'' and assign a runtime budget based on these.
We use the ``import tokens''~\cite{canettiUC} notion of polynomial time and track runtime through tokens that are capped by a total and that can be passed between processes. 
%These tokens can be passed among the processes along with the messages sent on each tape, but the total amount of tokens must be conserved (neither created nor destroyed), and locally each process cannot take more steps than (a polynomial of) the amount of import tokens it has stored.
In this model, we say $P$ is \emph{locally polynomial time} if for \emph{any} evaluation context $e[\_]$, at any step $t$ during its execution,
\[
\#\textsf{stepsTaken}(e[P])_{t} \le T(n_{\textsf{in}} - n_{\textsf{out}})
\]
where $n_{\textsf{in}} - n_{\textsf{out}}$ is the net number of import tokens that $P$ possesses, and $T$ is an arbitrary polynomial.
This immediately ensures that if the system starts out with a total number of tokens bounded $n \in poly(k)$, then the overall number of steps taken by any of the processes in the system is also $poly(k)$.
The arbitrary polynomial $T$ serves as a slack parameter that allows, for example, the emulation of a universal turing machine program (which may incur up to quadratic overhead).
Our approach, described in Section~\ref{sec:basic}, is to encode this notion directly into the type system, by tracking import tokens and statically enforcing this polynomial runtime constraint.

