\begin{lemma}\label{lem:enoughimport}

The \textit{weakly-balanced} relaxation guarantees a simulator \mathc{S} will always have sufficient import to delay ideal world codeblocks \textit{at least} as long as the corresponding real world code blocks.

\end{lemma}

\textit{Proof of Lemma \ref{lem:enoughimport}.}

The crux of the argument lies in ensuring that, even when minimal import is provided to the adversary in the ideal world, it has sufficient import to meaningfully delay ideal world codeblocks to ensure simulatability. 
Therefore we set up an extreme scenario:
\begin{itemize}
\item A protocol $\pi$ which schedules $n$ codeblocks and outputs to the environment in the last codeblock. An ideal protocol $\phi$ schedules a single codeblock that outputs the same message to the environment.
\item The simulator must be able to ensure that the ideal world codeblock can execute at the same time as the final codeblock in $\pi$.
\end{itemize}

In the real world with dummy adversary $\mathcal{D}$, $\mathcal{Z}$ needs to \Advance \Wasync $n+1$ times in order to force execute all codeblocks.
Scheduling $n$ codeblocks requires at least $n$ unites of import on the part of the parties, and the adversary in both worlds also receives at least $n$ units of import.
Therefore, in order to ensure indistinguishability, the simulator must be able to delay the ideal world \Wasync $n$, which it can plainly do \footnote{The adverasry is also given $k$, the security parameter, amount of import at the start of execution ensuring that a polynomial time simulator is still polynomial under this new definition.}.

