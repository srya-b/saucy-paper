In this section we give more extensive code on how the multisession extension works. 
We also explore Theorems \ref{thm:squash} and \ref{thm:functor} in greater detail, and, specifically address the proof obligation for both.

\subsection{!\F}
The multisession extension runs instances of the wrapped functionality in a sandbox much like the \partywrapper.
Its type is simple and relies on a simple send/poll message passing introduced in the linear type \inline{comm}.
The reason for such a simple type (below) for the operator is that it needs to be agnostic to the underlying session types 
of the functionality. All it does it route messages and forward them to the shared part of the providerless channel
connecting to the internal instances of \F.

\begin{lstlisting}[basicstyle=\small\BeraMonottFamily, mathescape]
$\yo{type}$ p2ms[a] = P2MS of ssid ^ a ;
$\yo{type}$ ms2p[b] = MS2P of ssid ^ b ;
\end{lstlisting}

Like the \partywrapper virtualizaton is achieved just by spawnig functionalities with the virtual token type \inline{K1}.
From the code below for accepting messages fro the \partywrapper, we can see that the operator's code is alost identical the code in Appendix~\ref{app:arbparties}.
\begin{lstlisting}[basicstyle=\footnotesize\BeraMonottFamily, frame=single, mathescape]
$\nproc$ f_ms[K][K1][s,p2f,f2p,a2f,f2a] :
  $\tg{(* args *)}$
  ($\$$p2f: comm[p2ms[s][p2f]][K]), 
  ($\$$f2p: comm[ms2p[s][f2p]][K]),
  ($\$$a2f: comm[a2ms[s][a2f]][K]), 
  ($\$$f2a: comm[ms2a[s][f2a]][K]),
  (lp2f: [(SID[s], comm[P2F[p2f]][K1])]), 
  (lf2p: [(SID[s], comm[F2P[f2p]][K1])]),
  (la2f: [(SID[s], comm[a2f][K1])]), 
  (lf2a: [(SID[s], comm[f2a][K1])]) |- ($\$$ch: 1) =
{
  $\nmatch$ $\$$p2f, $\$$a2f (
    P2F(pid, P2MS(ssid, m)),* =>
      $\nget$ {p2msn} K $\$$p2f ;
      $\nif$ not exists ssid lp2f $\nthen$
        #p2f' $\leftarrow$ channel_init[K1][P2F[p2f]] ;
        #f2p' $\leftarrow$ channel_init[K1][F2P[f2p]] ;
        #a2f' $\leftarrow$ channel_init[K1][a2f] ;
        #f2a' $\leftarrow$ channel_init[K1][f2a] ;
        $\$$c' $\leftarrow$ Func[K][K1] $\leftarrow$ k rng sid #p2f' #f2p' 
                                 #a2f' #f2a'
        lp2f $\leftarrow$ append (ssid, #p2f') lp2f; 
        lf2p $\leftarrow$ append (ssid, #f2p') lf2p;
        la2f $\leftarrow$ append (ssid, #a2f') la2f; 
        lf2a $\leftarrow$ append (ssid, #f2a') lf2a;
      $\nelse$ ()
      #p2f' $\leftarrow$ search ssid lp2f ;
      $\nwithdraw$ K K1 p2fn ;
      #p2f'.SEND ; $\nsend$ #p2f' P2F(pid, m) ; 
      $\npay$ {p2fn} K1 #p2f' ;
    *,A2MS(ssid, m) => 
      $\tg{(* identical case *)}$
  )
}
\end{lstlisting}

\subsection{Squash} \label{sec:squash}
Recall that the Theorem~\ref{thm:squash} realizes !!\F, a functionality wrapped twice in the multisession operator an indexed by a pair of ssids $(\m{ssid}_1 \product \m{ssid}_2)$ instead of a single \msg{ssid}.
The real world protocol \msf{squash} realizes it with only a single $!\F$. 
As every instance of \F in $!!\F$ is referred to be a unique pair of \msf{ssid}s, it suffices for \msf{squash} to ``squash'' the two into a single \inline{ssid}.

Recall the type \inline{session[a] = SID of String ^ a} used for all session-ids. The \inline{String} is meant to be unique to the session and \inline{a} is protocol-specific and used to encode arbitrary information, such as global parameters, for the functionality in question. 
If the index of some \F in !!\F is $(\msf{SID}(s_1, x), \msf{SID}(s_2, x))$ then the squashed session id for the corresponding instance in !\F has $\msf{SID}(s_1 || s_2, x)$. 
In general we only need that the two strings $s_1, s_2$ are joined with a one-to-one and reversible function. 

\begin{theorem}[Squash Theorem] \label{thm:squash}
	\begin{mathpar}
		\inferrule*[right=squash]
		{
			\textit{well-resource-typed} \; \F
		}
		{
			!\F \xrightarrow{\msf{squash}} !!\F
		}
	\end{mathpar}
\end{theorem}

\begin{proof}
The simulator for this construction is a direct simulation where $\SIM{\m{squash}}$ takes as input one of 
\begin{itemize}
	\item \inline{Z2A2P(pid, P2MS(ssid, msg))}: $\SIM{\m{squash}}$ un-concatenates \inline{ssid} into a pair (\inline{ssid1}, \inline{ssid2}) and forward the message \inline{A2P(pid, P2MS(ssid1, P2MS(ssid2, msg)))} to the \idealP (forwards to !!\F).
	\item \inline{Z2A2F(A2MS(ssid, msg))}: Similar to above, $\SIM{\m{squash}}$ splits the given \inline{ssid} and sends \\ \inline{A2F(A2MS(ssid1, A2MS(ssid2, msg)))}. 
\end{itemize}

On input from \Z, \SIM{\m{squash}} does the following; 
\begin{lstlisting}[basicstyle=\footnotesize\BeraMonottFamily, frame=single, mathescape]
$\nproc$ sim_squash[K][p2f,f2p,a2f,f2a]{p2fn,f2pn,a2fn}:
  $\tg{(* default + usual args *)}$
  ($\$$ztoa: comm[z2amsg]), ($\$$atoz: comm[a2zmsg]) 
  $\vdash$ ($\$$ch: 1) =
{
  ...
  $\ncase$ $\$$ztoa (
    Z2A2P(pid, P2MS(ssid, msg))) =>
      ssid1, ssid2 $\leftarrow$ split ssid
      $\$$atop.SEND ;
      $\npay$ $\$$atop {a2pn} ;
      $\nsend$ $\$$atop A2P(pid, 
                  P2MS(ssid2, P2MS(ssid2, msg)) ;
    Z2A2F(A2MS(ssid, msg)) =>
      ssid1,ssid2 $\leftarrow$ split ssid
      $\$$atof.SEND ; 
      $\npay$ $\$$atof {a2fn} ;
      $\nsend$ $\$$atof A2P(A2MS(ssid1, 
                  A2MS(ssid2, msg))) ;
  ...
}	  
\end{lstlisting}
\end{proof}
The \msf{split} function the inverse of the one-to-one function used to combine two \msf{ssid}s.

\section{Multisession Composition} 
We further require Theorem~\ref{thm:functor} in order to conclude full UC-style composition.
\begin{theorem}[Multisession Composition]\label{thm:functor}
	\begin{mathpar}
		\inferrule*[right=MultiComp]
		{
			\F_1 \xrightarrow{\pi} \F_2
		}
		{
			!\F_1 \xrightarrow{!\pi} !\F_2
		}
	\end{mathpar}
\end{theorem}
%\begin{theorem}[Multisession Composition]
%	\begin{mathpar}
%		\inferrule*[right=MultiComp]
%		{
%			\F_1 \xrightarrow{\pi} \F_2
%		}
%		{
%			!\F_1 \xrightarrow{!\pi} !\F_2
%		}
%	\end{mathpar}
%\end{theorem}
The simulator construction for the Multisession Composition Theorem~\ref{thm:functor} is trivial.
UC guarantees that different session of a protocol or ideal functionality can not share state. 
In the multisession operator, this means that different instances of \F within $!\F$ also don't share any information.
This allows us to construct a simulator for Theorem~\ref{thm:functor} that, for every subsession of \F, creates a corresponding
instance of $\Sim_\pi$.
Messages are routed to the correct instance of $\Sim_\pi$ based on the \inline{ssid} of the message. 

The complete proof for this theorem is more involved than \emph{just} the simulator construction.
Instead, it relies on a hybrid argument that translates an environment $\Z$ that can distinguish the multi-session experiment into an environment $\Z^*$ that can distinguish real from ideal in the single session experiment.

The high level approach is as follows: $\Z^*$ can only interact with only a single instance of $\F_1$ in the real execution, but it must present a view of multiple sessions $!\F_1$ to its sandbox execution of $\Z$.
To do this, $\Z^*$ will locally sandbox subsessions of $\F_1$ for all but one subsession, the one in the real execution.

To complete the precise statement of this proof, we have to resort to (a variable number of) hybrid worlds~\cite{canettiUC,variablehybrids}. We sketch the proof below.
We define a family of environments $\Z^*_i$, indexed by a polynomial $i$,
\begin{itemize}
\item the first $i(k)-1$ subsessions shown to $\Z$ are sandboxed instances of $\F_1$.
\item the $i(k)$'th subsession is the instance running in the ``real'' exection, i.e. not in a sandbox. 
\item the $i(k)+1$'th and later subsessions are sandboxed instance of $\F_1$ replaced by $pi$
\end{itemize}
Notice that $\Z^*_0$ presents a simulation of exactly the multi-session real world, while $\Z^*_q$ presents a simulation of exactly the multi-session ideal world where $q$ is a polynomial bound on the runtime resulting from $\Z$.
What we can show is that by triangle rule, if $\Z$ is a distinguisher for the multi-session experiment, there must exist some polynomial $i$ such that the real world with $\Z^*_i$ is distinguishable from the real world with $\Z^*_{i+1}$. Since the real world with $\Z^*_{i+1}$ is exactly the same as the ideal world with $\Z^*_i$, this gives us that $\Z^*_i$ is a distinguisher for real and ideal in the single-session experiment.

%\subsection{Squash Theorem}
%\begin{theorem}[Squash Theorem]\label{thm:squash}
%	\begin{mathpar}
%		\inferrule*[right=squash]
%		{
%			\textit{well-resource-typed} \; \F
%		}
%		{
%			!\F \xrightarrow{\msf{squash}} !!\F
%		}
%	\end{mathpar}
%\end{theorem}
%The Squash theorem, is more straightforward to prove than Theorem~\ref{thm:functor}.
%
%\begin{proof} (Theorem~\ref{thm:squash})
%On examination the \msf{squash} theorem does constant work per activation and concantenates two \inline{ssid}s into one for messages going to the functionality and does the inverse for messages incoming from the functionality.
%In other words, it is a direct simulation and the same ideal functionality exists if both worlds, and it is clear to see that the this theorem holds with a trivial simulator.
%First we describe the \msf{squash} protocol where $!!\F$ are nested $!$ operators.
%The protocol accepts messages intended for $!!\F$ of type \inline{P2MS[p2ms[a]][ms2p[b]]}, i.e. of the form $(\msf{ssid}_1, (\msf{ssid}_2, msg))$, and ``flattens'' them into a single message of type $\inline{P2MS[a][b]}$, i.e. of the form $(\msf{ssid}_3, msg)$.
%
%In $(\idealP, !!\F)$, \idealP~expects to receive messages of the form $(\msf{ssid}_1, (\msf{ssid}_2, m))$ where $\msf{ssid_2}$ is a sub-session of $\F$ (i.e. instance) inside some $!\F$ with sub-session id $\msf{ssid}_1$ inside of $!!\F$ (the message accesses functionality $\F[\msf{ssid}_1][\msf{ssid}_2]$).
%The \msf{squash} protocol flattens the indexing of instances of \F and combines session ids $\msf{ssid}_1$ and $\msf{ssid}_2$ into a single \msf{ssid}: $\msf{ssid}_3 := \msf{ssid}_1 \cdot \msf{ssid}_2$.
%If follows intuitively that the view for the environment remains the same. 
%
%We construct a simulator such that:
%\[
%\msf{execUC} \, \Z \, \idealP \, !!\F \, \SIM{\msf{squash}} \approx \msf{execUC} \, \Z \, \msf{squash} \, !\F \DA 
%\]
%The simulator is very simple. 
%Inputs to/from parties/\Z for a corrupt party is forwarded unmodified.
%Input intended for $!\F$ of the form $(\msf{ssid}_1 \cdot \msf{ssid}_2, msg)$ is sent as $(\msf{ssid}_1, (\msf{ssid}_2, msg))$ to $!!\F$. 
%Output from $!!\F$ is modified inversely and sent to \Z.
%
%The proposed simulator is trivially analyzed to be \textit{well-resource-typed}.
%It performs constant work per activation and does ``real'' simulation other than message modification to/from $!!\F$.
%\end{proof}
