\paragraph*{\textbf{Choice Operators}}
The internal choice $\ichoice{\ell : A_\ell}_{\ell \in L}$ constructor
is an $n$-ary labeled generalization of the additive disjunction $A \oplus B$.
A process that provides $x : \ichoice{\ell : A_\ell}_{\ell \in L}$ can send
any label $k \in L$ along $x$ and then continue by providing $x : A_k$. The
corresponding process is written as $(\esendl{x}{k} \semi P)$, where
$P$ is the continuation that provides $A_k$.
On the other end of the channel, the client branches on the label received along $x$.
The provider and client are typed according to the following $\oplus R$ and $\oplus L$
rules respectively.
\begin{mathpar}
  \infer[{\oplus}R]
  {\B{\Tokens \semi \Psi} \semi \wt, \D \entailpot{\B{q}}{\B{q'}} (\esendl{x}{k} \semi P) ::
    (x : \ichoice{\ell : A_\ell}_{\ell \in L})}
  {(k \in L) \qquad \B{\Tokens \semi \Psi} \semi \D \entailpot{\B{q}}{\B{q'}} P :: (x : A_k)}
\and
  \infer[{\oplus}L]
  {\B{\Tokens \semi \Psi} \semi \D, (x : \ichoice{\ell : A_\ell}_{\ell \in L})
    \entailpot{\B{q}}{\B{q'}} \ecase{x}{\ell}{Q_\ell}_{\ell \in L} :: (z : C)}
  {(\forall \ell \in L) \qquad \B{\Tokens \semi \Psi} \semi \wt, \D, (x : A_\ell)
    \entailpot{\B{q}}{\B{q'}} Q_\ell :: (z : C)}
\end{mathpar}
Additionally, the provider should possess the write token to be able to send the
label $k$. Dually, the client receives the write token with the label to continue
execution.

Operationally, since communication is asynchronous, the process
$(\esendl{c}{k} \semi P)$ sends a message $k$
along $c$ and continues as $P$ without waiting for it to be received.
As a technical device to ensure that consecutive messages on a
channel arrive in order, the sender also creates a fresh continuation
channel $c'$ so that the message $k$ is actually represented as
$(\esendl{c}{k} \semi \fwd{c}{c'})$ (read: send $k$ along $c$ and
continue as $c'$). The provider substitutes $c'$ for $c$ enforcing
that the next message is sent on $c'$.
The work counter of the process remains unaltered, and the new message
is created with work $0$.
\begin{tabbing}
$(\oplus S) : $ \=$\proc{c}{w, \esendl{c}{k} \semi P} \step$ \qquad \qquad \qquad \qquad \fresh{c'}\\ 
\>$ \quad \proc{c'}{w, [c'/c]P},\msg{c}{0, \esendl{c}{k} \semi \fwd{c}{c'}}$
\end{tabbing}
When the message $k$ is received along $c$, the client selects branch
$k$ and also substitutes the continuation channel $c'$ for $c$, thereby
ensuring that it receives the next message on $c'$. This implicit
substitution of the continuation channel ensures the ordering of the
messages.
The client process also collects the work performed by the message, if
there is any.
\begin{tabbing}
$(\oplus C) :$ \= $\msg{c}{w, \esendl{c}{k} \semi \fwd{c}{c'}},
\proc{d}{w', \ecase{c}{\ell}{Q_\ell}}$ \\ 
$\step \proc{d}{w+w',[c'/c]Q_k}$
\end{tabbing}

The dual of internal choice is \emph{external choice} $\echoice{\ell :
A_\ell}_{\ell \in L}$, the $n$-ary labeled generalization of the
additive conjunction $A \with B$. This dual operator simply reverses
the role of the provider and client. The provider process of
$x : \echoice{\ell : A_\ell}_{\ell \in L}$ branches on receiving a label
using the expression $\ecase{x}{\ell}{Q_\ell}_{\ell \in L}$,
while the client sends one such label in $L$ using the expression $(\esendl{x}{k} \semi P)$.

\begin{mathpar}
  \infer[\with R]
  {\B{\Tokens \semi \Psi} \semi \D \entailpot{\B{q}}{\B{q'}} \ecase{x}{\ell}{P_\ell}_{\ell \in L} ::
    (x : \echoice{\ell : A_\ell}_{\ell \in L})}
  {(\forall \ell \in L) \qquad \B{\Tokens \semi \Psi} \semi \wt, \D
    \entailpot{\B{q}}{\B{q'}} P_\ell :: (x : A_\ell)}
\and
  \infer[\with L]
  {\B{\Tokens \semi \Psi} \semi \wt, \D, (x : \echoice{\ell : A_\ell}_{\ell \in L})
    \entailpot{\B{q}}{\B{q'}} \esendl{x}{k} \semi Q :: (z : C)}
  {\B{\Tokens \semi \Psi} \semi \D, (x : A_k) \entailpot{\B{q}}{\B{q'}} Q :: (z : C)}
\end{mathpar}
Dual to internal choice, the client contains the write token which is
sent to the provider along with the label.

The provider receives the branching label $k$ sent by the provider. Both
processes perform appropriate substitutions to ensure the order of messages
sent and received is preserved.
\[
\begin{array}{lll}
(\with S) & \proc{d}{w, \esendl{c}{k} \semi Q} \step \msg{c'}{0, \esendl{c}{k} \semi \fwd{c'}{c}}, \\
& \qquad \proc{d}{w, [c'/c]Q} \quad \qquad \qquad \qquad \fresh{c'} \\
(\with C) & \proc{c}{w, \ecase{c}{\ell}{Q_\ell}_{\ell \in L}}, \\
& \qquad \msg{c'}{w', \esendl{c}{k} \semi \fwd{c'}{c}} \step \\
& \qquad \proc{c'}{w+w', [c'/c]Q_k}
\end{array}
\]

\paragraph*{\textbf{Termination}}
The type $\one$, the multiplicative unit of linear logic, represents
termination of a process, which is not allowed to use
any linear channels. A terminating process offering on $x : \one$ simply
closes channel $x$ using the expression $\eclose{x}$ while the client waits
for this close message to arrive using the expression $\ewait{x} \semi Q$
and then continues executing $Q$.
\begin{mathpar}
  \infer[{\one}R]
  {\B{k \semi \Tokens \semi \Psi} \semi \wt \entailpot{\B{q}}{\B{q'}} \eclose{x} :: (x : \one)}
  {\B{q = 0}}
  \and
  \infer[{\one}L]
  {\B{k \semi \Tokens \semi \Psi} \semi \D, (x : \one) \entailpot{\B{q}}{\B{q'}} (\ewait{x} \semi Q) :: (z : C)}
  {\B{k \semi \Tokens \semi \Psi} \semi \wt, \D \entailpot{\B{q}}{\B{q'}} Q :: (z : C)}
\end{mathpar}
Similar to internal choice, the closing process transfers the write
token to its waiting client along with the close message.
Additionally, the terminating process does not store
any potential since it cannot take any further execution steps
(explained more in Section~\ref{sec:import}).

Operationally, the provider converts into a closing message
with no continuation since the offered channel terminates.
\begin{tabbing}
$(\one S) : \proc{c}{w, \eclose{c}} \step \msg{c}{w, \eclose{c}}$ \\
$(\one C) : \msg{c}{w, \eclose{c}}, \proc{d}{w'\ewait{c} \semi Q} \step$ \\
$ \qquad \qquad \qquad \proc{d}{w+w', Q}$
\end{tabbing}



\paragraph*{\textbf{Exchanging Functional Data}}
So far, we have discussed the channels in $\D$ in the typing judgment for NomosUC.
Now, we turn out attention to the functional layer $\Psi$ that contains the
traditional data structures and values.
Communicating a \emph{value} of the functional fragment along a channel
is expressed at the type level by adding the following two session types.
\begin{center}
\begin{minipage}{0cm}
\begin{tabbing}
$A ::= \ldots \mid \tau \arrow A \mid \tau \product A$
\end{tabbing}
\end{minipage}
\end{center}
Here, $\tau$ describes a functional type, e.g. $\m{int}, \m{bool}, \tau \; \m{list}$, etc
(we assume the language contains standard functional types).
The type $\tau \arrow A$ prescribes receiving a value of type $\tau$
with continuation type $A$, while its dual $\tau \product A$ prescribes
sending a value of type $\tau$ with continuation $A$. The corresponding
typing rules for arrow ($\arrow R, \arrow L$) are given below.
\begin{mathpar}
  \infer[\arrow R]
  {\B{\Tokens} \semi \Psi \semi \D \entailpot{\B{q}}{\B{q'}}
  \erecvch{x}{v} \semi P :: (x : \tau \arrow A)}
  {\B{\Tokens} \semi \Psi, (v : \tau) \semi \wt, \D \entailpot{\B{q}}{\B{q'}}
  P :: (x : A)}
  %
  \and
  %
  \inferrule*[right = $\arrow L$]
  {\B{r' = p+q'} \qquad
  \B{\Psi \share (\Psi_1, \Psi_2)} \qquad
  \Psi_1 \exppot{\B{p}} M : \tau \\
  \B{\Tokens} \semi \Psi_2 \semi \D, (x : A) \entailpot{\B{q}}{\B{q'}}
  Q :: (z_k : C)}
  {\B{\Tokens} \semi \Psi \semi \wt, \D, (x : \tau \arrow A)
  \entailpot{\B{q}}{\B{r'}} \esendch{x}{M} \semi Q :: (z : C)}
\end{mathpar}
As indicated in the $\arrow R$ rule, receiving a value $y : \tau$ on a channel
$x : \tau \arrow A$ adds it to the functional context $\Psi$. On the
other hand, sending (value of) expression $M$ on channel $x : \tau \arrow A$
requires that $M$ has type $\tau$ (third premise).
The premises indicated in blue describe how potential is divided across
the functional and session-typed layers and will be described further in Section~\ref{sec:import}.
Intuitively, the potential in functional context $\Psi$ is \emph{shared}
between $\Psi_1$ and $\Psi_2$ (second premise); $\Psi_1$ is used to type
$M$ while $\Psi_2$ is passed on to the continuation $Q$.
The $\product$ operator is dual to $\arrow$ reversing the roles of provider and client,
and we omit those rules for brevity.



\paragraph*{\textbf{Shared Channels}}
Shared session types impose an \emph{acquire-release} discipline on processes; 
a client must acquire the channel offered by a shared process to interact with it
and must release this channel after the interaction.
The corresponding typing rules are
\begin{mathpar}
  \infer[\up L]
  {\Tokens \semi \Psi \semi \wt, \D, (x : \up A_L)
  \entailpot{q}{q'} \eacquire{y}{x} \semi Q :: (z : C)}
  {\Tokens \semi \Psi \semi \D, (y : A_L)
  \entailpot{q}{q'} Q :: (z : C)}
  %
  \and
  %
  \infer[\up R]
  {\Tokens \semi \Psi \semi \D \entailpot{q}{q'}
  \eaccept{y}{x} \semi P :: (x : \up A_L)}
  {\Tokens \semi \Psi \semi \wt, \D \entailpot{q}{q'} P :: (y : A_L)}
\end{mathpar}
The $\up L$ rule describes a client acquiring a shared channel $x$
and obtaining a private linear channel $y$ along which it can communicate
with the corresponding acquired process.
Correspondingly, the $\up R$ rule describes the shared process
accepting the acquire request and creating the fresh linear channel $y$.
The release-detach rules corresponding to the $\down$ type constructor
are exact dual of acquire-accept.
