In this section we expand on the real world protocol that realizes \Fcom in the \Fropp-hybrid model.
The type of \Fropp in Figure~\ref{fig:fropptype} builds on the \Fro type presented in Section~\ref{sec:commitment}. 
Because the receiver must receive a message \Fropp and potentially send it a hash query, we split communication between two uni-directional channels.
The adversary type with the functionality which that lets it query hashes isn't shown but requires 1 unit of import to be sent along with a query. 

\begin{figure}
\begin{center}
	\parbox{0cm}{
	\begin{tabbing}
		$\m{sender}[a] =  \ichoice{ $\=$ \mb{query}: \textcolor{red}{\paypot^1} \m{PID} \product \tgr{\m{Int}} \product \m{sender}[a],$ \\
		\>$\mb{sendmsg}: \textcolor{red}{\paypot^1} \m{ PID} \product \m{a} \product \m{sender}[a]}$ \\
		$\m{shash} = \echoice{ \mb{hash}: \m{ PID} \arrow \tgr{\m{Int}} \arrow \m{hash}}$ \\
		$\m{rquery} =  \ichoice{ \mb{query}: \textcolor{red}{\paypot^1} \m{PID} \product \tgr{\m{Int}} \product \m{rquery} }$ \\
		$\m{receive}[a] = \echoice{ $\=$\mb{hash}: \m{PID} \arrow \tgr{\m{Int}} \arrow \m{receive}[a],$ \\
		\>$\mb{msg}: \textcolor{red}{\getpot^1} \m{PID} \arrow \m{a} \arrow \m{receive}[a]}$
	\end{tabbing}}
\end{center}
\caption{The two types for each of the committer and receiver. The first to types are the senders type to \Fropp and the next is received from \Fropp, and the same holds for the next two and the receiver. Two units of import are sent with a message from the committer. One is sent to the receiver so that it has enough import to query the oracle and check the commitment.}
\label{fig:fropptype}
\end{figure}

Due to the fact that there is a single channel connecting the \partywrapper and the functionality, the amount of import send from the \partywrapper to the wrapped \F is constant. Similarly, a constant amount of import is sent back from \F to the \partywrapper.
If we sendt the most import ever needed, as specified by the session type in Figure~\ref{fig:fropptype}, it would result in the \partywrapper sending 1 import to \Fropp with every message and receiving 1 import with from \Fropp with every message--resulting \Fropp having 0 net import.
Therefore, alongside the functional message types the user must specify for the shared parts of providerless channels they must also determine an assignment of import with the messages that satisfies the session types of the wrapped processes and allow for them to have enough import to send to othes.

In the case of the commitment protocol, the assignment might look something like this.
The intended behavior of the session type is for a random oracle query to cost 1 unit of import, and for the committer to send 1 import to the receiver so that it can ``spend'' it to check the commitment with the random oracle. 
Firstly the import sent to/from the \partywrapper must ensure that it has enough for both the sender and receiver's queries to the random oracle. 
A satisfying import assignment is that $\pi_\msf{com}$ requires 4 units of import on \mb{commit} from \Z to the \partywrapper, and it requires 2 units of import to be sent from \partywrapper to \Fropp. Why?
We need to ensure that the generic random oracle always has at least 1 net import token to handle a possible polynomial number of queries and storage. 
Furthermore, after the committer queries the \Fro and sends a message to the receier the \partywrapper has sent 4 import tokens and received 1 in return leaving it a net of 1 import tokens. 
When the message to the receiver is delievered, the \partywrapper gets back 1 unit of import. When the receiver queries the random oracle to check the commitment, the \partywrapper now has the 2 import tokens required to send a message to \Fropp and complete the protocol. 

Though complicated, the assignment of import above attempts to use minimal number of tokens. In fact, a simple heuristic for assignments ensures that all processes always have 1 net import token left over so that they can always perform polynomial work and any additional tokens are used to send to other processes. 
Recall from Section~\ref{sec:basic} why this isn't a concern: capturing tight runing bounds or timing guarantees aren't an intended goal of the import mechanism in UC or NomosUC.

\begin{figure}
\begin{lstlisting}[basicstyle=\footnotesize\BeraMonottFamily, frame=single, mathescape]
$\tg{(* committer code after receiving a 'commit'}$
        $\tg{message from the environment *)}$
b = $\nrecv$ $\$$z2p ;
$\nget$ $\$$z2p {2} ;
bits = sample (k-1) rng ;
$\$$p2f.query ;
$\npay$ K {2} $\$$p2f ;
$\nsend$ $\$$p2f pid ;
$\nsend$ $\$$p2f (b || bits) ;
$\ncase$ $\$$f2p (
  hash => pid = $\nrecv$ $\$$p2f ; 
    h = $\nrecv$ $\$$p2f ;
    $\$$p2f.sendmsg ;
    $\nsend$ $\$$p2f pid 2 hash;
	$\npay$ K {2} $\$$p2f ;
\end{lstlisting}
\caption{The code for the committer in $\prot{com}$ when it receives a \msf{commit} message from \Z. It obtains a hash of the message from \Fropp over \msf{p2f} and sends it to the receiver (pid=2) through the same functionality.}
\label{lst:committer}
\vspace{-2mm}
\end{figure}

\begin{figure}
\begin{lstlisting}[basicstyle=\footnotesize\BeraMonottFamily, frame=single, mathescape]
$\tg{(* receiver waiting for the commitment opening}$
        $\tg{from the random oracle channel *)}$
sender = $\nrecv$ $\$$f2p ;
p = $\nrecv$ $\tm{recv}$ $\$$f2p ;
b:hs = p
$\nget$ $\$$f2p {1} ; 
$\tg{...}$
$\tg{(* query the hash of b || hs with 1 import *)}$
$\tg{...}$
h = $\nrecv$ $\$$p2f ;
$\nif$ h == hash
$\nthen$
  $\$$z2p.open
$\nend$
\end{lstlisting}
\caption{The code for the receiver checks for a new message and receives the bit and nonce from the committer. If the hash of the bit and nonce matches the commitment it received, it returns \msf{open} to \Z to confirm the commitment.}
\label{lst:receiver}
\vspace{-3mm}
\end{figure}

In Figures~\ref{lst:committer} and \ref{lst:receiver} we see the code for the committer reacting to a $\mb{commit}$ message from \Z and the receiver reacting to an open commit from the committer, respectively. 

%\subsection{Simulation}
%Finally, we present a simulator \simcom, for the dummy adversary, for which the \Fcom is realized by \prot{com} in the \Fropp-hybrid world.
%Recall that the import requirements for the ideal world, in this case for \Fcom. Therefore, the simulator is parameterized by import parameters required in the real world for the parties of $\pi_\m{com}$ and \Fro.
%The simulator is straightforward and internally maintains a table like \Fro and responds to the environments queries for hashes. 
%When the receiver is corrupt:
%\begin{itemize}
%\item \simcom responds with \inline{P2A2Z(2, no)} to all messages by \Z to get a message from the functionality
%\item On \inline{Committed} by the ideal receiver, \simcom generates a random $r$ and sends \inline{P2A2Z(2, RHash(h))} with no import.
%\item On \inline{Open(b)} from the ideal receiver, \simcom generates a random nonce $x$ and stores \inline{b+x : h} in its \Fro table, and sends \inline{Yes(1, (b,x))} to \Z when asked for messages for the corrupt receiver.
%\end{itemize}
%
%The corrupt committer is not much different from the above case. In this case
%the simulator stores the bit $b$, the none $x$, the corresponding hash $h$, and the import that \Z uses to create a commitment.
%When the simulator receives the message to send the commitment to the receiver, it tells the ideal world committer to commit to $b$ along with 2 import given by \Z, and when it's told to open the commitment it opens it in the ideal world. 
%
%It is immediately clear that this simulator satisfied $\Fro \xrightarrow{\prot{com}} \Fcom$ for the dummy adversary.

